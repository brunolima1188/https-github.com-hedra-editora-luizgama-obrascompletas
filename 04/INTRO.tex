\chapter*{Introdução}
\addcontentsline{toc}{chapter}{Introdução, \emph{por Bruno Lima}}

\begin{flushright}
\textsc{bruno lima}
\end{flushright}

Nas linhas quase apagadas de um velho jornal carioca, lê-se uma
revelação que joga luz sobre a obra do jornalista e advogado Luiz Gama:
segundo Lúcio de Mendonça, no ano de 1868, Gama assinava textos com o
pseudônimo Afro.

Mas quais textos? Onde eles estão? O que eles dizem?

Até hoje, os especialistas não os encontraram. A afirmação categórica de
Mendonça permanece, todavia, no vácuo da dúvida historiográfica. De
todos os esforços, apenas um texto apareceu. Porém, isolado e sem
contexto, não se levou à frente qualquer conclusão séria sobre sua
autoria.

As respostas, contudo, moram nos detalhes. E aqui ganham valor as tais
linhas quase apagadas do velho jornal carioca. Afinal, elas registram o
depoimento da única testemunha que relatou os fatos que ora se desvelam.

Puxando os fios da memória como quem anda num quarto escuro, Mendonça,
amigo e confidente de Gama, contou num folhetim que marcou época alguns
lances que presenciou e outros que ouviu dizer, todos referentes à vida
do líder abolicionista. Alguns acontecimentos contavam mais de dez anos.
É natural, portanto, que a memória ora aplique de seus truques e ora
reviva com clareza nuances antes fugidias. Para o momento, nos
interessam aqueles fatos que Mendonça testemunhou e que só ele trouxe a
público.

Atentos aos detalhes, então, vejamos como Mendonça recorda ter conhecido
o amigo: ``Nesse ano de 1868, conheci Luiz Gama. Vi-o, se bem me lembra,
a primeira vez, na tipografia do diário liberal \emph{O Ypiranga}''. Se a
primeira frase é taxativa, identificando 1868 como o ano exato, a frase
que vem em seguida vacila --- ``se bem me lembra'' --- quanto ao local do
encontro (e ao que Gama fazia lá). A afirmação que vem na sequência
reitera o ano do encontro: ``No ano seguinte, lembro-me dele entre os
redatores do \emph{Radical Paulistano}'', jornal republicano que teve
vida curta e agitada ao longo de 1869 e início de 1870. Aqui, como se
vê, a lembrança --- ``lembro-me dele'' --- não escorrega: Luiz Gama, de
fato, foi um dos redatores do \emph{Radical Paulistano}
(voltaremos a isso mais adiante) entre abril de 1869 e janeiro de 1870.

Mas se Mendonça acerta a linha do tempo, erra no arremate. Para ele,
teria sido no \emph{Ypiranga} que Gama ``foi colaborador da folha, onde
assinava com o pseudônimo Afro''. Ao menos desde a década de 1930,
especialistas na obra de Gama reabrem as páginas amareladas do 
\emph{Ypiranga} procurando os artigos assinados por Afro. Trabalho 
em vão. O testemunho de Mendonça falha justamente no ponto em que admite
não estar seguro, isto é, quanto ao local do encontro e, por extensão,
quanto à forma com a qual Gama colaborava com o jornal.

Depois de reviradas as páginas do \emph{Ypiranga}, sem maior sucesso na
busca por Afro, por que não esmiuçar os outros jornais
paulistanos publicados em 1868? É uma boa pergunta e que apenas
pesquisas futuras podem dar conta em toda a amplitude e minúcia que se
requer. No entanto, por critérios temáticos e temporais, isto é, pela
escolha de alguns veículos de imprensa e a partir de determinados
debates sociais em evidência, pode-se chegar ao menos a mais dois textos
assinados por um certo Afro, totalizando agora três artigos,
somados com o único antes localizado pelos especialistas. É pouco? Sim,
muito pouco, não encorajando que se tome nenhuma conclusão a respeito da
autoria. Ademais, nenhum dos três artigos é de 1868, mas sim de 1866 e
1867, o que abre janelas para uma nova periodização, por um lado, mas
escapa, por outro lado, do testemunho de Mendonça no que ele tem de mais
assertivo: o ano em que Gama escrevia como Afro.

Realmente, tratar do problema da autoria na imprensa brasileira da
segunda metade do século \textsc{xix} é como caminhar em um território pedregoso.
Num mundo de nomes, pseudônimos, conflitos, assuntos e interesses
partidários difíceis de se compreender e caracterizar, o leitor deve
redobrar a atenção. Periódicos surgiam e sumiam em semanas. Alguns
jornais mais longevos, por sua vez, mudavam de linha editorial
repentinamente, quase sempre em razão de algum evento político, como
alguma sacudida no parlamento, troca de comando na administração
provincial ou mesmo uma simples eleição de juiz de paz que acabava em
sangue e troca de tiros. Enquanto as máquinas dos partidos do Império se
revezavam nos ministérios, no Legislativo e nas províncias, a imprensa,
geralmente a reboque do partido da ocasião, vacilava entre um e outro,
liberais e conservadores, todos convergentes no fundamental quando o
assunto era a nefasta prosperidade da escravidão negra.

\section{São Paulo, 1866--1868}

Se o triênio 1866--1868 pode ser indicado, de modo geral, como um ponto
de inflexão na luta político-partidária do Império, também pode ser
visto, em particular, como uma nova etapa do debate de ideias na
imprensa, sobretudo a partir do surgimento do movimento republicano como
uma terceira força política relevante. A guerra no Paraguai, a
dissolução traumática do gabinete de Zacarias de Góis com a imediata
promoção dos conservadores na chefia do Executivo, além do cenário
internacional refeito pela abolição da escravidão nos Estados Unidos da
América, colocavam na ordem do dia temas espinhosos como o papel do
Estado na guerra, a soberania nacional do Brasil, os limites da
representação política no parlamento, assim como a expansão da
cafeicultura e a novas exigências para a sustentação da política da
escravidão.

Em São Paulo, cidade que começava a alcançar os trinta mil habitantes,
um jornal humorístico e ilustrado, coisa rara naquele tempo, capturava
essas e outras questões sociais pelo viés liberal-progressista e
antimonarquista. As imagens e os textos satíricos do \emph{Cabrião}
divertiam seus leitores e incomodavam fundo seus opositores, que
inclusive os processaram numa fracassada tentativa de censura. Hoje
as páginas do \emph{Cabrião} são documentos de uma época. Suas
crônicas testemunham de perto um período desse triênio, entre setembro de
1866 e outubro de 1867, no qual durou o semanário humorístico, e
deixam pistas de um outro que lhe sucederia na parte restante do
triênio: o jornal \emph{Democracia}, publicado de dezembro de 1867 até
julho de 1868.

Se é correto relacionar a temporalidade de veículos de imprensa com a
ascensão de determinados grupos políticos no poder, podemos traçar uma
linha entre a posse de Zacarias de Góis na chefia do Executivo, em
agosto de 1866, e a criação do \emph{Cabrião} no mês seguinte, em
setembro de 1866. Se a correlação entre temporalidades procede, podemos
estender essa linha até a queda do gabinete liberal-progressista, via
intervenção direta do imperador Pedro \textsc{ii}, e veremos cair ao mesmo tempo
o domínio liberal-progressista e o jornal \emph{Democracia}, espécie de
sucessor do \emph{Cabrião}, no mês de julho de 1868.

Assim, a voz do liberalismo radical paulista nos debates públicos
coincidiria exatamente com o tempo que Zacarias de Góis presidiria o
gabinete dos ministros e, por extensão, supervisionava as províncias,
visto que as indicações locais --- presidente de província, chefes de
polícia, juízes de direito, etc. --- passavam por sua caneta.

Em outras palavras, o \emph{Cabrião} surgiu com a ascensão
liberal-progressista ao poder central, cresceu na turbulência política
que avassalava o país, rachou aos estilhaços como o próprio Partido
Liberal nos finais de 1867, e uma dessas frações reorganizou-se em outro
veículo de imprensa, agora chamado \emph{Democracia}, que, por sua vez,
duraria tão somente oito meses, isto é, o tempo final que os liberais
ficaram no poder.

A linha temporal do início ao fim do ciclo
\emph{Cabrião}-\emph{Democracia} conectada com eventos da política
nacional é mais fácil de se traçar. Difícil, porém, é captar as
dinâmicas da luta intrapartidária que levaram os liberais a se
fragmentarem em grupos distintos, num movimento que se revelou
irreversível com o surgimento de associações republicanas locais, como
clubes e jornais, até a fundação do Partido Republicano, em 1873.

Uma imagem, contudo, expressa com nitidez a cisão interna do Partido
Liberal às vésperas da ruptura. O lendário artista Angelo Agostini teve
a rara felicidade de retratar esse instante político com a maestria que
o tornou conhecido como um ``poeta do lápis''. Estampada no \emph{Cabrião}
em fevereiro de 1867, a ilustração apresenta as principais figuras do %colocar a ilustração no texto talvez?
Partido Liberal divididas em dois grandes grupos: os liberais moderados
e os liberais radicais. Ao centro, a personagem-símbolo que dava nome ao
jornal, o \emph{Cabrião}, fazia que apartava a iminente briga com a
bandeira da unificação partidária desfraldada com os seguintes dizeres:
``Viva o Partido Liberal\,/\,A União faz a força''.

Ao lado direito do \emph{Cabrião}, entre outros chefes do partido, os
moderados José Bonifácio, o Moço, ex-ministro e então deputado, além de
Silva Carrão e Joaquim Floriano, ambos ex-presidentes da província de
São Paulo. Ao lado esquerdo, para variar, Luiz Gama à frente de uma
pequena multidão de liberais dissidentes em que se achavam, recuados, o
jornalista Américo de Campos e Martim Francisco, ministro da Justiça do
gabinete Zacarias.

A litogravura de Agostini é rica em sinais. Todos na imagem carregam um
porrete. Apenas um deles ameaça a outra ala: o de Luiz Gama. Todos na
tela estão de gravata ou camisa fechada: só Gama a tem aberta e, além
disso, com a manga já arregaçada. Enquanto José Bonifácio, líder do
bloco dos liberais moderados, segura sua respectiva bandeira fechada, do
lado oposto tremula a bandeira dos ``Liberais Dissidentes'' carregada por
Luiz Gama. Todos, por fim, estão com suas bocas fechadas. Menos o
\emph{Cabrião} e Gama.

Lá atrás, o \emph{Cabrião} abria a boca para pedir calma para os
liberais radicais e, quem sabe, salvar a unidade partidária. Hoje,
contudo, segue dizendo algo incômodo para os que minimizam o papel de
Gama na formação das ideias republicanas no Brasil. Na pena de Agostini,
o único negro do quadro branco assumia a liderança da dissidência
liberal, insistindo que o Partido Liberal investisse em bandeiras-chave
para o desenvolvimento nacional, como a reconquista da soberania
popular, surrupiada pelo imperador desde a Carta outorgada de 1824, a
separação absoluta entre Igreja Católica e Estado, além da extinção da
escravidão. O~líder do liberalismo radical em São Paulo no triênio
1866--1868 era, sem dúvidas, Luiz Gama.

\section{Afrodemocracia}

Foi a ala dissidente do Partido Liberal que fundou o periódico
\emph{Democracia,} em 1º de dezembro de 1867. As eleições locais, a
composição da nova Assembleia Provincial que assumiria os trabalhos no
início de 1868, e a troca do presidente da província de São Paulo,
saindo Tavares Bastos para a entrada de Saldanha Marinho, liberais de
longa data, porém, de fileiras no momento inconciliáveis, influíram
certamente na decisão de fundar um jornal que pretendesse radicalizar o
debate público.

O relógio político tem suas astúcias: se um ponteiro mais lento marca
mudanças mais duradouras, outro mais rápido distribui pontadas no
dia a dia da política. Enquanto os liberais-progressistas com todas as
suas divergências e fricções duravam nos ministérios, antes que os
militares e conservadores os assaltassem no ``golpe de Estado de 16 de
julho'', alas liberais rebeladas preparavam o dia de amanhã --- antes que
a eventual perseguição os alcançasse ---, forçando os ponteiros da
aceleração histórica com a inclusão da indesejável pauta republicana na
esfera pública de um país monarquista.

Assim, na velocidade do tempo político de finais de 1867, fechou-se um
jornal, abriu-se outro, e parte da redação de um pulou para o seguinte
em questão de semanas, as mesmas semanas que noticiam a substituição de
comando no Executivo paulista. Do \emph{Cabrião} à \emph{Democracia,}
uma mesma tipografia em comum: a Imparcial, de Azevedo Marques,
jornalista e editor português radicado em São Paulo. Um novo formato,
contudo, escancarava diferentes projetos e objetivos entre ambos. Se o
ilustrado \emph{Cabrião} malhava os costumes da província, a
\emph{Democracia} era pragmática, tinha uma linguagem programática para
o fim da monarquia e da escravidão, não investindo na sátira
sequer como recurso retórico. Ambos, ao fim e ao cabo, mais do
que compartilharem as mesmas ideias liberais-radicais, eram formas
distintas para um programa em comum.

Mas que jornal é esse que não se vê citado em canto algum, nem mesmo na
excelente História da Imprensa no Brasil?

Apenas um estudo, curto e magistral, da historiadora Raquel Glezer, abre
pistas, perguntas e respostas.

Comecemos por aplainar o terreno que lá atrás se viu pedregoso. Glezer
constatou que o jornal \emph{Democracia} era ``uma publicação rara e %trazer referência completa da citação
pouco conhecida, quer pelos especialistas em história da imprensa, quer
pelos estudiosos da história das ideias, da história literária ou da
história da cultura no Brasil''. Mais à frente, notou três pistas úteis
--- a terceira delas fatal para a conclusão que aqui se elabora.

Primeiro, que ``\emph{Democracia}, de modo bastante original, não traz
expediente de redação, nome de proprietário ou responsável pela edição'',
assim como não publica anúncios. Segundo, que ``pela época em que
foi editado podemos deduzir que não fazia parte dos jornais acadêmicos,
pois estes viviam durante o período letivo e morriam nas férias
escolares. Um jornal publicado de dezembro em diante devia ter em mente
um outro tipo de público que não o exclusivamente acadêmico''. E
realmente tinha em mente outro público. Como se percebe desde seus
primeiros números, os leitores que se objetivava alcançar eram aqueles
que se interessavam pelos debates legislativos de 1868, sobretudo um
projeto de lei de que falaremos adiante.

``Quanto aos colaboradores'', arremata Glezer, ``há necessidade de estudos
mais aprofundados, pois via de regra usaram pseudônimos --- Afro, Ultor,
Graccho, O Sertanejo --- que tornam difícil a identificação imediata''. A
essa hora, estamos a um passo de apurar a autoria textual de ao menos um
dos pseudônimos, Afro, seguindo afinal a orientação da historiadora para
se aprofundar os estudos sobre os colaboradores do semanário.

Sem imediatismos, portanto, façamos um break e saltemos juntos doze
anos, até o início da década de 1880.

No \emph{Almanaque literário de S.\,Paulo para o ano de 1881},\footnote{Publicação periódica anual com informações variadas da vida política,
  administrativa, comercial, cultural e literária. O \emph{Almanaque
  literário de S.\,Paulo para o ano de 1881}, edição a que Gama se
  refere, foi lançado por José Maria Lisboa, figura de destaque na
  imprensa paulista do século \textsc{xix}.} Luiz Gama resolve dar publicidade a
uma carta que trazia consigo guardada há muito tempo. Muita água já
havia rolado debaixo da ponte do Tamanduateí; possíveis feridas
cicatrizadas e amizades rompidas, quem sabe, refeitas. O ``fervoroso
empenho'' do velho Lisboa convenceu Gama a enviar-lhe qualquer escrito em
prosa de José Bonifácio que tivesse em seus arquivos. A resposta sóbria
e concisa vale ser lida na íntegra.

\begin{quote}
\forceindent{}Meu caro Lisboa,

Ao fervoroso empenho que hoje manifestaste-me, de publicares no teu bem
aceito \emph{Almanaque de S.\,Paulo} algum escrito em prosa da pena do
exmo. conselheiro José Bonifácio, correspondo enviando-te de pronto o
único que possuo, que tenho como riqueza e que guardo como avarento;
\textit{é uma carta datada de 26 de abril de 1868, um precioso documento
literário e político, endereçado a um amigo, quando redator da}
Democracia\textit{, periódico partidário que aqui se publicava.}

Essa carta acompanhou a célebre poesia --- ``Primus inter pares''\footnote{Primeiro entre iguais.} --- por ele escrita e dedicada ao bravo capitão
Arthur Silveira da Motta;\footnote{Arthur Silveira da Motta (1843--1914)
  foi escritor, historiador e militar. Considerado herói na Guerra do
  Paraguai (1865--1870), reformado como almirante, foi também membro da
  Academia Brasileira de Letras (1907). O fato de Gama o citar enquanto
  capitão reforça que a admiração não é exatamente pelo almirante, isto
  é, pela carreira posterior à Guerra do Paraguai, que até inclui a
  outorga de um título de baronato, espécie de comenda que Gama,
  antimonarquista convicto, refutava. A admiração de Gama --- e de José
  Bonifácio --- é ao mais jovem capitão de mar e guerra promovido por
  atos de bravura da história da Marinha brasileira.} é gema preciosa
pouco conhecida e que por certo te dará no goto.\footnote{Cair nas
  graças, cair no gosto, conquistar a simpatia.}

\begin{flushright}
Teu

\textsc{luiz gama}
\end{flushright}
\end{quote}

Agora é definitivo: podemos abrir as páginas da \emph{Democracia} com a
certeza de que Gama não só conhecia o ``periódico partidário'', como o
conhecia por dentro, possuindo ``como riqueza'' um ``precioso documento
literário e político''. Que amigo é esse que tinha em mãos documentos
privados de uma empresa extinta e liquidada doze anos antes? Na edição
de 2 de maio de 1868, o \emph{Democracia} publicou a íntegra da carta e
da ``célebre poesia'' tal qual Gama enviou para Lisboa. O publicado no
\emph{Almanaque} em 1881 corresponde exatamente ao publicado pelo
\emph{Democracia} em 1868. Nota-se, contudo, uma única diferença: na
publicação de 1868, não há o nome nem nada que remeta a José Bonifácio.
Apenas um pseudônimo assume a carta: Cincinatus. O documento,
além de tudo, era secreto. Afinal, que amigo é esse que saberia a
autoria de uma carta que saiu com a firma cifrada? Que amigo saberia a
identidade por trás da figura literária desconhecida?

O que é fora de dúvida é que, no início da década de 1880, Gama atribuiu
a autoria de Cincinatus a José Bonifácio, que, vivo à época, nada
contestou, num assentimento típico dos literatos; exatamente o
assentimento que Gama prestou ao público quando Mendonça lhe atribuiu o
pseudônimo de\ldots{} Afro!

Voltemos do break temporal, tornando, enfim, ao 1º dezembro de 1867.
Naquela data, uma coisa inédita ocorria na história da imprensa
brasileira. Pela primeira vez no mundo das ideias políticas desse país,
um redator de jornal, que não pode ser categorizado como esporádico ou
lateral, surgia na cena pública como dirigente de um semanário e
reivindicava, a um só tempo, a raça negra como voz ativa e um programa
político que tratasse da ``abolição da escravatura, de exército
permanente, da Guarda Nacional, da pena de morte e da religião do
Estado''. Tinha como bandeira ``a liberdade de consciência e de cultos, de
ensino, de imprensa, (\ldots{}) de associação e reuniões pacíficas'', assim
como se levantava ``pela regeneração dos tribunais, poluídos pela
cobiça dos juízes''. Ao fim, \emph{Democracia} sintetiza seu programa:
``em política sustenta as ideias republicanas; como socialista, a
democracia cristã''.

Na capa de sua primeira edição, a 1º de dezembro de 1867,
\emph{Democracia} trazia uma única e sugestiva assinatura: Afro 1º. Dessa data em diante, Afro 1º, ou simplesmente Afro,
cravou sua assinatura por 15 vezes, até abril de 1868. Com essa marca,
Afro figura como o pseudônimo que mais vezes aparece em todas as
31 edições do \emph{Democracia}. Ainda assim, como veremos adiante, há
outros textos que lhe são relacionados. O texto de Afro de abril
de 1868, por exemplo, estabelece uma linha contínua, embora sem
assinatura, com outros quatro textos publicados sequencialmente entre os
meses de maio e junho, concluindo uma série de artigos em que se discute
a educação pública na província de São Paulo. São, portanto, 15 textos
assinados por Afro ou Afro 1º e mais quatro textos
diretamente ligados, totalizando 19 textos, que dão unidade ao conjunto
da obra que se inicia desde o primeiro número do \emph{Democracia}.
Desse montante, quase todos versam sobre o direito à educação, razão
pela qual veremos o tema mais de perto.

\section{Afroeducação}

Atento à correlação de forças partidárias e ao debate legislativo que
atravessou os primeiros meses da legislatura provincial, instalada em
fevereiro de 1868, Afro centrou esforços na análise de um tema ---
a instrução pública, termo equivalente ao que atualmente se chama de
educação pública ---, propondo soluções e discutindo um projeto de lei
que pretendia assentar novas bases para a educação básica na província.

Embora a escolha do debate educacional deva ter obedecido a critérios e
estratégias políticas do calor da hora, isto é, relativas à agitada
conjuntura partidária local, Afro demonstrou conhecer o assunto
por experiência e por diversas perspectivas teóricas, seja a do direito
constitucional, da administração pública ou do que podemos chamar hoje
de política comparada. Não era a primeira vez, contudo, que um certo
Afro dissertava sobre o estado da educação na província. Em
meados de 1866, no conservador \emph{Diário de S.\,Paulo}, Afro
dirigiu uma carta aberta ao diretor da instrução pública da província, o
liberal moderado Diogo de Mendonça Pinto, destrinchando o relatório
oficial que havia publicado sobre a situação da instrução pública em São
Paulo no ano de 1864.

A carta é uma aula de direito público e uma análise contundente sobre
história política e hermenêutica constitucional, especialmente no que o
autor caracteriza como ``hiperbólica apreciação da nossa Constituição
política'' que, ``anacrônica e absurda'', não passava de ``um agregado
disforme de textos contraditórios; rapsódia\footnote{Fragmento de um
  escrito.} indigesta extraída de outros, doutamente escritos, na qual
se procurou, com estudada hipocrisia, harmonizar princípios
heterogêneos, que se repelem''. Leitor de Pimenta Bueno, Afro
tinha em mente as mistificações do formalismo de uma ``Constituição
simbólica'', sem eficácia normativa em garantir a ``instrução primária
gratuita a todos os cidadãos'' de que falava o perdido inciso 32 do art.
179 da Constituição autocrática de 1824.

Afora essa afiada crítica jurídica que denuncia a erudição de nosso
conhecido autor, o que nos chama atenção é que Afro enxerga a
instrução pública enquanto ``direito inalienável do homem'' e o liberto
como destinatário de direitos. Numa quadra histórica em que aprender a
ler e escrever era um privilégio restrito a uma parcela ínfima da
população, mesmo entre a população livre, Afro incluía o liberto
efetivamente como cidadão, reforçando direitos e reconhecendo-os
verdadeiramente como parte do corpo político da nação.

O projeto de inclusão social via popularização da escola pública de
Afro foi exposto em três etapas, uma seguida da outra: a série
``Instrução pública'', dividida em sete trechos; a ``Carta ao
exmo.\,sr.\,deputado dr. Tito A.\,P.\,de Mattos'', em três partes e,
finalmente, ``A nova lei de instrução primária'', também em três
partes.

Em síntese, Afro tinha em mente duas ideias centrais para
reformar a educação pública: ``a instrução gratuita e obrigatória e a
liberdade de ensino''. A primeira deixaria as ``portas da ciência
inteiramente francas a todas as inteligências''; a segunda garantiria a
pluralidade de circulação de ideias nas escolas, quebrando o rígido
controle do pensamento operado pelo Estado e pela Igreja Católica, a
religião oficial do Estado e mantenedora subsidiada de numerosos
estabelecimentos de ensino. Tirar o ensino público do raio de ação da
Igreja era uma obsessão que Afro elevava ao patamar de reforma
civilizatória e democrática que o Brasil, seguindo o exemplo de países
que se desenvolveram, não poderia se furtar a fazer. A liberdade de
ensino, portanto, seria uma expressão da liberdade de consciência e de
pensamento. Sobre a participação estatal, todavia, tratava-se de equação
mais difícil de sanar. Ao tempo que defendia a expansão do ensino
primário obrigatório e gratuito, mantido e custeado pelo Estado,
criticava a centralização administrativa ``em que as sugestões capciosas
do governo, emissário da corrupção que impera no alto, podem facilmente
infeccionar os sãos preceitos da lei e nulificar completamente as
legítimas aspirações populares''.

Nem centralização administrativa, nem concentração do conhecimento. O
projeto de Afro corria em duas frentes: regionalização da rede de
ensino público por todos os rincões da província (e do país) e
atendimento escolar gratuito para crianças de todas as classes sociais.
``A escola pública é um grande e poderoso elemento de igualdade social.
Seu objeto, instruindo gratuita e indistintamente a todos, é elevar,
pelo cultivo da inteligência, o filho do mendigo à posição do filho do
milionário.'' E~continuava, já não se sabendo o que era utopia e o que
era meta concreta: ``Nenhuma aldeia sem uma escola, nenhuma vila sem um
colégio, nenhuma cidade sem um liceu, nenhuma província sem uma
academia. Um vasto todo, ou, para melhor dizer, uma vasta textura de
oficinas intelectuais, escolas, liceus, colégios, bibliotecas e
academias, ajuntando sua irradiação na superfície do país, despertando
por toda a parte as aptidões e animando por toda a parte as vocações''.

Mas o autor tinha os pés bem fincados na crua realidade política da
província. O país estava em guerra. A política da escravidão dava sinais
de esgotamento. Os partidos se esfacelavam. O horizonte de expectativas
estava aberto como nunca esteve nos anos imediatamente precedentes. Era
sim possível --- calculava --- pôr fim à escravidão desde o transe nas
bases da população livre, liberta, escravizada, do Império.
Afro-Gama jogava suas fichas na desestabilização da monarquia, no
``enfraquecimento da autocracia administrativa'', a começar pela tentativa
original de, sem mandato, sublevar a Assembleia Provincial de São Paulo
e arregimentar aliados localistas com o discurso de fortalecimento dos
municípios, a partir da restituição de ``importantes funções, usurpadas
pelo imperialismo''.

De mangas arregaçadas, Gama levantava o seu porrete e hasteava sua
bandeira.

Afro conhecia a fundo os contrastes abissais de um país em que o
negro, escravizado ou liberto, morria ``delirante nos campos de batalha,
ao som inebriante dos clarins e dos epinícios\footnote{Cântico
  para comemorar uma vitória ou regozijar um feliz acontecimento.}
divinos entoados à pátria para perpetuar a tenebrosa hediondez da
escravidão de seus pais''. Gama igualmente sabia que ``recebiam uma
carabina envolvida em uma carta de alforria, com a obrigação de se
fazerem matar à fome, à sede e à bala nos esteiros paraguaios'' e que,
``nos campos de batalha, caíam saudando risonhos o glorioso pavilhão da
terra de seus filhos''.

O delírio no campo de batalha paraguaio era também o delírio imperial
brasileiro da promessa da liberdade condicionada à certeza da morte em
combate. A pátria que perpetuava a escravidão, argumenta Afro, só
poderia ser desafiada pela difusão em massa da instrução primária
obrigatória e gratuita, acompanhada da liberdade de ensino. Liberdade
sem direitos, acesso à educação, cidadania, sem ``sufrágio universal e
eleição direta'', seria uma liberdade frágil, precária, sem substância.
Assim, o direito à educação básica com pluralidade de ideias e sem
distinção social --- ``onde houver um espírito, que haja também um livro''
---, distribuído em uma ampla rede escolar de todos os níveis, seria a
chave para o fim da escravidão e consequente construção da democracia no
Brasil. Com o quadro nacional em vista, muito embora estrategicamente
fale de modo geral, Afro crava que ensino obrigatório e liberdade
de ensino seriam inconciliáveis com a vigência do Império brasileiro.
O desenrolar dos eventos políticos do final do século se
encarregaria de reforçar a razão de seus assertos. Em uma síntese
lapidar:

\begin{quote}
A liberdade de ensino é o complemento do ensino obrigatório.

Estas duas instituições, nos países democráticos, únicos que podem
comportá-las, constituem a base da grandeza e da felicidade dos povos.

A sustentação de tais princípios é a declaração de guerra às monarquias.

Nós escrevemos em nome do povo e da liberdade.

\begin{flushright}
\textsc{afro} 1º
\end{flushright}
\end{quote}

\section{Luiz Gama e a educação}

Quando se fala da obra de Luiz Gama, é comum que se destaque sua ação
jurídica para alforria de escravizados ou mesmo embates forenses de
outras naturezas processuais; sua produção poética e jornalística; bem
como sua vida político-partidária, associativista e abolicionista. Esses
três mundos --- em síntese, o direito, as letras e a política, que
obviamente se entrelaçam, cruzam e sobrepõem --- ocultam um outro espaço
de ação a que dedicou-se vivamente: a educação.

Há registros que indicam que Gama foi professor de português em colégio
particular para meninos, professor de alfabetização de adultos --- homens
e mulheres --- em escola comunitária, e até mesmo diretor de biblioteca.
Para pensarmos o conturbado final da década de 1860, uma pista
reveladora é lermos um dos relatórios da loja maçônica América, fundada
em São Paulo em novembro 1868, na qual a presença constante
de Luiz Gama se nota por mais de dez anos. Publicado no \emph{Correio
Paulistano}, o relatório, que tinha como destinatário final o presidente
da província (e o público em geral), é assinado por uma comissão de sete
dirigentes da Loja, entre eles Luiz Gama, a segunda assinatura de cima
para baixo. No entanto, após exame grafológico do relatório original,
que confere exatamente com o publicado na imprensa, conclui-se que a
escrita do relatório é inteiramente do punho de Gama. Através de sua letra, a comissão explica nesse documento que a Loja ``resolveu
trabalhar no intuito de promover a propagação da instrução primária'' e
``difundir o ensino popular'' para ``tornar uma realidade a igualdade dos
homens no gozo de seus direitos naturais indebitamente postergados''.

Um trecho do relatório dá a dimensão da estrutura e alcance da escola de
alfabetização da Loja América:

\begin{quote}
Em relação ao ensino popular, ela fundou e sustenta nesta capital
\textit{uma escola noturna de primeiras letras, onde se acham
matriculados 214 alunos, sendo efetivamente frequentes 100.}

Os trabalhos correm ali \textit{com toda a regularidade} e com grande
proveito para os alunos, que em geral mostram a melhor vontade em
aprender e comportam-se com toda a conveniência, sem que entretanto
estejam sujeitos a punição alguma.

Além dos esforços do professor para o preenchimento de seus deveres, há
o concurso dos auxílios de um dos membros da loja, o qual, durante a
semana que lhe é designada, tem de assistir todas as noites à escola.

Além desta, há em várias localidades da província outras instaladas por
adeptos da oficina e por ela pecuniariamente auxiliadas.
\end{quote}

Pelo excerto, podemos ter ideia do funcionamento da ``escola noturna''
sediada na rua 25 de Março, então periferia da área nobre da cidade,
pelo menos desde maio de 1869. Quem seria o professor de primeiras
letras ou mesmo os fiscais da Loja incumbidos de auxiliar as atividades,
são ainda questões inconclusas, muito embora haja indícios que sugiram a
participação direta de Gama também nesse assunto. Um exemplo instigante
encontra-se na mesmíssima edição do \emph{Correio Paulistano} que
publicou o relatório da Loja América. Imediatamente abaixo do documento,
lê-se o artigo intitulado ``Luiz G. P. Gama'', no qual se defende de
opositores, em nome próprio, embora fale indiretamente em defesa
da Loja América, em evidente sinal de liderança pública daquele grupo
maçônico. O artigo é uma peça histórica. Acusado de agente da
Internacional Comunista --- ``esta Loja maçônica
trabalharia sob os influxos de agentes da Internacional'' ---, Gama
revidou como experiente militante político na desfavorável posição de
combate em que se encontrava. Os planos para uma ``tremenda insurreição
de escravos'' que lhe atribuíam seriam ``boatos humorísticos'',
insinuações infundadas. Até segunda ordem, trabalhava estritamente pela
legalidade para concretizar dois objetivos que eram da Loja América e
também seus: ``promover a propagação da instrução primária e emancipação
dos escravos pelos trâmites legais''.

Propagar, portanto, educação e alforrias. O medo senhorial dos
conservadores (e parte dos liberais) da província não se media apenas
pelos processos de Gama e da Loja América nos tribunais, mas também por
suas ações na criação e fortalecimento de espaços de ensino, a exemplo
da gigantesca escola noturna da rua 25 de Março, com no mínimo uma
centena de alunos, ou de outras escolas comunitárias ``pecuniariamente
auxiliadas'' por esse grupo maçônico. Gama não era só encarregado das
causas de liberdade, mas alguém que também atuava de perto nos assuntos
relativos à instrução primária. Além dessas frentes, continuava a
construção partidária da alternativa republicana. Por isso, nesse mesmo
artigo, Gama contra-atacou ``a cuidada hipocrisia da imprensa
monarquista, que não cessa de propalar --- que o Partido Republicano
compõe-se de `comunistas, de abolicionistas, de internacionalistas'\,''.

Desse modo, o problema da desestabilização da monarquia, passando
necessariamente pela agitação das massas escravizadas, em particular, e
dos despojados de direitos e cidadania, em geral, não morava apenas nas
demandas de liberdades e direitos nos tribunais, se não também nas
escolas noturnas que começavam a surgir em toda a província.

Algumas pistas do potencial subversivo da educação em São Paulo podem
ser lidas no \emph{Radical Paulistano}, jornal que demarca uma nova
etapa do movimento republicano, após o término do \emph{Democracia}, em
julho de 1868 e que, no que se refere ao nome, tinha um baiano ---
soteropolitano, aliás --- na sua liderança.

\section{O radical soteropaulistano}

Lançado em abril de 1869, no início da longa hegemonia conservadora que
duraria quase uma década, o \emph{Radical Paulistano} levantava
bandeiras bastante similares às sustentadas pelo \emph{Cabrião} e
\emph{Democracia}, mas as defendia em um momento mais complicado para
consolidar um órgão de imprensa com ideias republicanas. Sob as mais
adversas condições políticas inauguradas com o domínio político da linha
dura do Partido Conservador, o \emph{Radical Paulistano} preparava o
terreno para a constituição do Partido Republicano, que se daria,
finalmente, em 1873.

Não que o \emph{Cabrião} e o \emph{Democracia} tivessem enfrentado
tempos fáceis, mas enquanto a luta política se travava entre os
liberais, afastados os conservadores do centro decisório, havia maiores
liberdades de opinião e imprensa. No entanto, da divisão semifratricida
dos liberais, o Partido Conservador capitalizou a crise e voltou ao
poder assumindo o protagonismo até mesmo das reformas sociais de
emancipação gradual do trabalho escravo para o trabalho livre.

\emph{Cabrião} e \emph{Democracia} tinham, cada qual, formas distintas
de expressar ideias em comum. Como vimos, o \emph{Cabrião} era um
periódico ilustrado e satírico que transitava entre a crítica política,
religiosa e literária, batendo pesado tanto em liberais quanto em
conservadores, ambos muitas vezes atirados no mesmo balaio cultural. Já
\emph{Democracia} optava pragmaticamente por cavar um espaço nos debates
da política local, pressionando sobretudo os liberais moderados nas
discussões do legislativo provincial, propondo reflexões teóricas e
aplicações de medidas de governo, principalmente relacionadas à
educação.

O \emph{Radical Paulistano} cumpriria outro objetivo imediato: manter
hasteada a duras penas a bandeira republicana, tensionando a arena
política para a entrada de um novo partido que, diferente dos demais,
prometia pôr fim ao regime monárquico. Um partido por todos os aspectos
inconciliável com a continuidade dinástica do Império. Órgão do
clube radical de São Paulo, espécie de fórum local que apareceu em
diversos municípios como preparatório da organização partidária futura,
o \emph{Radical Paulistano} circulou regularmente de abril de 1869 até
janeiro de 1870.

A redação do jornal era formada em sua maioria por jovens estudantes,
somada por dois experientes republicanos que, por sinal, eram os dois
únicos redatores fixos: Américo de Campos e Luiz Gama. Passaram pela
redação do \emph{Radical Paulistano} estudantes da Faculdade de Direito
de São Paulo, como Ruy Barbosa, Freitas Coutinho, Bernardino Pamplona e
Olympio da Paixão, e é de se supor que eles se revezassem em colunas de
opinião acompanhados pela orientação direta dos líderes do clube
radical, Luiz Gama e Américo de Campos.

O papel de cada um nas páginas do \emph{Radical} é difícil de precisar.
Um indício razoável, no entanto, são as ``conferências radicais'', grande
plenária política e carro chefe, junto do jornal, da propaganda das
ideias republicanas.

Nas memórias de Lúcio de Mendonça a respeito de Gama, uma frase se
destaca: ``Foi aplaudidíssima uma conferência sua no salão Joaquim Elias,
à rua Nova de S. José''.

Gama discursou para um salão apinhado de aproximadamente quatrocentas
pessoas. O tema da conferência foi o ``Poder Moderador'', mecanismo
político-constitucional que permitia ao imperador interferir nos poderes
políticos do Império, em fatal desequilíbrio dele sobre os demais,
Executivo, Legislativo e Judiciário.

É de se notar que aquela era a conferência inaugural, de uma série que
se seguiu praticamente mês a mês até o final de 1869. Líder do
liberalismo radical que formaria o movimento republicano em São Paulo,
Luiz Gama era, portanto, o responsável por abrir os trabalhos das
prestigiadas conferências públicas. Para matizar um pouco melhor a
direção política do movimento republicano, vejamos que, após Gama, a
segunda conferência seria feita por Américo de Campos.

Ocorria, porém, que não eram conferências isoladas. O jornal cumpria um
requisito importante de subsidiar o debate público. Fosse qual fosse a
temática, o \emph{Radical Paulistano} publicava textos referentes ao
tema da vez, seja preparando a atividade futura ou repercutindo a
conferência passada. A de Américo de Campos, sobre liberdade
de cultos, por exemplo, antecedeu um longo texto, dividido por trechos
em diferentes edições do \emph{Radical Paulistano}. Pode-se supor com
boa margem de acerto que Américo de Campos, orador do tema, estivesse
por trás dos textos intitulados, enfim, de ``Liberdade de cultos''.

Gama falaria sobre as ficções jurídicas do poder moderador, ``chave de
toda a organização política'' do Império, na definição do artigo 98 do
texto constitucional de 1824. É de se conjecturar que ao menos parte dos
textos sobre o tema publicados no \emph{Radical Paulistano} tenha sido
redigida por ele. Pela análise da divisão de trabalho interno da
redação, de que a organização temática por orador de conferência é
apenas uma das variáveis, pode-se apurar quais textos que, ao fim e ao
cabo, foram escritos, individualmente ou em coautoria, por um dado
autor. Assim, examinadas todas as colunas e edições do \emph{Radical
Paulistano} à luz das evidências encontradas, rastreamos a colaboração
de Gama que, ao que se sustenta, não figura como redator marginal, mas
como redator-chefe do periódico.

Há sugestivos exemplos da dobradinha com Américo de Campos, que se
ocupava mais da redação do \emph{Correio Paulistano}, e outros
indicativos da ação de Gama na supervisão do trabalho dos estudantes
novatos com os afazeres no chão da tipografia, de onde surgiam as
páginas impressas que noticiavam o mundo para aquele local.

Seguindo a análise temática, vê-se que, além do poder moderador, a
educação ocupou parte dos debates no \emph{Radical Paulistano} que
receberam a atenção de Gama. ``As aulas noturnas'', ``Em vez de
escola, tarimba'', ``A democracia e a instrução do povo'' e ``Escolas
populares'', por exemplo, são artigos que, embora não assinados,
apresentam uma leitura de realidade política e um estilo de argumentação
semelhantes ao que Afro dedicou nas páginas do \emph{Democracia}.
Ao mesmo tempo, o redator demonstrava estar muito bem informado da
crítica dos detratores das escolas comunitárias de letramento básico,
como a escola da rua 25 de Março. Defendendo a escola noturna da Loja
América, o \emph{Radical Paulistano} perguntava: ``Se nas aulas noturnas
ensinam princípios subversivos, por que não os apontam esses arautos do
absolutismo, esses apóstolos da ignorância do povo? Para que não vão
assistir ao ensino dessas aulas?'' E continuava a toda carga: ``Nessas
aulas se ensina a ler aos escravos, ainda dizem os inimigos encarniçados
da instrução; é verdade, mas com o consentimento de seus senhores; e
quem poderá impedir este ato? Que imoralidade e desrespeito às leis há
aqui?''

Se cotejarmos essas perguntas com o trecho destacado do relatório da
Loja América, começaremos a ver que escravizados estudavam na ``escola
noturna de primeiras letras, onde se acham matriculados 214 alunos,
sendo efetivamente frequentes 100''.

A escola da Loja América, portanto, aplicava dois princípios de
instrução primária que Afro, e agora o \emph{Radical Paulistano,}
defendiam nos planos teórico e prático: letramento de todos, sem
distinção social, haja vista a inclusão de escravizados como sujeitos de
direitos; e a liberdade de ensinar garantida à sociedade civil, nesse
caso exercida através de um grupo maçônico.

Propagar educação e alforrias, não custa dizer, faces de uma mesma
emancipação civilizatória.

\section{A virada de 1869}

O ano de 1869, contudo, significou uma clivagem na presença de Gama nos
jornais, razão até para sublinhá-lo como um ano à parte na turbulenta
década de 1860. Foi nesse ano que Gama iniciou sua escrita em nome
próprio na imprensa e no direito, anunciando a carreira profissional que
tomaria pelo resto de sua vida. A partir de fevereiro de 1869, ele só
pararia na morte, em agosto de 1882. Não se diz, com isso, que antes ele
não tivesse escrito e assinado um punhado de artigos em nome próprio,
mas agora se tratava de um caminho sem volta, que lhe custaria, no
curtíssimo prazo, o emprego como amanuense da Secretaria da Polícia e o
fim da proteção pública que lhe emprestava Furtado de Mendonça, ex-chefe
de polícia e professor da Faculdade de Direito de São Paulo.

De fevereiro a dezembro de 1869, a assinatura de Luiz Gama apareceria
quase todos os meses na imprensa paulistana: fevereiro no \emph{Ypiranga}, 
março e abril no \emph{Correio}, maio, julho, agosto e
setembro novamente no \emph{Radical}, novembro e dezembro no
\emph{Correio}. Lidos em conjunto, os artigos em nome próprio salpicados
na imprensa somados com o projeto editorial do \emph{Radical
Paulistano,} que continuava a pleno vapor, indicam as nuances de uma
inserção no debate público bastante arrojada, que marcaria a história do
direito, da imprensa, da democracia e do Brasil.

O artigo que melhor representa, a um só tempo, a opção estética ajustada
para uma nova realidade política, o aprimoramento do estilo de ativismo
e o manejo de técnicas argumentativas próprias dos tribunais do Império
--- em atenção aos tais ``costumes do foro'' --- chama-se ``Questão de
liberdade''. Pode ser visto, desde o prólogo, como uma performance cênica
que apenas dramaturgos experenciados conseguem adaptar para os tablados
dos melhores teatros. Mas também pode ser lido como um notável exemplar
de literatura normativo-pragmática, desses que só se erigem através do
amplo conhecimento do direito consistentemente alinhavado pela
combinação metódica de habilidade prática e erudição teórica. Em 1869,
Gama cabalmente possuía as duas. Calejado funcionário da Secretaria de
Polícia, dominava de cátedra o repertório da multinormatividade
administrativa local e brasileira. Como inquieto leitor de história,
literatura, política, poesia e direito, podia discutir qualquer tópico
desses campos de saberes com quem aparecesse habilitado para tal.

``Questão de liberdade'' anuncia um novo tempo no direito brasileiro
e na produção literária de Luiz Gama. Costurando referências entre a
crônica judiciária norte-americana, a doutrina civilística
luso-brasileira e a poesia satírica portuguesa, Gama tinha um objetivo
pragmático: estabelecer um referencial normativo emancipatório para
processamento e julgamento de causas de liberdade na província de São
Paulo. Pelo exemplo norte-americano, discutia aspectos do direito
natural que impediriam, em seus fundamentos filosóficos, a escravização
do homem pelo homem; pela interpretação dos compêndios de praxe
processual e de direito civil, traduzia por dentro da tradição jurídica,
isto é, relia normas e lições acadêmicas para efeito de intervenção no
juízo local; e, finalmente, pela mordaz poética lusitana, arrancava da
aparente inércia magistrática os julgadores mancomunados com a parte
contrária, a dizer, os proprietários de títulos de domínio fatalmente
ilegais ou ilegítimos, provocando-os --- juízes, jurisconsultos,
políticos, gente do povo, sociedade em geral --- a refletir e se indignar
com a administração da justiça no país.

A escrita de Gama ``em nome da parda Rita'' é, em suma, uma obra de arte,
porque sendo um monumento à liberdade, reinventou a dignidade do direito
por sobre os escombros da injustiça da escravidão. A estrutura objetiva
da demanda de liberdade, seguida da fundamentação normativa e exibição
de provas, passou a ser uma espécie de roteiro para a literatura
normativo-pragmática que Gama criou em 1869 e desenvolveu posteriormente
por toda a carreira como advogado. Um fator a mais entrava na equação: a
discussão da causa processual na imprensa, através da transcrição do
julgado e consequente exposição do julgador.

Após transcrever uma das decisões no processo da parda Rita, dizia Gama
que ``o despacho do benemérito juiz foi uma tortura imposta à desvalida
impetrante, que, para fazer valer o seu direito, implorava segurança de
pessoa, perante a justiça do libérrimo país em que ela desgraçadamente
sofre ignominiosa escravidão''. Mais: o despacho seria ``violação
flagrante dos preceitos característicos do julgador'' por ``singular
capricho do respeitável juiz''. Gama recorreu da ``grave e escandalosa
extorsão'' de que o juiz municipal Santos Camargo caprichosamente tomava
partido. Um novo juiz, um ``novo assalto jurídico''. Rego Freitas,
cumulativamente presidente da Câmara Municipal de São Paulo e juiz de
direito, cobriu o seu parceiro Santos Camargo, dando-lhe respaldo e
proteção. Para Gama, não passava de outro roubo, ``assalto que, conquanto
diversifique do primeiro, segundo a forma, lhe é, em fundo,
completamente idêntico''.

``Questão de liberdade'' propõe uma forma de interpelação a um só
tempo judicial e pública. Também revela quais seriam seus principais e
encarniçados opositores no próximo ciclo que se abria, agora apenas como
solicitador e depois como advogado de fato e de direito: os juízes
Antonio Pinto do Rego Freitas e Felício Ribeiro dos Santos Camargo. Do
primeiro, Rego Freitas, viria em 1870 a acusação pelo crime de injúrias,
que passou à história judiciária pelo célebre processo em que Gama
defendeu-se no Tribunal do Júri e foi absolvido por unanimidade de
votos. Do segundo juiz, Santos Camargo, basta que se leia a série de
artigos escrita por Gama no ano de 1872, intitulada ``Cousas do
sapientíssimo sr.\,dr.\,Felício''.

``Questão de liberdade'', portanto, é um abre-alas para se
compreender a formação das estratégias que Gama empregou na longa
trajetória de advogado da liberdade, haja vista os casos seguintes ao de
Rita, debatidos nas páginas do \emph{Radical Paulistano}: os direitos
manumissórios dos ``\emph{sete} infelizes, que se acham em cativeiro,
como vítimas da santidade do nosso finado e adorado bispo'' Antonio
Joaquim Melo, publicado em maio; a prisão ilegal do ``infeliz Antonio %publicado concorda com o que aqui?
José da Encarnação'', em julho; a quebra unilateral de contrato que
atingiu o ``infeliz Francisco Pereira Thomaz'', em agosto; a alforria
testamentária de Benedicto, em setembro; e o paradigmático caso dos
africanos livres ilegalmente traficados e trancafiados, Jacyntho e Anna,
em novembro.

``Época difícil é a que atravessamos para as causas judiciárias'',
escrevia sobre o caso Jacyntho e Anna, uma semana antes de ser demitido
da Secretaria de Polícia. A notícia alcançou a Corte. Possivelmente, o
próprio ministro da Justiça José de Alencar tenha saudado a demissão
como medida há muito esperada. Por outro lado, os ingleses do
\emph{Anglo-Brazilian Times} denunciaram o ato de demissão como uma
arbitrariedade contra os direitos de Gama e um aviso de potencial
represália aos demais liberais radicais que formavam o movimento
republicano. É o indício de participação de José de Alencar, ao menos
como entusiasta da demissão, que leva a estendermos o ano de 1869 até os
primeiros dias de janeiro de 1870, incluindo nesse volume um inédito
artigo de Luiz Gama, na última edição do \emph{Radical Paulistano},
respondendo Alencar e dando continuidade ao que parecia ser o ponto
final da discussão, o artigo ``Pela última vez''.

Assim, a demissão de Luiz Gama, contada por ele próprio, tem uma nova
demarcação: do artigo ``Um novo Alexandre'' até ``Calúnia
calculada''. Ganha, afinal, a historiografia, com uma peça a mais que
complexifica a análise do já intrincado tabuleiro político que levou à
demissão de Luiz Gama.

Antes desse exame, que certamente se dará num futuro próximo, voltemos
nossa atenção para o tópico final, analisando a reveladora metáfora que
Gama emprega para relatar a indecência de sua demissão: a história de
Alexandre e o nó górdio.

\section{O obscuro Luiz Gama}

Leitor voraz das mitologias, fábulas e poesias dos mundos grego e
romano, Gama tinha uma especial simpatia pela história de Alexandre, o
Grande, e o nó górdio. Evocou a passagem mitológica ao menos em cinco
textos autorais.

Conta a lenda que o general Alexandre chegou a Frígia, província romana
na Ásia, e encontrou uma carroça amarrada em uma das colunas de um
templo de Zeus. A carroça pertencera a Górdio, antigo camponês, que
resolveu atá-la ao templo em agradecimento a Zeus, que fez do humilde
servo um rei. Enraizada na tradição oracular local, a profecia corrente
dizia que se tornaria um novo rei para toda aquela região quem
conseguisse desatar o nó que amarrava a carroça ao templo havia já muito
tempo. Alexandre, por sua vez, ciente da história, desembainhou a espada
e de um só golpe rompeu a corda, desatando, à sua maneira, o nó górdio.
Existem muitas interpretações do significado do gesto e dos sinais da
profecia, visto que o promissor militar em campanha se tornaria
conquistador e imperador de povos e grandes territórios. Uma leitura
possível é a que vê na espadada um gesto simples e definidor de um
problema complexo; outra, a que enxerga no corte à espada um desfecho
grosseiro para o enigma que reclamava solução diferente.

Seja como for, Gama elegeu a metáfora como representação da sua
demissão. ``Digamos a verdade sem rebuço. A minha demissão era um nó
górdio que há tempos preocupava muitos espíritos. E para cortá-lo,
achou-se, ao fim, um inculpado Alexandre de cataratas!''\footnote{Por
  metonímia, a referência a Alexandre, o Grande (356--323 a.C), assume
  contornos burlescos e substitui o todo-poderoso chefe de polícia que
  assinou a portaria de demissão, Vicente Ferreira da Silva Bueno
  (1815--1873). Enfurecido, o autor insinua que Bueno ``sofre da vista'' e
  portava ``cataratas'', não se sabendo, contudo, se empregava, uma vez
  mais, o recurso retórico da metáfora de que o chefe de polícia não
  enxergava bem, ou se explorava uma condição física desfavorável.}
Mora aí a razão de três artigos consecutivos intitulados:
``Um novo Alexandre'', ``O novo Alexandre'' e,
finalmente, ``Ainda o novo Alexandre''.

Lá atrás, quando Afro perdia a batalha da causa da educação
básica, obrigatória e gratuita para todos, sem distinção de classe e
raça, dirigiu nas páginas do \emph{Democracia} uma longa carta aberta ao
deputado Tito Mattos. Inconformado com a opção política tomada pela
Assembleia Provincial de São Paulo que, ao fim, soterraria as pretensões
populares de acesso à educação, Afro viu nisso não apenas um
simples ``erro político'', mas ``um descomunal atentado contra as legítimas
aspirações da província'' e ``uma traição imperdoável à confiança
pública''. Seria o ``fraternal aperto de mão'' de liberais moderados ``por
cima do túmulo da liberdade, aos sórdidos algozes do Partido
Conservador''.

Justamente quando agonizava em praça pública, vencido em um ponto-chave
do programa político que liderava, Afro recorria à
metáfora-síntese da derrota que sofria: ``Novo Alexandre,\footnote{Alexandre \textsc{iii} da Macedônia (356--323 a.C), popularmente conhecido como
  Alexandre, o Grande, foi rei da Macedônia, Pérsia e faraó do Egito.}
quando posto em aperturas, pretende V. Excia. solver a questão cortando
o nó górdio, com a estudada resposta:
`Voto contra o projeto, por inconstitucional'\,''.

Ambos Alexandres, o protagonista do ato de demissão e o deputado
símbolo do embate que travou extramuros do legislativo, cortaram a
``gordiana urdidura jurídica'' que amarravam a questão. Chamá-los de
Alexandre, no contexto da metáfora, expunha a mesquinhez de
homens públicos que distorciam, por catarata ou miopia política, o
tamanho de seus respectivos poderes. Tinha ``Vicente Ferreira\footnote{Vicente Ferreira da Silva Bueno (1815--1873) teve longa carreira
  administrativo-judiciária, exercendo cargos de delegado de polícia,
  juiz municipal, juiz dos órfãos, juiz de direito e desembargador em
  diversas províncias, como Bahia, Paraná, São Paulo e Rio de Janeiro.
  Em 1869, era chefe de polícia interino da província de São Paulo,
  cabendo a ele papel de algoz no espetáculo da demissão de Luiz Gama do
  cargo de amanuense da Secretaria de Polícia.} bem desempenhado o seu
papel de Alexandre'', fulminando à espadada Luiz Gama da Secretaria de
Polícia; Tito Mattos, Alexandre caricato, que ``vacila taciturno entre as
raias da democracia e os marcos do despotismo, levando aos ombros, aliás
robustos, o pesado fardo da péssima causa que espontaneamente aceitou'',
também deu a sua espadada na constitucionalidade do projeto
liberal-radical de reforma do ensino primário. Na ciência, dizia
Afro, ``não há lugar para os Alexandres'', porque ``não se cortam as
dificuldades com o gládio\footnote{Punhal.} ultrice\footnote{Vingador.} dos homicidas coroados: resolvem-se pelo raciocínio que
enobrece''.

Despedindo-se de Tito Mattos, Afro deixou uma camada a mais de
tinta preta na folha branca, sugerindo maiores conexões com sua já
intrigante assinatura:

\begin{quote}
Ao terminar estas linhas, devo esclarecer à V. Excia. {[}Tito Mattos{]}
que \textit{sou forçado a ocultar o meu nome próprio, que lhe é assaz
conhecido, por ser menos obscuro o pseudônimo de que uso, o qual encerra
uma tradição memorável}; e que, ao escrevê-las, tive sempre à vista a
distância que medeia\footnote{Separa, divide.} entre a pessoa sempre
respeitável do deputado e as suas ideias, que são de propriedade
pública.
\end{quote}

Não havia segredo. Afro era ``assaz'', muitíssimo bem conhecido na
imprensa e na política. A ``tradição memorável'' que evoca é a mesmíssima
que cantava Getulino, o também assaz conhecido pseudônimo de que
Luiz Gama lançou mão em 1859 e 1861, na qualidade de primeiro poeta
afro-brasileiro a publicar um livro autoral, o \emph{Primeiras trovas
burlescas de Getulino}. Em uma palavra: a ``tradição memorável'' de sua
mãe Luiza Mahín, maior exemplo de luta, coragem e justiça que Gama teve
e enalteceu.

As razões que o teriam levado a esconder o nome próprio --- mais obscuro
que o pseudônimo --- escapam ao propósito da introdução desse volume,
bastando que os leitores se lembrem dos limites funcionais de um
empregado público, sobretudo de repartições policiais. No entanto, é de
se notar a ênfase numa espécie de palavra-mágica, a ``obscuridade'' que,
em contraste ao luminoso, aponta para uma peculiaridade do temperamento
de Gama e que poucos descreveram tão bem quanto Raul Pompeia, na singela
sentença: ``soube excluir-se''.

Silvio Roberto Oliveira leu essa passagem com brilhantismo e
profundidade de análise:

\begin{quote}
O jogo de raciocínio operado por Pompeia foi bem sagaz. Assinalou mais
precisamente que Gama se excluiu para ``incluir-se'', pois o baiano teria
percebido que assimilar as discriminações sofridas (dos estudantes de
direito, por exemplo) foi fundamental para sobressair-se. Gama teria
usado inteligentemente os estereótipos criados pela cultura predominante
(europeizada em extremo) que o excluía, sabendo, em profundidade,
incluir-se manipulando os próprios fundamentos dessa cultura.
\end{quote}

A carreira de Gama está repleta de sinais de como ``se excluiu para
incluir-se''. Um deles se vê na variação inventiva de pseudônimos,
``ocultando o nome próprio (\ldots{}) por ser menos obscuro o pseudônimo de
que uso''. No jogo sagaz do menos ou mais obscuro, Gama dava suas
espadadas nas tão sutis quanto violentas normas e conveniências sociais,
morais e raciais que quase o alijaram por completo do direito a
pensamento, voz e voto na arena pública. Alijamento ao qual ele
criativamente revidou a partir da reivindicação da legitimidade de um
negro não acadêmico em ``dizer o direito'', isto é, estabelecer respostas
normativo-pragmáticas para problemas sociais. Ou simplesmente nas
palavras do poeta e tão ao gosto das metáforas crísticas que Gama
mobiliza, ``de propor justiça ao mundo pecador''.

No citado episódio da demissão, Gama caiu atirando e por fim declarou:
``Eis o estado a que chegou o discípulo obscuro do exmo. sr. conselheiro
Furtado de Mendonça''. Meses antes, na ``Carta ao muito ilustre e
honrado sr. comendador José Vergueiro'', Gama antecede a dura crítica
jurídica à formação da Sociedade Democrática Limeirense --- democrática
porém escravocrática, liberal porém limeirense --- com a tirada que se
revelaria parte de seu repertório de defesas: ``Eu, por meu turno, se bem
que o mais obscuro de entre todos, venho de minha parte\ldots{}''.

Afro, por sua vez, comentando o projeto de lei de reforma do
ensino primário em São Paulo, esquivou-se performaticamente:
``Foliculário\footnote{O mesmo que jornalista, aquele que escreve em
  periódicos. O termo, contudo, era usualmente empregado no sentido
  pejorativo, de modo que se falaria, nesse caso, de um jornalista de
  técnica limitada ou baixa erudição. Quem resolver tirar a limpo a
  escrita de Gama verá numerosos exemplos de aparente autodesprezo, em
  se descrevendo como ``obscuro'', ou calculadamente relativizando a
  importância de sua obra, resumindo, por exemplo, sua consistente
  atividade poética à expressão única ``fiz versos''.} obscuro, porém,
deixo ao critério de cada um apreciar como lhe convier este poliedro
curioso, fruto bem amadurecido da ilustração política de seus autores''.

Mais, muitíssimo mais: Afro se desvelava por inteiro, para que
nem uma pá de dúvida restasse. Em suas palavras:

\begin{quote}
\textit{Homem do povo, obscuro pelo nascimento, pela inteligência e
pela pobreza, que não detesto, posto no último grau da escala social},
tenho hoje, \textit{conduzido pela fatalidade, de cumprir}, perante V.
Excia., a tarefa importante e honrosa de refutar os dois pontos capitais
do seu belo discurso.
\end{quote}

Agora é a vez de Luiz Gama, onze meses após Afro, apresentar-se
em ``Questão de liberdade'':

\begin{quote}
\textit{Homem obscuro por nascimento e condição social, e de apoucada
inteligência, jamais cogitei, no meu exílio natural, que a cega
fatalidade pudesse um dia arrastar-me à imprensa}, nestes afortunados
tempos de venturas constitucionais, para, diante de uma população
ilustrada, como é seguramente a desta moderna Atenas brasileira,
sustentar os direitos conculcados de pobres infelizes, vítimas
arrastadas ao bárbaro sacrifício do cativeiro pelos ingênuos caprichos e
pela paternal caridade dos civilizados cristãos de hoje, em face de
homens notáveis, jurisconsultos reconhecidos e acreditados legalmente, a
quem o supremo e quase divino governo do país, em hora abençoada,
confiou o sagrado sacerdócio da honrosa judicatura.
\end{quote}

Não é o caso de repisar o que se mostra cristalino aos leitores que até
aqui chegaram. As citações pareadas uma a uma são suficientes. É como a
imagem de Luiz Gama de porrete na mão. Pelo sim e pelo não, apenas uma
diferença singular destaca Afro de Gama: a árdua tarefa que cada
um se propunha a encarar em cada um dos momentos. Afro discutiria
a fundo a questão da educação e Gama a questão da liberdade, ambas, dito
lá atrás, faces de uma mesma emancipação civilizatória. O primeiro
hasteando bandeira do ensino primário obrigatório e gratuito conectado à
liberdade absoluta de ensino; e o segundo o direito à liberdade,
cidadania e dignidade.

A ``moderna Atenas brasileira'' de Luiz Gama coincidia em exato com o chão
em que Afro pisava, afinal, dizia o \emph{menos} obscuro dos
pseudônimos na dose certa do sarcasmo, ``vivemos em um grande país,
maravilhosamente constituído, onde as vastas e muito esclarecidas
províncias, povoadas de Cíceros\footnote{Marco Túlio Cícero (106--43 a.C.), advogado, filósofo, orador e estadista romano, que
  marcou definitivamente a história da literatura e das ideias
  políticas.} e Demóstenes,\footnote{Demóstenes (384--322 a.C.) foi
  um advogado, filósofo, orador e estadista ateniense que exerceu imensa
  influência intelectual no mundo greco-romano, assim como na Europa
  renascentista e na modernidade.} se disputam orgulhosas o título
famoso de `Atenas'\,''. Ao passo que daí Afro concluiria ser a
``província de S. Paulo (Atenas, por antonomásia)'' um espaço mitológico
vivo, o que se notaria pelo povoamento imaginário que Gama lhe daria por
toda a vida, com seu criativo panteão afro-greco-latino onde
deuses digladiavam no Quartel de Linha; titãs duelavam na
academia jurídica; ninfas invocadas pousavam por sobre suas
denúncias de juízes hérostratos, que incendiavam templos e
códigos agindo feito licurgos e minos no inalterável
sentido de esmagar os desvalidos de sempre; o povo herculeamente
resistia enquanto esopos fabulavam sobre a fauna política da
várzea do Tamanduateí e terâmenes não se abalançavam da sorte que
as taças de crítias fatalmente impunham; prometeus
roubavam o fogo, lampião e querosene nas esquinas da capital; e grandes
Alexandres ensimesmados desfiavam na espada o enigmático nó da
outorga de privilégios, comendas e baronatos.

Que não se perca de vista que São Paulo seria, para Afro, Atenas
``por antonomásia''. Sugestiva figura de linguagem, a antonomásia é uma
espécie de metonímia que consiste em substituir um nome de pessoa,
cidade ou objeto, entre diversas possibilidades, por outra denominação,
que lhe agregue sentido, explicação ou conotação moral. Tanto
Afro quanto Gama substituem São Paulo por Atenas. A aposta por
essa criativa antonomásia também pode ser lida na ``Carta ao muito
ilustre e honrado sr. comendador José Vergueiro'' que, escrita por Luiz
Gama no ínterim dos artigos citados, substitui São Paulo por ``moderna
Jerusalém''. Vistas em conjunto, elas comunicam uma ideia satírica de
análise da realidade social. Entre o riso, a ironia e a rebeldia,
Afro e Gama descrevem uma cidade, um país e seu povo.

Estamos de acordo com Silvio Roberto Oliveira: ``As faces iluminadas dos
heróis são, em paradoxo, faces obscuras. No caso de Gama, certas
suspeitas motivadas pelas narrativas acerca de sua origem reafirmam a
assertiva''. Estamos diante do mais obscuro pensador brasileiro. Razão
pela qual quiçá se fez o mais luminoso pensador do Brasil. Que \emph{Ça
Ira!} já escreveu: ``A trajetória desse misterioso astro se dirige a uma
grande alvorada. Tranquilizemo-nos''.
