\part{\texorpdfstring{"\textbf{A QUESTÃO É DE
DIREITO"}}{"A QUESTÃO É DE DIREITO"}}

\textbf{*didascália* (argumento)}

\begin{quote}
\emph{Na conclusão de um dos artigos que compõe essa seção --} Foro da
Capital \emph{--, Gama e Brito, coautor do texto, sintetizavam a demanda
em curso com a frase que intitula esse conjunto de artigos: ``A questão
é de direito''. Formada por 29 textos de literatura
normativo-pragmática, a seção é dividida em seis tópicos, que esmiuçam o
tal do direito de que se ocupa a questão. Isto é, se a questão é de
direito, é preciso colocar na mesa de qual direito estamos falando. A
seção começa pela definição de jurisidição, conceito estruturante do
raciocínio jurídico de Gama. Nela, se encontram três textos que tratam
da jurisdição e suas competências, seguidos, então, por textos agrupados
por tipificação criminal: injúria, abuso da liberdade de opinião e de
imprensa; falsificação de moeda; roubo; e homicício. A classificação não
pretende padronizar os textos e engessá-los conforme um tipo ideal,
sobretudo porque tal repartição, além de artificial, representaria, no
limite, uma mistificação. Não se trata disso. A ideia de jurisdição, por
exemplo, perpassa quase todos os tópicos e nem por isso optou-se em
arranjá-la como se o buscador de palavras-chave fosse um ordenador
eficaz. Para o triênio de que se ocupa o volume, contudo, parece
adequado apresentar o trabalho de Gama não como uma linha sequencial
inflexível, no que resultaria uma pilha de textos sem comunicação
temática, mas também não seria lá tão apropriado tratar cada bloco como
um mundo com um fim em si mesmo. O critério, em síntese, foi estabelecer
ligações temáticas como que por áreas afins, calhando serem as tais
afinidades os tipos criminais da época. Dito isso, vejamos que o tópico
da injúria é composto por três artigos; o de abuso da liberdade de
opinião e de imprensa, por apenas um, porém, como se explicará, com
condições de intitular um tópico específico; o de falsificação de moeda
é constituído por dois textos; o de roubo, o maior dos tópicos, por doze
textos de Gama e duas réplicas a ele, totalizando, portanto, catorze
artigos; e o tópico do homicídio, por seis textos. São, como dito, 29
textos no total. As duas seções seguintes, especialmente a próxima --
``\emph{Ladrão que rouba ladrão}'' --, possuem pontos de contato com
essa que é a coluna do volume. Não é demais repisar que este quinto
volume, ``\emph{Crime}'' (1877-1879), embora separado pela
``\emph{Sátira}'' (1876), representa uma continuidade das ideias
centrais e do conhecimento normativo desenvolvido por Gama nos quase
sessenta textos do terceiro volume, ``\emph{Direito}'' (1870-1875). }
\end{quote}

\part{A JURISDIÇÃO**}

\textbf{*didascália* (didascália)}

\begin{quote}
\emph{Três artigos compõem essa seção. O primeiro trata de um ``conflito
de jurisdição'' entre um juiz de órfãos e um juiz de direito no
município de Araraquara (SP). O segundo responde a uma pergunta-chave
para demandas de liberdade, a saber, se o escravizado poderia escolher o
juízo local onde ingressar com uma ação de liberdade. O terceiro artigo
é sobre uma causa de liberdade iniciada na comarca de Santos, que
defendia que o escravizado alforriado não poderia ser posteriormente
vendido ou doado. Decidir qual o juiz competente para julgar uma causa;
cravar qual o local de propositura de uma ação; ou a ideia de vinculação
de precedentes num mesmo tribunal passavam por conceituar o que
constituía a jurisdição e quem possuía a competência para exercê-la. O
tema era de primeira importância. Gama via na ``jurisdição um dever ou
obrigação'' do Estado e a competência como uma incumbência legal, que se
caracterizava por ser um ``dever político'' marcado na forma da lei.
Juízes que exerciam poderes fora da órbita definida na legislação
seriam, por definição, legalmente incompetentes. Identificar os limites
de uma jurisdição, discutir quem possuía a competência para exercê-la e
``as normas salutares da invariabilidade e certeza dos julgamentos''
numa dada jurisdição, portanto, são assuntos teóricos tratados nos três
artigos. Estes, longe de se ocuparem de uma discussão hipotética ou
abstrata, tinham a reflexão teórica pensada e organizada para solucionar
casos concretos que corriam, sobretudo, em juízos e tribunais da
província de São Paulo. }
\end{quote}

\chapter{CONFLITO DE JURISDIÇÃO -- ARARAQUARA}

\footnote{In: \emph{A Província de S. Paulo} (SP), Seção Judiciária,
  Tribunal da Relação, 19/05/1877, pp.~1-2.}

\textbf{*didascália*}

\emph{Literatura normativo-pragmática. Gama critica uma decisão do
Tribunal da Relação de São Paulo e desenvolve o tema da competência
jurisdicional para decidir um litígio. A fundamentação do acórdão --
``perniciosa e injurídica doutrina'' -- é duramente atacada. Os
desembargadores basearam-se em dois decretos do Poder Executivo para
decidir qual juiz -- se um juiz de órfãos ou um juiz de direito --
deveria julgar uma causa que tratava da possibilidade de licença para
casamento de menor de idade. Embora o caso concreto pareça de pouca
repercussão pública, Gama aproveitava para discutir um tema caríssimo
que permeava um sem número de disputas judiciais: a separação (e a
desigualdade) de poderes. ``Sabido é, de há muito tempo'', concluía
Gama, ``que o Poder Judiciário se vai tornando em subserviente
mandatário do Executivo''. As ``privadas conveniências'' levavam juízes
a fazer vista grossa à legislação da matéria e a recepcionar, fora da
competência traçada em lei, decretos do Executivo. Os acórdãos, com
isso, tornavam-se meros reprodutores da vontade do Executivo, quando,
por harmonia dos poderes, deveriam espelhar a vontade da lei do país.
Tal fenômeno certamente não surpreendia o advogado, que escrevia
indignado com a má formação e a fundamentação aplicada pelos
desembargadores. No entanto, ele fazia questão de fincar sua bandeira em
defesa da dignidade do direito. ``Em S. Paulo, porém'', dizia Gama, ``a
verdade é outra: não é a da lei; é a do Poder Executivo, que a revogou;
a incoerência fez-se direito; aqui a jurisprudência é a incerteza; a
incerteza é a razão da justiça!''}

\begin{center}\rule{0.5\linewidth}{\linethickness}\end{center}

Acórdão\footnote{Decisão de tribunal que serve de paradigma para
  solucionar casos semelhantes.} em Relação, etc.

Julgam competente o juiz de direito para concessão ou denegação de
licença, para casamento do menor, compreendendo, nessa autorização,
todos os casos, ainda mesmo o especial de que trata o art. 15, nº 12, do
Decreto de 15 de Março de 1842, em vista da expressa disposição do
Decreto nº 5.467 de 12 de Novembro de 1873\footnote{Embora de modo
  indireto, o acórdão fazia referência ao art. 4º, § 6º, do citado
  decreto, em que havia ``expressa disposição'' sobre concessão e
  licença para casamento de menor de idade.

  Vejamos: Art. 4º: "Pertencem à ordem das decisões que põem termo ao
  feito, e devem ser proferidas pelos juízes de direito das comarcas
  gerais, nas causas que lhes compete julgar, as sentenças seguintes,
  quer delas caiba agravo, quer apelação:

  § 6º: De concessão ou denegação de licença para casamento do menor".}.

S. Paulo, 20 de Junho de 1875.

Aquino e Castro\footnote{Olegário Herculano de Aquino e Castro
  (1828-1906), nascido em São Paulo (SP), foi promotor público, juiz,
  desembargador, presidente do Tribunal da Relação de São Paulo,
  ministro e presidente do Supremo Tribunal Federal. Ocupou, também,
  cargos no Legislativo, como deputado (1867-1870 e 1878-1881), e no
  Executivo, como chefe de polícia das províncias de Goiás, Rio de
  Janeiro e São Paulo, além de presidente da província de Minas Gerais
  (1884-1885).}, presidente.

C. Lima.\footnote{Antonio Cerqueira Lima Júnior (1832-1876), natural da
  Bahia, foi juiz de direito em sua província natal (1856), além das
  províncias do Ceará (1857), Rio Grande do Sul (1858) e Minas Gerais
  (1861, 1872-1873), além de desembargador do tribunal da Relação de São
  Paulo (1874-1876).}

Faria.\footnote{José Francisco de Faria (1825-1902), natural do Rio de
  Janeiro (RJ), foi político e magistrado. Foi chefe de polícia da Corte
  (Rio de Janeiro), juiz de direito, desembargador dos tribunais da
  Relação de Ouro Preto e de São Paulo, procurador da Coroa, Soberania e
  Fazenda Nacional e ministro do Supremo Tribunal de Justiça. Teve
  muitos embates com Luiz Gama na parte contrária, sendo o mais célebre
  aquele em que Gama advogou \emph{habeas-corpus} para o africano congo
  Caetano. Como Gama relata na abertura de seu estudo sobre os efeitos
  manumissórios da proibição do tráfico de escravos, foi a partir de uma
  arguição do desembargador e procurador da Coroa José Francisco de
  Faria que ele resolveu responder ao público a gravidade da matéria.
  ``Este perigoso discurso, este enviesado parecer do respeitável
  magistrado'', respondia Gama, ``obrigou-me a escrever este artigo.''}

Villaça.\footnote{Joaquim Pedro Villaça (1817-1897), nascido na
  província de São Paulo, foi promotor público, juiz municipal e de
  órfãos, juiz de direito, desembargador dos tribunais da relação de
  Ouro Preto (1873) e de São Paulo, onde também foi presidente do
  tribunal (1879), além de ministro do Supremo Tribunal de Justiça
  (1888).}

\begin{center}\rule{0.5\linewidth}{\linethickness}\end{center}

Como bem se vê, o egrégio Tribunal da Relação, pelo acórdão supra, e sob
inscrição de -- \emph{conflito de jurisdição} -- entre os juízes de
órfãos e de direito de Araraquara, resolveu sobre uma \emph{averiguação
de competência}.

Generaliza-se infelizmente, entre nós, na ilustrada província de S.
Paulo, por força de Arestos\footnote{Acórdão, decisão de tribunal que
  serve de paradigma para solucionar casos semelhantes.}, mais
autoritários do que judiciosos\footnote{Sensatos.}, esta nova,
perniciosa e injurídica doutrina, que aos juízes de direito, por
dilatação de esfera, confere atribuições que a lei só em casos especiais
lhes concedeu.

E posto que na província de S. Pedro do Rio Grande do Sul fosse já a
matéria assaz\footnote{Suficientemente, bastante.} discutida, e com
subido critério jurídico, em um recurso crime, interposto para a Relação
de Porto Alegre pelo dr. juiz dos órfãos do termo de Santo Antonio da
Patrulha\footnote{Município do Rio Grande do Sul.}, recurso que obteve
provimento por unânime votação dos julgadores, {[}a{]}praz-me levantar,
aqui, de novo, a questão, para que não passe em silêncio um ato
calculado, que me parece atentatório da sã jurisprudência e da evidente
disposição da lei.

Considerada diretamente a questão, por seus fundamentos, em face do
direito escrito, vem de molde reproduzir a opinião autorizada do
distinto e muito acatado sr. dr. Vicente Ferreira da Silva, que copiamos
de um parecer, por ele dado:

"A jurisprudência que pretendeu firmar o colendo Tribunal da Relação de
S. Paulo, no Acórdão de 20 de Junho do ano precedente, decidindo -- que,
nas comarcas gerais, em todas as hipóteses, cabe ao juiz de direito
\emph{conceder licença} para casamento de órfãos, não pode passar sem
reparo.

Esse acórdão foi proferido para pôr termo à conflito de jurisdição
levantado entre o juiz de direito e o de órfãos de Araraquara.

Escusado é demonstrar que não se deu conflito de jurisdição, nem se
podia dar, entre autoridades de uma mesma jurisdição, que exercem
funções no mesmo território, e sobre a mesma matéria; o que há, e está
bem patente, é --\emph{divergência sobre extensão de competência}.

Dois são os fundamentos do acórdão: \emph{Autoridade} dos avisos e a
\emph{expressa disposição} do Decreto de 12 de Novembro de 1873.

O primeiro, \emph{ratio judicandi}\footnote{Razão do julgamento.}, nem
merece ser refutado em país onde poderes do estado acham-se divididos e
têm a órbita de suas atribuições claramente traçada na Constituição; não
cabendo ao Executivo interpretar leis, máxime\footnote{Principalmente,
  especialmente.} no ramo do direito privado. Tal faculdade já lhe era
recusada no regime antigo, como demonstra Corrêa Telles\footnote{José
  Homem Corrêa Telles (1780-1849) foi jurista, juiz, desembargador,
  historiador do direito e político português. Autor fundamental para o
  desenvolvimento do conhecimento jurídico do século XIX, Corrêa Telles
  foi, como se tem uma amostra nesse artigo, um dos doutrinadores de
  direito civil mais referenciados por Gama em suas argumentações no
  curso dos processos.} com a Resolução de 3 de Novembro de 1792
(Comentário à Lei da Boa Razão, nº 3, pág. 6).

Aquilo que no domínio do absolutismo era vedado observar (Ordenação do
L. 2º, Tit. 41)\footnote{A ordenação disciplinava o uso das portarias de
  subordinados do rei. Em síntese, determinava que agentes do reino,
  nomeadamente da Justiça, não fizessem ``obra alguma por portaria''.}
constitui, hoje, a mais copiosa fonte em que vão haurir suas decisões os
nossos magistrados, de todas as hierarquias. Nem como modo de entender a
lei doutrinariamente podem os avisos ser aceitos: \emph{non exemplis,
sed legibus judicannondum est}\footnote{Julgue-se em obediência às leis
  e não aos casos precedentes; ou, deve-se julgar com leis, não com
  exemplos.}. O juiz deve formar a sua convicção pelo próprio estudo e
não pelas opiniões ministeriais.

O Decreto de 12 de Novembro de 1873, art. 4º, § 1º, nº 6, rege só a
hipótese de haver recusa de consentimento, por parte do pai ou do tutor,
pois só então tem lugar a \emph{decisão} do magistrado, \emph{decisão}
que é determinada por controvérsia e terminada por \emph{sentença}.
Quando, porém, o tutor, pai, ou curador presta o seu consentimento, não
sendo mais possível a oposição por parte de ninguém, a autorização do
juiz é um ato de jurisdição graciosa; não há \emph{decisão} a proferir;
e, portanto, tal hipótese não pode ser regida pelo citado decreto de
1873.

A faculdade que a Ordenação do Liv. 1º, Tit. 88, § 19\footnote{O título
  da Ordenação referia-se aos juízes dos órfãos e o parágrafo citado
  tratava da possibilidade de casamento do órfão ou menor. Vejamos: ``E
  se algum órfão ou menor de vinte e cinco anos, que tiver tutor ou
  curador, se casar sem autoridade do juiz dos órfãos; e o casamento for
  feito por vontade do órfão, ou menor, sem induzimento de pessoa
  alguma; e for o casamento menos daquilo que o órfão ou menor pudera
  achar, segundo a qualidade de sua pessoa, e da fazenda que tiver, não
  lhe mandará o juiz entregar seus bens até chegar a idade de vinte
  anos. E posto que haja carta nossa, ou dos nossos desembargadores do
  paço para que lhe sejam entregues, se nela se não fizer expressa
  menção como assim se casou sem autoridade do juiz dos órfãos, o dito
  juiz não cumprirá tal carta, nem lhe mandará entregar seus bens, até
  chegar a idade de vinte anos. E esta pena haverá outro qualquer que,
  sem autoridade do juiz, casar alguma órfã, ou menor de vinte e cinco
  anos, que tutor ou curador tiver''.}, e o art. 5º, § 8º do Regulamento
de 15 de Março de 1842 confere aos juízes de órfãos não lhe foi tirada
pela reforma e seus regulamentos.

O espírito desta foi aproximar-se, quanto às circunstâncias do país o
permitiam, do preceito constitucional, que vê na perpetuidade dos
magistrados uma condição de independência, para o bom \emph{julgamento
das causas}.

Nada tem, pois, com as funções de caráter administrativo. Tanto o
legislador reconheceu que os juízes de órfãos ficariam com as
atribuições, que não lhes fossem tiradas, que deu-se ao trabalho de
enumerar as que passou para os juízes de direito, como -- o julgamento
de partilhas, contas de tutores, e decisões que ponham termo ao feito
(Reform{[}a{]} Jud{[}iciária{]}, art. 24).\footnote{Refere-se
  indiretamente à Lei nº 2.033 de 20/09/1871, cujo artigo 24, § 1º,
  disciplinava exatamente sobre o tema que Gama discute nesse parágrafo,
  isto é, sobre o ``julgamento das partilhas, conta de tutores, bem como
  qualquer outra decisão definitiva que ponha termo à causa em primeira
  instância''.}

Não podendo separar-se desse pensamento capital, que presidiu à
distribuição da competência, o Regulamento de 1873, artigo 4º, fiel aos
princípios da lei, fala em \emph{feitos}, \emph{decisão} e
\emph{sentença}.

No caso de simples concessão de licença não há \emph{decisão}, nem
\emph{pleito}, nem \emph{sentença}; mas só despacho em requerimento de
curador.\footnote{Aquele que está, em virtude de lei ou por ordem de
  juiz, incumbido de cuidar dos interesses e bens de quem se acha
  judicialmente incapacitado de fazê-lo.}

A este respeito, como nos mais atos da orfanologia\footnote{Multinormatividade
  e doutrina referente à assistência e proteção dos órfãos.}
administrativa, ficaram intactas as atribuições dos juízes de órfãos e
subsiste a competência como estava estabelecida; é o próprio governo
quem o confessa (Aviso de 6 de Abril de 1872).\footnote{Gama se reporta
  ao Aviso nº 97 de 6 de Abril de 1872. Assinado pelo ministro da
  Justiça, o aviso se propunha a sanar dúvidas sobre a reforma
  judiciária de 1871, tratando, em particular, de atribuições de juízes
  municipais e juízes de direito em causas comerciais e sumárias. Ao que
  parece, Gama invocava um diploma legal bastante específico mais por
  expediente retórico do que por força normativa. Afinal, embora o aviso
  respondesse uma consulta de um juiz de órfãos, o governo pouco
  ``confessava'' sobre o estabelecimento da competência em causas
  orfanológicas.}

Autorização para casamento de órfão é atribuição do juiz de órfãos que,
em face da lei, não é lícito pôr em dúvida;

Tal autorização é concedida sem forma nem figura de juízo, de plano e
pela verdade sabida;

Este é o procedimento, desde que o tutor conscienciente\footnote{Na
  eventualidade de erro tipográfico, restam duas possíveis leituras.
  Pode ser lido como o mesmo que conscientemente, ou consenciente, isto
  é, aquele que permite, que está de acordo.} o impetra e concorda o
curador geral (Acc. da Rel. de Porto Alegre, 28 de Março de 1876. Vid.
\emph{Gaz Jurid}. vol.~11, nº 162, pág. 570 a 577)".\footnote{Embora não
  tenha acessado o acórdão na íntegra, é possível dizer que o julgado em
  segunda instância tratava de matéria de escravidão, entre elas, modos
  de aquisição do pecúlio. Sobre o acórdão na doutrina, Cf. Antonio
  Joaquim Ribas\emph{, Consolidação das leis do processo civil}, vol.~2,
  1879, pp.115-116.}

\begin{center}\rule{0.5\linewidth}{\linethickness}\end{center}

Consideremos agora a questão sob outro ponto de vista, o do direito
público, e principalmente do criminal.

Competência, em acepção jurídica, é a incumbência legal de entender em
negócios públicos, ou julgar dos contenciosos.

Toda a incumbência legal, para aquele que a exercita, encerra um dever
político; e todos os deveres políticos cumpre que sejam restritos e
expressamente estatuídos; porque importando tais deveres, para os
incumbidos de sua observância, atos positivos da jurisdição; e sendo a
jurisdição - um poder legal ou autoridade de aplicar as leis, não pode
provir de meras induções, nem deduzir-se de interpretações doutrinárias
do direito, nem ser assumida, \emph{ad nutum}\footnote{Discricionariamente,
  ao arbítrio.}, por presunções despertadas pela pública utilidade; mas
deve resultar de preceitos preestabelecidos pelo poder competente.

É, portanto, a competência uma incumbência social, política e legal; a
incumbência uma jurisdição e a jurisdição um dever ou obrigação; e
porque, na vertente hipótese, a transgressão ou inobservância do dever
importe o cometimento de um delito, sujeito à sanção penal, teríamos, em
face da doutrina do citado Acórdão de 20 de Junho, que os delitos podem
resultar não só de previsões positivas da lei, como de indução
filosófica e de interpretações doutrinárias do direito!

A base do citado acórdão, fora de contestação, é o Decreto nº 5.467 de
12 de Novembro de 1873, art. 4º, nº 6; esta base, porém, quando
contivesse a \emph{expressa disposição} que lhe atribui o acórdão, seria
ela de todo ponto falsa, por ser manifestamente ilegal esta parte do
mencionado decreto.

Aos juízes dos órfãos, como autoridades administrativas, exercendo
jurisdição graciosa, nas comarcas gerais, conferia a Ordenação do Liv.
1º, Tit. 88, § 19, a atribuição de conceder aos órfãos licença para
casarem-se; e, pela legislação em vigor, os juízes de direito, em tais
comarcas, têm faculdade para judiciar a respeito, mediante recurso,
quando o ato assume caráter contencioso.

Aquela atribuição graciosa dos juízes dos órfãos resulta de expressa
disposição de lei pátria, visto que tais foram tornadas as Ordenações do
Reino, por explícita adoção da Lei de 20 de Outubro de 1823\footnote{Aprovada
  no bojo do processo constituinte de 1823, esta lei declarava em vigor
  uma série de normas portuguesas que possuíam inquestionável força
  normativa no Brasil até abril de 1821. O art. 1º da lei fazia
  explícita menção às Ordenações como um desses conjuntos normativos que
  voltavam oficialmente a ter vigência no Brasil. Com a citação de lei
  nacional, Gama procurava realçar a força normativa das Ordenações, o
  que, por extensão, reforçava o seu argumento.}, até hoje não derrogada
por qualquer outra. E se este é o preceito legal integralmente mantido,
é certo igualmente que o Poder Executivo carece de competência para
revogá-lo, ou distendê-lo ampliando, de tal arte, e de \emph{motu
proprio}\footnote{Iniciativa própria, espontaneamente.}, as atribuições
dos juízes de direito; e consequentemente os fatos constitutivos da sua
responsabilidade que, se não fica sujeita a novo e mais perigoso
arbítrio, ficará desastradamente isenta de sanção penal!

Disto resulta, portanto, que, se o Decreto de 1873 encerra a positiva
disposição que lhe empresta o venerando acórdão, contém o mesmo decreto
necessariamente uma interpretação autêntica da lei; e, se contém tal
interpretação, é incontestável que o Poder Executivo também legisla; e
sendo o Poder Executivo delegado à Coroa, que o exercita pelos seus
ministros, resulta de modo inevitável, que os ministros legislam, sem
parlamento, por ordem da Coroa; que o governo do Brasil é despótico; e
que, sem convocação de constituinte, foi revogada, assim como foi
promulgada, a carta constitucional do Império!...\footnote{Embora
  incidental ao argumento central, é de se notar a crítica que Gama faz
  ao processo constituinte brasileiro. A carta constitucional do Império
  -- que o autor parece fazer questão de não chamar de Constituição --
  carecia de proceso constituinte legítimo, haja vista ter sido reunida
  e promulgada sem convocação pelo titular do poder político originário.}

Sabido é, de há muito tempo, que o Poder Judiciário se vai tornando em
subserviente mandatário do Executivo; e que, por causa desta notável
dobrez\footnote{Ambiguidade, também no sentido de dissimulação.}, sob o
peso de privadas conveniências, os esdrúxulos avisos e as desaforadas
exorbitâncias dos regulamentos vão cotidianamente transformando todo o
sistema e plano jurídico da nossa legislação.

A razão, a metafísica, a etnografia social, o progresso moral, a
civilização e a política, já não constituem elementos de hermenêutica;
os juízes deixaram os labores do jurisconsulto; há um sábio no país,
pensa o governo; rege o aviso; reflete o regulamento; está em paz a
ciência; está a pátria salva; tripudiam de júbilo os tribunais!

A Relação de Porto Alegre manda processar o juiz de órfãos do termo da
Patrulha, porque ilegalmente concedera licença para o casamento de um
menor; o crime é reconhecido em primeira instância, o juiz é
pronunciado; recorre da sentença e mostra que, perante a lei, era ele o
juiz competente, único, para, no caso dado, conceder a licença; o
tribunal aceita a doutrina das alegações, reconhece a verdade da lei, e,
por votação unânime, absolve o juiz!

Em S. Paulo, porém, a verdade é outra: não é a da lei; é a do Poder
Executivo, que a revogou; a incoerência fez-se direito; aqui a
jurisprudência é a incerteza; a incerteza é a razão da justiça!

S. Paulo, Março de 1877.

L. GAMA.

\chapter{Tem o escravo escolha de foro para a propositura de ação
manumissória?}

\footnote{In: \emph{A Província de S. Paulo} (SP), Seção Judiciária,
  Tribunal da Relação, 02/08/1877, p.~2. O \emph{Correio Paulistano}
  repercute a publicação desse estudo chamando-o de ``uma análise
  jurídica do advogado sr. Luiz Gama (...).'' A ação manumissória era
  uma das formas processuais pelas quais se demandava a liberdade.}

\textbf{*didascália*}

\begin{quote}
\emph{Literatura normativo-pragmática. Gama disseca os fundamentos
jurídicos de um acórdão do Tribunal da Relação de São Paulo que decidiu
pela limitação do direito do escravizado em propor uma causa de
liberdade. Ao negar um recurso de uma pessoa escravizada, os
desembargadores do tribunal paulista fixavam uma doutrina que
aniquilaria a possibilidade de alguém escravizado demandar sua liberdade
fora do domicílio em que vivia. Diziam os desembargadores que não
caberia exceção -- mesmo em matérias de liberdade -- ao ``princípio
geral que estabelece a competência do juiz do domicílio do réu''. Se
esta parece uma questão menor, basta pensar em uma pessoa escravizada em
fuga, ou seja, alguém distante do domicílio do réu (nesse caso, contra
quem se demandava). O tribunal decidia, portanto, que o escravizado, se
quisesse lutar por sua liberdade, deveria voltar ao local de onde havia
fugido. Para bom entendedor, o que era certamente o caso de Gama, a
decisão do tribunal significaria antes a morte brutal do que a
possibilidade de um julgamento razoável. Cabia, nesse sentido, desmontar
os pressupostos do acórdão e construir uma repsosta normativa que
favorecesse, na prática, demandas de liberdade em qualquer jurisdição. }
\end{quote}

\begin{center}\rule{0.5\linewidth}{\linethickness}\end{center}

"Acórdão\footnote{Decisão de tribunal que serve de paradigma para
  solucionar casos semelhantes.} em Relação, etc.

Negam provimento ao agravo\footnote{Recurso a uma instância superior
  interposto a fim de se reformar ou modificar decisão interlocutória de
  juiz ou membro de tribunal inferior.} interposto do despacho de fls.
3,\footnote{Por estar no plural, refere-se à frente e ao verso da folha}
por quanto, vistos os autos, foi o mesmo proferido de conformidade com o
direito. O princípio geral que estabelece a competência do juiz do
domicílio do réu, para conhecer das ações contra este intentadas, não
acha exceção na espécie de que se trata.

Ainda nas causas de liberdade, movidas de conformidade com a Lei de 28
de Setembro de 1871\footnote{Refere-se à conhecida Lei do Ventre Livre,
  que declarava livres os filhos da mulher escravizada nascidos a partir
  da promulgação daquela lei. A lei também regulava outras matérias, a
  exemplo do processamento e julgamneto de causas de liberdade.}, e seu
regulamento, prevalece o princípio de só deverem elas ser intentadas no
foro do domicílio do réu. O privilégio de escolha de juiz, invocado pelo
agravante, é insustentável no regime judiciário, que vigora. A Ord. Liv
3º, Tit. 5º,\footnote{A ordenação trata ``dos que podem trazer seus
  contendores à Corte por razão dos seus privilégios''. Embora possuísse
  força normativa, haja vista o agravante tê-la invocado, o título 5º
  confrontava o disposto na Constituição -- nomeadamente o art. 179, §
  16 --, que aboliu privilégios que não tivessem ``utilidade pública''.
  Assim como os desembargadores consideravam essa ordenação
  ``insustentável no regime judiciário'', Cândido Mendes compreendia, no
  mesmo sentido, que a ``prática tem dado como revogada essa Ord.''. Cf.
  \emph{Ordenação e leis do Reino de Portugal}, Terceiro Livro, p.~10,
  1870. Cândido Mendes de Almeida.} em que se funda o
agravante\footnote{Quem interpõe o recurso de agravo.}, nenhuma
aplicação tem ao caso e, quando tivesse, é sempre certo que na prática
se tem dado como revogada a mesma Ord., em face do disposto no art. 179,
§ 16, da Const{[}ituição{]} do Império.\footnote{Art. 179. "A
  inviolabilidade dos direitos civis e políticos dos cidadãos
  brasileiros, que tem por base a liberdade, a segurança individual e a
  propriedade, é garantida pela Constituição do Império, pela maneira
  seguinte:

  § 16º. Ficam abolidos todos os privilégios que não forem essenciais e
  inteiramente ligados aos cargos por utilidade pública".

  O modo de construção da frase não deixa dúvida de que os
  desembargadores consultavam a edição de Cândido Mendes para formular o
  acórdão. Cf. Cândido Mendes de Almeida, \emph{Ordenação e leis do
  Reino de Portugal}, Terceiro Livro, 1870, p.~10.}

Os favores que a Legislação atual tem outorgado à liberdade não importam
o desconhecimento dos direitos do senhor. Tão garantido é pela lei o
direito de propriedade como o de liberdade. A doutrina sustentada pelo
agravante tornaria desigual a posição dos litigantes, e iria de encontro
ao preceito legal. E assim mandam que para os devidos efeitos subsista o
despacho de que se agrava, pagar as custas ex-causa.\footnote{Pela
  causa.}

S. Paulo, 20 de Março de 1874".

ALENCAR ARARIPE.\footnote{Tristão de Alencar Araripe (1821-1908),
  nascido em Icó (Ceará), foi político, magistrado e escritor. Ocupou
  diversos cargos no Judiciário, sendo juiz municipal e de direito,
  desembargador e presidente dos tribunais da relação da Bahia e de São
  Paulo, além de ministro do Supremo Tribunal Federal, onde se
  aposentou. Foi chefe de polícia das províncias do Espírito Santo,
  Pernambuco e Ceará, presidente das províncias do Pará e de São Pedro
  do Rio Grande do Sul e ministro da Justiça.}

AQUINO E CASTRO.\footnote{Olegário Herculano de Aquino e Castro
  (1828-1906), nascido em São Paulo (SP), foi promotor público, juiz,
  desembargador, presidente do Tribunal da Relação de São Paulo,
  ministro e presidente do Supremo Tribunal Federal. Ocupou, também,
  cargos no Legislativo, como deputado (1867-1870 e 1878-1881), e no
  Executivo, como chefe de polícia das províncias de Goiás, Rio de
  Janeiro e São Paulo, além de presidente da província de Minas Gerais
  (1884-1885).}

J. N. DOS SANTOS\footnote{José Norberto do Santos (?-?) foi político e
  magistrado. Presidiu a província do Rio de Janeiro e foi desembargador
  nos tribunais do Maranhão, Bahia, Rio de Janeiro e São Paulo, onde
  também foi presidente desse tribunal (1874-1875).}\\
A. L. DA GAMA.\footnote{Agostinho Luiz da Gama (?-1880), nascido na
  província do Mato Grosso, foi político e magistrado. Exerceu os cargos
  de juiz municipal, juiz de direito e desembargador do Tribunal da
  Relação de São Paulo. Foi chefe de polícia das províncias da Bahia,
  Pernambuco e na Corte (Rio de Janeiro), além de presidir a província
  de Alagoas.}

\begin{center}\rule{0.5\linewidth}{\linethickness}\end{center}

Nas discussões, em geral, como ainda na de que ora nos ocupamos, para
que bem se possa argumentar, e melhor concluir, preciso é bem assinalar,
e com critério distinguir, os pontos cardeais da questão.

Como se vê do venerando acórdão, que deixamos transcritos, o escravo não
tem escolha do foro, para propositura de ação manumissória; e não tem
tal escolha, pelas seguintes razões, que vamos reproduzir,
enumerando-as, com escrupulosa fidelidade:

1º: Porque o princípio geral, em que estabelece a competência do juiz do
domicílio do réu, para conhecer das ações contra ele intentadas, não
acha exceção na espécie de que se trata;

2º: Porque ainda nas causas de liberdade, movidas de conformidade com a
Lei de 28 de Setembro de 1871, e seu regulamento, prevalece o princípio
de só deverem ser elas intentadas no foro do domicílio do réu;

3º: Porque o privilégio de escolha de juiz, invocado na vertente
hipótese, é insustentável no regime judiciário, que vigora; visto como,

4º: A Ord. Liv. 3º, Tit. 5º, invocada, nenhuma aplicação tem ao caso; e,

5º: Quando tivesse aplicação ao caso, é sempre certo que, na prática, se
tem dado como revogada a mesma Ord., em face do disposto no art. 179, §
16, da Constituição do Império;

6º: Porque os favores que a Legislação atual tem outorgado à liberdade
não importam o desconhecimento dos direitos do senhor; e tanto que,

7º: Tão garantido é, pela lei, o direito de propriedade, como o de
liberdade;

8º: Porque a doutrina contrária tornaria desigual a posição dos
litigantes, e iria de encontro ao preceito legal.

Estes fundamentos, porém, não procedem; porque, além de carecerem de
razão jurídica, não se esteiam em disposição legal; e antes são
evidentemente contrários ao direito escrito, e atacam, de modo
inconveniente, se não desastroso, a própria moral judiciária.

E não procedem estes fundamentos:

O primeiro -- Porque o princípio geral, que estabelece a competência de
juiz do domicílio do réu, para conhecer das ações contra ele intentadas,
tem limites na lei; tais limites encerram exceções à regra geral; e as
exceções acham razão e fundamento na moral, no direito, e no público
interesse; o limite está na Ord. do Liv. 3º, Tit. 5º, § 3º,\footnote{Gama
  busca na mesma ordenação citada no acórdão, muito embora em outro
  parágrafo, o 3°, fundamento para seu argumento. Nessa passagem da
  ordenação, Gama encontra fulcro para sustentar que o escravizado
  possuía, sim, o favor de escolher o local da propositura da ação
  manumissória. Afinal, conforme tal parágrafo, ``o órfão varão menor de
  catorze anos e a fêmea menor de doze, e a viúva honesta, e pessoas
  miseráveis, ainda que sejam autores, têm privilégio de escolher por
  seu juiz os corregedores da corte, ou juiz de ações novas (...)''. A
  continuidade do raciocínio, ao que passaria a explicar, tratava de
  equiparar o escravizado à pessoa miserável, o que Gama fazia, por
  outra parte, com igualmente sólido repertório doutrinário.} que aos
MISERÁVEIS outorga o favor de trazerem aos seus contendores à corte;
isto é ao foro da capital, foco de maior civilização, onde está situado
o colendo Tribunal, fora da perniciosa influência de localidade, onde
predomina indebitamente a rude vontade do grosseiro potentado; esta
Ordenação é lei vigente do Império, pela de 20 de Outubro de
1823\footnote{Aprovada no bojo do processo constituinte de 1823, esta
  lei declarava em vigor uma série de normas portuguesas que possuíam
  inquestionável força normativa no Brasil até abril de 1821. O art. 1º
  da lei fazia explícita menção às Ordenações como um desses conjuntos
  normativos que voltavam oficialmente a ter vigência no Brasil.}, que
explicitamente admitiu-a; são pessoas miseráveis, na frase da lei, -- as
viúvas, os órfãos, \emph{os escravos}, \emph{que litigam pela sua
alforria}, e outras que, por certas circunstâncias, a estas possam ser
comparadas (Ago. Barbos. \emph{appelativ}. 152, nº 5; Novar. \emph{de
privileg}. \emph{miserabil}. \emph{person}. \emph{prelud}. 8, nº 6;
etc.\footnote{Mantenho excepcionalmente a referência abreviada haja
  vista a dificuldade, até o momento, em cravar qual a citação exata. De
  todo modo, é bastante provável que Gama se reporte ao jurista
  português Agostinho Barbosa (1589-1649) e uma de suas obras
  civilísticas.} -- Repert. Ord. \emph{verb}. - \emph{Mis}. pág. 543 e
not. (a)\footnote{Provavelmente, refere-se ao verbete ``miserável'' do
  \emph{Repertório das Ordenações e Leis do Reino de Portugal} (1795).};
a causa manumissória é considerada, em direito, -- \emph{causa
pia}\footnote{Caridosa, piedosa.}; porque, no dizer dos jurisconsultos
-- ``é o cativo uma pessoa miserável de condição, que necessita, pelo
seu estado lamentável, da eficaz proteção da lei, para fazer valer os
seus direitos naturais, dos quais foi casualmente privado pela lei
civil''; -- e tão sagrado era considerado este rigoroso preceito da lei
civil, pelo qual foi conferida ao escravo a faculdade de escolher juiz,
para propositura de ação manumissória, que, em Portugal, no ano de 1615,
movendo-se dúvida, porque um escravo propusera ação contra um
fidalgo-cavalheiro, com vencimento de moradia, que também tinha
privilégio, para escolher juiz, perante o qual fosse demandado (Ord.
Liv. 3º, Tit. 61, § 1º)\footnote{Por provável erro tipográfico, a
  referência não corresponde ao teor do argumento.} -- julgou-se, que
nas causas sobre liberdade \emph{tinha o escravo maior privilégio}, e
podia escolher o juiz que lhe parecesse, sem permissão de declinatória
da parte do demandado senhor, ainda mesmo quando fidalgo-cavalheiro
fosse, com privilégio de moradia; e assim julgavam os sábios juízes do
absolutismo, que, nas árduas interpretações do direito político,
desatendiam, com critério, os privilégios emanados de concessões régias,
para observar, com restrição e civismo, os ditames piedosos da reta
moral e sã consciência;

O segundo -- Porque ainda mesmo nas causas de liberdade, movidas de
conformidade com a Lei de 28 de Setembro de 1871, e seu Regulamento nº
5.135 de 13 de Novembro de 1872\footnote{Com mais de 100 artigos, esse
  regulamento, aprovado pelo decreto de número indicado no corpo do
  texto, regulava e modulava os efeitos da Lei do Ventre Livre (Lei nº
  2.040 de 28/09/1871).}, não há limitação alguma ao princípio
sabiamente estatuído na citada Ordenação do Liv. 3º, Tit. 5º, § 3º,
adotada pela Lei de 20 de Outubro de 1823; os casos manumissórios
estabelecidos na Lei de 28 de Setembro de 1871, que respeitou e manteve
os preceitos da legislação anterior, \emph{são especiais}; ainda quando,
portanto, tal limitação houvesse, por ela se não derrogava o
\emph{princípio geral}, como bem determinam os Assentos de 16 de
Novembro de 1700, e 3º de 9 de Abril de 1772\footnote{Os julgados
  mencionados, provenientes das Casas da Suplicação e do Cível, em
  Lisboa, não parecem ter relação direta com o argumento que Gama
  constrói no parágrafo. Ambos tratam de temas distintos e alheios à
  matérias que levassem a um ``princípio geral'' derrogado, razão pela
  qual não apresentarei adiante as ementas dos assentos. Ao que me
  parece, salvo melhor juízo, Gama trouxe os assentos como expediente
  retórico ornamental para intrincar o argumento e quiçá confundir
  potenciais replicantes. Os assentos podem ser consultados no excelente
  repositório digital:
  \href{http://www.governodosoutros.ics.ul.pt/}{{http://www.governodosoutros.ics.ul.pt}}.};
o egrégio Tribunal da Relação da Corte, que, como os demais do Império,
é porção da grande Babel judiciária do país, mesmo depois da promulgação
da Lei de 1871, e ainda em o ano precedente, em mais de um acórdão,
reconheceu e confirmou, em benefício do escravo, o direito de escolha de
juiz, fora do conhecido domicílio do senhor; o Decreto nº 4.835 de 1º de
Dezembro de 1871\footnote{Para execução do art. 8º da Lei do Ventre
  Livre, o decreto definia o regulamento para a matrícula especial dos
  escravizados e dos filhos da mulher escravizada.}, para o caso da
matrícula especial do escravo, e para o efeito da manumissão por conta
do Estado, concede ao escravo, em certa condição, \emph{residência
especial}, e distinta da do senhor; a Lei nº 2.040 de 1871, outorga ao
escravo, para manumissão, por meio de pecúlio\footnote{Patrimônio,
  quantia em dinheiro que, por lei (1871), foi permitido ao escravizado
  constituir a partir de doações, legados, heranças e diárias
  eventualmente remuneradas.}, \emph{direito de petição}; o Decreto de
12 de Abril de 1832\footnote{O decreto regulava a execução da Lei de 7
  de Novembro de 1831. Gama, por sua vez, fazia referência indireta ao
  art. 10 do decreto que reconhecia de modo bastante enfático a
  capacidade jurídica do preto (sublinhe-se, não escravizado) requerer
  sua liberdade com base no tráfico ilegal. Gama equipara categorias
  jurídicas que sabia bastante distintas -- ``preto'' e ``escravo'' --
  para reforçar seu argumento, isto é, a formação e extensão de um
  direito de ação ao escravizado, assim como discutir a questão nos
  termos da lógica senhorial a um só tempo escravista e racista. Dada a
  força normativa do artigo, que Gama exploraria outras vezes, leiamos-o
  na íntegra desde já. Art. 10. ``Em qualquer tempo, em que o preto
  requerer a qualquer juiz, de paz ou criminal, que veio para o Brasil
  depois da extinção do tráfico, o juiz o interrogará sobre todas as
  circunstâncias que possam esclarecer o fato, e oficialmente procederá
  a todas as diligências necessárias para certificar-se dele, obrigando
  o senhor a desfazer todas as dúvidas que se suscitarem a tal respeito.
  Havendo presunções veementes de ser o preto livre, o mandará depositar
  e proceder nos mais termos da lei.''}, expedido para execução da Lei
de 7 de Novembro de 1831\footnote{Considerada uma lei vazia de força
  normativa, recebendo até o apelido de ``lei para inglês ver'', a
  conhecida ``Lei de 1831'' previa penas para traficantes de
  escravizados e, de maneira não tão assertiva como a historiografia
  crava, declarava livres os escravizados que chegassem ao Brasil após a
  vigência da lei.}, respeitando o disposto na Ordenação do Liv. 3º,
Tit, 5º, citada, e na Lei de 10 de Março de 1682\footnote{O alvará
  regulava a liberdade e a escravidão de negros apreendidos na guerra
  dos Palmares, na antiga capitania de Pernambuco. Conhecido da
  historiografia sobretudo pela regulação da prescrição do cativeiro
  após cinco anos de posse da liberdade, nesse texto Gama se reporta a
  outro comando normativo do alvará -- possivelmente o quinto parágrafo
  --, no qual o rei de Portugal outorgava que os cativos poderiam
  demandar e requerer liberdade, ainda que contra o interesse de seus
  senhores.}, expressamente declara o escravo hábil para requerer a sua
manumissão, perante qualquer juiz de paz ou criminal, que lhe convenha;
e, consequentemente, obriga o senhor a vir responder perante o juiz
escolhido pelo escravo; e o mesmo princípio foi repetido na Lei de 4 de
Setembro\footnote{A conhecida Lei Eusébio de Queiroz -- Lei de 4 de
  Setembro de 1850 -- estabelecia medidas, ritos e punições para
  reprimir o tráfico atlântico de escravizados.}, e no Decreto de 14 de
Outubro de 1850\footnote{Regulava a execução da Lei Eusébio de Queiroz,
  definindo como se dariam a repressão, o processamento e o julgamento
  dos contrabandistas.}; é claro, pois, e claro até à evidência, que o
segundo fundamento do venerando acórdão não passa de mera invenção
poética, e de todo ponto contrária ao direito e à jurisprudência dos
Tribunais;

O terceiro -- Porque o privilégio -- de escolha do juiz --, invocado na
vertente hipótese, é incontestável, no regime judiciário que vigora;
porque, estatuído em lei, como se acha, e fica plenamente demonstrado,
só poderá desaparecer por disposição positiva, de nova lei, que
precisamente o revogue;

O quarto -- Porque a Ordenação do Liv 3º, Tit. 5º, invocada, é lei
brasileira, feita pelo Poder Legislativo, e está em pleno vigor; e para
que não tenha aplicação ao vertente caso, do que se infere - \emph{que
terá para outros}, como afirma-se arbitrariamente, no venerando acórdão,
indispensável é que se demonstre que a miséria tornou-se indigna do
favor público, ou que os preceitos de piedade incompatibilizaram-se com
os bons sentimentos, e tornaram-se alheios às regras e princípios de
direito, e normas de sociabilidade, ou que a lei é contraditória, ou que
o manumitente não é pessoa miserável!...;

O quinto -- Porque é certo, e fora de contestação, que, se na prática,
não tem sido observada esta Ordenação, o que aliás não é exato porque a
Relação da Corte, como já o dissemos, há pensado inversamente, será
antes por incúria\footnote{Negligência, desleixo ou falta de iniciativa.}
ou por inadvertência dos julgadores, do que por exceção, como
equivocadamente pretende-se no venerando acórdão; nem tampouco porque
tenha sido revogada pelo art. 179, § 16 da Constituição do Império; a
carta constitucional, abolindo resolutamente os privilégios, fez
claríssima exceção dos que fossem \emph{julgados essenciais} e
inteiramente ligados aos cargos, por utilidade pública; e evidente é,
que, em tais termos, referiu-se precisamente o poder, com a imposta
abolição, às concessões honoríficas, graciosas e pessoais, e às regalias
de ordem privada; e não interessou às de ordem pública, que foram, do
modo o mais escrupuloso ressalvados; e muito menos as especialíssimas,
consagradas no direito civil, por princípios benéficos de piedade, para
apoio e justa proteção da miséria; à menos que os modernos juristas, com
entono\footnote{Sentimento de amor-próprio, que pode ser entendido como
  orgulho, vaidade.} pindárico\footnote{Por sentido figurado, suntuoso,
  magnífico.}, se bem que baldos\footnote{Desprovidos, carentes.} de
senso jurídico, não pretendam, de um só jato, que a carta\footnote{Isto
  é, a Carta de 1824, que Gama habilmente se esquivava em chamar de
  Constituição.} eliminando aqueles privilégios, abolisse também a
miséria, e com ela, no radiado golpe capitolino\footnote{No sentido de
  imponente, triunfal.}, a viuvez, a orfandade e o cativeiro! -- quanto
à mantença\footnote{Manutenção, custeio.} dos privilégios de ordem
pública, quer interessem diretamente aos serventuários do Estado, quer
particularmente aos indivíduos, que, por sua condição excepcional,
necessitam do auxílio peculiar da autoridade, para a defesa regular da
sua causa, é fato inconcusso\footnote{Inquestionável, indiscutível.},
que avulta em a nossa legislação; (Deixamos de citar, para não
alongarmos inutilmente este escrito, as disposições que concedem
privilégio de foro aos militares, aos legisladores, aos presidentes de
Província, aos ministros, aos bispos, etc.; com relação ao mandato --
aos príncipes, arcebispos e bispos, aos duques, marqueses, condes,
doutores, militares, etc.; com relação aos miseráveis -- aos ofendidos,
etc.);

O sexto -- Porque os favores que a Legislação atual, que é a mesma em
que nos esteiamos, tem outorgado {[}que{]} aos manumitentes, não
importam negação dos direitos dominicais\footnote{Senhoriais.}; e
apenas, por eles, cuidou o legislador de coibir inveterados\footnote{Bastante
  antigos, arraigados.} abusos;

O sétimo -- Porque \emph{em direito} é desconhecida a propriedade do
homem sobre o homem; o cativeiro é um fato anormal, transitoriamente
mantido pelos governos, porém repelido formalmente pelo direito; a
liberdade é de direito natural (\emph{Lei 30 de Julho de
1609})\footnote{Embora se trate de lei relativa à proibição do cativeiro
  de índios no Brasil do início do século XVII, Gama cita-a para
  reforçar seu argumento sobre o direito natural à liberdade. O motivo
  para escolhê-la como um dentre os fundamentos normativos do direito à
  liberdade devia-se mais ao efeito persuasivo de coligir uma lei que já
  contava com quase três séculos de existência, do que ao seu conteúdo
  normativo ambíguo que vacilava sobre as razões de se manter ou não o
  cativeiro no Brasil.}; nas causas que sobre ela versarem, \emph{pode o
juiz dispensar na lei}, para mantê-la (Ord. Liv. 4º, Tit. 11, §
4º)\footnote{O longo parágrafo quarto começa com a célebre sentença que
  se leria em muitas ações de liberdade no Brasil do século XIX: ``E
  porque em favor da liberdade são muitas cousas outorgadas contra as
  regras gerais''.}; \emph{porque o cativeiro é contra a natureza} (cit.
Ord. Tit. 42)\footnote{A ordenação citada cuida de assunto diverso -- da
  não obrigação da pessoa morar em local onde não queira ficar. No
  entanto, em rápido relance, se admite que o cativeiro ``é contra
  {[}a{]} razão natural''.}; no Brasil não há lei alguma que instituísse
o cativeiro; o suposto direito dominical é uma ficção odiosa,
ilegalmente mantida, por circunstâncias imperiosas, que os poderes do
Estado, compelidos pela vontade pública, tratam com afano\footnote{O
  mesmo que afã, empenho.} de remover.

O oitavo, finalmente -- Porque, sendo essencial a igualdade de posições
dos litigantes, em juízo para a regular propositura e desenvolvimento
dos pleitos, são indispensáveis os favores da lei, em prol dos
miseráveis, que, na ausência de tais favores, serão vítimas da
prepotência dos grandes, que tudo dominam; o que o venerando acórdão
denominou -- \emph{desigualdade --}, é, pelo contrário, o que o
legislador, com muita sabedoria, instituiu, para equilíbrio das
posições, em juízo.

*

Julgamos ter discutido e demonstrado, à face da lei, que a resolução
jurídica, a resolução legal, a resolução que não ataca os verdadeiros
fundamentos do direito, nem os preceitos de moral, nem os puros
sentimentos de piedade, que tanto enobrecem o elevado caráter dos
legisladores e dos juízes dos povos cultos, em questões gravíssimas, de
máximo interesse social, como a de que nos ocupamos, não é certamente a
que, com mais paixão do que civismo, adotou o venerando Acórdão de 20 de
Março de 1874; se não a que, talvez por equívoco, em menos
desafortunados pleitos, seguiu o colendo Tribunal da Relação da Corte; a
que, por muitas vezes, com admirável isenção, e menosprezo de
favoneados\footnote{Protegidos.} preconceitos, firmaram os doutíssimos
juízes, e os severos Tribunais de Portugal, que sabiam, em nome da
razão, e em homenagem aos direitos naturais do homem, sem infração da
lei civil, antepor o justo interesse do escravo, causa nobilíssima da
redenção, ao orgulho exulado\footnote{Exilado.} de hiperbólicos
senhores, aos privados interesses dos dominadores do Estado, às graças
do Rei, que era bastante poderoso para criar nobrezas, para instituir
privilégios, para decretar e revogar as leis; porém somenos\footnote{Inferiores,
  irrelevantes.} para dominar altivas e retas consciências, e para
impedir os ditames da justiça: eram juízes e tribunais livres, que, à
semelhança do Sol, erguiam-se mais alto do que as cúpulas dos tronos.

S. Paulo, 25 de Junho de 1877.

L. GAMA.

\chapter{Escravo alforriado em testamento pode ser vendido ou doado
pelo manumissor?} % Bruno cap. 3. 

\footnote{In: \emph{A Província de S. Paulo} (SP), Seção Judiciária,
  Tribunal da Relação, Foro de Santos, 14/11/1877, p.~2.}

\textbf{*didascália*}

\emph{Literatura normativo-pragmática. Embora assinado por Antonio
Carlos Andrada Machado e Silva, credito excepcionalmente esse artigo, ao
menos em parte, a Luiz Gama. É certo que eles eram sócios e amigos
íntimos havia já muitos anos. Gama e Antonio Carlos dividiam o mesmo
escritório, trabalhavam juntos diariamente, tinham afinidades políticas
e até mesmo convivência familiar. No entanto, Antonio Carlos nunca foi
assíduo em debates jurídicos nos jornais. Custa encontrar, mesmo no
longo espaço de uma década, um grupo de artigos doutrinários que ele
tenha publicado na imprensa. Que dirá ainda se se reduzir o enfoque
especificamente para literatura normativo-pragmática relacionada com
matérias de escravidão. Para não dizer impossível, certamente será
difícil encontrar uma reflexão autoral sobre o tema, principalmente,
sublinhe-se, por se tratar de um jurisconsulto especializado em causas
comerciais e empresariais, haja vista que era professor catedrático de
Direito Comercial da Faculdade de Direito de São Paulo. Isso posto, por
que Antonio Carlos iria agora -- uma única vez -- aos jornais tratar de
uma alforria testamentária, repisa-se, tema que não dominava e, mais
ainda, era competência do seu sócio? Por que Antonio Carlos adentraria
em território senão desconhecido, ao menos inóspito e de todo irregular
para ele? As perguntas são muitas e pararei por aqui. As hipóteses,
aliás cotejadas em espaço apropriado para elas, se afunilam para uma
possibilidade que se afigura verossímel, sobretudo se se destaca a data
da escrita desse texto e os eventos paralelos que corriam no mesmo
Tribunal da Relação de São Paulo. A hipótese mais forte, em rápida
síntese, sugere que Gama não poderia, excepcionalmente, dada a outra
ação em curso, sustentar embargos no Tribunal sem comprometer a outra
causa em andamento. Entre o final de outubro e meados de novembro, Gama
estava na linha de tiro dos desembargadores. Um passo em falso e ele
poderia pôr a perder o processo Largacha. Desde junho que Gama pelejava
publicamente com alguns desembargadores sobre o famoso crime da
alfândega de Santos, sendo que a coisa ganhara complexidade com o
acórdão de 19/10/1877, emitido em desfavor de Largacha. Como se verá, a
causa de Largacha mereceu de Gama a escrita do maior texto que se tem
notícia de toda a sua literatura normativo-pragmática, lançado,
inclusive, tão somente quatro dias após esse que é assinado por Antonio
Carlos. Contudo, o problema para Gama não parecia ser debruçar-se sobre
outra causa, esruturar-lhe os argumentos e instruir-lhe os documentos
concomitantemente ao preparo de ação distinta. De modo algum. O problema
central seria que a discussão em sessão e possíveis repercussões na
imprensa poderiam entornar o caldo e aumentar rusgas com desembargadores
em momento inapropriado. Assim, Gama contaria com Antonio Carlos, seu
sócio, para sustentar os tais embargos por ele preparados. Outras
hipóteses, todavia, concorrem para a dobradinha Antonio Carlos e Gama.
Mas deixemos para debatê-las em hora e espaço oportunos. Por ora,
tenhamos em vista o conhecimento normativo assombroso organizado para
responder a instigante questão jurídica: o ``escravo alforriado
emtestamento pode ser vendido ou doado pelo manumissor?'' Através de um
repertório normativo gigantesco, tanto em escala temporal quanto
geográfica, Antônio Carlos / Gama, conceituam o que era uma manumissão,
ou alforria, quais os requisitos que a tornavam juridicamente perfeita
e, portanto, irrevogável, além de suas modalidades de concessão,
especialmente as de ``causa mortis'' e ``inter vivos''. O artigo é uma
aula de direito. Interpreta o lugar do cativeiro nas fontes do direito
brasileiro e não titubeia em afirmar a primazia da liberdade dentro,
inclusive, da tradição jurídica do ``direito civil pátrio''. }

\begin{center}\rule{0.5\linewidth}{\linethickness}\end{center}

\emph{Semel autem causa probata, sive vera sit, sive falsa, non
retractatur}

(Inst{[}itutas{]}, J{[}ustiniano{]}, L{[}ivro{]} 1º, T. 6º, § 6º{]}

\emph{Haverá juiz sisudo que acredite na legalidade da venda de um
escravo alforriado em testamento que não foi revogado pelo testador?}

\emph{(}Dr.~C. A. Soares)\footnote{Caetano Alberto Soares (1790-1867),
  nascido na ilha da Madeira, Portugal, foi um sacerdote católico e
  advogado radicado no Brasil. Foi um dos fundadores e presidente do
  Instituto dos Advogados do Brasil (1852-1857).}

\_\_\_\_

SUSTENTAÇÃO DE EMBARGOS NO TRIBUNAL DA RELAÇÃO

I

Senhor.

A manumissão ou alforria é a concessão legal e perpétua do cativeiro,
qualquer que seja o fundamento, qualquer que seja o modo porque {[}pelo
qual{]} se determine o ato.

A manumissão ou alforria, uma vez concedida, \emph{por qualquer meio
lícito}, é irrevogável (LL. 20, 26 {[}e{]} 33, Cód. \emph{de liber
causa}, etc. Lei nº 2.040 de 28 de Setembro de 1871, arts. 4º, § 9º;
Acórdãos {[}da{]} Relação {[}da{]} Corte, 24 de Abril de 1847, 29 de
Fevereiro, 21 de Outubro de 1848; Sup. Trib. Just., 5 de Fevereiro de
1850, 20 de Dezembro de 1873); porque faz do liberto cidadão (Inst.
Just. Liv. 1º, Tít. 5º, § 3º, pág. 2. \emph{Sed dedititorum quidem
pessima}, etc.; Const. Pol. do Imp., art. 6º, § 1º, art. 7º, §§ 1º, 2º,
3º; Cód. Filip. pág. 865, not. 3; pág. 866, n. 1).

A irrevocabilidade\footnote{Impossibilidade de revocar, anular, revogar.}
provém de que a escravidão tem a sua origem do direito das gentes
(\emph{Servitus autrem est constituto juris gentium}); e \emph{é apenas
mantida transitoriamente} pelo direito civil; e não existindo no direito
civil pátrio disposição alguma mediante a qual se instituísse o
cativeiro (Inst. cit. Liv. 1º, Tít. 3º, § 4º - \emph{aut jure civili},
etc.), é certo, é inconcusso, que a liberdade uma vez adquirida,
torna-se irrevogável (Inst. Just. cit. Liv. 1º, Tít. 3º, § 2º -
\emph{Serviutus autem est}, etc.; Const. Pol. do Imp. Loc{[}ução{]}
cit.).

A \emph{fórmula} de direito das gentes, por a qual estabeleceu-se a
escravidão entre os Romanos, é de todo ponto inaplicável ao Brasil;
portanto, a mesma \emph{fórmula}, que inadmissível é para o
estabelecimento da escravidão entre nós, não pode autorizar o
reestabelecimento daquilo que justamente foi deposto (Inst. Just. cit.
Liv. 1º, Tít. 3º, § 3º - \emph{Servi autem ex eo appellati sunt}, etc.).

Os libertos tornam-se \emph{cidadãos brasileiros}; e as qualidades
constitutivas deste estado político só se perdem pelos fatos previstos
em nossa lei fundamental (Const. Pol. Imp. Logs. Citad.).

O liberto não pode ser revocado\footnote{Restituído, retroagido.} à
escravidão; porque não pode ser vendido; visto como será criminoso
vender o liberto (Cód. Crim. art. 179).

A lei não admitiu distinções; ninguém, pois, as poderá estabelecer,
ampliando ou restringindo o preceito legal. (Assentos de 16 de Novembro
de 1700, e 3º de 9 de Abril de 1772; leis de 29 de Novembro de 1753, 6
de Julho de 1755, pr., e 18 de Agosto de 1769, § 11).

II

É válida a manumissão ou alforria \emph{qualquer que seja o modo da
concessão}; e o termo regulador da validade, e único, é a
\emph{capacidade} do manumissor; esta capacidade é legal e não
presuntiva; pelo que, a nulidade, como qualquer que seja, da forma do
documento ou meio, não afeta intrinsecamente a concessão (Inst. Just.
cit. Liv. 1º, Tít. 3º, § 1º - \emph{Multis autem modis manumissio
procedit.}, etc..; §§ 2º e 3º, Tít. 6º, § 7º; Acórdãos Rel. Corte, 20 de
Outubro de 1872; 23 de Junho de 1873; Sup. Trib. Just., 20 de Dezembro
do mesmo ano; 29 de Outubro de 1864; Vid. \emph{Corr. Merc.} de 24 de
Novembro; Julgamento 1.486 - Bremeu - \emph{Univ. Jurid.} Tratad. 1º,
Tít. 7º, § 6º, Resoluções 24 e 25; Pothier, Pand{[}ectas{]}, L. 40, Tom.
3º, pág. 630 e 631).

A alforria, considerada como ato jurídico, carece de fundamentos ou de
características essenciais para ser regularmente considerada como doação
(L. 15, Dig. \emph{de manum.} 40, I - \emph{in extremum tempus
manumissoris vilxx,} etc.; Savigny, Dir. Rom. Tom. 4º, § 170).

A \emph{concessão manumissória} é uma \emph{fórmula jurídica} mediante a
qual o libertador devolve ao libertando o direito que tinha, ao valor do
seu trabalho, como seu escravo - \emph{sercure {[}servire{]}}. Dr.~C. A.
Soares. Comment. Às leis de 16 de Janeiro de 1756, 6 de Junho de 1755, e
Alvará 2º, de 16 de Janeiro de 1773).

\emph{Essencialmente considerada}, a manumissão é um ato psicológico, de
exclusivo domínio da lei moral; é o reconhecimento consciencioso do
estado natural do homem: não é, não pode ser um \emph{fato judicial}
propriamente dito (Inst. Just. cit. Liv. 1º, Tít. 3º... \emph{qua quis
dominio alieno contra naturam subjectur}).

Sob este ponto de vista a alforria é isenta de toda condição; e, como
\emph{ato jurídico}, todas as condições remíveis ou revogáveis (dr. C.
A. Soares - \emph{Apontamentos jurid. manumis.} Págs. 66 e 67 - Parecer
de 8 de Novembro de 1855).

A liberdade não pode ser objeto de propriedade; não pode ser considerada
móvel de \emph{doação} esta \emph{expressão jurídica} tem acepção
peculiar (\emph{libertas pecunia lui non potest, nec reparari} -
\emph{emi - potest}; - Ord. Liv. 4, Tít. 42; - Alv. 30 de Julho 1609).

III

Pelo antigo direito, hoje expressamente revogado pela Lei n.~2.040
{[}de{]} 28 de Setembro {[}de{]} 1871, art. 4º, § 9º, considerada a
alforria como \emph{doação inter vivos}, ou \emph{causa mortis} (Ord.
Liv. 4º, Tít. 63, §§ 7, 8 e 9) ao \emph{Patrono} era permitido
revogá-la; \emph{mas somente em certos e determinados casos},
especificados na Lei (Vid. Report. - Verb. - \emph{Si enxxx donatur}
etc., Tom. 2º, Págin. 391{[}394{]}, Not. - b -).

E a revogação \emph{era judicial}, obtida em juízo contraditório, por
fatos provados e julgados por sentença de juiz (dr. Loureiro -
\emph{Dir. Civ. Brazil.,} § 9º; - Cód. Filip. Pág. 866, Not. 1ª ao § 7º
da Ord. Liv. 4º, Tít. 63; - T. Freitas - \emph{Cons. Leis} Not. ao art.
419; - Cód. da Luisiana, art. 189; Lei 15, Cit., Dig. \emph{De manum.}
40, 1 - \emph{in extremum tempus manumissoris vitae}; Savigny, Dir.
Rom., Tom. 4º, § 170).

Fora dos casos peculiares da lei, e sem as formalidades judiciárias,
era, portanto, inadmissível a revogação.

Estas disposições foram sempre mantidas pelos Colendos Tribunais do
Reino, como atestam os eminentes praxistas Lima e Barbosa -, nos seus
comentários à Ord. Liv. 4º, Tít. 63, §§ 7, 8 e 9; assim como pelos do
Brasil, segundos os Arestos venerandos retrocitados.

A concessão de alforria, ainda quando se dê em testamento cerrado,
\emph{se dela houver prova regular, não é revogável ad nutum} (Vid.
Perdig. Malh. - Vol. 1º, § 146 - 2º -, Not. 797; Acc. Rel. Corte Cits.
29 {[}de{]} Outubro {[}de{]} 1872, 23 {[}de{]} Junho {[}de{]} 1873).

IV

Nestes autos a espécie é clara e o direito evidente.

O apelante foi alforriado em testamento regular, \emph{licitamente}, por
quem o podia fazer\emph{.}

A substância do testamento é jurídica e a forma constitui instrumento
público incontestável: \emph{é o que é; e mantém quanto foi feito}.

Antes, porém, da abertura do testamento, a libertadora, por escritura
pública, fez doação do liberto a um seu afilhado; e, destarte, por
arbítrio próprio, sem fundamento, nem razão legal, revogou a concessão
manumissória!...

Se este procedimento é lícito, como há quem o pretenda, é certo que o
libertador pode revogar arbitrariamente a manumissão; e, com ela, o
direito escrito!... (\emph{Scriptum jus est lex, plebiscita, senatus
consulta, principum placita*, magistratuum edicta, responsa prudentium};
Ord. Liv. 3º, Tít. 64, pr.; Lei 18 {[}de{]} Agosto {[}de{]} 1769; -
Borg. Carn. - \emph{Dir. Civ.} p.~3. Da \emph{introduc.} § 14, etc.),
se, porém, indubitável é, que o \emph{ex-senhor} não pode
arbitrariamente revogar a manumissão, porque a Lei expressamente o
proíbe, nula, e imprestável deve ser a doação declarada, para subsistir
inteira e exclusivamente a liberdade preconcedida no testamento, se bem
que verificada depois; não só como ato irrevogável, \emph{que é}, como
está demonstrado, como porque, na vertente hipótese, prevalece o
documento anterior, com exclusão do posterior, por contrário ser à
liberdade (\emph{In libertatibus levissima scriptura spectanda est} - L.
5º, Dig. \emph{de manum. testam.; -} Vid. Pothier, Pand.).

V

Está demonstrado, em face do direito, e, portanto, em termos
incontestáveis:

Que a manumissão foi concedida em testamento perfeito; e que, quando
mesmo perfeito não fosse, uma vez conhecido e judiciado o testamento,
desde que \emph{lícita fosse a concessão}, por partir de \emph{pessoa
capaz}, era válida;

Que válida a concessão, por ser o meio lícito e partir de pessoa capaz,
é, por direito, irrevogável;

Que, em essência, irrevogável é, porque o são as concessões
manumissórias \emph{inter vivos}, ou, \emph{causa mortis};

Que são irrevogáveis as concessões manumissórias \emph{inter vivos} ou
\emph{causa mortis}, \emph{porque o são a termo ou a título oneroso};

Que o são a termo ou a título oneroso porque os serviços prestados por o
manumitente, até o falecimento do manumissor, \emph{qualquer que seja o
prazo}, constituem tacitamente contrato bilateral, e importam resgate; e
o resgate uma indenização satisfatória;

Que, isto posto, a manumissão, assim concedida, é jurídica e legal, e
irrevogável.

Tal será a verdade da Lei, enquanto o fato, que faz objeto desta causa
não for devidamente contrariado.

\_\_\_

Senhor!

O venerando Acórdão embargado está em contradição manifesta com o
direito, com as práticas de julgar, e com outro Aresto \emph{deste mesmo
egrégio Tribunal}, invocado pelo embargante, nas sus razões de Apelação,
proferido em pleito semelhante!

\_\_\_

Um dos mais peregrinos talentos da Europa moderna, que, por largo tempo,
com a irradiação do seu gênio, iluminou os auditórios de Portugal,
lamentando, um dia, a confusão prejudicialíssima, causada pelos Arestos
contraditórios, que emaranham a jurisprudência e obscurecem o foro, em
vez de esclarecê-lo, por as normas salutares da invariabilidade, e
certeza dos julgamentos, disse:

``Esta espécie de Babel judiciária, que resulta da contradição dos
julgados, apadrinhada pela independência individual dos juízes, que
aliás constituem a coletividade necessária dos Tribunais, em nome dos
quais, e sob os auspícios da majestade da Lei, que é a forma da unidade
da justiça, devem ser meditadas, e são preferidas as sentenças, põe os
litigantes em triste condição dos ébrios recalcitrantes, dominados da
perigosa mania de cavalgar, que galgam arrojadamente a cela, por um
lado, para cair, com rapidez, pelo outro, de onde à socapa, retiram-lhe
o estribo!!!...''

\_\_\_

O embargante, senhor, cumpriu o seu dever: é um mísero escravo que, em
nome do direito, implora liberdade e pede que se lhe faça

Justiça.

O curador, Dr.~ANTONIO CARLOS.\footnote{Antonio Carlos Ribeiro de
  Andrada Machado e Silva (1830-1902) nasceu em Santos (SP) e pertence à
  segunda geração dos Andradas, sendo sobrinho de José Bonifácio, ``O
  Patriarca'', e filho de pai homônimo. Foi político, advogado,
  professor de Direito Comercial na Faculdade de Direito de São Paulo e
  sócio de Luiz Gama por aproximadamente uma década em um escritório de
  advocacia.}

\textbf{A INJÚRIA}

\textbf{*didascália*}

\emph{Em três textos relacionados a uma mesma causa, Luiz Gama, junto a
um ou mais advogados, vem a público defender os interesses de seus
clientes. No primeiro, encontramos uma carta aberta a um juiz -- que
anos mais tarde se tornaria notório desafeto de Gama e do movimento
abolicionista --, assinada por muitos advogados e professores de direito
de São Paulo. Embora cada signatário pudesse ter seu interesse
particular ao firmar a carta aberta, o} timing \emph{e a razão do
desagravo ligavam-se diretamente ao processo dos clientes de dois desses
advogados e autores: Luiz Gama e Laurindo Aberlado de Brito. O segundo e
o terceiro textos, por sua vez, deixam evidente uma das motivações do
primeiro, qual seja, a defesa de seus clientes para além do juízo e do
tribunal, compreendendo a imprensa como fórum de salvaguarda de
direitos, ideia, como se sabe, ampla e habilmente utilizada por Gama
desde há muito tempo. O núcleo da demanda envolve a caracterização do
tipo criminal da injúria. Num litígio que cuidava da disputa de posse de
terras entre vizinhos, uma das partes, um certo Marques Capão, resolveu
despejar ``imundícias'' -- lixos, fezes, entre outros dejetos de igual
valia -- no limite do terreno em que a outra parte, Vicori e Chicherio,
habitavam. Em resposta, os ofendidos atiraram os mesmos dejetos em uma
das propriedades de Marques Capão, acrescentando um igrediente a mais na
acalorada briga de vizinhos. Diziam eles que as imundícias lançadas
``estavam frescas, {[}prontas{]} para serem devoradas por o dito Capão e
sua família''. Ou seja, não só pagavam na mesma moeda como agregavam um
insulto na contenda. Afinal, poderia o insulto verbal ser qualificado
como injúria, na forma do Código Criminal? Havia na frase destacada
algum conteúdo que desabonasse a conduta de Marques Capão? Ou, noutros
termos, havia nas palavras de Vicori e Chicherio ``a imputação de crime,
vício ou defeito'' ou expunham o queixoso ``ao ódio ou desprezo
público''? Questões, como se vê, de natureza jurídico-criminal, sobre as
quais, como se verá, os autores se debruçaram para descaracterizar o
entendimento controverso de que, de fato, aquelas eras palavras
injuriosas.}

\begin{center}\rule{0.5\linewidth}{\linethickness}\end{center}

\textbf{4. AO EXMO SR. DR. BELLARMINO PEREGRINO DA GAMA E
MELLO}\footnote{In: \emph{A Província de S. Paulo} (SP), Seção Livre,
  23/10/1877, p. 2.}

\textbf{*didascália*}

\emph{Tão suscinta quanto sem razão aparente, a carta aberta é um
exemplo das relações de amizade e interesses entre advogados,
praticantes do foro, juízes e professores de direito de São Paulo.
Assinada por vinte e sete signatários, o desagravo ao juiz de direito
Bellarmino Peregrino da Gama e Mello era capitaneado -- ao que se infere
pela primeira assinatura -- pelo político, advogado e professor de
direito José Bonifácio, o Moço. Outros dois dos irmãos Andradas --
Martim e Antonio Carlos, este último sócio de Luiz Gama -- também
assinavam a carta. O nome de Gama surge mais adiante. Todavia, se a
ordem dos signatários não altera a razão do artigo, é de se notar que
``este voto de apreço'' expressa apoio a um juiz que estava sendo alvo
de demasiadas e injustas críticas. A homenagem ao caráter indiviual do
juiz, contudo, não seria condescendente com possíveis erros jurídicos.
Aliás, a examinar as entrelinhas do contexto, o desagravo parecia ter
como objetivo a revisão de um erro já cometido. Assim, prestigiar a
``ilustração e honradez do magistrado'' não parecia ser mera questão de
congratulação por algum feito recente, ao contrário, sugeria que o juiz
havia errado e, pelas virtudes que possuía, estaria obrigado a
corrigir-se. }

\begin{center}\rule{0.5\linewidth}{\linethickness}\end{center}

Os abaixo-assinados, advogados na cidade de S. Paulo, julgam de seu
dever dirigir uma manifestação de apreço ao juiz da 2ª Vara desta
capital, o exmo. sr. dr. Bellarmino Peregrino da Gama e Mello.\footnote{Bellarmino
  Peregrino da Gama e Mello (?-?) foi advogado, juiz de direito, chefe
  de polícia e desembargador dos tribunais da Relação de Ouro Preto e de
  São Paulo.}

Compreendem que V. Excia. se possa enganar na aplicação do direito; e
ninguém pode pretender a infalibilidade; rendem, porém, homenagem à
ilustração e honradez do magistrado, e sabem apreciar a retidão de suas
sentenças e a independência do seu caráter.

Cidadãos zelam um patrimônio que também é o seu; advogados respeitam a
justiça honesta, da qual são auxiliares no exercício de sua profissão.

Por estes motivos dirigem à V. Excia. este voto de apreço.

S. Paulo, 21 de Outubro de 1877.

J. Bonifácio.

Leoncio de Carvalho.

José Maria Corrêa de Sá e Benevides.

Francisco Justino G. de Andrade.

Martim Francisco R. de Andrada.

João Theodoro Xavier.

Antonio Carlos Ribeiro de Andrada Machado e Silva.

José Candido de Azevedo Marques.

Antonio A. de Bulhões Jardim.

Joaquim Ignacio Ramalho.

Joaquim José Vieira de Carvalho.

Pedro Vicente de Azevedo.

Frederico J. C. de A. Abranches.

Joaquim Augusto de Camargo.

João Alves de Siqueira Bueno.

Manoel Augusto de M. Brito.

José Fernandes Coelho.

Henrique A. Bernarbé Vincent.

João da Silva Carrão.

Luiz Gama.

Laurindo Abelardo de Brito.

Vicente Ferreira da Silva.

Paulo Egydio de Oliveira Carvalho.

Arthur de Carvalho.

José Rubino de Oliveira.

Antonio Januario Pinto Ferraz.

Antonio Dino da Costa Bueno.

\textbf{5. FORO DA CAPITAL}\footnote{In: \emph{A Província de S. Paulo}
  (SP), Seção Livre, 30/10/1877, p. 2. Republicado na capa da edição
  seguinte, \emph{A Província de S. Paulo} (SP), Seção Livre,
  31/10/1877, p.~1.}

\textbf{*didascália*}

\emph{Tão somente cinco dias após a carta pública ao juiz Bellarmino
Peregrino da Gama e Mello, dois de seus autores, Luiz Gama e Laurindo
Abelardo de Brito, escreveram uma defesa pública de três clientes, que
estavam metidos na ruidosa causa que ocupava tanto o juízo cível da
capital quanto o Tribunal da Relação de São Paulo. Ao que parece, muito
embora não haja conclusão categórica, o magistrado que julgaria uma
dessas causas era o próprio Gama e Mello, juiz da 2ª vara cível de São
Paulo. Na carta precedente, por exemplo, os autores e signatários
afirmavam que o juiz poderia se ``enganar na aplicação do direito''; já
na presente, mencionam que a consciência de um juiz -- como a de um
advogado, aliás -- estava sujeita ``às condições do erro involuntário''.
Um texto, portanto, se liga ao outro. Literatura normativo-pragmática
por excelência, estilo ao qual Gama dedicava-se desde há muito, este
artigo funciona como uma preliminar de mérito, isto é, introduz questões
gerais concernentes ao processo de que trata; sustenta de antemão a
inocência de seus clientes; provoca a parte contrária; e, por fim,
anuncia que discutirá a fundo -- o que de fato se deu -- detalhes do
caso concreto. O artigo é bastante sóbrio, refletindo, por um lado, a
recente derrota que os advogados Gama e Brito haviam sofrido no Tribunal
da Relação e, por outro lado, preparando o caminho para melhor sorte em
julgamento futuro.}

\begin{center}\rule{0.5\linewidth}{\linethickness}\end{center}

\emph{Por meras presunções os néscios julgam;}

\emph{Os sábios, por verdades que divulgam}

Se fossemos juízes e alguém, por amizade ou civismo, escrevesse, sem
incumbência nossa, alguma defesa dos nossos atos, qualificados com
injúria do nosso caráter, julgando ofendida a nossa dignidade,
repeliríamos a oficiosa defesa, como calculado presente de
gregos\footnote{No sentido de ardil, armadilha.}; e assim procederíamos
porque a defesa do nosso brio e da nossa honra é da nossa exclusiva
atribuição.

Escrevemos estas linhas como defesa própria; para ressalva dos nossos
nomes e resguardo da nossa profissão; e como protesto solene a um
escrito do sr. Bernardo Marques Capão, publicado na \emph{Província} de
15 do corrente, e agora reproduzido, com estrépito\footnote{Estardalhaço.},
nos jornais da Corte, com o fim sinistro de prejudicar a clientes
nossos, que tentaram recursos legais que ainda pendem de decisão...

Somos advogados dos srs. Julio Vicori, Carlos Vicori e Carlos Chicherio,
contendores do sr. Capão, que, por a \emph{vitória} obtida, com a
condenação daqueles, veio à imprensa, ardente do mais expansivo
entusiasmo, entoar loas\footnote{Elogios.} ao egrégio Tribunal da
Relação!...

Acreditamos, e tal é a nossa convicção, que o egrégio Tribunal menos bem
julgou a questão de injúrias; assim como há julgado mal, e até com
violação manifesta do direito, em outros pleitos; mas não ousamos
qualificar de prevaricadores\footnote{Corruptos, aqueles que
  faltam~ao~cumprimento~do~dever~por~interesse~ou~má-fé.} os juízes;
porque na consciência de cada um deles, se bem que, como a nossa,
sujeitas às condições do erro involuntário, irradia-se a majestade da
lei, avulta o emblema da justiça, e o símbolo sagrado da autoridade
nacional.

O sr. Capão, se inspirado em bons sentimentos acatasse, com a devida
prudência, os conselhos dignos do seu douto advogado, não daria à
estampa o tristíssimo escrito a que aludimos...

Louvamos, entretanto, em parte, o seu precipitado procedimento; porque
ele justificará a publicação, que faremos em tempo, das principais peças
do processo, e da incontestável improcedência do gabado\footnote{Enaltecido.}
acórdão\footnote{Decisão de tribunal que serve de paradigma para
  solucionar casos semelhantes.} da Relação.

Ao sr. Bernardo Capão afirmamos que, pela causa dos nossos constituintes
tomamos o mais vivo interesse; não poupamos o menor esforço em seu
benefício; mas, nós o garantimos, \emph{ninguém} praticou um único ato
ofensivo da lei, ou do decoro dos juízes: eles e nós somos livres no
cumprimento dos nossos deveres, e sinceros no mútuo respeito que nos
tributamos.

Em mãos do exmo. sr. dr. juiz de direito, para julgamento, está uma
causa cível, em que o sr. Capão contende\footnote{Disputa.} com os
nossos clientes Vicori, relativamente à propriedade de um terreno, que é
o motivo da desmoronada balbúrdia criminal.

Dar-se-á que a publicação do sr. Capão seja um ardil para
inquinar\footnote{Macular, manchar.} de viciosa a futura sentença?...

Lamenta o sr. Capão a sua pobreza, que, bem como a riqueza, não é, de
per si, título de honra para ninguém; e acreditamos que o faz
industriosamente\footnote{Por sentido figurado, astuciosamente.}; visto
como estas pensadas lamentações estão em contraste notável com a sua
soprada\footnote{No sentido de insuflada, propagada.} basófia\footnote{Presunção,
  vaidade exacerbada.} de há pouco, quando em lugares públicos dizia:
``que para meter na cadeia os Vicori gastaria \emph{os poucos contos de
réis que possui, e venderia até a camisa!}''...

Pomos termo a este artigo, escrito em defesa dos nossos clientes, e
nossa, sem ofensa ao sr. Bernardo Marques Capão, que, com razão, arde em
festas, pelo haver o colendo Tribunal da Relação felicitado com a sorte
de um inesperado Acórdão!

A questão é de direito.

S. Paulo, 26 de Outubro de 1877.

DR. LAURINDO A. DE BRITO.\footnote{Laurindo Abelardo de Brito
  (1828-1885), nascido em Montevideu, Uruguai, foi advogado, promotor
  público, deputado pelo Paraná, onde foi presidente da Assembleia
  Provincial (1862-1863), e também por São Paulo, província da qual foi
  presidente (1879-1881).}

L. GAMA.

\textbf{6. TRIBUNAL DA RELAÇÃO -- O fato de alguém deitar imundícias à
porta ou à casa de outrem constitui crime de injúria em face do artigo
236 do Código Criminal?}\footnote{In: \emph{A Província de S. Paulo}
  (SP), Seção Judiciária, 31/01/1878, p.~1.}

\textbf{*didascália*}

\emph{Literatura normativo-pragmática. Dividido em quatro tópicos, o
artigo conceitua qual o melhor entendimento doutrinário sobre o delito
de injúria. Distinguindo-o da qualificação genérica de ofensa e/ou
provocação, Gama e Brito dão uma definição autoral sobre a injúria --
``o ataque difamatório dirigido contra alguém, com ofensa ou prejuízo da
sua reputação'' --, sem perder de vista a defesa de seus clientes. Para
os advogados, não estava provado nos autos que Vicori e Checherio haviam
cometido tal delito, nem segundo o Código Criminal brasileiro, nem sob a
perspectiva do Código das Duas Sicílias, provavelmente invocado em razão
da procedência dos indivíduos incriminados pelo Tribunal da Relação de
São Paulo. Gama e Brito descaracterizam a ideia de que o fato tenha sido
injurioso, mais até, de que o fato teria sido criminoso. O modo pelo
qual ``deitaram as imundicías à porta'' de uma propriedade de Marques
Capão, para Gama e Brito, escapava à tipificação criminal que se
emprestava. Justamente por isso, diziam, ``se escapa à qualificação
jurídica, excede a condição imprescindível do direito escrito; e, assim
sendo, não é crime de injúria, porque, para que o seja, carece de
existência legal, atenta a limitação posta nos artigos 1º e 229 do
Código Criminal''. O raciocínio técnico jurídico é lógico e apelava, no
limite, para o princípio da legalidade, qual seja, que em não se havendo
lei anterior que defina o crime, não há, fatalmente, a configuração
legal daquele crime. }

\begin{center}\rule{0.5\linewidth}{\linethickness}\end{center}

I

\emph{Injúria}, na peculiar acepção do sistema legal moderno, é
exclusivamente toda e qualquer \emph{ofensa}, cometida por palavras,
escritos, impressos, desenhos, gravuras, emblemas e outros meios
semelhantes, próprios para produzir uma manifestação imediata do
pensamento, na intenção de \emph{ofender} a honra, a consideração ou o
melindre de uma determinada pessoa ou corporação (Código das Duas
Sicilias, artigo 365\footnote{Gama e Brito não reproduziram textualmente
  o artigo citado mas, de modo criativo, sem dúvida, adaptaram-no e
  preservaram o sentido normativo na interpretação que fizeram. Em
  tradução livre, o art. 365 da Parte Penal do Código das Duas Sicílias,
  de 1819, dispunha: ``Injúria é qualquer ofensa pública ou privada
  expressa em palavras, gestos, inscrições ou de qualquer outra forma,
  desde que tenha o objetivo de causar a perda ou diminuição da estima
  da pessoa contra a qual é dirigida''.}).

A injúria, pois, como todos os delitos ativos, tem um caráter genérico;
é uma \emph{ofensa} que constitui ação voluntária, infringente do
preceito da Lei criminal (Código Criminal, art. 2º, § 1º\footnote{Art.
  2º. "Julgar-se-á crime ou delito:

  § 1º. Toda a ação ou omissão voluntária contrária às Leis penais".});
e tem igualmente uma face especial e distintiva, não só quanto ao objeto
que determina a sua existência, como ao modo ou meios de sua
perpetração, e à intensidade refletiva do fato e sua aplicação
(Cód{[}igo{]} cit{[}ado{]}, art. 236).

Este preceito-legal, bem como os demais da mesma natureza, em matéria
criminal, é complexo, prescrito, claro e restrito (Zuppetta,
Metaphys{[}ica{]} da Scienc{[}ia{]} das leis criminais, Liv{[}ro{]} 1º,
Cap. 1 a 5\footnote{Luigi Zuppetta (1810-1889) foi advogado, político e
  professor de direito penal da Universidade de Nápoles, Itália.
  Publicou o \emph{Corso completo di Diritto Penale} \emph{comparato},
  sendo a primeira parte intitulada \emph{Metafisica della scienza delle
  leggi penali} (1868).}).

II

Constituem delito de injúria:

A \emph{imputação} de um fato criminoso não compreendido no artigo 229
do Código Criminal; e assim,

A \emph{imputação} de vícios ou defeitos que possam expor ao ódio ou
desprezo público;

A \emph{imputação} vaga de crimes ou vícios, sem fatos especificados;

\emph{Tudo quanto pode prejudicar a reputação de alguém}.

Os \emph{discursos}, \emph{os gestos} ou os sinais, reputados
insultantes \emph{na opinião pública} (Cód{[}igo{]} Crim{[}inal{]}, art.
236).

É injúria, portanto, em face da expressa disposição do nosso direito:

A imputação de fato criminoso, de ação particular ou privada;

A imputação de crimes ou vícios de ação privada ou pública, não havendo
especificação de fatos, principalmente na segunda hipótese;

A imputação de vícios ou defeitos que possam expor o increpado\footnote{Acusado,
  tachado.} ao ódio ou desprezo público, sejam os fatos verdadeiros ou
não; e de tudo quanto possa prejudicar a sua reputação (Cód{[}igo{]}
Crim{[}inal{]}, arts. 229 e 236).

Temos, portanto, que a injúria \emph{é a difamação, originada em fatos
de ordem privada, segundo os preceitos da lei penal}; delito este que
pode ser perpetrado por palavras, ou por sinais, ou gestos,
\emph{reputados insultantes na pública opinião}.

Disto conclui-se necessária e evidentemente, que injúria é o ataque
difamatório dirigido contra alguém, com ofensa ou prejuízo da sua
reputação; pelo que não só se poderá considerar \emph{injúria} toda a
ofensa ou provocação, que diretamente não vise o ataque difamatório à
reputação, como, principalmente, tendo o legislador previsto os fatos
materiais, e qualificado excepcionalmente o delito, indicado, por
exemplos, os meios de sua perpetração, tornou certo, e é incontestável,
que não haverá injúria fora dos casos indicados, e muito menos quando se
não der o concurso dos meios previstos, salvas as exceções consignadas
na lei.

III

Posto o fato em confronto com a lei, e comparado, ponto por ponto, com
os seus preceitos, lógica e única é a conclusão.

O que se indaga não é da espécie de ofensa ou desacato que resulta do
fato de um indivíduo lançar imundícias à porta ou na casa de outrem;
quer-se saber se tal ocorrência constitui o delito de injúria previsto
no artigo 236 do Código.

Do detido exame dos preceitos positivados na Lei, evidencia-se que o
fato, nem mesmo em sentido translato\footnote{Figurado, metafórico.},
quando considerado como \emph{sinal}, pois que não é desenho, nem
emblema, nem gravura, nem gesto, nem discurso, não contém, de modo
algum, \emph{imputação} de ato criminoso ou alusão manifesta de vícios
ou de defeitos que possam expor alguém ao ódio ou desprezo público, ou
de qualquer outra ação que possa prejudicar a reputação de alguém.

Ora, se o fato que consideramos, em face das prescrições da Lei, não
pode ser considerado meio de perpetrar injúria, é incontestável que, por
ele, se não pode determinar a existência de tal delito; e se se não pode
determinar, por esse fato, a existência do delito, por não constituir
móvel\footnote{Motivo, razão.} ou meio para a sua perpetração, é que
certamente escapa à qualificação jurídica; se escapa à qualificação
jurídica, excede a condição imprescindível do direito escrito; e, assim
sendo, não é crime de injúria, porque, para que o seja, carece de
existência legal, atenta a limitação posta nos artigos 1º e 229 do
Código Criminal\footnote{Ao citar o primeiro artigo do Códido Criminal,
  Gama novamente fundamenta uma opinião no princípio da reserva legal,
  onde só pode haver crime se lei anterior assim o definir. Quanto ao
  art. 229, leiamos os termos: ``Julgar-se-á crime de calúnia o atribuir
  falsamente a algum {[}alguém{]} um fato que a lei tenha qualificado
  {[}como{]} criminoso, e em que tenha lugar ação popular ou
  procedimento oficial de Justiça''.}.

IV

Na hipótese que consultamos, deu-se que Bernardo Marques Capão, porque
trouxesse dúvidas\footnote{Reparem como Gama e Britto hábil e
  taticamente eufemizam uma briga de vizinhos para um ocasional
  desentendimento -- ``dúvidas'' -- entre eles.} com os seus vizinhos
Julio Vicari e Carlos Vicari, originadas de contestações sobre a
servidão de terrenos, que a estes pertencem, mandara como calculado
desforço\footnote{Vingança, retaliação, represália.} lançar imundícias
em um valo divisório, com o intuito de o inutilizar, em prejuízo dos
seus vizinhos, e de lhes causar incômodos; e que os contendores Vicari,
por seu turno, mandassem como repulsa de momento extrair do valo essas
imundícias, e, com outras semelhante, lançá-las em uma olaria de Marques
Capão, dizendo ``que estavam frescas, para serem devoradas por o dito
Capão e sua família'' {[}??...{]}

Se o fato em si, como ficou demonstrado, não constitui crime de injúria,
menos ainda o poderão constituir as expressões todas relativas de que
foi acompanhado, porque tais expressões não importam insulto. E não
importam insulto porque não envolvem imputação de crime, vício ou
defeito, nem expõem aqueles a quem foram dirigidas, ao ódio ou desprezo
público.

É certo, entretanto, que a despeito da disposição da Lei, da natureza
dos fatos, segundo a prova judicial do sumário\footnote{Provas aduzidas
  no sumário de culpa, que apresentam os indícios da acusação.}, e das
alegações dos acusados, o colendo Tribunal da Relação do
distrito\footnote{Tribunal de segunda instância.}, por uma formal
inversão da ocorrência, revogou uma sentença absolutória da primeira
instância\footnote{Sentença penal que põe fim ao processo com resolução
  do mérito, absolvendo ou condenando o réu.}, se bem que fundada em
diverso fundamento, e condenou os acusados por votação unânime, como
autores de crime de injúria! E basearam o seu venerando Acórdão, que é
de 28 de Setembro do ano precedente, em que o ato praticado pelos
acusados é reputado insultante na opinião pública, mas não demonstraram,
porque não o podiam fazer, as relações do fato com os ditames da Lei; e
muito menos a prova, aliás indispensável, de que o aludido fato é
reputado insultante na \emph{opinião} pública, principalmente quando dos
autos consta o contrário.

Agitamos esta questão pela imprensa por entendermos que o julgamento do
colendo Tribunal, sobre ser injurídico, foi injusto; e para provocarmos
novo e acurado estudo da matéria, que nos parece digna de maior
ponderação.

S. Paulo, 18 de janeiro de 1878.

DR. ABELARDO DE BRITO.\footnote{Laurindo Abelardo de Brito (1828-1885),
  nascido em Montevideu, Uruguai, foi advogado, promotor público,
  deputado pelo Paraná, onde foi presidente da Assembleia Provincial
  (1862-1863), e também por São Paulo, província da qual foi presidente
  (1879-1881).}

L. GAMA.

\textbf{O ABUSO DA LIBERDADE DE OPINIÃO - E DE IMPRENSA}

\textbf{*didascália*}

\emph{A defesa que Gama faz do seu cliente Justiniano Silva é digna das
páginas da história do direito e da política. A disputa entre Silva e
Ribeiro de Lima -- que prestou queixa contra Silva, movido, segundo
Gama, por ``odiosa demanda para a cobrança de quantia superior à que lhe
é devida'' -- foi o pano de fundo para a lição de direito que Gama deu
às autoridades judiciárias e ao público leitor de São Paulo. É certo que
o litígio entre ambos, Silva e Lima, era mais um capítulo de uma
história que já ia comprida. No entanto, o valor jurídico da petição de
Gama -- revertendo decisão anterior de juiz que, depois, fora declarado
incompetente para o feito -- reside em uma interpretação singular sobre
um tema em que pairavam ``dúvidas no foro, e dúvidas gravíssimas, porque
interessam elas à ordem e às fórmulas substanciais do processo criminal
e, portanto, às garantias e segurança do cidadão, máxime tratando-se do
sagrado direito e exercício da liberdade constitucional de comunicar os
pensamentos''. E continuava o raciocínio até o centro normativo da
demanda: ``Dúvidas que atingem ao ponto importantíssimo de saber-se se a
Lei de 20 de Setembro de 1830 foi revogada ou apenas derrogada por a
legislação posterior''. A questão envolvia interesses graúdos, que
simplesmente poderiam restringir a liberdade de imprensa e, por
extensão, pensando no contexto de Gama, o direito político em constituir
uma posição abolicionista e republicana da imprensa. ``Consultado
levianamente o Poder Executivo'', asseverava Gama -- aliás, insistia --,
"como sempre acontece em casos idênticos, resolveu este {[}o Poder
Executivo{]} indebitamente, exorbitando das suas atribuições legais,}
por interpretação autêntica\emph{, em Aviso de 15 de Janeiro de 1851,
que a mencionada Lei de 20 de Setembro de 1830 está inteiramente
revogada, pelo que os crimes cometidos, por via da imprensa, devem ser
processados e punidos por as leis posteriores!..." O leitor verá o caso
em detalhes. Por ora, basta notar como um simples processo é tomado a
sério não só para resolver uma demanda pontual, mas para se criar um
precedente relevante, sobretudo aos homens de imprensa, no foro de São
Paulo. A partir da leitura normativa sobre responsabilidade criminal no
delito de ``abuso da liberdade de comunicar os pensamentos'', Gama
desenvolve um argumento que discerne o campo de ação de cada agente de
imprensa -- autor, editor, impressor e distribuidor -- como requisito
base para poder-se apurar qual a eventual conduta criminosa de cada um.
Em síntese, Gama exigia um processo crime minucioso e, pode-se até
dizer, garantista, ainda que tornando o trabalho da acusação muito mais
complexo. Quem disse, aliás, que a vida da defesa é facilitar o jogo
acusatório? }

\begin{center}\rule{0.5\linewidth}{\linethickness}\end{center}

\textbf{7. FORO DA CAPITAL}\footnote{In: \emph{Correio Paulistano} (SP),
  Seção Particular, 07/07/1878, pp.~1-2. Justiniano Silva, cliente de
  Gama, publicou esse pequeno texto à guisa de introdução. Vejamos: ``O
  sr. alferes João Antonio Ribeiro de Lima, que jamais pretendeu
  perseguir-me, como apregoa, por costume, deu contra mim segunda
  queixa, por crime de injúrias. Para que o respeitável público avalie
  da sinceridade do sr. Ribeiro de Lima, e da elevação da sua justiça,
  nos processos que contra mim promove, publico, em seguida, as razões
  produzidas nos autos, em minha defesa, pelo meu advogado, e a sentença
  proferida pelo íntegro sr. dr. Sebastião José Pereira, juiz de direito
  do 1º Distrito Criminal da Comarca.'' Com esta publicação pretendo
  mostrar, como sempre, que em todas questões judiciais em que tenho-me
  achado, por graça do sr. Ribeiro Lima, o direito, mau grado alguns
  julgamentos irregulares, tem sempre estado por mim. S. Paulo, 4 de
  julho de 1878. C. JUSTINIANO SILVA."}

\textbf{*didascália*}

\emph{Literatura normativo-pragmática. Gama elabora uma monumental peça
de defesa de seu cliente, Justiniano Silva, que enfrentava a acusação de
um crime tipificado à época como ``abuso da liberdade de comunicar os
pensamentos''. Foi Silva, aliás, quem decidiu publicar a petição de seu
advogado e a sentença que acolhia os argumentos de Gama. A querela se
dava porque o alferes Ribeiro de Lima se sentiu injuriado por palavras
escritas por Justiniano Silva na imprensa de São Paulo. Silva teria dito
que Ribeiro de Lima extorquia clientes e, com violência, os obrigava a
pagar débitos vencidos. Tratava-se, em síntese, de uma espécie de briga
de feira, ou de armazém, se preferirem. Gama nem de longe trata de
discutir os dizeres -- ``aliás textuais'' -- de seu cliente. Sua
estratégia foi mais longe e, para tal, mobilizou o conhecimento
normativo que sabidamente possuía, interpretando a Constituição, as
Ordenações, a lei de imprensa da época, obras doutrinárias de
referência, assentos da Casa da Suplicação, acórdãos de tribunais de
Relação, decretos e avisos executivos, para então aportar na tipificação
do delito conforme regramento do Código Criminal e do Código de Processo
Criminal. Nesse sentido, Gama descaracteriza a jurisdição que
inicialmente processara o feito, justificando, em sequência, qual seria
a jurisdição competente para a matéria; disseca e discrimina quais
responsabilidades caberiam aos potenciais agentes criminosos,
descartando, contudo, a existência do crime de injúria. Isso,
ressalve-se, se houvesse qualquer configuração criminal no fato narrado
pelo queixoso Ribeiro de Lima. Realmente, a defesa de Gama --
constituída por diferentes tópicos, entre eles, o ``fundamento da
causa'' e a ``demonstração de contrariedade -- é uma aula de direito.
Gama tanto cuida da tradição jurídica, quanto observa as minúcias do
rito do proceso crime. Disserta sobre condutas criminosas e suas
respectivas responsabilidades no delito de''abuso da liberdade de
comunicar os pensamentos``, coteja provas documentais e testemunhos,
especula hipóteses, concluindo pela inocência de seu cliente. Não
faltaria, todavia, aquela verve crítica que enquadrava juízes ignorantes
ou negligentes no ofício que exerciam.''Improcedente é o presente
processo, deforme, monstruoso, nulo e imprestável, perante o direito e a
lei", resumia -- e fulminava -- Luiz Gama. }

\begin{center}\rule{0.5\linewidth}{\linethickness}\end{center}

MERITÍSSIMO JUIZ

À imparcialidade do Juízo, que é a razão do direito e o critério da lei;

À ilustração do emérito julgador, que é o símbolo da sabedoria e o verbo
interjectivo\footnote{Que expressa sentimento, apelo, ordem.} da
justiça;

Por a manutenção da verdade dos autos, que é o dever do magistrado, e a
suma segurança dos direitos do cidadão, expomos as seguintes
considerações.

I

\emph{Fundamento da causa}

Pretende o querelante\footnote{Que apresentou queixa, ofendido.}

- Alferes J. A. Ribeiro de Lima que, com infração da lei criminal, lhe
fizesse injúria o querelado,

- Cidadão Candido Justiniano Silva;

E indica, em sua petição de queixa, como objeto ou fundamento material
do delito, as expressões seguintes, atribuídas ao querelado, com
referência dolosa ao caráter do autor:

``Exigências exageradas do mesmo senhor (o queixoso) que, de mim (o
acusado), queria cobrar desarrazoadamente quantia maior que a devida;
vender gêneros de má qualidade aos fregueses; trocar os gêneros vendidos
por outros inferiores; invadir violentamente a casa dos compradores para
ajustar as contas.''

Estas expressões, aliás textuais, são extraídas de trechos de um artigo
inserto na \emph{Gazeta de Notícias} de 20 de dezembro do ano
precedente, com esta inscrição -- S. Paulo, Foro da Capital -- e aqui
reproduzido em o jornal \emph{Província de S. Paulo}, nº 859, de 27 de
dezembro do mesmo ano.

Pretende, portanto, o queixoso, em vista da ocorrência que refere, a
condenação do querelado, como incurso no art. 236, §§ 1º e 3º, combinado
com o {[}art.{]} 237, § 3º, do Código Criminal.\footnote{Art. 236.
  "Julgar-se-á crime de injúria:

  § 1º. Na imputação de um fato criminoso não compreendido no art. 229.

  § 3º. Na imputação vaga de crimes, ou vícios sem fatos especificados.

  Art. 237. O crime de injúria cometido por algum dos meios mencionados
  no artigo 230.

  § 3º. Contra pessoas particulares ou empregados públicos, sem ser em
  razão do seu ofício".}

O querelado, porém, contestando a perpetração do aludido delito e
negando absolutamente a sua responsabilidade, ainda mesmo na hipótese de
criminação\footnote{Imputação, acusação.} do fato, afirma:

1º: Que não cabe nas atribuições dos juízes singulares, com exclusão do
foro comum, o conhecimento e o julgador {[}julgamento{]} dos delitos por
abuso de liberdade de comunicar os pensamentos por a imprensa (Lei de 20
de Setembro de 1830\footnote{Gama se reporta de modo geral à lei que
  tratava dos delitos cometidos por meio da imprensa e, em particular,
  ao processo de julgamento através de júri (cf.~art. 14 e seguintes).});

2º: Que assim o têm entendido e decidido juízes distintos no Foro da
Corte;

3º: Que, à vista do exposto, a decisão deste processo, e dos
semelhantes, observando-se a regra geral, compete ao Tribunal do Júri,
pelo que, a forma adotada contra o acusado, sobre ser irregular, é
ilegal;

4º: Que, além do exposto, em face da lei e dos princípios de
jurisprudência, não é este Juízo competente para conhecer da presente
causa, por fatos que se hão de provar;

5º: Que o acusado não é nem pode ser o responsável legal do impresso
criminado;

6º: Que, em tal impresso, em tese, como deve necessariamente ser
considerado, há uma agregação de atos lícitos, constitutivos de justa
defesa, que, à vista do direito, não determinam existência de crime;

7º: Que quando mesmo tais fatos pudessem constituir delito, não seria,
por certo, o de injúrias, de todo ponto inadmissível, na vertente
hipótese, porque os crimes decorrem dos fatos previstos por a lei, e não
podem ser imputados arbitrariamente.

II

\emph{Demonstração da contrariedade}

Movendo-se dúvidas no foro, e dúvidas gravíssimas, porque interessam
elas à ordem e às fórmulas substanciais do processo criminal e,
portanto, às garantias e segurança do cidadão, máxime\footnote{Principalmente,
  especialmente.} tratando-se do sagrado direito e exercício da
liberdade constitucional de comunicar os pensamentos. Dúvidas que
atingem ao ponto importantíssimo de saber-se se a Lei de 20 de Setembro
de 1830 foi revogada ou apenas derrogada por a legislação posterior.
Consultado levianamente o Poder Executivo, como sempre acontece em casos
idênticos, resolveu este indebitamente, exorbitando das suas atribuições
legais, \emph{por interpretação autêntica}, em Aviso de 15 de Janeiro de
1851\footnote{O aviso executivo, ``por interpretação autêntica'', na
  fina ironia do autor, opinava pela revogação da lei de abuso de
  liberdade de imprensa.}, que a mencionada Lei de 20 de Setembro de
1830 está inteiramente revogada, pelo que os crimes cometidos, por via
da imprensa, devem ser processados e punidos por as leis posteriores!...

Acontecendo, porém, que nada resolvesse esta extravagante resolução do
Poder Executivo, violadora da lei comum e infringente de preceitos
constitucionais, progrediram as dúvidas, dando, como natural resultado,
disparatados julgamentos e uma jurisprudência caótica, se bem que
rigorosamente lógica.

O Supremo Tribunal de Justiça, por Acórdão de 22 de Agosto de 1848, e o
egrégio Tribunal da Relação de Pernambuco, por Acórdão de 9 de Março de
1849, resolveram ``que o delito de abuso de liberdade de imprensa
somente pode ser julgado no Tribunal do Júri, em virtude de expressa e
não revogada disposição do art. 68 da citada lei de 1830''. E desta
jurídica e libérrima\footnote{Superlativo de livre, algo como muitíssimo
  livre, muitíssimo liberal.} opinião também é o eminente jurisconsulto
sr. Marquês de S. Vicente (vide Direito Público Brazileiro, Tít{[}ulo{]}
8, cap. 2, Sec{[}ção{]} 3, § 3º, nº 545, \emph{in fine}).

E, ou porque nos páramos\footnote{Planaltos.} do direito errem os
levitas\footnote{Sacerdotes.}, delirantes e tomados do santo espírito
das leis, ou seja balda\footnote{Pode ser lida como falta de juízo ou
  tipo de mania nociva.} antiga dos mórbidos Themistas\footnote{Possível
  referência a juristas, jurisconsultos e juízes, alçados, ainda que
  ironicamente, à posição de sacerdotes da justiça, cultores da lei.}
ouvirem, de preferência, conselhos, para evitar os labores de enfadonho
estudo, novamente consultaram o Poder Executivo sobre o melhor caminho a
seguir nestas lôbregas\footnote{Sombrias, nebulosas, assustadoras.}
agruras do direito escrito...

E o Poder Executivo, depois de prudentemente ouvir a respectiva Seção do
Conselho do Estado\footnote{Refere-se a uma das seções do Conselho de
  Estado, órgão consultivo ao imperador formado por uma seleção de
  ministros de Estado e outras figuras-chave do direito e da política
  nacional. Para o Segundo Reinado, suas atribuições estão marcadas na
  Lei nº 234 de 23 de Novembro de 1841.} \emph{sobre a mesma questão
decidida por Aviso de 15 de Janeiro de 1851}, julgou-se incompetente
para desatar o gordiano nó\footnote{Remete à passagem lendária em que
  Alexandre, o Grande (356-323 a.C), cortou o nó da corda que atava a
  carroça do antigo rei Górdio a uma das colunas do templo de Zeus. A
  metáfora, nesse caso, indica um problema de difícil solução.}: e,
assim, devolveu o caso para a jurisprudência dos tribunais!... (Vide
Aviso nº 83 de 6 de Fevereiro de 1866).

E o colendo Tribunal da Relação da Corte, tomando ao sério, ao que
parece, \emph{a regra de delegação de poderes}, por Acórdão de 15 de
Setembro de 1865, declarando em vigor a legislação posterior, reconhece,
em termos explícitos, que a especial disposição do art. 68 da Lei de 20
de Setembro de 1830 subsiste, porque não foi revogada clara, positiva e
expressamente por lei alguma (vide Av{[}iso{]} nº 262 de Agosto de
1857)!...

O art. 68 da Lei de 20 de Setembro de 1830\footnote{Art. 68. ``É nula
  toda a sentença proferida por outro tribunal ou juízes que não forem
  os do júri competente, e nunca produzirá efeito algum, nem mesmo para
  servir de fundamento à nova ação no juízo a que competiria''.} encerra
conceitos preciosos, da mais alta magnitude política, regulamenta
peculiarmente um preceito constitucional, estabelece, com sólidas
cautelas, de modo jurídico, a garantia de um direito natural sabiamente
aceita e imposta pela Constituição. Estatui sobre a forma do processo e
firma, em termos claros e inalteráveis, o preceito altamente liberal de
que, em tais delitos, só o Tribunal do Júri é competente para julgar o
cidadão. É uma disposição expressa que somente por outra igual pode ser
derrogada (Av{[}iso{]} de 21 de Junho de 1877).

Se, como reconheceu implicitamente o governo, e, com evidência,
proclamam os tribunais, este artigo da lei, tão peculiar em seu sistema,
não foi clara, positiva e expressamente revogado, é certo, é
incontestável que ele está em pleno vigor.

- Porque os preceitos legais, garantidores do exercício de direitos, e
maiormente de direitos constitucionais, só por absurdo se podem
considerar revogados por meras induções ou por fórmulas indiretas de
inqualificável hermenêutica.

- Porque somente quando cessa a razão da lei é que cessa a sua
disposição (Ord{[}enações{]}, Liv{[}ro{]} 2, Tít{[}ulo{]} 29, §
últ{[}imo{]}; Liv{[}ro{]} 4, Tít{[}ulo{]} 103, §§ 2º e 3º; Alv{[}ará{]}
de 17 de Outubro de 1768).

- Porque na hipótese vertente, se o preceito não foi clara, positiva e
expressamente revogado, subsiste, e deve ser rigorosamente guardado; ou
se está em contradição com disposições análogas deve ser autenticamente
interpretado (Ass{[}ento{]} de 16 de Novembro de 1700; Ord{[}enações{]},
Liv{[}ro{]} 4, Tít{[}ulo{]} 45).

- Porque se o preceito depende de interpretação autêntica, não são
competentes para dá-la nem os ministros, nem os magistrados.

- Porque as leis só podem ser feitas, interpretadas, suspensas e
revogadas pelo Poder Legislativo (Constituição, art. 15, §
8º).\footnote{Art. 15. "É da atribuição da Assembleia Geral:

  § 8º: Fazer leis, interpretá-las, suspendê-las e revogá-las.}

- Porque o Poder Legislativo é exclusivamente delegado à Assembleia
Geral, com sanção do imperador (Constituição, art. 13).\footnote{Art.
  13. ``O Poder Legislativo é delegado à Assembleia Geral com a Sanção
  do Imperador''.}

- Porque a lei, qualquer que ela seja, só deixa de vigorar quando é, por
outra, expressamente revogada (Lei de 12 de Maio de 1840, art.
8º\footnote{Art. 8º. ``As leis provinciais, que forem opostas à
  interpretação dada nos artigos precedentes, não se entendem revogadas
  pela promulgação desta lei sem que expressamente o sejam por atos do
  Poder Legislativo Geral''.}).

- Porque, isto posto e bem ponderado, resulta, e é certo, que não foi
competentemente revogado o art. 68 da Lei de 20 de Setembro de 1830;
subsiste a sua disposição e, portanto, nulo é completamente o presente
processo (Constituição, art. 179, § 11; Decreto nº 4.824 de 22 de
Novembro de 1871, art. 50; Ord{[}enações{]}, Liv{[}ro{]} 1º,
Tít{[}ulo{]} 58, § 17, e Tít{[}ulo{]} 66, § 29\footnote{Respectivamente,
  art. 179. "A inviolabilidade dos direitos civis e políticos dos
  cidadãos brasileiros, que tem por base a liberdade, a segurança
  individual e a propriedade, é garantida pela Constituição do Império,
  pela maneira seguinte:

  § 16º. Ninguém será sentenciado, senão pela autoridade pública
  competente, por virtude de lei anterior e na forma por ela
  prescrita``; Art. 50.''A queixa ou denúncia que não contiver os
  requisitos legais não será aceita pelo juiz, salvo o recurso
  voluntário da parte``; Tit. 58, § 17, sobre possibilidades de
  revogação de normas, especialmente a expressão''posto que sejam feitas
  com a solenidade devida``; Tit. 66, § 29, especialmente a primeira
  frase:''E as posturas e vereações, que assim forem feitas, o
  corregedor da comarca não lhes poderá revogar, nem outro algum oficial
  ou desembargador nosso, antes as façam cumprir e guardar (...)".}).

\_\_\_\_\_\_\_\_

Está determinado no Código Criminal, em termos imperativos, e de modo
indeclinável em o art. 7º:

"Que, nos delitos de abuso de liberdade de comunicar os pensamentos, são
criminosos, e, por isso, responsáveis:

1º: O impressor, o qual ficará isento de responsabilidade, mostrando,
por escrito, obrigação de responsabilidade do editor, sendo este pessoa
conhecida, residente no Brasil, que esteja no gozo dos direitos
políticos; salvo quando escrever em sua causa própria, caso em que se
não exige esta última qualidade;

2º: O editor, que se obrigou, o qual ficará isento de responsabilidade,
mostrando obrigação, pela qual o autor se responsabilize, tendo este as
mesmas qualidades exigidas no editor para escusar o impressor;

3º: O autor, que se obrigou".

É, pois, evidente que, por força da lei, no Juízo, é precisamente o
impressor o primeiro responsável, como autor presumido e intuitivo do
impressor{[}impresso{]} criminado.

Impressor ou tipografário, em acepção jurídica, e na frase técnica da
lei, é o dono, o senhor, o proprietário, o que, pelo direito, tem o
domínio da tipografia (Cód{[}igo{]} Crim{[}inal{]}, art. 303; dr. M. da
Cunha, Annot{[}ações{]} ao Cód{[}igo{]} Crim{[}inal{]}, pág{[}ina{]} 55;
Ferr{[}eira{]} Borg{[}es{]}, Dicc{[}ionário{]} Jur{[}ídico{]},
pág{[}ina{]} 136\footnote{Art. 303. ``Estabelecer oficina de impressão,
  litografia ou gravura sem declarar perante a Câmara da cidade, ou
  vila, o seu nome, lugar, rua e casa em que pretende estabelecer, para
  ser escrito em livro próprio, que para esse efeito terão as Câmaras; e
  deixar de participar a mudança de casa, sempre que ela aconteça''.
  Gama faz referência exata ao trecho e página em que se discute as
  responsabilidades penais do impressor, editor e autor nos delitos de
  imprensa. Gama, cabe destacar, optou em citar a obra de maneira
  simplicada e própria de quem referencia um livro \emph{inter pares},
  ou seja, entre especialistas na matéria. Assim, embora o livro possa
  ser chamado de ``Anotações ao Código Criminal'', o que de fato é, seu
  título oficial é \emph{Código Penal do Imperio do Brazil com
  observações sobre alguns de seus artigos pelo doutor Manoel Mendes da
  Cunha Azevedo} (1851). Mendes da Cunha (1797-1858), como era
  conhecido, foi professor de Direito Romano da Faculdade de Direito do
  Recife; Gama consultou o verbete ``dono'' no \emph{Diccionario
  juridico-commercial} (1856) de José Ferreira Borges (1796-1838),
  jurisconsulto português e autor do primeiro Código Comercial de
  Portugal (1833). A citação confere exatamente com a página indicada e,
  mais, Gama reproduz a ordem dos termos -- ``senhor'', ``proprietário''
  {[} o que tem o{]} ``domínio'' -- conforme apresentada por Ferreira
  Borges.}).

Editor é o cidadão que, no gozo de direitos políticos, sob sua própria
responsabilidade, ou de outrem, se faz cargo da publicação de escritos
alheios (Cód{[}igo{]} Crim{[}inal{]}, art. 7º, § 2º).\footnote{O art.
  7º, § 2°, que se lê no corpo do texto, qualifica o editor e define sua
  responsabilidade penal.}

Autor é o cidadão que, no gozo de direitos políticos, produz ou exibe
trabalho escrito, para ser publicado (Cód{[}igo{]} cit{[}ado{]}, art.
7º, § 3º).\footnote{A definição é do próprio Gama, haja vista o Código
  Criminal, na parte que se lê no corpo do texto, não esmiuçar, como o
  fez com a figura do editor, quem seria juridicamente o autor.}

Temos, pois, segundo as prescrições legais, que:

- O primeiro responsável por a publicação dos escritos, em razão do seu
ofício, é o impressor;

- O impressor só poderá ser escusado da responsabilidade provando
imediata e legalmente a do editor;

- Esta responsabilidade, do editor, só é aceitável quando conjuntamente
seja provada a sua idoneidade;

- A escusa do impressor, ou do editor, é judicial, provocada por queixa
ou denúncia e julgada pelo juiz, pois que constitui auto de corpo de
delito (P{[}imenta{]} Bueno, Dir{[}eito{]} Crim{[}inal{]}\footnote{Possivelmente,
  Gama se refere ao célebre \emph{Apontamentos sobre o processo criminal
  brasileiro} (1857), de José Antonio Pimenta Bueno (1803-1878),
  conhecida autoridade política que desempenhou os cargos de juiz,
  desembargador (1844-1847), ministro da Justiça (1849), presidente do
  Conselho de Ministros (1870-1871) e senador do Império (1853-1878).}).

Do mesmo modo, poderá o editor escusar-se da responsabilidade, exibindo,
em juízo, a do autor idôneo, que se o obrigou.

Tal é a ordem legal do processo.

\_\_\_\_\_\_\_\_\_\_\_\_\_\_

Terá a \emph{Província de S. Paulo} um ``impressor'' e um ``editor'' que
possam regularmente assumir a responsabilidade legal dos escritos que
imprimem-se nesse jornal?

Na falta dessas duas entidades legais, poder-se-á, \emph{ipso
facto}\footnote{Por isso mesmo, necessariamente.}, devolver a
responsabilidade criminal do escrito ao autor, ainda quando este se
tenha obrigado?

A \emph{Província de S. Paulo} é propriedade de uma associação
comanditária\footnote{O mesmo que sociedade em comandita, forma
  associativa que possui duas classes de sócios: os comanditados e os
  comanditários. Os comanditados têm responsabilidades ilimitadas frente
  a terceiros, maiores obrigações sociais, trabalham e contribuem
  financeiramente; os comanditários, ao contrário, têm responsabilidade
  limitada, são alheios a obrigações na administração do negócio, não
  contribuem com trabalho, apenas com capital. Gama utiliza os atributos
  da pessoa jurídica que representa, o jornal \emph{A Província de S.
  Paulo}, para descaracterizar sua responsabilidade na denúncia
  oferecida.}, da qual o representante ``se não conhece nestes autos.''

São redatores deste jornal os drs.: Américo de Campos\footnote{Américo
  Brazilio de Campos (1835-1900), nascido em Bragança Paulista (SP), foi
  advogado, promotor público, jornalista e diplomata. Entre diversas
  colaborações na imprensa, foi redator d'\emph{O Cabrião}, diretor do
  \emph{Correio Paulistano} e fundador d'\emph{A Província de São
  Paulo}. Desde os seus tempos de estudante na Faculdade de Direito de
  São Paulo, na turma que se formou em 1860, até a ruptura pública dos
  finais de 1880, Américo de Campos foi um dos parceiros mais próximos
  de Luiz Gama, podendo ser encontrado em diversas fontes atuando ao
  lado de Gama na imprensa, na política ou na tribuna.} e F. Rangel
Pestana; \footnote{Francisco Rangel Pestana (1839-1903), natural de Nova
  Iguaçu (RJ), foi jornalista e político. Fundador do jornal \emph{A
  Província de São Paulo} (1875), foi também deputado e senador por
  sucessivas legislaturas.}é administrador J. Maria Lisboa\footnote{José
  Maria Lisboa (1838-1918), nascido em Lisboa, Portugal, foi jornalista,
  tipógrafo, editor e empresário radicado em São Paulo. Fundou, dirigiu
  e foi redator de dezenas de periódicos, destacando-se o \emph{Correio
  Paulistano}, \emph{A Província de São Paulo} e o \emph{Almanach
  Litterario de S. Paulo}.}, cidadão português, encarregado da parte
econômica da empresa.

Está, portanto, demonstrado, a toda luz, que este jornal não tem
``impressor'' ostensivo e menos ainda ``editor'' conhecidos (vid{[}e{]}
Dec{[}laração{]}, f{[}olha{]}. 8, pág{[}ina{]} 1ª, princ{[}ipal{]}).

A petição inicial de f{[}olha{]} 4, cuja forma é desconhecida em
direito, de próprio arbítrio, e com menoscabo da lei, deu patente de
impressor, sob denominação de editor ao dr. Francisco Rangel Pestana, e
tal petição, aliás inaceitável em juízo, por não conter os requisitos do
art. 79 do Código de Processo Criminal,\footnote{Os requisitos eram: a
  descrição do fato criminoso com todas as suas circunstâncias
  envolvidas; o valor provável do dano que se alegava sofrido; o nome do
  autor, ou seus sinais característicos, se a autoria fosse
  desconhecida; as razões de convicção ou presunção; a nomeação de todos
  os informantes e testemunhas do fato; e, finalmente, o tempo e o lugar
  em que foi o crime perpetrado.} para obrigar o impressor, como
acusado, na conformidade do art. 7º, § 1º, do Cód{[}igo{]}
Crim{[}inal{]}, e que, entretanto, foi deferida, com infração manifesta
do art. 50 do Dec{[}reto{]} nº 4.824 de 22 de Novembro de 1871, deveria
ter sido rejeitada, e, não o sendo, criou mais uma insanável deformidade
nos autos.

O autor desta causa, preterindo soberanamente a fórmula da lei, chamou a
juízo não o impressor, como devera, mas o editor, para responder em
primeiro lugar!... E não contente com este ato de ilegal inversão,
qualificou de editor ao dr. Francisco Rangel Pestana!! (Vid{[}e{]}
f{[}olha{]} 4 v{[}erso{]}).

E o dr. Pestana, sem atender a irregularidade do chamado, e sem refletir
na ilegalidade do ato, enviou indevidamente o imprestável documento de
f{[}olha{]} 9!...

A pessoa indicada e citada, e que não veio a juízo, como editor, para
exibir intempestivamente o autógrafo, é o dr. Francisco Rangel Pestana;
e a ele somente se refere a certidão de f{[}olhas{]} 4 e 5; o oficioso
apresentante do documento, no entanto, é o dr. Américo de Campos!...
(Vid{[}e{]} f{[}olhas{]} 10 e 11).

Exibido o autógrafo de f{[}olha{]} 9, sem que se desse a verificação da
idoneidade do autor, e nem julgada fosse a exoneração do inventado
editor, foram os autos entregues ao querelante, que ressurgiu no juízo
com a petição de f{[}olha{]} 2!...

E deu-se o chamamento do editor, de modo irregularíssimo, e sem que
acusado fosse o impressor; e em ausência de QUEIXA do querelante, meio
único pelo qual se poderia legalmente instaurar o processo, que tem
formas impreteríveis; e foi o improvisado editor exonerado de
responsabilidade, mediante a falsa obrigação de um estrangeiro,
notoriamente conhecido, como tal qualificado no processo, que nunca foi
guarda nacional, nem juiz de fato,\footnote{O mesmo que jurado, julgador
  no Tribunal do Júri.} nem votante, nem eleitor de paróquia, nesta
cidade importante, onde há mais de 20 anos tem efetiva residência; e sem
que se mostrasse, quando admissível fosse, a sua obrigação, que ele está
no gozo de direitos políticos, ou que em qualquer tempo tivesse.

Assim, fica provado, e de modo inconcusso\footnote{Inabalável,
  indiscutível.}, que a \emph{Província de S. Paulo} não tem impressor,
nem editor; que, se os têm, não são conhecidos; e que, portanto, nulo é
este processo, porque a queixa é nenhuma.

\_\_\_\_\_\_\_\_\_\_\_\_\_\_

Há muito tempo que o querelado reside no termo da capital, paróquia da
Sé, e distrito do sul (Vid{[}e{]} doc{[}umentos{]} sob n{[}úmeros{]} 6,
7 e 8).

E, tendo a sua residência, como tem, efetiva e legal neste distrito,
segundo a prova que exige o direito que exibe o querelado, não pode ele
ser processado nem julgado pelo juiz de direito do 2º Distrito Criminal
da comarca desta cidade\footnote{Jurisdição que corresponde ao distrito
  norte da freguesia da Sé.}.

Porque o governo da província, em cumprimento do seu dever, executando o
preceito legal, dividiu a comarca da capital em dois distritos criminais
e, por este ato, o distrito do sul da paróquia da Sé, onde reside o
acusado, foi incluído no primeiro da comarca (doc{[}umento{]} nº 3).

É certo, entretanto, que a tipografia da \emph{Província de S. Paulo}
está situada à rua da Imperatriz, no distrito do norte da paróquia da
Sé, distrito este que foi incluído, pelo ato mencionado, no 2º Distrito
Criminal da comarca, onde verificou-se a propositura do pleito. Mas este
fato, de \emph{per si}, não constitui, nem pode juridicamente
constituir, o que na expressão da lei se chama foro do delito. Porque a
existência do foro de delito, fato legal, do qual decorre a competência
do juízo, em razão da escolha do querelante (Cód{[}igo{]} de
Proc{[}esso{]} Crim{[}inal{]}, art. 160, § 3º, 2ª parte)\footnote{Art.
  160. "O denunciado, ou aquele contra quem houve queixa, não será
  ouvido para a formação da.

  § 3º, 2ª parte. É distrito da culpa aquele lugar em que foi cometido o
  delito, ou onde residir o réu, ficando à escolha do queixoso".}, é
fatalmente uma designação objetiva, que prescreve determinado local, com
exclusão de qualquer outro em que se tenha cometido o delito.

É, porém, igualmente certo que na presente peculiar hipótese, não é a
tipografia o local do cometimento ou berço do delito, na frase dos
juristas. O fato material ou elemento objetivo do crime, neste caso,
consiste na publicidade, esta existe por a distribuição dos impressos, e
esta distribuição realizou-se em diversos pontos de diferentes paróquias
(vid{[}e{]} Cód{[}igo{]} Crim{[}inal{]}, art. 230 e 7º, § 4º).\footnote{Art.
  230. ``Se o crime de calúnia for cometido por meio de papeis
  impressos, litografados ou gravados, que se distribuirem por mais de
  quinze pessoas contra corporações que exerçam autoridade pública'';
  art. 7º, § 4°, ver acima.} Não se pode precisar o lugar da publicação,
nem designar o foro do delito. A queixa, portanto, só podia ser dada no
foro do réu, perante o juízo de direito do 1º Distrito. Fez-se o
contrário. Nulo é o sumário por incompetência do juiz\footnote{Arremata,
  portanto, arguindo que o sumário de culpa está prejudicado por obra do
  juiz que conheceu da queixa de que não poderia, uma vez que não
  possuía competência para julgá-la.}.

\_\_\_\_\_\_\_\_\_\_

Nos processos por abuso da liberdade de comunicar os pensamentos é
condição essencial, para existência do delito, que o impresso criminado
seja distribuído por mais de 15 pessoas (Cód{[}igo{]} Crim{[}inal{]},
art. 230).

E não é admissível a existência de tal condição por simples ou mesmo por
fundada presunção, que indiretamente resulte de cogitações imaginosas
das testemunhas, ou de fatos não averiguados judicialmente, que bem
podem ocultar inexatidões ofensivas da verdade ou dos direitos de defesa
(Cód{[}igo{]} Crim{[}inal{]}, art. 36).

E os depoimentos de f{[}olhas{]} 18, 22 e 24, e seguintes, sobre serem
todos prestados por pessoas suspeitas, se não legalmente incapazes -- os
distribuidores do jornal acusados e o administrador da tipografia --,
responsáveis de fato, e segundo o direito, pelo delito,
mormente\footnote{Sobretudo, principalmente.} quando não são conhecidos
o impressor, nem o editor e o autor é estrangeiro, são nenhuns perante a
lei, por não conterem afirmação clara, positiva e
inobliterável\footnote{Inapagável, impossível de não se levar em conta.}
do fato principal da distribuição (Cód{[}igo{]} Crim{[}inal{]}, art. 7º,
§ 4º, art. 239; Ac{[}órdão{]} do {[}Tribunal{]} da Rel{[}ação{]} do
Recife de 5 e 8 de Abril de 1862).\footnote{O art. 7º, § 4º, definia
  também como criminoso do delito de abuso da liberdade de comunicar os
  pensamentos àquele que estava na ponta da cadeia da comunicação, isto
  é, na forma da lei, ``o vendedor e o que fizer distribuir os
  impressos, ou gravuras, quando não constar quem é o impressor, ou este
  for residente em país estrangeiro, ou quando os impressos e gravuras
  já tiverem sido condenados por abuso e mandados suprimir''; O art.
  239, por sua vez, prescrevia que ``As imputações feitas a qualquer
  corporação, depositário ou agente de autoridade pública, contendo
  fatos ou omissões contra os deveres dos seus empregos, não sujeitam a
  pena alguma, provando-se a verdade delas''.}

\_\_\_\_\_\_\_\_

O querelado é vítima de uma exigência exorbitante do autor, que contra
ele traz odiosa demanda para a cobrança de quantia superior à que lhe é
devida.

O querelado impugnou o petitório\footnote{Refere-se à parte da petição
  inicial em que se elabora o pedido.} e está usando de recursos legais
para tirar-se da iminente violência que lhe faz o autor, à sombra da
lei, sob indevida proteção da justiça, e patrocinado por um
ajeitado-direito que realmente não existe.

Condenado por uma injurídica sentença que obriga o acusado ao pagamento
do que não deve, apelou para o Superior Tribunal e, revoltado, mui
justificadamente, por a injustiça de tal julgamento, veio à imprensa
invocar a opinião autorizada dos doutores e a imparcialidade dos
judiciosos cidadãos. E, para isso, repetiu, em defesa da sua causa e
como prova do seu incontestável gravame\footnote{Prejuízo, dano sofrido.},
o que, em alegações judiciais, em peças-forenses, já tinha afirmado,
\emph{sem reclamação} \emph{alguma} do querelante, que só agora,
decorridos meses, deu-se por ofendido de uma simples transcrição!...

É aforismo de direito, e muito antigo, ``quem se defende não faz
injúria, porque não ofende.''

Além do que fica exposto, separar trechos de um escrito, deslocar
frases, e/ou isolar assertos ou pensamentos conexos, como fez o autor, é
infringir escandalosamente a lei, para criar crimes de injúrias
(vid{[}e{]} Cód{[}igo{]} Crim{[}inal{]}, art. 8º, 240, 241;
Acc{[}órdão{]} da Rel{[}ação{]} da Corte, 1862; doc{[}umento{]} nº
5).\footnote{Art. 8º. ``Nestes delitos {[}referentes ao abuso da
  liberdade de comunicar os pensamentos{]} não se dá cumplicidade; e
  para o seu julgamento, os escritos e discursos em que forem cometidos
  serão interpretados segundo as regras de boa hermenêutica, e não por
  razões isoladas e deslocadas''.

  Art. 240. ``Quando a calúnia ou injúria forem equívocas, poderá o
  ofendido pedir explicações em juízo, ou fora dele''.

  Art. 241. ``O juiz que encontrar calúnias ou injúrias escritas em
  alegações, ou cotas de autos públicos, as mandará riscar a
  requerimento da parte ofendida, e poderá condenar o seu autor, sendo
  advogado, ou procurador, em suspensão do ofício por oito anos a trinta
  dias, e em multa de quatro a quarenta mil réis''.}

\_\_\_\_\_\_\_\_\_\_

Os fatos de que se queixa o querelante, e que ele próprio extraiu de um
escrito, com a calculada intenção de perseguir ao querelado, e forçá-lo
a um acordo na temerosa demanda cível, são:

1º: Que o querelante é exagerado nas suas exigências, querendo cobrar
mais do que lhe deve o querelado;

2º: Que o querelante vende gêneros de má qualidade aos seus fregueses;

3º: Que troca os gêneros vendidos por outros inferiores;

4º: Que invade violentamente a casa dos compradores para ajustar contas.

Os dois primeiros fatos, não contendo, como realmente não contém,
expressões diretas e positivamente ofensivas, só poderiam ser
considerados injuriosos mediante as diligências determinadas
expressamente em o artigo 240 do Código Criminal.

O terceiro, se delito envolve, é o previsto no artigo 264, § 4º, do
Código Criminal.\footnote{Previsão normativa para crimes de estelionato,
  sendo a hipótese do § 4º assim definida: ``Em geral, todo e qualquer
  artifício fraudulento pelo qual se obtenha de outrem toda a sua
  fortuna, ou parte dela, ou quaisquer títulos''.} E, neste caso, não
constitui crime de injúrias (doc{[}umentos{]} n{[}úmeros 1 e 2).

O quarto encerra uma verdade judicialmente provada pelo próprio
querelante. Não há, nem pode haver, crime na referência de fatos
verdadeiros, sinceramente feita, em defesa de direitos, sem dolo e sem
má fé. Queixe-se o querelante de si mesmo, e lembre-se ``que quem não
quer ser lobo não lhe veste a pele'' (doc{[}umento{]} nº 4).

III

Improcedente é o presente processo, deforme, monstruoso, nulo e
imprestável, perante o direito e a lei. Porquanto:

Não cabe nas atribuições dos juízes singulares, por exceção não
autorizada, o seu julgamento, a despeito das modificações que lhe não
são aplicáveis, estabelecidos na Lei nº 261 de 3 de Dezembro de 1841, e
no Regulamento nº 120 de 31 de Janeiro de 1842\footnote{Respectivamente,
  lei de reforma do Código de Processo Criminal (1832); e regulamento da
  parte policial e criminal da mencionada lei de reforma (1841).};

Não foi, nem podia ser revogada, por \emph{disposição genérica}, a
peculiar e privada, expressamente preceituada o artigo 68 da Lei de 20
de Setembro de 1830 (Ass{[}ento{]} de 16 de Novembro de 1700;
Ord{[}enações{]}, Liv{[}ro{]} 4º, Tít{[}ulo{]} 45);

Não é destituída de autoridade filosófica esta jurídica opinião, aliás
confirmada até pelo Supremo Tribunal de Justiça;

É, pelo direito e pela lei, da competência exclusiva do Júri o
julgamento da causa;

Assim sendo, nula é a queixa, imprestável o processo e improcedente a
ação (Const{[}ituição{]} Pol{[}ítica{]}, art. 179, § 11; Decr{[}eto{]}
nº 4.824 de 22 de Novembro de 1871, art. 50);

O distrito da subdelegacia do Sul da paróquia da Sé pertence ao 1º
Distrito Criminal da comarca da capital;

A queixa foi dada e processada perante o dr. juiz de direito do 2º
Distrito Criminal, incompetente, portanto, para dela conhecer. Nenhum,
por nulidade insanável, é conseguintemente o pleito por ele ordenado.

Sem queixa, nem denúncia, foi exibida a responsabilidade perante o juiz
incompetente. Com irregularidade de forma, inobservância do direito e
infração da lei, foi acusado o editor, em vez do impressor. Em falta de
responsável legal, criou-se, de improviso, um editor. Do mesmo modo, foi
este substituído pelo autor. O autor é estrangeiro e não pode ser
criminado;

O escrito criminado não contém matéria infringente das leis penais.
Quando a contivesse, o delito não seria o de injúrias. Não contém
matéria criminal porque encerra justa defesa, produzida sem má fé, em
juízo contencioso, sem reclamação do autor. E, se quem se defende não
ofende, a defesa justa não pode constituir injúria. Além de que, as
expressões malsinadas\footnote{Denunciadas, repreendidas.} são trechos
truncados, adrede\footnote{Intencionalmente, de propósito.} extraídos de
um escrito complexo; frases mutiladas, assertos destacados para
determinar cavilosa\footnote{Maliciosa, capciosa.} interpretação, com
violação notória da lei. E o delito não seria de injúria, porque o fato
atribuído, quando criminoso fosse, importaria delito que tem
procedimento oficial de justiça;

As testemunhas chamadas a depor são os distribuidores do jornal
querelado, são os perpetradores do ato material, são os agentes da
publicação, os responsáveis legais dela, os delinquentes qualificados,
se tal publicação encerra ofensas.

A distribuição necessária, feita por mais de 15 pessoas, não está
provada, porque os depoimentos concluem por presunções e as presunções
não fazem prova em juízo;

A causa está completamente perdida, para o autor, por nulidades quanto
às fórmulas; por absoluta improcedência, quanto aos fatos; por absurda,
quanto ao direito; e por atentatória, quanto à lei.

E, pois, em nome da justiça e da moralidade dos tribunais, em honra da
ciência, para manutenção da lei, em respeito à liberdade individual e
para garantia da segurança dos cidadãos:

Pede-se ao meritíssimo julgador a absolvição do acusado e condenação do
autor nas custas do sumário.\footnote{Refere-se ao processo.}

S. Paulo, 7 de abril de 1878.

O advogado,

LUIZ GAMA.

\_\_\_\_\_\_\_\_\_\_\_\_\_\_\_\_

SENTENÇA

Vistos estes autos em que são partes a{[}o{]} alferes João A. Ribeiro de
Lima, autor, e Candido Justiniano Silva, réu. Considerando que nos
crimes de abuso de imprensa são responsáveis, em primeiro lugar, os
impressores ou donos das tipografias, os quais só ficam isentos de
responsabilidade mostrando por escrito obrigação da responsabilidade do
editor, sendo este pessoa conhecida, residente no Brasil e que esteja no
gozo de seus direitos políticos (art. 7º, § 1º do Cód{[}igo{]});

Considerando que o editor só é escusado mostrando obrigação pela qual o
autor se responsabilize, devendo esse autor ter as mesmas condições
exigidas no editor para isentar o impressor (art. citado, § 2º);

Considerando que o autor de um escrito só é responsável quando
obrigue-se pela publicação do mesmo (art. citado, § 3º);

Considerando que nestes autos não consta a citação do impressor, nem que
este exibisse obrigação escrita do editor;

Considerando que, sendo requerida a citação do dr. Francisco Rangel
Pestana, compareceu o dr. Américo Brazílio de Campos, sem que conste que
um ou outro seja editor que se obrigou, ou impressor;

Considerando que, na hipótese de serem editores os drs. Pestana e
Campos, o documento por eles apresentado não prova que o querelado seja
o autor do artigo, e que pela publicação dele se obrigou, porquanto o
documento de f{[}olha{]} 8 refere-se à publicação ou transcrição de um
artigo inserto na \emph{Gazeta de Notícias} e nenhuma prova há nos autos
de que esse artigo seja o de que trata o querelante;

Considerando que, dos escritos em que forem cometidos abusos, não se
devem isolar e destacar frases, mas se os deve considerar em todo
contexto;

Considerando que o artigo de f{[}olhas{]}, tomado em seu todo,
representando um só ato, uma só intenção, não pode conter atos diversos
publicados pelo mesmo agente que motivem a acumulação de penas, se fora
lícito destacar as frases para classificar umas de injuriosas e de
caluniosas outras, também o seria considerar tantos crimes de injúria ou
de calúnia quantas fossem as frases que de uma e outra classe pudessem
ser encontradas no mesmo escrito;

Considerando que tomado o artigo em sua integridade deve preponderar o
crime de natureza mais grave, sendo por ele absorvidos ou{[}os{]}
outros, e que na hipótese destes autos deve-se considerar o artigo como
calunioso, e portanto devendo o processo ser o comum e da competência do
Júri, e não do juízo singular.

Julgo improcedente a queixa de f{[}olha{]} 2 e condeno o autor nas
custas.

S. Paulo, 28 de maio de 1878.

Sebastião José Pereira.\footnote{Sebastião José Pereira (1834-1881),
  nascido em São Paulo (SP), foi advogado, juiz de direito e presidente
  da província de São Paulo (1875-1878).}

\textbf{A FALSIFICAÇÃO DE MOEDA}

\textbf{*didascália*}

\emph{A prisão do fotógrafo Victor Telles e mais cinco artistas mexeu
com a cidade de São Paulo, aliás, nos dizeres de Gama, com ``todo o
país''. Fosse apenas figura retórica, ou não, o suposto crime alcançou,
de fato, uma proporção fora do comum. A polícia armou um aparato de
guerra para invadir o modesto estúdio de fotografia da rua Direita,
centro de São Paulo, onde Telles trabalhava. A partir da denúncia de uma
só testemunha, Telles e seus companheiros se viram alvo de uma batida
policial que os tomavam como suspeitos de um crime gravíssimo contra o
Tesouro Nacional: eram acusados sumariamente pelo crime de falsificação
de papel moeda. O pequeno estúdio do fotógrafo, portanto, abrigaria
máquinas e mais máquinas voltadas para fabricação de dinheiro falso.
Gama assume a defesa dos artistas no tribunal, requerendo ordem de}
habeas-corpus\emph{, e também na imprensa, através de dois artigos, que
se leem a seguir. }

\begin{center}\rule{0.5\linewidth}{\linethickness}\end{center}

\textbf{8. MOEDA FALSA -- FATOS E BOATOS}\footnote{In: \emph{A Província
  de S. Paulo} (SP), Seção Livre, 01/02/1878, p.~2.}

\textbf{*didascália*}

\emph{Literatura normativo-pragmática. Já na primeira frase -- ``Sabe
todo o país...'' --, tem-se a dimensão da repercussão pública que a
causa havia alcançado na imprensa e nas ruas de diversas cidades do
Brasil. A descrição suscinta do fato de que se discutia a criminalidade
é lapidar: ``Victor Telles e mais cinco artistas foram presos como
suspeitos de fabrico e introdução de moeda-papel falsa na circulação
monetária do império''. O fotógrafo Victor Telles e os demais artistas
estavam presos há aproximadamente um mês. Gama, por sua vez, contava o
caso com sua habitual maestria narrativa. A ``misteriosa reclusão de
seis homens, que, há quase um mês, esperam por formação de culpa!...'',
ganhava foros de luta épica, bem ao gosto do poeta, advogado e literato.
Num inquérito viciado e amparado num testemunho contaminado, argumentava
Gama, ``Victor Telles tinha adquirido proporções de herói de romance;
era o novo Samuel Gelb, mesmo sem licença do velho Dumas!'' O inquérito
policial, contudo, apontava a materialidade do crime e a autoria dos
mesmos artistas como falsários: o simples fotógrafo era apontado como
mentor intelectual de um crime ousado. O promotor público ordenou mais
diligências, entre elas, um exame nas máquinas que seriam destinadas à
fabricação de papel-moeda falso. Este ``elemento de prova criminal'',
ainda que a defesa tenha sido de algum modo cerceada de acompanhar a
perícia, tornou-se peça-chave da estratégia de Gama, que passou a
discutir alguns quesitos da perícia neste artigo. De maneira hábil,
certamente visando a decisão do Tribunal da Relação de São Paulo, que
pautaria o caso na semana seguinte, Gama conclui o texto convencido -- e
tratando de convencer seus leitores, especialmente seus leitores no
tribunal... -- de que ``é evidente a não existência do delito'' de
falsificação de papel moeda. Se houve algo falsificado, foi a lei. O
protesto de Gama, afinal, era ``contra o arbítrio que é a falsificação
criminosa da lei'', ocorrida, nesse caso, pelo ``equívoco e a ilusão do
juiz'', que, ``violando o direito, tortura sem motivo ao cidadão, em
nome da segurança comum''. }

\begin{center}\rule{0.5\linewidth}{\linethickness}\end{center}

Sabe todo o país que o sr. Victor Telles\footnote{Victor Telles de
  Rebello e Vasconcellos, brasileiro naturalizado, viveu em Montevidéu,
  Uruguai, Pelotas (RS) e morava em São Paulo, onde tinha um estúdio de
  fotografia estabelecido na rua Direita.} e mais cinco artistas foram
presos como suspeitos de fabrico e introdução de moeda-papel\footnote{Dinheiro
  e/ou título de crédito conversível em ouro ou moeda.} falsa na
circulação monetária do império; e que, em razão de tal suspeita, estão
presos há perto de um mês, sem formação de culpa!...

O sr. dr. Henrique Antonio Barnabé Vincent,\footnote{Embora não tenha
  informações pessoais de Barnabé Vincent, sabe-se que ele assinou,
  junto com Gama, ainda em 1878, um desagravo público ao juiz Gama e
  Mello. Cf., nesse volume, \emph{Ao exmo. sr. dr. Bellarmino Peregrino
  da Gama e Mello}.} promotor público da comarca, não se satisfazendo
com o resultado das diligências policiais, requereu novos exames, do
modo seguinte:

"O promotor público interino, porque seja necessário, para marchar com
passo seguro, e completar a prova de moedeiros-falsos dos presos Victor
Telles, e outros, necessita que se faça exame em diversos objetos em que
os exames anteriores não foram completos e em outros em que se não fez
exame, como nas duas máquinas de numerar o mal examinado rolo de papel
de linho encontrados na casa de Victor, e chapas metálicas encontradas
na casa de Victor e de Esprik de Verny,\footnote{Esprik de Verny, ou
  João Esprek de Verny, era alemão e morava na ladeira de Piques, São
  Paulo.} por ser este exame de grande alcance para a denúncia dos
mesmos.

Requer, por isso, que se faça o exame por pessoas profissionais, não de
fotografia, e com urgência.

QUESITOS:

1º: Se o papel de linho apresentado é da mesma natureza ou idêntico ou
imita o papel das notas de papel-moeda do tesouro nacional;

2º: Se o dito papel serve, ou preparado poderá servir para estampar, sem
fazer diferença alguma, notas do tesouro nacional, de cem mil réis, de
cinquenta, de vinte, de dez, de cinco, de dois, de mil ou de quinhentos
réis;

3º: Se as máquinas de numerar servem para numerar notas do tesouro
nacional, se os algarismos estampados por qualquer das duas máquinas são
idênticos em forma aos algarismos dos números das notas do tesouro
nacional;

4º: Se acharam ou existem recibos da casa de Victor Telles numerados
pelas ditas máquinas;

5º: Qual a largura, comprimento e grossura das chapas metálicas
encontradas nas casas de Victor Telles e Esprik de Verny.

6º: Se as chapas têm tamanho suficiente para abrir-se uma forma de
qualquer nota do Tesouro Nacional.

RESPOSTAS

\emph{Ao 1}º \emph{quesito:}

Que pelo exame feito, e conforme os dados ao seu alcance, respondem que
o papel de linho de que se trata, parecendo da mesma natureza do papel
de algumas notas do Tesouro Nacional, não é, contudo, idêntico;

\emph{Ao 2}º \emph{quesito:}

Que o dito papel, mesmo preparado, não pode servir para serem nele
estampadas notas do Tesouro Nacional, de qualquer valor, sem haver
diferenças;

\emph{Ao 3}º \emph{quesito:}

Que as duas máquinas de numerar não servem para as notas do Tesouro
Nacional, cujos algarismos não são idênticos aos estampados por qualquer
das referidas máquinas;

\emph{Ao 4}º \emph{quesito:}

Não respondem por não terem conhecimento do objeto de que aí se trata;

\emph{Ao 5}º \emph{quesito:}

Que entre as chapas apresentadas a exame, existem três com as seguintes
dimensões:

Uma com 188 milímetros de comprimento e 83 ditos de largura;

Outra, com 183 milímetros de comprimento e 74 ditos de largura;

E a terceira com 192 milímetros de comprimento e 75 ditos de largura;

\emph{Ao 6}º \emph{quesito:}

Finalmente, que essas três chapas são as únicas, das apresentadas, que
têm tamanho e espessura suficientes para abrir-se uma forma de qualquer
nota do Tesouro, de 5\$000 réis, 2\$000 réis, 1\$000 e 500 réis
americanas.

(Assinados)

ANTONIO D. DA. C. BUENO - (juiz)

F. H. TRIGO DE LOUREIRO - (perito)

JOÃO R. DA. F. ROSA - (idem)

H. A. B. VINCENT - (promotor)

J. MOREIRA LYRIO - (testemunha)

M. C. QUIRINO CHAVES - (idem)

E. DE OLIVEIRA MACHADO - (escrivão)

\_\_\_\_\_\_\_\_\_

Este exame, que deve ser considerado da maior importância, como elemento
de prova criminal, e que, entretanto, muito favorece a causa dos
supostos fabricantes de moeda falsa, embora obscuro em diversos pontos,
no que concerne à defesa dos acusados, efetuou-se em ausência destes,
cujos direitos não foram devidamente acatados.

Todos conhecem esta lamentável ocorrência, se não calculado embuste, com
que foi surpreendida até a perspicácia da autoridade, e que deu em
resultado a misteriosa reclusão de seis homens, que, há quase um mês,
esperam por formação de culpa!...\footnote{Fase do processo em que se
  apura os indícios mínimos da existência, natureza e circunstâncias do
  crime e de seus potenciais agentes.}

Todos conhecem, por a leitura dos periódicos e do relatório firmado pelo
exmo. sr. dr. chefe de polícia,\footnote{Embora não nominado
  expressamente, o chefe de polícia era o próprio Furtado de Mendonça.
  Francisco Maria de Sousa Furtado de Mendonça (1812-1890), nascido em
  Luanda, Angola, foi subdelegado, delegado, chefe de polícia e
  secretário de polícia da província de São Paulo ao longo de quatro
  décadas. Foi, também, professor catedrático de Direito Administrativo
  da Faculdade de Direito de São Paulo. A relação de Luiz Gama com
  Furtado de Mendonça é bastante complexa, escapando, em muito, aos
  limites dos eventos da demissão de Gama do cargo de amanuense da
  secretaria de polícia, em 1869. Para que se ilustre temporalmente a
  relação, tenhamos em vista que à época do rompimento público, aos
  finais da década de 1860, ambos já se conheciam e trabalhavam juntos
  há cerca de duas décadas; e, mais, Gama não rompeu definitivamente com
  Furtado de Mendonça, como erroneamente indica a historiografia, visto
  o presente artigo, \emph{Aos homens de bem}, que é uma espécie de
  defesa moral e política da carreira de Furtado de Mendonça.} os
indícios fundados em presunções, e as presunções destruídas pelas
próprias testemunhas da acusação e pelos exames policiais, que serviram
de base à ilegal detenção de seis cidadãos, com flagrante violação da
lei!...

Há em todo vasto inquérito organizado pela polícia \emph{um só
depoimento} que faz carga aos acusados; e é tal depoimento prestado pelo
sr. Joaquim Fernandes da Cunha, negociante da cidade de Santos; mas este
sr. Fernandes da Cunha, na considerada opinião dos distintos senhores
tenentes Gaspar e Dias Baptista (\emph{está escrita nos autos!}) É
INDIGNO DE FÉ; porque, pelo seu caráter e irregular procedimento, tem má
reputação; era íntimo amigo de Victor Telles, e seu hospedeiro em Santos
veio a S. Paulo, de propósito, para denunciar à polícia Victor Telles e
os seus companheiros; deu como causa da denúncia o fato de não querer
Telles pagar-lhe a quantia de 300\$000 réis!

E a autoridade, seguramente por inadvertência, em vez de mandar que a
denúncia fosse tomada por termo, no sôfrego intuito de arranjar prova,
invertendo as posições, converteu o \emph{denunciante} em
\emph{testemunha}!!...

Neste memorável inquérito tudo tem corrido ao sabor da autoridade; à
mercê dos boatos; ao som das inventivas\footnote{Alegações inventadas,
  invencionices, fantasias.} as mais extravagantes; e das calúnias
desaforadas: a moeda falsa, as chapas, as gravuras, as máquinas, a
química, e até a sublimada alquimia avultaram na encantada fotografia da
rua Direita! Victor Telles tinha adquirido proporções de herói de
romance; era o novo Samuel Gelb, mesmo sem licença do velho
Dumas!\footnote{Alexandre Dumas (1802-1870), o pai, nascido em
  Villers-Cotterêts, França, foi jornalista, dramaturgo e romancista de
  grande sucesso. Autor de obras consagradas como \emph{Os três
  mosqueteiros} (1844) e \emph{O Conde de Monte Cristo} (1844-1846),
  também escreveu \emph{Dieu Dispose} (1851), publicada como folhetim no
  \emph{Jornal do Commercio} (1851-1852) sob o título de \emph{Deus
  Dispõe}, e que tem Samuel Gelb como protagonista. Para ler outro
  artigo em que Gama cita um protagonista de um romance de Dumas,
  cf.~\emph{Resposta à redação do Diário de S. Paulo}, 29/01/1867.} Para
complemento do quadro dava-se o edifício como minado; e todo o
quarteirão prestes a ir pelos ares!!...

Tudo isto se disse; afirmou-se; a polícia ouviu e não contestou; e a
imprensa repetiu sobressaltada!...

Tudo, porém, tem o seu tempo; depois dos boatos, os fatos.

O sr. Joaquim Fernandes da Cunha, que é o protagonista deste drama, já
representou os seus papéis; fez de \emph{testemunha denunciante},
entidade nova no direito criminal; todos devem dar-se por divertidos; é
tempo de baixar o pano, para que as vítimas do embuste possam voltar aos
lares; e, sem culpas e sem penas, cuidar do trabalho e da família.

Guardamos silêncio enquanto a polícia, tomada de sincero civismo, embora
errando, procurava os vestígios de um crime gravíssimo; de um atentado
contra a fortuna pública e particular; contra a propriedade nacional;
hoje, porém, que é clara, que é evidente a não existência do delito;
hoje que o equívoco e a ilusão do juiz, por sua indesculpável
insistência, violando o direito, tortura sem motivo ao cidadão, em nome
da segurança comum, protestamos contra o arbítrio que é a falsificação
criminosa da lei.

S. Paulo, 31 de Janeiro de 1878.

O advogado, LUIZ GAMA.

\textbf{9. TRIBUNAL DA RELAÇÃO}\footnote{In: \emph{A Província de S.
  Paulo} (SP), Seção Livre, 10/02/1878, p.~3.}

\textbf{*didascália*}

\emph{Literatura normativo-pragmática. Gama rebate a redação da} Tribuna
Liberal\emph{, que havia criticado a decisão do Tribunal da Relação de
São Paulo em conceder ordem de} habeas-corpus \emph{em favor de Victor
Telles e os outros cinco artistas presos sem formação de culpa. O artigo
tem passagens que detalham os bastidores da causa e alguns detalhes da
sessão no Tribunal da Relação de São Paulo. Revela, também, como Gama se
constitui em advogado dos clientes aprisionados, agindo, conforme conta,
``por inspiração própria, e não por conselhos ou sugestões de outrem''.
Após 33 dias presos, Telles e os demais artistas conseguem, por
intermédio de Gama, a tão desejada soltura. }

\begin{center}\rule{0.5\linewidth}{\linethickness}\end{center}

A notícia relativa à concessão de \emph{habeas-corpus} em favor de
Victor Telles e outros, dada pela \emph{Tribuna Liberal} de hoje, é
inexata em grande parte.

Fui eu quem requereu \emph{habeas-corpus} em prol dos pacientes; e o fiz
em meu nome; por inspiração própria, e não por conselhos ou sugestões de
outrem.

Serviram de fundamento à petição as ilegalidades incontestáveis de que
foram vítimas os custodiados.

É verdade que o paciente Carvalho Amarante foi advertido, quando estava
sendo interrogado pelo exmo. sr. conselheiro presidente do Tribunal, por
se haver encostado na balaustrada\footnote{Nesse caso, fileira de
  pequenas colunas que divide o espaço do tribunal ocupado por
  advogados, promotores, juízes, testemunhas, réus, serventuários, do
  público do auditório.}; assim como é verdade haver o mesmo Carvalho
Amarante procurado o sr. dr. Aquilino para seu advogado; mas é
igualmente certo que o sr. dr. Aquilino recusara a causa e aconselhou ao
paciente de procurar outro advogado, incluindo o meu humilde nome entre
os considerados que declarou.

Não é também exato que o exmo. sr. conselheiro Gama\footnote{Agostinho
  Luiz da Gama (?-1880), nascido na província do Mato Grosso, foi
  político e magistrado. Exerceu os cargos de juiz municipal, juiz de
  direito e desembargador do Tribunal da Relação de São Paulo. Foi chefe
  de polícia das províncias da Bahia, Pernambuco e na Corte (Rio de
  Janeiro), além de presidir a província de Alagoas.} insinuasse a
qualquer dos acusados o recurso de \emph{habeas-corpus}. Carvalho
Amarante, sabendo que a polícia o queria prender, por ignorância das
leis do processo, e antes de tomar advogado, foi à casa do sr.
conselheiro Gama, em procura do dr. Aquilino. Encontrou o dono da casa e
narrou-lhe o caso. A resposta do sr. conselheiro Gama foi esta:

``\emph{Vá se apresentar à autoridade, ou espere que o prendam.}''

Foi isto o que narrou perante o Tribunal o sr. Carvalho Amarante, e não
o que lhe é atribuído pela \emph{Tribuna}.

O voto contrário do exmo. sr. conselheiro Gama, aliás improcedente, não
tem a origem que a \emph{Tribuna} lhe empresta; S. Excia. votou para que
fosse de novo ouvido o dr. juiz de direito, por ser deficiente e pouco
clara a informação prestada.

Sou reconhecido como acérrimo\footnote{Obstinado, inflexível.} inimigo
de arbitrariedades; não dispenso favores nem aos meus próprios amigos;
porque acima da amizade está a lei, a verdade e o público interesse; mas
não posso, por isso mesmo, autorizar, com o meu silêncio, censuras
injustas esteiadas em inexatidões.

S. Paulo, 9 de fevereiro de 1878.

O advogado,

LUIZ GAMA.

\textbf{O ROUBO}

\textbf{*didascália*}

\emph{Gama avisava ao público: ``vou em cumprimento do meu dever''.
Talvez o leitor da época não se desse conta do que estava por vir. Mas,
do primeiro texto localizado, em junho de 1877, até o último texto
encontrado, em março de 1878, correram nove meses e um total de doze
artigos diferentes. Todos os textos, por sua vez, relacionados ao mesmo
caso: o roubo da alfândega de Santos e a prisão do principal suspeito do
crime, justamente o tesoureiro da alfândega, Antonio Eustachio Largacha.
Não foi um roubo qualquer. O país inteiro noticiou o crime. Logo se
soube que mais de 185 mil contos de réis -- o equivalente aproximado a
atuais vinte milhões de reais! -- foram surrupiados do cofre-forte da
alfândega. Todas as evidências colhidas em diligências sumárias
apontavam para o tesoureiro Largacha. Imediatamente posto em prisão,
Largacha procurou um advogado e, não se sabe como, chegou a Luiz Gama.
Em realidade, Gama trabalharia com outro advogado -- o também jornalista
e advogado Ribeiro Campos --, mas não está claro, todavia, quem assumiu
a causa primeiro. Seja como for, trabalharam em equipe enquanto Largacha
mofava desesperado numa cela da cadeia de Santos. Ambos, Gama e Ribeiro
Campos, escreveram individualmente sobre o caso. Escreveram também em
coautoria. Contudo, por razões de método, a seleção de artigos que
compõem essa seção reúne aqueles firmados individualmente por Gama e os
assinados em conjunto, i.e., por Gama e Ribeiro Campos, além de duas
réplicas de um contendor que se sentiu pessoalmente injuriado e ao qual
Gama resolveu responder dirigidamente. Ao todo, são doze textos, todos
firmados por Gama, sozinho ou em dupla, e as duas mencionadas
contestações de terceiros. Dentre todos os artigos, um deles --
``\emph{Egrégio Tribunal da Relação -- Processo da Alfândega de
Santos}'' -- possui uma estrutura singular e uma natureza que lhe torna
histórico já de saída: é simplesmente o mais extenso artigo escrito por
Luiz Gama. Devido ao tamanho, não saiu, como os demais, publicado nas
colunas convencionais do jornal} A Província de S. Paulo\emph{, senão
como encarte especial do jornal e, ao que parece, também como livreto
avulso. Além disso, o artigo foi republicado no espaço mais caro de um
dos principais e mais lidos jornais do país à época, o} Jornal do
Commercio \emph{(RJ). Assim, se é verdade que esse artigo é o mais
extenso de toda literatura normativo-pragmática de Luiz Gama, é também,
provavelmente, aquele que mais leitores alcançou, haja vista a
repercussão geral do roubo milionário na alfândega de Santos e a ampla
circulação do texto na imprensa de São Paulo e do Rio de Janeiro. Embora
o ``\emph{Egrégio Tribunal da Relação -- Processo da Alfândega de
Santos}'' seja bastante sólido e tenha unidade em si mesmo, é de se
notar que os demais textos não só não lhe são estranhos, como convergem
todos para o mesmo assunto, elaboram hipóteses trabalhadas anteriormente
e versam sobre ideias semelhantes. Desse modo, pode-se dizer que a
discussão do célebre crime da alfândega de Santos ocupou um espeço
central na reflexão jurídica de Gama entre os anos 1877 e 1878. Lidos em
conjunto, os doze textos podem ser vistos como uma articulação entre
perícia criminal, conhecimento normativo e sede de justiça. Combinação,
aliás, em que Gama era mestre.}

\begin{center}\rule{0.5\linewidth}{\linethickness}\end{center}

\textbf{10. \emph{HABEAS-CORPUS}}\footnote{In: \emph{A Província de S.
  Paulo} (SP), Seção Livre, 20/06/1877, p.~2.}

\textbf{*didascália*}

\emph{Gama avisava ao público: ``vou em cumprimento do meu dever''.
Talvez o leitor da época não se desse conta do que estava por vir. Mas,
dessa primeira parte, que receberia continuidade já no dia seguinte, até
o que parece ser o último texto da série, em março de 1878, nove meses
depois dessa publicação, iriam mais doze textos. Todos relacionados ao
mesmo caso, qual seja, a prisão e o} habeas-corpus \emph{em favor do
tesoureiro da alfândega de Santos, Antonio Eustachio Largacha. Já no
primeiro parágrafo, Gama revela ter sido derrotado no Tribunal da
Relação de São Paulo, que havia negado uma ordem de soltura que ele
tinha requerido, e qual a síntese da decisão dos desembargadores. Aliás,
Gama saía da sessão no tribunal direto para a escrivaninha onde redigiu
o texto. Este, portanto, é escrito no calor da hora, imediatamente após
a denegação da ordem de soltura. Ainda assim, não se vê um Gama
colérico, como talvez fosse de se esperar de um advogado recém saído da
tribuna de defesa. Ao contrário, a verve sóbria combina com a forma
solene que o comentário normativo-pragmático assume ao longo das
páginas. ``Tenho por injurídica, ilegal e insubsistente esta decisão'',
cravava Gama, partindo para a discussão técnica, por um lado, e
principiológica, por outro, ambas solidamente encravadas na tradição do
conhecimento normativo brasileiro. O Tribunal da Relação, como
argumentaria, não era competente ``para conhecer a procedência de
prisões administrativas'', como era, afinal, o tipo da ordem de prisão
expedida contra seu cliente, Largacha. Mas se esse era um dos
argumentos, não se pode dizer que era o que dava coesão e estrutura para
a obra que passaria a escrever por mais de dez textos. Gama iria ao
fundamento da ordem; ao fundamento da norma. ``A questão, para mim'',
advertia Gama, ``não só por amor da ciência do direito, como em relação
às garantias legais, mantenedoras da honra e da liberdade do cidadão, é:
se a ordem de prisão expedida contra o tesoureiro da alfândega de Santos
tem fundamento legal''. Da simples questão -- se a ordem de prisão tinha
fundamento legal --, uma obra de arte. O que Gama encontraria no fundo
da ordem, no fundo da norma, é próprio dos livros de história do
direito.}

\begin{center}\rule{0.5\linewidth}{\linethickness}\end{center}

I

O colendo Tribunal da Relação, em sessão de hoje, depois de ampla
discussão, negou a ordem de soltura, por mim requerida, em favor do
major Antonio Eustachio Largacha, tesoureiro da alfândega de Santos, por
o motivo de entender que é legal a prisão requisitada pela autoridade
administrativa, e realizada pelo sr. dr. juiz municipal da cidade de
Santos contra o mesmo tesoureiro.

Tenho por injurídica, ilegal e insubsistente esta decisão; e, por isso,
sem faltar ao acatamento devido aos provectos\footnote{Experientes.}
jurisconsultos, membros conceituados do egrégio Tribunal, vou em
cumprimento do meu dever, e para esclarecimento da questão, perante o
público, discuti-la à face das disposições vigentes.

No correr do meu escrito omitirei, por conveniência de método, a opinião
singular do exmo. sr. conselheiro Gama\footnote{Agostinho Luiz da Gama
  (?-1880), nascido na província do Mato Grosso, foi político e
  magistrado. Exerceu os cargos de juiz municipal, juiz de direito e
  desembargador do Tribunal da Relação de São Paulo. Foi chefe de
  polícia das províncias da Bahia, Pernambuco e na Corte (Rio de
  Janeiro), além de presidir a província de Alagoas.}, presidente do
Tribunal, se bem que, em muitos pontos contraria aos princípios
filosóficos do direito, e infringentes das leis inquebrantáveis da
lógica; porque tal opinião foi vitoriosamente combatida pelos seus
ilustrados colegas, e principalmente pelos exmos. srs. desembargadores
Uchôa\footnote{Ignacio José de Mendonça Uchôa (1920-1910), nascido na
  província de Alagoas, foi promotor público, juiz municipal e de
  órfãos, juiz de direito, desembargador dos tribunais da relação de
  Porto Alegre e de São Paulo, além de procurador da Coroa, Soberania e
  Fazenda Nacional e ministro do Supremo Tribunal de Justiça.},
Faria\footnote{José Francisco de Faria (1825-1902), natural do Rio de
  Janeiro (RJ), foi político e magistrado. Foi chefe de polícia da Corte
  (Rio de Janeiro), juiz de direito, desembargador dos tribunais da
  Relação de Ouro Preto e de São Paulo, procurador da Coroa, Soberania e
  Fazenda Nacional e ministro do Supremo Tribunal de Justiça. Teve
  muitos embates com Luiz Gama na parte contrária, sendo o mais célebre
  aquele em que Gama advogou \emph{habeas-corpus} para o africano congo
  Caetano. Como Gama relata na abertura de seu estudo sobre os efeitos
  manumissórios da proibição do tráfico de escravos, foi a partir de uma
  arguição do desembargador e procurador da Coroa José Francisco de
  Faria que ele resolveu responder ao público a gravidade da matéria.
  ``Este perigoso discurso, este enviesado parecer do respeitável
  magistrado'', respondia Gama, ``obrigou-me a escrever este artigo.''}
e Accioli\footnote{Luiz Barbosa Accioli de Brito (1825-1900) nasceu no
  Rio de Janeiro (RJ), foi juiz municipal e de órfãos, juiz de direito,
  desembargador e ministro do Supremo Tribunal de Justiça.}.

E assim procedo em razão de prevalecer o princípio, aliás incontestável,
de não carecer o Tribunal da Relação de competência para conhecer a
procedência de prisões administrativas, ainda quando ordenadas pelo
ministro da Fazenda, presidente do Tribunal do Tesouro Público Nacional,
em vista da disposição expressa e evidente do art. 18 da Lei n° 2.033 de
20 de Setembro de 1871.\footnote{Art. 18. ``Os juízes de direito poderão
  expedir ordem de \emph{habeas-corpus} a favor dos que estiverem
  ilegalmente presos, ainda quando o fossem por determinação do chefe de
  polícia ou de qualquer outra autoridade administrativa, e sem exclusão
  dos detidos a título de recrutamento, não estando ainda alistados como
  praças no exército ou armada''.}

A questão, para mim, a questão que cumpre ventilar, não só por amor da
ciência do direito, como em relação às garantias legais, mantenedoras da
honra e da liberdade do cidadão, é: se a ordem de prisão expedida contra
o tesoureiro da alfândega de Santos tem fundamento legal.

Sustento que a autoridade administrativa, na vertente hipótese, não
tinha jurisdição para requisitar a prisão; não tinha jurisdição porque o
alcance atribuído ao tesoureiro não é administrativo; requisitando a
prisão, a autoridade administrativa cometeu um erro; e sendo o erro
ofensivo da disposição legal, a realização da prisão, por parte da
autoridade judiciária, requisitada, importa ilegalidade e violação da
liberdade do funcionário: é o que passo a demonstrar.

S. Paulo, 19 de junho de 1877.

(\emph{Continua})

LUIZ GAMA.

\textbf{11. \emph{HABEAS-CORPUS}}\footnote{In: \emph{A Província de S.
  Paulo} (SP), Seção Livre, 21/06/1877, p.~2.}

\textbf{*didascália*}

\emph{Publicado no dia do aniversário de 47 anos de seu autor, a segunda
parte de ``\emph{Habeas-Corpus}'' sobe o tom da primeira, agregando
informações então desconhecidas pelo público, passando a discutir o
fundamento legal da ordem prisão de Antonio Largacha. Para começar, Gama
discute o testemunho do inspetor da tesouraria -- cargo da alta
burocracia fazendária -- perante os desembargadores do Tribunal da
Relação. Habilmente, afirma que o depoente agiu com ``inqualificável
arrogância'' contra o Poder Judiciário -- notemos a destreza em fazer
dessa causa um conflito de juridisções entre poderes distintos, i.e.,
Poder Executivo e Poder Judiciário. Ao desqualificar a palavra do
inspetor da tesouraria, que ordenara a prisão, muito embora a expedição
do ato oficial tivesse a firma do ministro da Fazenda, Gama atacava o
cerne da alegação das autoridades administrativas que haviam posto
Largacha, tesoureiro da alfândega, na prisão. O repertório normativo --
especialmente alvarás, decretos, leis, Códigos -- impressiona. Gama
esmiuça a questão e constrói um argumento, sintetizado em doze tópicos,
pela ``improcedência da prisão administrativa'', já que era ``fora de
dúvida que irregular, arbitrária e violenta foi a prisão do tesoureiro
da alfândega de Santos''. Ao final dessa parte da história, Gama
arrematava numa retórica que, antes de reiterar o estilo que lhe deu
fama, evidencia uma vez mais pressupostos de um raciocínio jurídico
singular. Vejamos: ``Se a lei não pode ser contrariada; se o direito tem
uma razão filosófica; se a lógica não é um contrassenso; se os fatos não
foram por mim falsificados; se a narração não foi por mim feita com
preterição da verdade; e se o caso é, como descrito, fica a votação, a
denegação unânime da ordem de soltura pelo colendo Tribunal da Relação,
a manutenção do arbítrio imposto pelo ministro da Fazenda, requisitado
pelo inspetor da tesouraria, e realizado pelo sr. dr. juiz municipal de
Santos, um ato de injustiça solene, com todas as honras fúnebres de um
saimento magno, é um ato de injustiça régia, imponente, em grosso,
o{[}u{]} por atacado''. }

\begin{center}\rule{0.5\linewidth}{\linethickness}\end{center}

II

A autoridade administrativa, disse eu ao terminar a primeira parte deste
artigo ontem publicada, errou requisitando a prisão do tesoureiro da
alfândega de Santos; e hoje acrescento -- exorbitou; porque não tinha
jurisdição para fazê-lo. E na informação que prestou ao colendo Tribunal
da Relação atentou, com inqualificável arrogância, contra a soberania e
independência do Poder Judiciário, que não é subordinado ao presidente
do Tribunal do Tesouro Público Nacional, ainda quando tal poder seja
representado pelo mais humilde juiz de paz de aldeia.

O sr. inspetor da tesouraria, com calculado intuito, disse na sua
resposta \emph{que a prisão fora ordenada pelo ministro da Fazenda}!...
E eu, de minha parte declaro que, se o ministro tal fez, o que devo
crer, confiado na palavra do sr. inspetor, violou a disposição da lei; e
o papel que, com tão alegre satisfação desempenhou o sr. inspetor da
tesouraria perante o dever e a lei não é, por certo, dos mais honrosos.

Em especial observância do Tít{[}ulo{]} 3°, § 2° e Tít{[}ulo{]} 7°, §§
9°, 10° e 11° do Alvará de 28 de Junho de 1808, por força do disposto no
art. 88 da Lei de 4 de Outubro de 1831 e art. 310 do Cód{[}igo{]}
Crim{[}inal{]}, determinou-se, no Decreto n° 657 de 5 de Dezembro de
1849, art. 2°, que o ministro e secretário de Estado dos Negócios da
Fazenda e presidente do Tribunal do Tesouro Público Nacional, NA CORTE,
e os inspetores das tesourarias, NAS PROVÍNCIAS, \emph{podem e devem
ordenar} a prisão dos tesoureiros, recebedores, coletores\footnote{Diz-se
  dos funcionários do Ministério da Fazenda encarregados do lançamento
  e/ou arrecadação de tributos.}, etc., quando forem \emph{omissos} ou
\emph{remissos} em fazer as entradas dos dinheiros a seu cargo, nos
prazos que pelas leis e regulamentos lhe estiverem marcados.\footnote{Respectivamente,
  título 3°, § 2°. ``Pelo que pertence aos bens e rendas, cuja
  arrecadação é diaria e finaliza no último {[}dia{]} de cada um mês,
  ordeno que a entrada se faça no meu Real Erário logo nos primeiros
  dias do mês próximo seguinte; que a cobrança dos subsídios, alfândegas
  e Casa da Moeda, onde as conferências, exames e contagens têm mais
  demora, a entrega se faça nos primeiros oito dias seguintes; que pelo
  que pertence a contratos, bilhetes da alfândega, arrendamento dos
  próprios reais, e outros reditos desta natureza, venham os cômputos ao
  dito Erário até quinze depois do vencimento; e que havendo negligência
  dos tesoureiros, recebedores, almoxarifes, contratadores ou rendeiros,
  retardando as remessas ou entregas, além dos prazos que por este meu
  AIvará lhes são concedidos, se expeçam logo no meu real nome contra
  eles, pelo presidente do Erário, as necessárias ordens de suspensão
  dos lugares, seqüestros, prisões e mais diligências que julgar
  oportunas para a segurança da minha Real Fazenda, e para se fazerem
  pronptas e efectivas as entradas que formarem o objeto de tais
  ordens''. Título 7°, § 9°. ``Os tesoureiros das alfândegas mandarão
  nos primeiros oito dias de cada mês ao Real Erário, ou às tesourarias
  gerais das Juntas, ou das provedorias da minha Fazenda, onde as
  houver, com guia assinada pelo juiz e administrador, e certidão do que
  houverem tido de rendimento às ditas Casas de Arrecadação no mês
  próximo antecedente, todo o recebimento que nele tiveram, assim em
  dinheiro como em bilhetes sobre os assinantes, na parte onde até agora
  se admitiram; e isto debaixo das penas de suspensão seqüestro e
  prisão, pelo simples fato da demora da dita entrada''. Título 7°, §
  10. ``Os recebedores e administradores do subsídio da aguardente da
  terra, do equivalente do contrato do tabaco, dos dízimos do açúcar, do
  subsídio literário, ou de outra qualquer das minhas rendas que tenha
  entrada diária, farão as entregas do seu recebimento mensal na
  tesouraria-mor do Erário nos primeiros dias do mês próximo seguinte,
  na conformidade do que acima fica dito a respeito dos tesoureiros das
  alfândegas e debaixo da mesma cominação''. Título 7°, § 11. ``Os
  tesoureiros, recebedores ou administradores de iguais ou semelhantes
  rendas, assim nas províncias deste Estado, como nas dos meus domínios
  ultramarinos, ficam da mesma sorte obrigados a fazer as entregas dos
  seus recebimentos nos tesouros ou cofres gerais das rendas públicas,
  nos sobreditos prazos, incorrendo nas penas que ficam referidas os que
  o contrário praticarem; concedendo, porém, a espera de 15 dias aos
  recebedores ou administradores que, pelas distâncias das suas
  residências, fizerem as entregas das minhas rendas por quarteis''.
  Art. 88.~``Todas as disposições do Alvará de 28/06/1808, nos títulos
  3º, 4º, 5º, 7º, e 8º, continuam em vigor, fazendo parte desta lei em
  tudo que por ela não fica revogado. Art. 310. Todas as ações ou
  omissões que, sendo criminosas pelas leis anteriores, não são como
  tais consideradas no presente Código, não sujeitarão à pena alguma que
  já não esteja imposta por sentença; que se tenha tornado irrevogável;
  ou de que se não conceda revista''. Art. 2°. ``Em especial observância
  do Tít. 3º, § 2º, e Tít. 7º, §§ 9º, 10º e 11º do referido Alvará, o
  ministro e secretario de estado dos Negócios da Fazenda e presidente
  do Tribunal do Tesouro Público Nacional, na corte, e os inspetores das
  tesourarias nas províncias, podem e devem ordenar a prisão dos
  tesoureiros, recebedores, coletores, almoxarifes, contratadores e
  rendeiros quando forem remissos ou omissos em fazer as entradas dos
  dinheiros a seu cargo nos prazos que pelas leis e regulamentos lhes
  estiverem marcados''.}

No mesmo Decreto n° 657 de 5 de Dezembro de 1849, arts. 3°, 4°, 5° e 6°
estatuiu-se mais:\footnote{Respectivamente, art. 3°. ``Para se efetuarem
  estas prisões nos casos do artigo antecedente, o presidente do Tesouro
  na corte ordenará, e os inspetores das tesourarias nas províncias
  deprecarão por seus ofícios às autoridades judiciárias que as mandem
  fazer por seus oficiais, e lhes remetam as certidões delas''. Art. 4°.
  ``Estas prisões assim ordenadas serão sempre consideradas meramente
  administrativas, destinadas a compelir os tesoureiros, recebedores,
  coletores ou contratadores ao cumprimento de seus deveres, quando
  forem omissos em fazer efetivas as entradas do dinheiro público
  existente em seu poder; e por isso não obrigarão a qualquer
  procedimento judicial ulterior''. Art. 5°. ``Verificadas as prisões, o
  presidente do Tesouro e os inspetores das tesourarias marcarão aos
  presos um prazo razoável para dentro dele efetuarem as entradas do
  dito dinheiro públicos a seu cargo, e dos respectivos juros devidos na
  conformidade do art. 43 da Lei de 28 de Outubro de 1848''. Art. 6°.
  ``Se os tesoureiros, recebedores, coletores e contratadores depois de
  presos não verificarem a entrada do dinheiro público no prazo marcado,
  se presumirá terem extraviado, consumido ou apropriado o mesmo
  dinheiro e, por conseguinte, se lhes mandará formar culpa pelo crime
  de peculato, continuando a prisão no caso de pronúncia e mandando-se
  proceder civilmente contra seus fiadores''.}

Que, para efetuarem-se as prisões nos casos previstos, de \emph{omissão}
ou \emph{remissão}, o presidente do Tribunal do Tesouro, \emph{na
Corte}, ordenará, e os inspetores das tesourarias, nas províncias,
deprecarão\footnote{Responderão à deprecada, isto é, ao ato escrito pelo
  qual um juiz ou tribunal pede a outro, ou autoridade que o valha, que
  cumpra algum mandado ou ordene alguma diligência.}, por seus ofícios,
às autoridades judiciárias, que as mandem fazer, por seus oficiais;

Que \emph{estas prisões assim ordenadas} serão sempre consideradas
meramente administrativas e destinadas a compelir os tesoureiros,
recebedores, coletores, etc., ao cumprimento dos seus deveres,
\emph{quando forem omissos ou remissos em fazer efetivas as entradas dos
dinheiros públicos existentes em seu poder}; e por isso NÃO OBRIGARÃO A
QUALQUER PROCEDIMENTO JUDICIAL ULTERIOR;

Que, verificadas estas prisões, o presidente do Tribunal do Tesouro e os
inspetores das tesourarias marcarão aos presos um prazo razoável para,
dentro dele, efetuarem as entradas dos dinheiros públicos a seu cargo, e
dos respectivos juros devidos na conformidade do art. 43 da Lei de 28 de
Outubro de 1848;\footnote{Art. 43º. ``A dívida ativa proveniente de
  alcances de tesoureiros, coletores, ou outros quaisquer empregados, ou
  pessoas a cujo cargo estejam dinheiro público, será sujeito ao juro
  anual de nove por cento em todo o tempo da indevida detenção''.}

Que se os tesoureiros, recebedores, coletores, etc., depois de presos
não verificarem as entradas de dinheiros públicos no prazo marcado,
\emph{se presumirá terem nos extraviado, consumido, ou apropriado}; e,
por conseguinte, \emph{se lhes mandará formar culpa por crime de
peculato}, continuando a prisão no caso de pronúncia.

É certo, portanto, à vista das disposições deste decreto, que, na
vertente hipótese, o tesoureiro, coletor, etc., só podem ser presos
administrativamente por omissão ou remissão; que omissos ou remissos
serão eles considerados não só quando, com inobservância das leis e
regulamentos, deixarem de fazer as entradas dos dinheiros cujo
recebimento ou guarda lhes caiba, senão quando o presidente do Tribunal
do Tesouro e os inspetores das tesourarias saibam, ou tenham motivos
para crer, que tais funcionários \emph{conservam em si os dinheiros},
\emph{e não os entregam por simples falta}, isenta de culpa; e tanto
assim é que, depois de advertidos, e mesmo presos, dando-se a entrada
dos dinheiros, \emph{não há lugar procedimento algum judicial ulterior};
e quando não realizam as entradas, depois da advertência, e findo o
prazo para isto marcado, \emph{presume-se} a existência de extravio,
consumição e apropriação dos dinheiros. Dá-se, por conseguinte, que
neste caso a prisão administrativa é uma coerção limitada, condicional e
peculiar.

É coerção porque não só compele o funcionário ao cumprimento do dever
preterido, como porque pune a falta cometida; é limitada porque não
alcança toda a extensão do fato, e só se aplica enquanto o alcance não
excede os limites da simples omissão; é condicional porque a sua
existência depende da exibição pecuniária do alcance; e é peculiar
porque dada a omissão, nos limites preventivos, constitui a sanção
exclusiva.

Se, portanto, antes da prisão do funcionário, antes de ele ser
advertido, antes da existência ou conhecimento do alcance
administrativo, o presidente do Tribunal do Tesouro ou os inspetores das
tesourarias souberem da existência de fatos que suscitar possam a
\emph{presunção} de que os dinheiros fossem extraviados, consumidos ou
apropriados, não poderão nem deverão ordenar ou requisitar a prisão de
tais funcionários; porque à vista das disposições citadas do Decreto n°
657 de 5 de Dezembro de 1849, não existe a simples \emph{omissão} ou
\emph{remissão} que constitui o \emph{alcance administrativo}; há o
extravio, há consumição, há apropriação, que constitui{[}em{]} o
\emph{alcance criminal}, ou peculato, nos termos do artigo 170 do Código
Criminal\footnote{Art. 170. ``Apropriar-se o empregado público,
  consumir, extraviar, ou consentir que outrem se aproprie, consuma ou
  extravie, em todo ou em parte, dinheiro ou efeitos públicos, que tiver
  a seu cargo''.}. E o peculato não pode ser processado e julgado pela
autoridade administrativa; é crime de responsabilidade; e, quando
cometido por empregado não privilegiado, corre o respectivo processo
perante os juízes de direito.

A doutrina contrária conduz ao absurdo; anula a disposição da lei; gera
invasão de poderes; viola o direito do cidadão; atenta contra a
segurança individual; torna a prisão em meio ordinário, e indispensável,
para ajuste de contas; leva a anarquia aos tribunais e ao seio da
sociedade.

Se, preso o funcionário, é bastante a \emph{presunção} da existência do
peculato para que a autoridade administrativa demita de si o
reconhecimento do fato e o devolva à autoridade judiciária;
incontestável é que a jurisdição da autoridade administrativa limita-se
ao caso de simples omissão; e não menos óbvio também é, ao menos para os
que entendem que o direito não anda divorciado da lógica, que o
conhecimento prévio, isto é, o conhecimento da existência do peculato,
ou de qualquer outra ocorrência de força maior, que vede ao funcionário
de fazer a entrada de dinheiros, \emph{que não tenha em seu poder},
inabilita necessariamente o presidente do Tribunal do Tesouro e os
inspetores da tesouraria de ordenarem ou requisitarem a prisão dos
funcionários.

Ou isto é uma verdade inconcussa\footnote{Inabalável, incontestável.} ou
a Lei é contraditória, ou o alcance administrativo e o criminal são
idênticos, ou as autoridades administrativas e judiciárias são uma mesma
cousa, ou a prisão administrativa é indispensável para que se dê o
processo judiciário, e neste caso, a autoridade judiciária é um corpo,
que tem por cabeça a administração e o Decreto n° 657 é uma
fantasmagoria!...

A prisão ordenada ou requisitada pelo presidente do Tribunal do Tesouro,
ou pelos inspetores das tesourarias, depois de constar nas respectivas
repartições a existência do peculato, ou de caso de força maior, é um
ato de violência e arbítrio; não é um ato administrativo regular; é uma
exorbitância de atribuições; é um atentado formal que nenhuma relação
tem, nem pode ter, com \emph{fiscalização da receita e despesa pública,
arrecadação, distribuição e contabilidade das rendas}, como se expressa
a lei; é o conhecimento de fatos de ordem diversa, para o que não tem
jurisdição nem competência.

Assim considerada a questão, e demonstrada como fica a improcedência da
prisão administrativa nas circunstâncias em que as discuto, é fora de
dúvida que irregular, arbitrária e violenta foi a prisão do tesoureiro
da alfândega de Santos:

1°: Porque é de notoriedade pública que, de 17 a 19 de fevereiro deste
ano, foi arrombado o cofre daquela repartição, do qual foi subtraído o
dinheiro que falta;

2°: Porque, no dia 19 de fevereiro, foi o tesoureiro suspenso pelo
inspetor da alfândega \emph{em razão do roubo do cofre} até ulterior
deliberação;

3°: Porque esta medida preventiva foi ordenada pelo próprio presidente
do Tribunal do Tesouro;

4°: Porque, conhecida a causa da não entrada do dinheiro, ainda quando
conivente fosse o tesoureiro na subtração, a designação de prazo para
que ele realize tal entrada importaria um ato de comédia;

5°: Porque todo o ato de autoridade é um fato de jurisdição; não há
exercício de jurisdição sem lei que o determine; não se pode dar
exercício de jurisdição sem um fato que o autorize; e a lei não pode ter
aplicação senão relativamente aos fatos por ela previstos;

6°: Porque o fato previsto no Decreto n° 657, cuja disposição é invocada
pelo sr. inspetor da tesouraria, que se proclama portador de uma ordem
ilegal do ministro da Fazenda, é de \emph{omissão}, ou de alcance
administrativo; mas o fato verdadeiro, o fato que está provado e
reconhecido pelo próprio ministro, pelo inspetor da alfândega, e pelo da
tesouraria, é o de subtração, que, quando se pudesse atribuir ao
tesoureiro, constituiria o crime de peculato, para cujo conhecimento não
tem competência a administração;

7°: Porque considerado todo este concurso de circunstâncias, ponderadas
as disposições da lei, e examinadas as atribuições da autoridade
administrativa, resulta que a deprecada, para a prisão, foi ilegal e
arbitrária;

8°: Porque existindo, como existe, a detenção do tesoureiro, verificada
por mandado do dr. juiz municipal de Santos; não se podendo considerá-la
como ato administrativo, em face da lei, torna-se tal detenção um ato
judicial;

9°: Porque não é a requisição em si que determina a natureza do ato;
senão a espécie jurídica ou legal que dá causa à sua existência;

10°: Porque na ausência do fundamento jurídico ou legal, e do objeto
correlativo que o justifique, é a ocorrência julgada de per si, como
fato especial;

11°: Porque, visto quanto fica expendido, e aplicada a doutrina ao caso,
não sendo o alcance administrativo, não podendo, por isso, autorizar a
prisão requisitada, torna-se ela ato próprio singular e exclusivo do
juiz que indebitamente a decretou;

12°: Porque sendo a prisão judiciária e não administrativa, por força da
lei, e da natureza do fato, não podia ser decretada sem ordem do sr.
juiz de direito da comarca, autoridade competente para a formação da
culpa; sem a deposição de duas testemunhas, \emph{que jurassem de
ciência própria}; sem a exibição de prova documental; sem a confissão do
culpado, feita em juízo competente, como prescreve a Lei n° 2.033 de 20
de Setembro de 1871, art. 13, § 2°,\footnote{Art. 13°. "O mandado de
  prisão será passado em duplicata. O executor entregará ao preso, logo
  depois de efetuada a prisão, um dos exemplares do mandado, com
  declaração do dia, hora e lugar em que efetuou a prisão, e exigirá que
  declare no outro havê-lo recebido; recusando-se o preso, lavrar-se-á
  auto assinado por duas testemunhas. Nesse mesmo exemplar do mandado, o
  carcereiro passará recibo da entrega do preso com declaração do dia e
  hora.

  § 2°. À exceção de flagrante delito, a prisão antes da culpa formada
  só pode ter lugar nos crimes inafiançáveis, por mandado escrito do
  juiz competente para a formação da culpa ou à sua requisição; neste
  caso, precederá ao mandado ou à requisição declaração de duas
  testemunhas, que jurem de ciência própria, ou prova documental de que
  resultem veementes indícios contra o culpado ou declaração deste
  confessando o crime".} e o Código do Processo Criminal, artigo
94.\footnote{Art. 94. ``A confissão do réu em juízo competente, sendo
  livre, coincidindo com as circunstâncias do fato, prova o delito; mas,
  no caso de morte, só pode sujeitá-lo à pena imediata quando não haja
  outra prova''.}

Se a lei não pode ser contrariada; se o direito tem uma razão
filosófica; se a lógica não é um contrassenso; se os fatos não foram por
mim falsificados; se a narração não foi por mim feita com preterição da
verdade; e se o caso é, como descrito, fica a votação, a denegação
unânime da ordem de soltura pelo colendo Tribunal da Relação, a
manutenção do arbítrio imposto pelo ministro da Fazenda, requisitado
pelo inspetor da tesouraria, e realizado pelo sr. dr. juiz municipal de
Santos, um ato de injustiça solene, com todas as honras fúnebres de um
saimento\footnote{Enterro.} magno, é um ato de injustiça régia,
imponente, em grosso, o{[}u{]} por atacado.

S. Paulo, 20 de junho de 1877.

LUIZ GAMA.

\textbf{12. TRIBUNAL DA RELAÇÃO -- \emph{HABEAS-CORPUS}}\footnote{In:
  \emph{A Província de S. Paulo} (SP), Seção Livre, 14/08/1877, pp.~1-2.}

\textbf{*didascália*}

\emph{O artigo é simples: resume o caso de maneira objetiva e reitera os
pontos defendidos nos dois textos anteriores. Porém, ele guarda uma
informação a mais, que nos será valiosa para a travessia sobre o caso
Largacha, processo ao qual Gama dedicou longos meses de estudo e
trabalho, tornando-o, sem dúvida, um dos mais importantes em que
advogou. A informação valiosa mora no detalhe. Tanto na epígrafe quanto
na conclusão do artigo, Gama faz referência a um julgamento ocorrido no
Tribunal da Relação da Corte. Trata-se de uma decisão, em sede recursal,
sobre uma ação em que se acusava um empregado público do crime de
peculato. A ementa do acórdão -- posta à guisa de epígrafe -- poderia
ser lida como precedente aplicável ao caso Largacha. Era isso o que Gama
buscava ao citar a ``palavra unânime de quatorze insuspeitos juízes'',
i.e., os juízes da Relação da Corte, que acolheram o argumento do
empregado público, que se defendia da acusação de peculato. ``Neste
acórdão'', reforçava Gama, ``está integralmente mantida a doutrina que
sustentei, que é a consagrada na Lei, a única verdadeira''. É claro que
a equiparação entre um caso e outro partia de uma leitura interessada.
Mas há nela, também, a ideia de uniformização dos julgados, princípio
caro para a organização judiciária, e a força do precedente, que, se não
vinculante, ao menos constitutivo como baliza hermenêutica. Gama não só
ganhava pontos com o público, demonstrando que a doutrina que defendia
era acatada em outras juridições, mas também sinalizava aos
desembargadores do tribunal paulista que o entendimento de seus pares da
Corte era, de fato e de direito, a melhor doutrina ao caso Largacha.
Seja como for, Gama abria frentes e repertórios para sustentar o direito
de seu cliente. }

\begin{center}\rule{0.5\linewidth}{\linethickness}\end{center}

``Não é possível constituir em responsabilidade criminal a empregado
público qualquer, por crime de peculato, sem que previamente preste ele
contas, e seja verificado o seu alcance, com apropriação sua do dinheiro
público''. (Gazeta Jurídica, nº 176, 1º de Agosto de 1877, Ano V, vol.
16, página 370, Apelação nº 428)\footnote{A citação e a referência são
  exatas. O acórdão da Apelação Crime nº 428, do Tribunal da Relação da
  Corte, transcrito na \emph{Gazeta Jurídica}, possui um outro trecho
  que vale destacar, tendo em vista o argumento que Gama vai se dedicar
  a construir nesse e nos demais artigos sobre o famoso roubo da
  Alfândega de Santos. Para os desembargadores, o ``fato criminoso de se
  haver ele {[}autor{]} apropriado de quantias pertencentes à Câmara
  Municipal, era indispensável, para ter lugar a dnúncia, que houvesse a
  referida Câmara tomado contas ao Acusado, e dessas contas resultasse,
  de modo líquido e fora de toda a dúvida, a verificação do alcance do
  Acusado, para então se lhe fazer efetiva a responsabilidade criminal;
  o que não consta dos autos, e antes destes se depreende que tais
  contas não foram tomadas, resultando desta omissão a falta de prova
  concludente e satisfatória do crime imputado ao Acusado, prova essa
  indispensável para basear a justa condenação''. Embora fossem casos
  distintos, Gama tinha esse precedente como parte do seu repertório
  normativo e buscaria, em sua estratégia judicial, patentear a
  deficiência das provas produzidas para aferir a responsabilidade
  criminal do acusado que ele defendia.}

Em as colunas da \emph{Província,} números 704 e 705, de 20 e 21 de
Junho precedente, discuti o fato importantíssimo da denegação de
\emph{habeas-corpus}, por mim requerido, em favor do major Antonio
Eustachio Largacha, tesoureiro da Alfândega de Santos.

Foram objeto da petição por mim endereçada ao colendo Tribunal da
Relação desta província, e da discussão, pela imprensa, a incompetência
da autoridade administrativa para, na hipótese, requisitar a prisão do
tesoureiro; por se não ter verificado alcance administrativo; e a
exorbitância da autoridade judiciária, por cuja conta exclusivamente
corria a ilegal detenção.

Para o primeiro caso apoiei-me nas disposições expressas do Decreto nº
657 de 5 de Dezembro de 1849; e, para o segundo, nas da Lei nº 2.033 de
20 de Setembro de 1871.\footnote{Conforme se lê nos artigos precedentes,
  Gama citou os arts. 2º, 3º, 4º, 5º e 6º do decreto que tratava da
  administração da Fazenda Nacional; e os arts. 13, § 2º, e 18 da lei de
  reforma judiciária de 1871.}

Não era competente a autoridade administrativa porque ela própria, por
atos oficiais seus, provados e incontestáveis, reconhecera que o alcance
arbitrariamente atribuído ao tesoureiro não era administrativo; porque o
alcance tinha por origem a perpetração irrecusável de um crime público;
se o tesoureiro fosse responsável, a responsabilidade seria a prevista
no artigo 170 do Código Criminal;\footnote{Art. 170. ``Apropriar-se o
  empregado público, consumir, extraviar, ou consentir que outrem se
  aproprie, consuma ou extravie, em todo ou em parte, dinheiro ou
  efeitos públicos, que tiver a seu cargo''.} e, em tal caso, o
procedimento caberia à autoridade judiciária; esta, porém, não poderia
decretar a prisão, por faltarem-lhe os elementos indispensáveis, tão
peremptoriamente exigidos na citada Lei nº 2.033 de Setembro de 1871:
prova material do fato, e da sua imputabilidade, por testemunhas ou
documentos irrecusáveis.

Na discussão aludida afirmei que o colendo Tribunal da Relação,
encantoando\footnote{Isolando, afastando.} o direito, dispensando na
lei, e atendendo a meras conveniências governamentais, embora
justificáveis, excepcionalmente, em especialíssimas circunstâncias,
cometera flagrante injustiça.

O respeito que sei guardar, sempre que me dirijo aos provectos\footnote{Experientes.}
representantes da lei, é inquebrantável garantia de que só discuto por
amor da verdade; em defesa do que é justo; contra todo o arbítrio; em
prol da manutenção da liberdade.

Este mesmo interesse generoso; esta mesma defesa imparcial; e este mesmo
respeitoso sentimento, que sempre existiu, ao lado da nativa altivez,
traz-me de novo à imprensa para repetir: a denegação de
\emph{habeas-corpus} ao tesoureiro da Alfândega de Santos foi um ato de
clamorosa injustiça!

Não sou eu quem o declara; é o egrégio Tribunal da Relação da Corte, em
o venerando acórdão\footnote{Decisão de tribunal que serve de paradigma
  para solucionar casos semelhantes.} de 4 de Maio deste ano, pela
palavra unânime de quatorze insuspeitos juízes.\footnote{Cf.
  \emph{Gazeta Jurídica}, nº 176, 1º de Agosto de 1877, Ano V, vol.~16,
  Tribunal da Relação da Corte, Apelação nº 428, pp.~370-371. Vale
  conferir, igualmente, carta pública do apelante Damaso Jacintho de Sá
  Carvalho, o empregado público vitorioso no acórdão da Relação da
  Corte, contando detalhes do caso. Cf. \emph{O Cruzeiro} (RJ),
  Ineditoriaes, Rio Bonito, 20/06/1878, p.~2.}

Neste acórdão, que mereceu os aplausos entusiásticos de um distinto
jurisconsulto do Rio de Janeiro, está integralmente mantida a doutrina
que sustentei, que é a consagrada na Lei, a única verdadeira.

S. Paulo, 13 de Agosto de 1877.

LUIZ GAMA.

\textbf{13. TRIBUNAL DA RELAÇÃO -- PROCESSO DA ALFÂNDEGA DE
SANTOS}\footnote{In: \emph{A Província de S. Paulo} (SP), Seção Livre,
  25/10/1877, p.~2.}

\textbf{*didascália*}

\emph{Após a negação da ordem de soltura, foi a vez do Tribunal da
Relação de São Paulo impor nova derrota a Luiz Gama, agora associado a
mais um advogado, Ribeiro Campos, na causa do tesoureiro da alfândega,
Largacha. Os desembargadores decidiram, dias antes desse artigo, por
pronunciar Largacha pelo crime de peculato. Ou seja, o tesoureiro iria a
julgamento como incurso no art. 170 do Código Criminal. Não restava
muito aos defensores constituídos de Largacha senão persistir na luta
judicial e através da imprensa, convertida em última trincheira de
defesa da dignidade de seu cliente. ``O processo será inteiro estampado
na imprensa'', prometiam os advogados, ``cada cidadão julgará por si
mesmo; os ladrões da Alfândega serão conhecidos através do mistério,
denunciados pela livre consciência do povo, perante a Nação, em peso,
convertida em Tribunal''. Como veremos, a promessa não era vã. Se não
deram a conhecimento público a íntegra do processo -- afinal, estamos
falando em mais de mil páginas! --, Gama e Ribeiro Campos cumpriram o
que diziam e fizeram uma seleção de eventos e documentos fundamentais do
processo, lançando trechos, por longos meses, em diferentes jornais de
diferentes cidades. O que se verá, por um lado, é uma ousada tática
processual que articulava imprensa e juízo, no caso, o Tribunal da
Relação de São Paulo e, por outro lado, uma defesa da dignidade do
direito e do ofício do magistrado. Os advogados não tinham tempo a
perder. Se dirigiam aos juízes com severidade e cobravam um julgamento
justo. O aviso era claro. ``Se as presunções, os indícios e as más
imputações, de per si, constituíssem em provas de crimes, os eméritos
juízes, em nome da própria dignidade, consultando as suas consciências,
dando bravos à calúnia, deveriam dilacerar as togas...'' }

\begin{center}\rule{0.5\linewidth}{\linethickness}\end{center}

Ontem à tarde foi a população desta cidade dolorosamente surpreendida
pela sentença de pronúncia\footnote{Decisão que conclui que há provas de
  materialidade do fato criminoso e indícios suficientes de autoria,
  i.e., identificado o autor, poderá ele responder ao Tribunal do Júri.}
proferida no egrégio Tribunal da Relação, contra os senhores major
Antonio Eustachio Largacha e Antonio Justino de Assis, tesoureiro e
inspetor da Alfândega de Santos.

À noite, geralmente em todos os círculos, era a veneranda sentença o
único objeto de espanto, e a exclusiva causa de todas as
conversações!...

As interrogações sucediam-se invariáveis: ``Pois a Relação
pronunciou?!...''

Nós, os advogados; nós que estudamos a causa; que conhecemos o processo
como as palmas das nossas mãos, não nos vexamos da confissão: fomos
também colhidos de surpresa por o Acórdão\footnote{Decisão de tribunal
  que serve de paradigma para solucionar casos semelhantes.} do colendo
Tribunal!

Quando aceitamos o patrocínio desta causa impusemo-nos um rigoroso
dever: não discutir os fatos pela imprensa antes de julgamento; temos
observado rigorosamente o nosso propósito, a despeito das
contrariedades.

O pleito está em via de julgamento; e a penosa jornada próxima do seu
termo. O processo será inteiro estampado na imprensa; cada cidadão
julgará por si mesmo; os ladrões da Alfândega serão conhecidos ao través
do mistério, denunciados pela livre consciência do povo, perante a
Nação, em peso, convertida em Tribunal.

A inocência dos nossos clientes está escrita nos autos; os fatos são
inalteráveis; as sentenças podem desconhecê-los; apagá-los nunca.

Se as presunções, os indícios e as más imputações, de per si,
constituíssem em provas de crimes, os eméritos juízes, em nome da
própria dignidade, consultando as suas consciências, dando bravos à
calúnia, deveriam dilacerar as togas...

Em identidade de circunstâncias, quem souber manter-se ileso da
corrupção social, zelando os foros da sua posição, diante da lei e da
justiça, será sempre o guarda fiel da honra e da liberdade dos seus
concidadãos. E, quer absolva, quer condene, poderá, como o magistrado
romano exclamar:

-- Folgue ou gema a natureza, sou juiz; hei de cumprir o meu dever.

S. Paulo, 24 de Outubro de 1877.

RIBEIRO CAMPOS\footnote{José Emílio Ribeiro Campos foi jornalista,
  fundador e redator do \emph{Diário de Santos} (1872), promotor público
  e advogado.}.

L. GAMA.

\textbf{14. TRIBUNAL DA RELAÇÃO -- PROCESSO DA ALFÂNDEGA DE SANTOS
{[}II{]}}\footnote{In: \emph{A Província de S. Paulo} (SP), Seção Livre,
  28/10/1877, p.~2.}

\textbf{*didascália*}

\emph{O julgamento de Largacha começaria logo após a sentença de
pronúncia. Gama e Ribeiro Campos dão ao público notícias sobre a
primeira sessão. De saída, alertavam os leitores de um ``grave
procedimento'' que os ``tomou de surpresa'' e que ambos advogados
qualificaram como ``misteriosa ocorrência'' de ``funestas
consequências''. Eles se referiam ao fato de um dos desembargadores, que
estranhamente se teria dado por suspeito, intervir nos debates para, de
caso pensado, comprometer a defesa dos acusados. A maneira como os
advogados destacaram esse comportamento do desembargador Candido da
Rocha logo no início do relato da sessão servia como advertência, ao
público e aos julgadores, de que eles não tolerariam nenhum cerceamento
de defesa. À advertência introdutória, segue-se um rosário de
ilegalidades processuais e materiais. Gama e Ribeiro Campos anotaram e
publicizaram uma série de transgressões praticadas pelo desembargador
relator, Accioli de Brito, o que tinha o efeito de denúncia sobre fatos
pregressos, mas também soava o alarme para evitar atropelamentos legais
nas audiências futuras. Se o artigo documenta ocorridos reprováveis
``digno{[}s{]} dos beócios mandões de aldeia; porém nunca de juristas
respeitáveis'', o texto também ganha tempo, lançando novas informações
sobre o processo, e formando uma opinião pública sobre a inocência do
tesoureiro Largacha. Agora, por exemplo, sabemos que mais de duzentas
testemunhas depuseram no inquérito e/ou no processo crime. O número é,
sobretudo tendo em vista os padrões da época, gigantesco! Além do mais,
sugere a repercussão geral que o caso havia alcançado. Mas Gama e
Ribeiro Campos estavam só no começo. ``Prometemos à face do país e a
promessa há de ser cumprida. O processo inteiro será impresso; o
mistério de iniquidade será desvendado; os culpados hão de ser
conhecidos''. Para Gama -- e provavelmente para Ribeiro Campos --,
promessa era dívida.}

\begin{center}\rule{0.5\linewidth}{\linethickness}\end{center}

Apresentado este notável processo no Tribunal da Relação, processo que,
na autorizada frase do exmo. sr. desembargador Accioli de
Brito\footnote{Luiz Barbosa Accioli de Brito (1825-1900) nasceu no Rio
  de Janeiro (RJ), foi juiz municipal e de órfãos, juiz de direito,
  desembargador e ministro do Supremo Tribunal de Justiça.}, ``é
monstruoso em tudo'', dele foi designado relator o exmo. sr.
desembargador Candido da Rocha\footnote{Antonio Candido da Rocha
  (1821-1882), nascido em Resende (RJ), foi promotor público, juiz
  municipal, juiz de direito, desembargador e político que, à época da
  demissão de Gama do cargo de amanuense da Secretaria de Polícia,
  exercia a presidência da província de São Paulo.}, que jurou
suspeição\footnote{Impedimento legal motivado por interesses ou
  circunstâncias que privariam o juiz de atuar imparcialmente numa
  causa.} na causa...

Ignoramos as razões deste grave procedimento, que certamente devem ser
da mais alta ponderação. Somos, entretanto, forçados a confessar, e tal
é o motivo que nos traz à imprensa, que nos tomou de surpresa o
inexplicável e singularíssimo procedimento de S. Excia. quando, na
memorável sessão de 23 do corrente, ``intervinha de contínuo na
discussão da causa, por meio de apartes, calculadamente comprometedores
da defesa dos acusados!''

Se não medíssemos as funestas consequências desta prática, tão
extraordinária quão admirável, do digno juiz, que bem sabe avaliar o
alcance dos seus atos, e a influência que, na decisão da causa, deveria
exercer a sua palavra autorizada, entre amigos que o estrenecem,
parentes que o prezam, e colegas que o veneram, não nos daríamos ao
trabalho de propalar esta misteriosa ocorrência.

O fato aqui fica mencionado, sem comentários; porque nem precisos são
comentários, quando o caso mais sem eles avulta.

Suspeitado voluntariamente o exmo. sr. desembargador Rocha, foi
designado para substituí-lo o exmo. sr. desembargador Accioli de Brito,
a quem deu-se, de pronto, vista dos autos.

Na sessão do dia 23, este provecto\footnote{Experiente.} juiz, em vez de
``relatar o feito'', como era do seu rigoroso dever, ``rompeu em
tenebrosa acusação'' contra os denunciados, discutindo, com
impertinência, as doutrinas de Mittermaier\footnote{Carl Joseph Anton
  Mittermaier (1787-1867), nascido em Heidelberg, Alemanha, foi um
  jurista, professor de Direito Criminal e político eleito sucessivas
  vezes para cargos legislativos. É considerado um dos juristas mais
  importantes do direito penal do século XIX.}, relativas à prova
indiciária em matéria criminal! E só se deteve na impetuosa torrente
pelas reclamações enérgicas dos exmos. srs. desembargadores
Uchôa\footnote{Ignacio José de Mendonça Uchôa (1920-1910), nascido na
  província de Alagoas, foi promotor público, juiz municipal e de
  órfãos, juiz de direito, desembargador dos tribunais da relação de
  Porto Alegre e de São Paulo, além de procurador da Coroa, Soberania e
  Fazenda Nacional e ministro do Supremo Tribunal de Justiça.} e
Villaça\footnote{Joaquim Pedro Villaça (1817-1897), nascido na província
  de São Paulo, foi promotor público, juiz municipal e de órfãos, juiz
  de direito, desembargador dos tribunais da relação de Ouro Preto e de
  São Paulo, onde também foi presidente do tribunal, além de ministro do
  Supremo Tribunal de Justiça.}, que formalmente exigiram a ``exposição
dos fatos, mediante a leitura e confronto das peças principais do
sumário.''

Cingindo-se\footnote{Restringindo-se, limitando-se.} à observância desta
justa exigência ainda o ilustrado relator não foi fiel ao cumprimento do
seu dever.

Leu, ao seu sabor, a extensa denúncia do Ministério Público; e, para
corroborá-la, ``os dois primeiros autos de exame'', peças constitutivas
do inquérito policial, ``feito em segredo de justiça'', sem conhecimento
nem assistência dos acusados! E dispensou-se de ler os demais exames com
anuência dos seus colegas; porque eram meras (palavras textuais) ``eram
meras reproduções ociosas dos dois primeiros!!!...''

Não foi lida a defesa do major Largacha, nem um só dos muitos e valiosos
documentos por ele exibidos; da defesa do sr. Assis foi apenas lido o
preâmbulo! NÃO FOI LIDA A FORMAÇÃO DA CULPA!! E leu-se ``com irônicas
observações'', a veneranda sentença do honrado sr. dr. juiz de direito
de Santos!!!

Cumpre, porém, observar, e o fazemos muito de caso pensado, para que o
público ilustrado bem possa avaliar do modo fútil porque foi judiciado,
no colendo Tribunal da Relação, este processo importantíssimo.

-- Que os dois primeiros exames, sobre serem incompletos, difusos e
contraditórios no que concerne às emaranhadas respostas dos peritos,
``são imprestáveis e nulos manifestamente''; porque foram realizados no
juízo municipal, incompetente, em face da lei, para a realização de tais
atos;

{[}--{]} Que os exames subsequentes, ``que não foram lidos por
ociosos'', se bem que incompletos e incoerentes, são retificações legais
e necessárias dos dois primeiros; e foram ordenados pela autoridade
policial competente, mandada à cidade de Santos pelo governo, para
regularizar o processo, e restaurar quanto havia sido nulamente feito;

{[}--{]} Que na defesa do major Largacha, mediante documentos
irrecusáveis, foram completamente destruídos os artificiosos indícios de
criminalidade, que porventura pudessem existir contra ele e os outros
acusados;

{[}--{]} Que com estudo e calma, sem prevenções nem ódio, com alguma
prudência e reflexão, se o exmo. sr. relator quisesse, acharia NO
PROCESSO O FIO DE ARIADNE, por o qual chegaria até aos felicíssimos
autores do roubo da Alfândega;

{[}--{]} Que agarrarem um tesoureiro pelo fato de ocupar ele o cargo de
clavicurário\footnote{Aquele que é responsável pela guarda de chaves.}
de um cofre, diante de uma subtração ousada, astuciosa, rodeada
misteriosamente de ``calculadas simulações, em um processo monstruoso'',
que conta mais de 200 testemunhas (!!), verdadeiro conjunto de
peripécias extraordinárias, que, de contínuo estão desafiando a
inteligência, o tino, e a perspicácia de amestrado\footnote{Treinado,
  habilitado.} observador, pronunciá-lo alvarmente\footnote{Ingenuamente
  ou de modo infundado.}, à guisa do grosseiro pescador, que tudo colhe
quanto lhe cai na rede, será digno dos beócios\footnote{Aqui no sentido
  de incultos, ignorantes.} mandões de aldeia; porém nunca de juristas
respeitáveis, de magistrados eminentes, que administram justiça na
capital da heroica província de São Paulo.

Prometemos à face do país e a promessa há de ser cumprida.

O processo inteiro será impresso; o mistério de iniquidade será
desvendado; os culpados hão de ser conhecidos.

Então os exmos. srs. desembargadores suspeito, e prevenido relator,
conhecerão que, involuntariamente, por fatal obediência à desastradas
impressões, faltaram aos ditames da justiça, para servir à causa do
delito.

RIBEIRO CAMPOS\footnote{José Emílio Ribeiro Campos foi jornalista,
  fundador e redator do \emph{Diário de Santos} (1872), promotor público
  e advogado.}.

L. GAMA.

\textbf{15. TRIBUNAL DA RELAÇÃO -- PROCESSO DA ALFÂNDEGA DE SANTOS
{[}III{]}}\footnote{In: \emph{A Província de S. Paulo} (SP), Seção
  Livre, 08/11/1877, p.~3.}

\textbf{*didascália*}

\emph{Gama e Ribeiro Campos analisam parte do fundamento do acórdão de
19/10/1877, especialmente quanto à punição do tesoureiro Largacha. De
modo bastante didático, os advogados expõem uma contradição manifesta
entre a pena cominada e o seu cumprimento. Se era verdade que o
tesoureiro Largacha respondia pelo crime de peculato, art. 170 do Código
Criminal, cujo grau máximo da pena era de quatro anos de prisão com
trabalho, ele poderia requerer e obter soltura mediante o pagamento de
fiança. Afinal, o crime de peculato era um crime, nos termos da lei,
afiançável. Gama e Ribeiro Campos solicitaram tal ``direito
incontestável, garantido pela lei'', em benefício de Largacha, mas o
Tribunal da Relação de São Paulo, ``por decisão unânime, resolveu que o
crime não admite fiança porque a prisão foi decretada por efeito da
pronúncia, mas de acordo com o decreto de 1849''. Gama e Ribeiro Campos
se quedaram incrédulos. Tal admissão, num acórdão de tribunal,
significava, no mínimo, uma leitura arbitrária e caprichosa da lei. A
conclusão dos advogados foi que os desembargadores expressamente
substituíram um artigo do Código Criminal por outro artigo, só que de um
decreto do Poder Executivo! A razão maior, contudo, era extra-legal.
Valia tudo para deixar Largacha trancafiado. ``Agora apenas esperamos um
fato'', arrematavam os advogados com a peculiar combinação de sarcasmo e
fúria que Gama dominava tão bem, ``depois da negação da fiança, que seja
consequentemente negada a existência do acusado!!!...''}

\begin{center}\rule{0.5\linewidth}{\linethickness}\end{center}

O colendo Tribunal da Relação, por acórdão\footnote{Decisão de tribunal
  que serve de paradigma para solucionar casos semelhantes.} de 19 do
mês precedente, que brevemente analisaremos, pronunciou o major Antonio
Eustachio Largacha como incurso no art. 170 do Código Criminal\footnote{Art.
  170. ``Apropriar-se o empregado público, consumir, extraviar, ou
  consentir que outrem se aproprie, consuma ou extravie, em todo ou em
  parte, dinheiro ou efeitos públicos, que tiver a seu cargo''.}; e o
sujeitou à prisão e livramento, de conformidade com o Decreto nº 657, de
5 de Dezembro de 1849!

O sr. major Largacha é tesoureiro da repartição da Alfândega e, nos
expressos termos dos arts. 2º, 3º, 4º, 5º e 6º do mencionado decreto,
foi, depois de administrativamente preso, \emph{entregue ao Poder
Judiciário}, e processado pelo crime de peculato.\footnote{Respectivamente,
  art. 2°. ``Em especial observância do Tít. 3º, § 2º, e Tít. 7º, §§ 9º,
  10º e 11º do referido Alvará {[}28/06/1808{]}, o ministro e secretario
  de estado dos Negócios da Fazenda e presidente do Tribunal do Tesouro
  Público Nacional, na corte, e os inspetores das tesourarias nas
  províncias, podem e devem ordenar a prisão dos tesoureiros,
  recebedores, coletores, almoxarifes, contratadores e rendeiros quando
  forem remissos ou omissos em fazer as entradas dos dinheiros a seu
  cargo nos prazos que pelas leis e regulamentos lhes estiverem
  marcados''. Art. 3°. ``Para se efetuarem estas prisões nos casos do
  artigo antecedente, o presidente do Tesouro na corte ordenará, e os
  inspetores das tesourarias nas províncias deprecarão por seus ofícios
  às autoridades judiciárias que as mandem fazer por seus oficiais, e
  lhes remetam as certidões delas''. Art. 4°. ``Estas prisões assim
  ordenadas serão sempre consideradas meramente administrativas,
  destinadas a compelir os tesoureiros, recebedores, coletores ou
  contratadores ao cumprimento de seus deveres, quando forem omissos em
  fazer efetivas as entradas do dinheiro público existente em seu poder;
  e por isso não obrigarão a qualquer procedimento judicial ulterior''.
  Art. 5°. ``Verificadas as prisões, o presidente do Tesouro e os
  inspetores das tesourarias marcarão aos presos um prazo razoável para
  dentro dele efetuarem as entradas do dito dinheiro públicos a seu
  cargo, e dos respectivos juros devidos na conformidade do art. 43 da
  Lei de 28 de Outubro de 1848''. Art. 6°. ``Se os tesoureiros,
  recebedores, coletores e contratadores depois de presos não
  verificarem a entrada do dinheiro público no prazo marcado, se
  presumirá terem extraviado, consumido ou apropriado o mesmo dinheiro
  e, por conseguinte, se lhes mandará formar culpa pelo crime de
  peculato, continuando a prisão no caso de pronúncia e mandando-se
  proceder civilmente contra seus fiadores''.}

O art. 170 do Código Criminal, em que se esteia a sentença de pronúncia,
na parte penal, reza assim:

``No grau máximo -- perda do emprego, quatro anos de prisão com
trabalho, e multa {[}em{]} 20 por cento da quantia ou valor dos efeitos
apropriados, consumidos ou extraviados.''

O Código do Processo Criminal, no art. 101, assim dispõe:

``A fiança não terá lugar nos crimes cujo máximo de pena for: morte
natural; galés; 6 anos de prisão com trabalho; 8 anos de prisão simples;
20 anos de degredo.''\footnote{À parte o uso de numerais, a citação é
  exata.}

O Decreto nº 657 de 5 de Dezembro de 1849, promulgado pelo Poder
Executivo, trata exclusivamente de matéria especial, de competência
administrativa; competência esta, que, com aplicação ao caso vertente,
cessou inteiramente, desde que o detido foi entregue à autoridade
judiciária, e por esta processado: assim determina o art. 6º do citado
decreto.

À vista do exposto é fora de dúvida que o crime do art. 170 do Código
Criminal é afiançável.

O major Largacha, certo do seu direito incontestável, garantido pela
lei, requereu fiança ao colendo Tribunal da Relação, e este, por decisão
unânime, \emph{resolveu que o crime não admite fiança porque a prisão
foi decretada por efeito da pronúncia, mas de acordo com o decreto de
1849!!!}

Conclusão lógica, irrecusável e necessária que decorre do acórdão da
data de ontem:

O DECRETO DE 1849, EXPEDIDO PELO PODER EXECUTIVO REVOGOU O ARTIGO 101 DO
CÓDIGO DO PROCESSO CRIMINAL!!!

Diante deste fato confessamos, de bom grado, a nossa inópia\footnote{Ignorância.
  O estilo da ironia, por sua vez, reforça a ideia de que Gama seja o
  autor dessas linhas.}: não entendemos a sublime doutrina do venerando
acórdão!

Agora apenas esperamos um fato: depois da negação da fiança, que seja
consequentemente negada a existência do acusado!!!...

S. Paulo, 7 de Novembro de 1877.

Os advogados,

DR. RIBEIRO CAMPOS.\footnote{José Emílio Ribeiro Campos foi jornalista,
  fundador e redator do \emph{Diário de Santos} (1872), promotor público
  e advogado.}

L. GAMA.

\textbf{16. EGRÉGIO TRIBUNAL DA RELAÇÃO} -- \textbf{Processo da
Alfândega de Santos}\footnote{In: \emph{A Província de S. Paulo} (SP),
  Suplemento ao n° 829, 18/11/1877. A redação do \emph{Correio
  Paulistano}, em 29/11/1877, informa e agradece ter recebido um
  ``folheto'' ``contendo a exposição do processo da Alfândega de Santos,
  pelos advogados, o sr. dr. Ribeiro Campos e Luiz Gama.'' Podemos
  compreender que esse ``folheto'' possuísse características
  tipográficas e critérios editoriais de um livreto comum. In:
  \emph{Correio Paulistano} (SP), Noticiário Geral, 29/11/1877, p.~2.
  Isso posto, é possível que além do formato de Suplemento ao número do
  jornal, esse estudo jurídico tenha circulado como folheto e/ou
  livreto. Em assim sendo, hipótese que exploro em minha tese de
  doutorado, a publicação do folheto pode ser compreendida como uma
  espécie de publicação de livro jurídico, ou, na terminologia mais
  acurada, um livro normativo-pragmático.}

\textbf{*didascália*}

\emph{Esta é a mais extensa obra jurídica escrita por Luiz Gama. Ao
melhor estilo de sua literatura normativo-pragmática, Gama discute
hipóteses, refuta indícios frágeis, coteja testemunhos, coleciona provas
documentais, identifica o erro material de uma dada perícia, vocifera
contra o vício processual, produzindo, enfim, conhecimento normativo
para estabelecer o direito na jurisdição onde o litígio se trava. Gama
contou com a colaboração do também advogado Ribeiro Campos --
possivelmente para coligir documentos e anotar informações do foro de
primeira instância -- para lançar este que é o histórico Suplemento nº
829 d'}A Província de S. Paulo\emph{. Publicado como encarte do jornal e
ao que parece também como livreto avulso, o} ``Egrégio Tribunal da
Relação -- Processo da Alfândega de Santos'' \emph{foi um divisor de
águas na tumultuada causa em que o tesoureiro da instituição era acusado
de roubar milhares de contos de réis -- que, por sinal, ele teria o
dever legal de zelar e guardar. Até esse livreto, Gama e Ribeiro Campos
acumulavam derrotas nos tribunais e o tesoureiro Largacha mofava
desesperançado na cadeia de Santos. A publicação alcançou tamanha
repercussão -- numa causa, aliás, que ganhava os holofotes em diversos
jornais, de norte a sul do país --, que mudou os rumos do processo e
impactou nos julgamentos -- de mérito e recursais -- que ainda restavam
a ser proferidos. Para tal resultado, sem dúvida que o esforço
gigantesco dos seus autores foi um dos ingredientes-chave. Mas não só o
esforço. A estratégia também foi estudada em detalhes. Gama e Ribeiro
Campos esmiuçam os fundamentos do acórdão de 19/10/1877, do Tribunal da
Relação de São Paulo, e tratam, por variados canais, de demonstrar que
os desembargadores foram levados a erro pelas más informações prestadas
pelas autoridades policiais e judiciárias subalternas, principalmente
pela colheita de indícios e provas de materialidade e autoria oriundas
de uma condução enviesada e viciada do inquérito e demais diligências
preliminares e preparatórias. Alguns dos desembargadores devem ter
ficado incrédulos com o que leram. Eram eles, cabe dizer, o público-alvo
da publicação. Havia tempo para reverter a injustiça contra Largacha,
certamente calculavam seus advogados, de modo que as baterias deveriam
se voltar uma vez mais aos desembargadores. Contudo, outra ação restava
pendente de julgamento, no juízo de direito de Santos. Assim, o juiz
competente para tal feito seria também um entre os leitores desejáveis
que Gama e Ribeiro Campos teriam em mente com o histórico Suplemento nº
829. É evidente, também, que o Suplemento visava o grande público,
afinal, a estratégia de pautar a repercussão geral do caso contiuava em
vigor. ``Vamos estampar todas as peças que serviram de base ao venerando
Acórdão de 19 de Outubro'', anunciavam os autores, ``e outras que, pelos
julgadores, foram desconsideradas, mediante as quais mostraremos, com
evidência, que o mencionado Acórdão carece de fundamento quanto aos
fatos; não se apoia em princípios jurídicos; e é contrário
manifestamente aos preceitos da lei''. Daí em diante, o que se vê é uma
sofisticada desconstrução do processo sumário, que os advogados
reputavam enviesado e viciado e que, por tal e qual, havia metido
Largacha no cárcere. Mas só a descontrução da narrativa policial e
judiciária não bastaria para mover os inamovíveis desembargadores. Era
preciso reconstituir a cena do crime: recolher objetos, colher indícios,
ouvir testemunhos -- esparsos nos autos ou dispersos à boca miúda --,
especular com a cabeça dos criminosos, enfim, reconstruir o caso e
propor uma versão dos fatos verossímel e juridicamente consistente. O
que se lê, então, é uma aula de direito. Uma aula de como pensar o
direito e como agir pelo direito. }

\begin{center}\rule{0.5\linewidth}{\linethickness}\end{center}

\emph{O Acórdão}\footnote{Decisão de tribunal que serve de paradigma
  para solucionar casos semelhantes.} \emph{de 19 de Outubro e as provas
do sumário.}\footnote{Processo rápido e simplificado com poucas
  formalidades para acusação e contestações iniciais.}

Muitas vezes o crime é

uma obliteração\footnote{Múltiplas possibilidades para o contexto:
  obstrução, eliminação, destruição.} do direito,

um capricho da Lei,

\emph{ou um erro fatal dos julgadores}.

\begin{center}\rule{0.5\linewidth}{\linethickness}\end{center}

Entregar ao leitor judicioso\footnote{Sensato, ponderado.} estas cópias,
coligidas\footnote{Reunidas.} com a mais escrupulosa fidelidade,
escoimadas\footnote{Desembaraçadas, livres.} das evoluções caóticas do
sumário, monstruoso em tudo, como afirmou o exmo. sr. desembargador
Accioli de Brito\footnote{Luiz Barbosa Accioli de Brito (1825-1900)
  nasceu no Rio de Janeiro (RJ), foi juiz municipal e de órfãos, juiz de
  direito, desembargador e ministro do Supremo Tribunal de Justiça.}, e
até no difuso relatório de S. Excia., e no espantoso julgamento
proferido no colendo Tribunal da Relação (segundo a nossa
humilíssima\footnote{Muito humilde.} opinião), coordenadas, com método,
em séries distintas, para que possam ser facilmente apreciadas, deveria
ser o nosso trabalho único.

Assim, porém, não procederemos. Somos advogados; perante esta célebre
causa, para com os nossos constituintes, e em face do país
inteiro\footnote{De fato, o crime da Alfândega de Santos atingiu grande
  repercussão, sendo noticiado em dezenas de periódicos de diversas
  localidades do país. Cf., por exemplo, \emph{Diário de Pernambuco
  (PE),} 10/03/1877, \emph{Notícias do sul do Império,} p.~1;
  \emph{Correio da Bahia (BA),} 28/06/1877, \emph{Revista das
  Províncias,} p.~1; \emph{Gazeta de Joinville (SC}), 13/11/1877,
  \emph{Província de S. Paulo,} p.~2.}, assumimos gravíssima
responsabilidade; metemos ombros\footnote{Atiramo-nos ao trabalho,
  trabalhamos com afinco.} à mais árdua das empresas, contraímos um
duplo dever: provar a inocência dos acusados neste pleito, e mostrar que
os dignos magistrados -- o preparador do inquérito e os membros do
egrégio Tribunal da Relação --, se às informações secretas da repartição
de fazenda, mais inspiradas pela vaidade do que pela experiência e pela
perspicácia, cerrassem os ouvidos\footnote{Argumenta, nas entrelinhas,
  que o processo não foi julgado pelos autos, mas sim sorrateiramente,
  fora dos autos, fora do direito.}; se apartassem as vistas dos
\emph{quadros aparentes}, adrede\footnote{Premeditadamente.} preparados
pelos atiladíssimos\footnote{Inteligentíssimos, muito sagazes.}
criminosos; se as fitassem, não no que \emph{parecia}, mas no que
\emph{deveria ser}, teriam posto a espada da justiça sobre os
verdadeiros culpados.

Vamos estampar todas as peças que serviram de base ao venerando Acórdão
de 19 de Outubro, e outras que, pelos julgadores, foram desconsideradas,
mediante as quais mostraremos, com evidência, que o mencionado Acórdão
carece de fundamento quanto aos fatos; não se apoia em princípios
jurídicos; e é contrário manifestamente aos preceitos da lei.

A causa da pronúncia\footnote{Despacho pelo qual o juiz declara que
  alguém está indiciado como autor ou cúmplice de um crime.} do
tesoureiro Largacha\footnote{O nome aparece assim mesmo, com essa
  grafia.} e do inspetor Assis\footnote{Em algumas passagens, o nome do
  inspetor -- Assis -- aparece grafado com ``z'' no final -- Assiz.
  Optei por padronizar conforme a primeira aparição para facilitar a
  leitura.} não assenta, por certo, nas ridículas banalidades
autoritariamente consignadas no sumário, que não encerram sequer
indícios remotos; mas nas informações secretas que hão sido habilmente
sopradas por um alto funcionário; na importância e na celebridade da
causa; na ingênua vaidade dos julgadores, que se prevalecem da
oportunidade para dar prova da sua elevação de caráter; e... na
lamentável fatalidade que persegue a uns para salvação de outros.

...........................................................................................\ldots{}.

AUTO DE CORPO DE DELITO

(PERANTE O JUÍZO MUNICIPAL)

A 19 de Fevereiro de 1877

\emph{Exame do cofre}

Peritos:

Antonio Paes da Costa \}

Antonio Clemente da Fonseca \} serralheiros

\emph{Exame do edifício:}

Thomaz Antonio de Azevedo\} mestres de

Nicolau Ignacio da Silveira \} obras

\emph{Testemunhas:}

Antonio Moreira Sampaio

Dr.~Luiz Manoel de Albuquerque Galvão,

(\emph{Engenheiro arquiteto)}

PRIMEIRA SÉRIE DE EXAMES

PARTE EXTERNA DO EDIFÍCIO

\emph{Quesitos do Juiz:}

1° Se há vestígios de violência às cousas ou objetos;

2° Quais eles sejam;

3° Se por essa violência foi vencido ou podia vencer-se o obstáculo que
existisse;

4° Se havia obstáculo;

5° Se se empregou força, instrumento ou aparelho para vencê-lo;

6° Qual foi essa força, instrumento ou aparelho;

7° Se encontraram vestígios de escalada;

8° Se esta podia dar-se independente de deixar vestígios.

PARTE INTERNA DO EDIFÍCIO

\emph{(Primeiro exame)}

Quesitos do Juiz:

1° Se há vestígios de violência às cousas ou objetos;

2° Quais eles sejam;

3° Se por essa violência foi vencido ou podia vencer-se o obstáculo que
existisse;

4° Se havia obstáculo;

5° Se empregou-se força, instrumento ou aparelho para vencer-se;

6° Qual foi essa força, instrumento ou aparelho;

7° Se do lugar em que encontraram as violências até chegar ao em que se
acha o cofre forte havia algum embaraço a vencer;

8° Qual ele seja;

9° E por qual meio foi vencido.

SEGURANÇA DO EDIFÍCIO

\emph{Segundo exame}

\emph{Quesitos do Juiz:}

1° Qual a segurança do edifício em geral, e em particular das peças
adjacentes e do salão em que se acha o cofre forte;

2° Quantas são as portas que dão entrada para o salão do cofre forte e
qual o estado delas;

3° Se denota ter conhecimento do edifício da Alfândega, e
particularmente da situação do cofre, quem penetrou no edifício;

4° Qual o valor do dano causado.

SEGUNDA SÉRIE DE EXAMES

COFRE DA ALFÂNDEGA

\emph{Quesitos do Juiz:}

1° Se há vestígios de violência às cousas ou objetos;

2° Quais eles sejam;

3° Se por essa violência foi vencido ou podia vencer-se o obstáculo que
existisse;

4° Se havia obstáculo;

5° Se empregou-se força, instrumento ou aparelho para vencê-lo;

6° Qual foi essa força, instrumento ou aparelho;

7° Qual o estado interior e os objetos nele encontrados.

\emph{Quesitos do dr. Promotor Público:}

1° Se a fechadura pequena, por onde a chave faz mover as peças que
impelem as linguetas\footnote{Peças da fechadura que, quando movidas
  pela chave, servem para trancar a porta.} grandes da porta do cofre,
estava em estado perfeito ou se oferecia sinais de violência; e, no caso
afirmativo, quais são eles;

2° Se o canhão\footnote{Peça cilíndrica, oca, na entrada de alguns tipos
  de fechadura.} da fechadura referida oferecia sinais de violência em
todo o seu comprimento, ou se somente no orifício exterior; e, no caso
afirmativo, em que consistem essas violências;

3° Se no comprimento do canhão havia alguma fenda, por onde pudesse cair
para dentro da fechadura algum fragmento ou pedaço de prego, ou
gazua\footnote{Instrumento de ferro, curvo, ou gancho de arame, com o
  qual se abrem fechaduras na falta de chave apropriada.}, que fosse
introduzido no mesmo canhão;

4° Se na fenda da fechadura, que está por baixo do canhão, por onde
entra o corpo da chave, oferecia sinais de violência;

5° Se os papéis e estampilhas\footnote{Selo postal ou de documento
  oficial.} encontrados fora do cofre tinham em si marcas de azeite.

\_\_\_\_\_\_\_\_\_\_\_\_\_\_\_\_\_\_\_\_\_\_\_\_\_\_\_

\emph{Descrição dos peritos concernente ao interior do edifício}

"Passando em revista todo o edifício, cujas portas da frente, fundo e
lados já tinham sido abertas pelos empregados, verificaram que nenhuma
violência fora nele praticada.

"Subindo então ao andaime da obra nova, que é próxima ao edifício da
Alfândega, quer de um quer de outro lado, reconheceram que no respalgo
(sic) da parede também não existiam indícios de violência.

"Examinando de cima do respalgo do telhado verificaram mais que não
havia destruição alguma, como telhas quebradas ou estragos no
emboço\footnote{Reboco.}; existindo, porém, perto do rincão\footnote{Peça
  de madeira que, na armação do telhado, ocupa a posição de aresta deste
  ângulo, e pela qual se interceptam as águas mestras do telhado.} do
ângulo externo dos fundos algumas telhas removidas, que reconheceram ser
em uma gateira\footnote{Fresta aberta para dar luz e arejar o vão entre
  o forro e o telhado.}.

"E como entendem que foram elas arredadas pela parte de dentro,
reservam-se para descreverem mais minuciosamente este fato no segundo
exame, e por isso respondem da maneira seguinte:

"Ao 1° quesito: Sim (há vestígios de violência às cousas ou objetos);

"Ao 2° quesito: Que consiste na remoção das telhas da gateira;

"Ao 3° quesito: Sim (por essa violência foi vencido ou podia vencer-se o
obstáculo existente);

"Ao 4° quesito: Sim (havia obstáculo);

"Ao 5° e 6° quesitos responderão no seguinte exame: (se empregou-se
força ou instrumento ou aparelho para vencer o obstáculo; e quais sejam
essa força, instrumento ou aparelho);

"Ao 7° quesito: Não (não encontraram vestígios de escalada);

"Ao 8° quesito: Sim, sendo praticado por indivíduo descalço, e que
andasse somente por cima dos espigões\footnote{Ângulos formados pelas
  intersecções de peças do telhado.} e cumeeira\footnote{Parte mais alta
  do telhado e também nome da peça-chave onde se apoiam caibros e ripas
  de madeira.}. (A escalada podia dar-se independente de deixar
vestígios).

\_\_\_\_\_\_\_\_\_\_\_\_\_\_\_\_\_\_\_\_\_\_

Passando, em seguida, ao exame do interior do edifício, e subindo por
uma pequena escada que, partindo de um dos corredores contíguos ao salão
do cofre-forte, conduz ao forro, na qual escadinha há uma porta
imprestável, desde longo tempo, por ter um dos gonzos\footnote{Tipo de
  dobradiça de duas peças articuladas.} quebrados, foram ter à gateira
referida. Aí encontraram, num paredão, sobre o qual se assenta uma das
vigas, justamente debaixo da gateira, uma lima de três quinas, com a
ponta partida de fresco; uma verruma\footnote{Broca, instrumento
  metálico que tem a sua extremidade inferior aberta em espiral e
  terminada em ponta, usada para abrir furos em madeira.}, um formão
pequeno, sem cabo, e sem sinal algum de haver servido, e mais um pequeno
instrumento em forma de chave, de arame de ferro, com uma das pontas
enleada\footnote{Atadas, amarradas.} de arame de cobre, mais fino.

"Examinando a gateira, verificaram ser ela formada pelo vão de duas
ripas que pareciam ter sido serradas no tempo em que se ripou\footnote{Remete
  ao processo de erguer ripas de madeira para estruturar a construção.}
a casa; e, mais, que estavam corridas para o lado de baixo as telhas que
a cobriam, as quais, na sua opinião, foram removidas por pessoa que se
achava da parte de dentro; voltando depois de verificarem que sobre o
forro não havia rastro de pessoa descalça ou calçada, e que nenhuma
violência existia no compartimento superior, dirigiram-se pelo
mencionado corredor à porta que dali abre-se para o salão do
cofre-forte. Esta porta, que tem uma bandeira de vidro\footnote{Espécie
  de painel de vidro numa porta de madeira.} e um metro e 30 centímetros
de largura, compõe-se de duas meias-folhas,\footnote{Peça de madeira que
  compõe a feitura de uma porta.} havendo em uma delas, à esquerda, dois
fechos pedreses\footnote{Espécie de peça de ferro utilizada para manter
  uma ou mais portas fechadas.}, um do lado superior e outro embaixo, o
qual firma-se no buraco de um tijolo, pois que é coberto, com ladrilho,
o chão nessa parte do edifício. Na outra meia-porta\footnote{Cada uma
  das duas folhas de uma porta, que se fecham sobre um painel central.}
há uma fechadura velha e ruim. No chão encontraram um pequeno formão,
sem cabo. A porta achava-se no estado seguinte: a meia-folha da esquerda
com o fecho de cima arriado, e o de baixo também descido, na posição de
fechar; e, estando fora do lugar o pedaço de tijolo em que devia
apoiar-se, achava-se preso ao tijolo seguinte, isto é, ao que se unia ao
pedaço removido. A outra meia-folha, em consequência de ter sido forçada
a da esquerda, e de ter por seu turno provavelmente sofrido força,
estava aberta, tendo na fechadura, cuja lingueta estava saída, a chave
pelo lado de dentro. Procedendo a uma experiência na mesma porta, para
verem se tendo cedido a folha da esquerda quanto cedeu, podia ter-se
escapado a lingueta da chapa, verificaram ser isso possível. Notaram que
a lingueta da fechadura entrava e saía sem auxílio da chave, o que
denotava estar sem mola; na porta, na altura do trinco superior e
inferior, e na soleira do tijolo, não encontraram vestígio algum de
violência.

"Procederam, também, à experiência seguinte: fecharam as portas com os
fechos e com as chaves e, do lado do corredor, conseguiram, forcejando
sobre a parte superior da folha direita, introduzindo pela fenda um
pedaço de ripa, fazer descer o fecho, deixando, porém,
escalavrada\footnote{Arranhada, danificada.} a tinta nessa operação; não
podendo, entretanto, remover o pedaço de tijolo que haviam colocado em
seu lugar, e cuja saída permitia a abertura de outra folha. Penetrando
no salão, encontraram em frente a duas mesas um instrumento de arame,
semelhante ao já descrito; e, mais acima, adiante do biombo\footnote{Divisor
  de espaços.} de balaústres\footnote{. Balaustrada, série de colunas
  que dividem o ambiente.}, onde está o cofre, um outro também de arame,
tendo {[}em{]} uma das suas extremidades cinco círculos de fio de cobre
em forma de uma flor, sujo de azeite ou óleo de amêndoas, que se
impregnara igualmente no assoalho. O salão compõe-se de duas peças
contínuas, divididas por um arco, sendo uma delas onde há 4 mesas de
trabalho mais baixa{[}s{]}, e guarnecida de três janelas, e mais uma
porta, que já foi descrita.

"A outra, cujo assoalho é mais alto, tem 4 biombos balaustrados, num dos
quais acha-se o cofre-forte, e é guarnecida de 5 janelas de lado e 3 na
frente, estando duas das 5 dentro do biombo do cofre.

"Do lado oposto ao da janela há uma porta sem folhas, que deita para a
saleta, de onde desce uma escada, que tem no topo uma porta forte com
fechadura de grande ferrolho. Nem um indício de violência encontraram no
salão descrito, cujas janelas estavam fechadas, à exceção de uma das que
dão para o biombo do cofre, a qual tinha uma das folhas aberta. Esta
janela é mais alta e menor que as outras; acha-se mais próxima do cofre.
No restante do edifício também não foi encontrado indício algum de
violência.

"Portanto, respondem aos quesitos do modo seguinte:

"Ao 1° quesito: Sim (há vestígios de violência às cousas e objetos);

"Ao 2° quesito: Sim; e consistem no arrombamento do telhado e da porta;

"Ao 3° quesito: Sim (que por essa violência foi vencido e podia
vencer-se o obstáculo existente);

"Ao 4° quesito: Sim (que havia obstáculo);

"Ao 5° quesito: Sim, quanto à porta (empregou-se força, instrumento e
aparelho para vencê-lo);

"Ao 6° quesito: Que a força foi a pressão exercida sobre a porta, não
podendo precisar quais os instrumentos ou aparelhos;

"Ao 7° quesito: Sim (que do lugar em que encontraram as violências até
chegar ao em que se acha o cofre-forte havia embaraço a vencer);

"Ao 8° quesito: Partindo da gateira à porta, no lado, e desta ao cofre,
o biombo, em cuja porta havia fechadura com chave e trinco;

``Ao 9° quesito: Pela força e instrumentos desconhecidos e o emprego da
chave do biombo.''

\_\_\_\_\_\_\_\_\_\_\_\_\_\_\_\_\_\_\_\_

EXAME DO COFRE

\emph{(2ª Série)}

Entrando no biombo, encontraram esparsos pelo chão maços de notas
miúdas, e outras não amassadas, também miúdas; folhas de estampilhas de
diversos valores, umas dilaceradas e outras em perfeito estado; um vidro
de óleo de amêndoas, sem rolha, com uma pena de galinha dentro; uma
caixa de folha vazia aberta; um alicate; vários instrumentos de arame
semelhantes aos já descritos; dois ferros curvos, em forma de gazua;
dois espelhos\footnote{. Chapas que circundam a fechadura.} do cofre,
untados de óleo; papéis diversos no chão e sobre a mesa; alguns pingos
de sebo; no soalho\footnote{. O mesmo que assoalho.} muitas manchas de
óleo; o cofre-forte estava aberto, apresentando o seu exterior o estado
seguinte:

-- O espelho da fechadura e do fingimento\footnote{. No sentido de peça
  de madeira vulgar.} tinham sido arrancados, achando-se os seus
parafusos, uns espalhados pelo chão e outros ainda presos aos mesmos
espelhos; no lugar destes, e nos trincos, fora posta grande quantidade
de óleo, tanto que passou de um lado para o outro da porta.

"Examinando o cofre, verificaram ter ele de altura um metro e 78 a 90
centímetros de largura, no ventre; de fundura, 75 centímetros; duas
linguetas do trinco na folha esquerda e quatro ditas na direita, todas
com 6,5 centímetros de largura e 2 de grossura. As chapas de ferro, que
formam o quadro, tem 2 centímetros de espessura; e as das duas folhas da
porta, 1,5. Dentro existem 3 compartimentos superiores; 2 armários e
três gavetas, tudo de ferro; \emph{Nas faces internas das portas nenhum
indício havia de violência}, e somente algumas nódoas\footnote{.
  Manchas.} de óleo se viam em torno dos orifícios existentes nos
lugares correspondentes à fechadura e trinco.

"O interior continha, na parte inferior, papéis em desordem; uma caixa
de folha, tendo em cima um livro, e papéis arrumados; perto, um maço de
notas do Tesouro, atado com um barbante, com o letreiro (1:000\$000
réis); e um outro -- mil réis -- fechado por parêntesis. No armário da
esquerda havia diversos maços de notas iguais ao primeiro, e menores; e
no da direita, nada existia. As três gavetas estavam vazias e os
compartimentos superiores apenas continham alguns papéis. As chaves
acharam-se nas fechaduras das gavetas abertas, \emph{e nenhum sinal de
violência}, ou nódoas de óleo apresentavam; e assim os mais papéis e a
lata encontrada dentro. A fechadura sobre a qual procederam a minucioso
exame, desparafusada a chapa interna, deixava-se ver no seguinte estado:

-- O orifício externo do canhão mostrava-se pendido e estragado, sendo a
violência feita no sentido de alargá-lo; e, por isso, foi mister
reduzi-lo ao estado primitivo, para ser desprendido da fechadura, da
chapa grande onde penetra o mesmo canhão. Dentro deste estava a broca de
quatro quinas, deformizada\footnote{. O mesmo que deformada,
  desfigurada.} pela violência nela praticada; e, no fundo, via-se um
pedaço de lima, que ali se quebrara, deixando patente a cor brilhante do
aço.

"Em todo seu comprimento, a fenda por onde passa o corpo da chave
oferecia, apesar de algumas arranhaduras, a dimensão natural. As peças
internas da fechadura \emph{nenhum sinal de violência, nem mesmo
arranhaduras}, deixavam ver, se bem que a mola, que guarda a forma dos
dentes da chave, e funciona perfeitamente, seja de metal amarelo ou
bronze. A fenda ou abertura que está por baixo do canhão, na parte
interna do cofre, tem a mesma largura ou comprimento que a da fenda
externa. Esta, nas faces verticais, apresenta sinais de força; mas a da
chapa interna acha-se em perfeito estado. Entre as peças da mesma
fechadura, foram encontrados soltos alguns fragmentos de ferro, como um
pedaço de ponta de paris\footnote{. Prego fino, com cabeça pequena,
  redonda e chata.}, um de prego forjado e dois outros que parecem ser
pontas de gazua. As molas de metal acima referidas parece que se compõem
de 5 pequenas chapas, representando, cada uma, a forma dos dentes da
chapa, para sobre elas girar esta; mas, em realidade, só duas funcionam
separadamente, não por desconcerto, \emph{mas por ser esta a forma
natural que lhes quis dar o artista} (!!!) que as confeccionou.
Respondem, portanto, dos quesitos, da maneira seguinte:

"Ao 1° quesito: Sim (há vestígios de violência às cousas e objetos);

"Ao 2° quesito: Consistem na dilaceração do orifício externo do canhão
da fechadura, no estrago feito na broca, no arrancamento dos fechos, e
nas arranhaduras, na fenda externa que está por baixo do dito canhão;

"Ao 3° quesito: Sim (por essa violência foi vencido o obstáculo
existente);

"Ao 4° quesito: Sim (havia obstáculo);

"Ao 5° quesito: Sim (houve emprego de força, instrumento ou aparelho);

"Ao 6° quesito: A força empregada pela chave de parafuso e por
ferramenta feita em forma de gazua;

"Ao 7° quesito: Está respondido pela descrição já feita (qual o estado
interior do cofre e dos objetos nele encontrados).

SEGURANÇA DO EDIFÍCIO, PARTE INTERNA

\emph{(2º exame)}

Respondem:

"Ao 1° quesito: Que está prejudicado pela descrição;

"Ao 2° quesito: Do mesmo modo;

"Ao 3° quesito: Sim; quem penetrou na alfândega denota conhecimento do
edifício);

"Ao 4° quesito: Que avaliam em 50\$000 {[}réis{]} o dano causado no
cofre e telhado; quanto às estampilhas e ao dinheiro roubado não podem
avaliar, por não terem dados para isso.

\_\_\_\_\_\_\_\_\_\_\_\_\_\_\_\_\_\_

QUESITOS DO DR. PROMOTOR PÚBLICO CONCERNENTES AO COFRE

Respondem:

"Ao 1° quesito: Que está prejudicado, pelo que já ficou dito, em
referência a essa fechadura;

"Ao 2° quesito: Que igualmente está prejudicado pela descrição feita do
estado do canhão;

"Ao 3° quesito: Que, antes de encravado pela lima, quebrada dentro do
canhão, podiam passar pela fenda nele existente os ferros encontrados
soltos na fechadura;

"Ao 4° quesito: Julgam-no prejudicado pela descrição feita a respeito;

"Ao 5° quesito: Não.

\_\_\_\_\_\_\_\_\_\_\_\_\_\_\_\_\_

Foram coligidos pelo dr. juiz municipal os seguintes objetos,
mencionados nos autos de exame:

Uma lima de três quinas com a ponta partida e o cabo ou espigão curvo;

Um alicate pequeno;

Uma verruma grande;

Dois formões pequenos, sem cabos, com os espigões partidos, sendo um de
fresco;

Um pedaço de ferro curvo nas extremidades, uma das quais partida de
fresco;

Um arame recurvado, grosso;

Seis instrumentos de arame de ferro e cobre, com formas diferentes;

Três pontas de paris tortas nas pontas;

Um vidro, com etiqueta de oriza, contendo óleo de amêndoas e uma pena de
galinha;

Dois espelhos, um de fechadura e outro de fingimento;

A fechadura do cofre forte, e a chapa grande correspondente à chave da
mesma, que foi entregue pelo tesoureiro Antonio Eustachio Largacha;

Um rótulo de papel almaço riscado com o dístico\footnote{. Letreiro.} -
100:000\$000 {[}réis{]} - escrito por extenso, com o número 100 em
algarismo, entre parêntesis, no ângulo inferior esquerdo.

\_\_\_\_\_\_\_\_\_\_\_\_\_\_

A 20 de fevereiro, \emph{porque na terra se achassem peritos hábeis}, a
requerimento do dr. promotor público, mandou o meritíssimo sr. juiz
municipal, presente o mesmo dr. promotor público da comarca, proceder a
segundo exame no cofre e respectiva fechadura.

AUTO DE CORPO DE DELITO

\emph{Exame no cofre e fechadura}

Peritos:

Adolpho Sydow, \emph{serralheiro}.

Frederico Guilherme Herstzberg, \emph{maquinista}.

\emph{Quesitos do juiz}

1° Se há vestígios de violência no cofre e fechadura?

2° Quais sejam?

3° Se por essa violência foi aberto ou podia abrir-se o mesmo cofre?

4° Se estando ele fechado havia obstáculo para ser aberto?

5° Se houve emprego de força, instrumento ou aparelho para vencê-lo?

6° Qual foi essa força, instrumento ou aparelho?

7° Se o cofre, que examinaram, podia ser aberto, em vista de sua
fechadura, com uma gazua qualquer, ou com aqueles instrumentos cujos
fragmentos foram encontrados dentro da mesma fechadura?

8° No caso afirmativo, era possível a abertura do cofre sem que as peças
internas da fechadura apresentassem sinal de violência?

9° Se não sendo possível a abertura do cofre por um gazua qualquer, ou
qualquer dos instrumentos, que forma, neste caso, deveria ter o
instrumento que o pudesse abrir?

10° Se a introdução da lima, cuja ponta encontrou-se dentro do canhão da
fechadura, podia ter servido para abertura do cofre?

11° Qual o juízo que formam da introdução da dita lima no canhão da
fechadura, e da aplicação do óleo de amêndoas que existe no exterior do
cofre?

\emph{Quesitos da promotoria pública}:

1° Existe marca ou sinal no cofre-forte pelo qual se conheça o nome do
seu fabricante?

2° Os fabricantes de cofres-fortes guardam forma especial para as
fechaduras dos cofres feitos em suas fábricas? (!!!)

3° Conhecido o nome do fabricante, é possível confeccionar, desde logo,
gazua ou qualquer instrumento que possa adaptar-se à fechadura e
facilitar a abertura do cofre, {[}à{]} quem for serralheiro, ou mesmo
sem sê-lo, {[}e{]} tiver conhecimento do ofício? (!!!)

\textbf{Respostas dos peritos}

QUANTO AO COFRE:

-- "Encontraram as duas portas do cofre untadas de óleo nos lugares dos
trincos, um dos quais achava-se sobre uma mesa.

"No interior encontraram dois armários, três gavetas pequenas e três
compartimentos superiores, tudo de ferro.

"As gavetas tinham fechaduras e as chaves; mas estas não indicaram
trabalho, em razão de estarem cobertas de ferrugem nos lugares em que
têm atrito.

"Todas estas peças interiores nem um sinal de violência apresentam.

``POR EXPERIMENTAREM, COLOCARAM A FECHADURA NO SEU RESPECTIVO LUGAR,
ISTO É, NA SUA CAIXA GRANDE, E RETIRADO O CANHÃO DO SEU ORIFÍCIO NATURAL
FOI INTRODUZIDA GAZUA, DAS COLIGIDAS PELO JUIZ; E, DEPOIS DE MUITO
ESFORÇO, A LINGUETA DESCEU, OU ANTES TOMOU A POSIÇÃO QUE DEVERIA TOMAR
PARA ABRIR-SE O COFRE''.

"Para conseguir-se isto, foi acomodado à peça de metal, ou mola, que era
visível, por não estar encostado à folha interior correspondente, que a
encobriria, uma gazua auxiliada por algumas pancadas dadas com a palma
da mão. Unidas, porém, a caixa grande, a folha externa ou principal, e
colocado o canhão no seu lugar, ficando assim a fechadura como devera
estar, quando intacto o cofre, verificaram ser impossível a introdução
das gazuas coligidas, por serem de diâmetro superior à largura da
entrada da fechadura que está abaixo do canhão.

"Fizeram de um dos instrumentos de arame de ferro uma gazua, com a forma
correspondente à parte da mola em que devia funcionar e, introduzida na
fechadura, não conseguiram, apesar de muitos esforços, fazer descer a
lingueta, o que verificaram, com uma luz introduzida na caixa grande,
por uma das entradas das linguetas de segurança, a qual {[}a{]} luz
deixava ver, por uma fenda, feita por um calço adaptado entre a caixa
grande e a folha exterior, a lingueta sempre fora da caixa pequena.

"Retirada novamente a fechadura, verificaram que dentro existiam
arranhaduras e amolgaduras\footnote{. Efeito de amolgar, amassar,
  deformar.} feitas com as gazuas, por eles peritos, nas experiências
referidas; porque antes de tentarem suas experiências, encontraram-na em
perfeito estado, sem o mais leve sinal de violência.

"TÊM A NOTAR QUE A MOLA DE METAL, CONQUANTO FUNCIONE PERFEITAMENTE, NÃO
TEM A FORMA PRIMITIVA, EM RAZÃO DE ESTAREM AS TRÊS CHAPAS INFERIORES
LIGADAS E FORMANDO UMA SÓ, AO PASSO QUE AS DUAS SUPERIORES NÃO FORMAM
TAMBÉM SENÃO UMA PEÇA.

"PARA TORNAREM MAIS CLARO ESTE PONTO, DECLARAM QUE A CHAVE QUE DEVERA
MOVER, COM SEUS 6 DENTES, 5 CHAPAS E A LINGUETA, MOVE SOMENTE DUAS
CHAPAS E A LINGUETA.

"Acrescentam que lhes parecia ter sofrido conserto a fechadura, em razão
de estar rebatido o eixo das molas, que em seu estado primitivo
ofereceria uma superfície igual à da chapa superior.

"Com algum esforço, conseguiram retirar de dentro do canhão um pedaço de
lima, que, unido ao que foi coligido, adaptava-se perfeitamente a ela.

"Estranharam a existência deste pedaço de lima dentro do canhão, porque
não podem explicar qual o fim com que foi ele ali introduzido, quando
seria mais conveniente, a quem quisesse forçar o cofre, ter
desembaraçado a fenda do canhão, para facilitar a introdução de qualquer
instrumento no interior da fechadura.

"Examinando uns pequenos instrumentos de arame, declaram que não sabem
que serventia podem eles ter, a menos que não seja para azeitar as peças
internas, de obras de ferro; mas notavam que não serviram, porque o
arame de cobre, das extremidades, estava em estado perfeito e
demonstravam não ter sido introduzido sequer na fenda da fechadura, o
que logo lhe mudaria as formas de chave e de flor.

"Encontraram dentro da fechadura diversos pedaços de gazua e um de ponta
de paris; mas não podem achar explicação pela falta de sinais que
deixariam esses ferros, se empregados, e com os quais deveriam ter
forcejado muito, pois que os quebraram.

``\emph{Respondem, portanto:}''

"Ao 1° quesito: Que apenas encontraram vestígios de violência na parte
exterior do canhão da fechadura, nada havendo no cofre, senão a remoção
dos espelhos;

"Ao 2° quesito: Que o vestígio encontrado é o torcimento do canhão;

"Ao 3° quesito: Não (\emph{por essa violência não foi aberto nem se
podia abrir o cofre});

"Ao 4° quesito: Que, estando o cofre fechado com a chave, deveria
oferecer obstáculo bastante forte;

"Ao 5° quesito: Que não houve emprego de instrumento ou aparelho;

"Ao 6° quesito: Prejudicado pela resposta anterior;

"Ao 7° quesito: Não;

"Ao 8° quesito: Que, se fosse possível, haviam de apresentar as peças
internas da fechadura sinais de violência;

"Ao 9° quesito: QUE PARA ABRIR A FECHADURA ERA NECESSÁRIO OU UMA CHAVE
APROPRIADA À FECHADURA OU UMA GAZUA MOLDADA PELO FEITIO DA CHAVE;

"Ao 10° quesito: Não; \emph{a lima não podia ter prestado auxílio algum
à abertura do cofre}).

"Ao 11° quesito: Quanto à introdução da lima, são de opinião ter ela
sido feita depois da abertura do cofre, \emph{e simplesmente para
simular violência}; e quanto ao óleo untado, julgam que podia ser
dispensado, MAS QUE FOI APLICADO PARA AMACIAR AS PEÇAS.

\_\_\_\_\_\_\_\_\_\_\_\_\_\_\_\_\_\_\_\_\_\_

\emph{Quesitos da promotoria}

RESPONDEM

"Ao 1° quesito: Existe no alto da folha esquerda do cofre a marca
seguinte - Obbs \& C. - N° 97 Chac. a pside. London;\footnote{. Como
  podem notar, a inscrição do auto de corpo de delito não faz muito
  sentido em idioma algum. À exceção da numeração e do nome da cidade,
  perfeitamente legíveis, pouco resta como ponto de partida. Contudo, a
  partir dessa informações, é possível reconstituir a inscrição exata.
  Trata-se da marca e do endereço da fabricante do cofre-forte: Hobbs \&
  Co.~- Nº 97 Cheapside, London. Fundada pelo inventor estadounidense
  Alfred Charles Hobbs (1812-1891), a marca Hobbs \& Co.~teve esse nome
  entre 1851 e 1855 -- sendo possível, portanto, conjecturar a data de
  fabricação do cofre-forte para esse quinquênio --, e seu endereço de
  funcionamento foi, por algum tempo, o nº 97 do bairro Cheapside,
  Londres.}

"Ao 2° quesito: Sim (os fabricantes de cofres-fortes guardam forma
especial para as fechaduras dos cofres feitos em suas fábricas);

``Ao 3° quesito: Não (pois o fabrico de um instrumento adaptado a abrir
a fechadura DEPENDE DO CONHECIMENTO DA DISPOSIÇÃO DAS PEÇAS
INTERIORES).''

\_\_\_\_\_\_\_\_\_\_\_\_\_\_\_\_\_

EXAME DO COFRE ORDENADO PELO EXMO. CONSELHEIRO CHEFE DE POLÍCIA

\emph{Quesitos:}

1° Que dimensões têm os compartimentos do cofre-forte quanto à largura,
comprimento e altura;

2° Se em alguns deles pode ser guardada, emaçada\footnote{. Empacotada,
  envolta em maço.} em um só volume, a quantia de 120:000\$000 réis,
sendo 50 em notas de 100, 30 em de 200, 20 em de 500, 5 de 50, 5 de 20,
2 de 10, 1 de 5 e 7 de 1\$000 réis.

\emph{Responderam os peritos}

"Ao 1° quesito: Que as 3 primeiras divisões superiores têm de fundo 45
centímetros; de altura, 60 centímetros; e 5 milímetros de largura. A
mais larga, 37 centímetros; e as outras duas, cada uma 18 centímetros.
As 3 gavetas, que são iguais, apresentam de fundo 51 centímetros; de
altura, 11 {[}centímetros{]}; e de largura 2 decímetros e 2 milímetros.
As duas divisões imediatamente inferiores, fechadas com portas,
apresentam de fundo 54 centímetros; de altura, 30 {[}centímetros{]}; de
largura, 32 {[}centímetros{]}. Finalmente, a divisão última inferior tem
de fundo 55 centímetros; de altura, 60 {[}centímetros{]}; de largura, 75
{[}centímetros{]}.

``Ao 2° quesito: Que a dita quantia pode, na forma perguntada, caber no
compartimento superior, mais largo, à esquerda, ou no inferior e
último.''

\_\_\_\_\_\_\_\_\_\_\_\_\_\_\_\_\_

EXAME NOVO ORDENADO PELO EXMO. CONSELHEIRO CHEFE DE POLÍCIA NA GATEIRA
DO TETO DA ALFÂNDEGA

\emph{Quesitos:}

1° Se as duas ripas da gateira, que se acham serradas, o foram de
recente ou antiga data;

2° Que distância medeia\footnote{. Divide ao meio.} entre os topos
serrados das mesmas ripas;

3° Se pela dita gateira podia alguém, ainda de corpo cheio, passar do
telhado para dentro do forro;

4° Que distância existe desde o ponto em que se acha a gateira até a
parede divisória, na sala do expediente;

5° Qual a extensão da dita sala do expediente, a contar da supra dita
parede divisória até a nova parede construída da alfândega nova; e como
os peritos atuais são os mesmos, que procederam o primeiro exame, a 19
de fevereiro próximo passado, foi-lhes mais perguntado;

6° Em que altura se achava a parede do novo edifício, contíguo à sala do
expediente, e se estava já esta última construção no ponto atual e
coberta.

RESPOSTA:

Ao 1° quesito: Que o corte das ripas é de antiga data;

Ao 2° quesito: Que o interstício é de 40 centímetros de largura sobre 54
{[}centímetros{]} de comprimento, formado pelo corte de duas ripas;

Ao 3° quesito: Que, pela gateira, podia alguém, ainda mesmo de corpo
cheio, passar do telhado para dentro do forro;

Ao 4° quesito: Que existe distância de 15 metros e 50 centímetros;

Ao 5° quesito: Que {[}é de{]} 17 metros e 80 centímetros;

Ao 6° quesito: Que quando procederam o exame na parede da construção
nova não estava a mesma no ponto em que se acha, nem coberta, mas no
nível do telhado da sala do expediente.

\_\_\_\_\_\_\_\_\_\_\_\_\_\_\_\_\_\_\_\_

\textbf{TELEGRAMA}

Vide \emph{Jornal do Commercio}, nº 51, de 20 de fevereiro de 1877.

"Santos, 19 de fevereiro, ONZE HORAS DA MANHÃ.

Tendo vindo do Governo ordem de remeter, para aí, o dinheiro que
houvesse na Alfândega, apareceu arrombado o respectivo cofre.
\emph{Desconfia-se que o arrombamento fosse feito só para encobrir o
anterior} DESVIO de dinheiro, QUE NÃO ERA POUCO.

\emph{Consta-nos que o Governo mandou suspender o tesoureiro da
Alfândega, procedendo-se a averiguações administrativas e policiais, e
que o roubo é avaliado em 180 contos."}

\emph{\_\_\_\_\_\_\_\_\_\_\_\_\_\_}

Ratificação e retificação de exames feitos a requerimento do tesoureiro
Largacha, pelo delegado de polícia; peritos, os mesmos.

QUESITOS:

1° Se, quando deu-se o primeiro exame do cofre, antes de qualquer outra
diligência, foi a chave respectiva aplicada à fechadura afim de
verificar se funcionava livremente, com o governo preciso, como até
então;

2° Se, apesar do estrago do canhão, podem acreditar ter sido aberto o
cofre com alguma gazua especial;

3° Se podem asseverar que os estragos feitos no canhão da fechadura
examinada aconteceram depois de aberta ela, e para simular arrombamento;

4° Se na ratificação que fizeram outra cousa pensaram, reformando hoje
seu juízo, ou idêntico foi o modo porque responderam os quesitos.

RESPOSTA:

Ao 1° quesito: Não -- quando deu-se o primeiro exame do cofre,
anteriormente a qualquer diligência, não foi a chave respectiva aplicada
à fechadura, afim de verificar se funcionava livremente, com o governo
preciso, como até então; (!...)

Ao 2° quesito: Que não afiançavam, mas podem acreditar que uma gazua
especial, movida por mão autorizada, introduzida no canhão, poderia
muito bem abrir a fechadura;

Ao 3° quesito: Não -- não podem esclarecer ou asseverar cousa alguma a
tal respeito, isto é, não sabem, nem podem asseverar que os estragos
feitos no canhão da fechadura examinada aconteceram depois de aberta ela
e para simular arrombamento;

Ao 4° quesito: \emph{Que o que dizem agora é o mesmo que declararam}, em
ratificação, perante o exmo. chefe de polícia, pois que nada em
contrário pode existir escrito, em vista do que expendido fica.

\_\_\_\_\_\_\_\_\_\_\_\_\_\_\_\_\_\_\_\_\_\_

TESTEMUNHAS

-- 47ª. \emph{Dr.~Luiz Manoel de Albuquerque Galvão:}

"Disse que as relações que tem com Antonio Eustachio Largacha resumem-se
\emph{nas 3 faturas de madeiras} QUE APRESENTA. (O juiz mandou-as juntar
aos autos).

"Que quanto à autoria e cumplicidade das subtrações havidas na
Alfândega, o depoente \emph{nada pode dizer de positivo}, pois que nada
sabe de positivo.

-- 50ª. \emph{Dr.~Moysés Rodrigues de Araujo Costa:}

"Disse que no dia 19 de fevereiro próximo passado, indo no bonde das 7
horas e meia, à barra, ouviu de um filho de Couto, empregado na
Alfândega, que o respectivo cofre fora arrombado, e que ia disso
prevenir ao inspetor.

"Que voltando à cidade ouviu muitas vezes falar do fato, \emph{atribuído
sempre} ao empreiteiro da obra da Alfândega, designando-se os nomes dos
srs. dr. Luiz Manoel de Albuquerque Galvão e Rodolpho Wursten; que no
dia 21 o \emph{Jornal do Commercio} trouxe um telegrama no qual era
atribuído o roubo e desfalques havidos na caixa; que a autoria deste
telegrama \emph{foi geralmente dada ao referido Rodolpho}, que, a esse
tempo, passava como correspondente do \emph{Jornal}; que ele Rodolpho
passava por autor do telegrama \emph{assim lho disse ele depoente},
almoçando em sua casa, em companhia de Emilio Airton; que, falando-se no
roubo da Alfândega, disse o depoente ao mesmo Rodolpho que a autoria do
roubo era geralmente ligada à do telegrama; que tem relações cortadas,
há muito tempo, com Antonio Eustachio Largacha; e vive em harmonia com
Rodolpho, que até é seu cliente.

-- 58ª. \emph{Antonio de Padua do Coração de Jesus}:

``Disse que foi chamado à Alfândega (não pode precisar a data, {[}mas{]}
no ano de 1859), porque se não podia abrir o cofre forte, que para ela
{[}lá{]} viera do Rio de Janeiro; que aí presentes o inspetor J. B. da
Silva Bueno e o tesoureiro Barroso, leram-lhe uma carta traduzida do
inglês, em que se dizia que o dito cofre estava aberto e a chave
principal dentro; que procurando abrir as folhas, elas não cederam;
\emph{porque as linguetas tinham corrido}, devido isto, de certo, a
tombos que o cofre tivesse levado, de maneira que viu-se forçado a
desmontar a meia folha superior da porta, que cobria a outra, furando o
lugar dos parafusos dos dois coices\footnote{. Peça de madeira onde se
  fixam cachimbos de metal sobre os quais se move uma porta.} da folha
onde existiam diversos parafusos; e assim, deslocada a dita meia folha,
verificou que a lingueta da mesma fechadura e mais linguetas tinham
corrido e fechado o cofre; \emph{que dentro dele apenas foi encontrada
uma chave da dita fechadura}, e outras mais pequenas, de diferentes
gavetas internas; que depois disto tornou a consertar tudo, para ficar
no mesmo estado em que o cofre tinha vindo, \emph{sendo ajudado em todo
este serviço por Benedicto José de Souza}, que ainda vive, segundo
pensa, em Santos; que passado mês e meio, ele, testemunha, foi chamado
novamente para desentralhar\footnote{. Desprender.} a broca da chave,
que era completamente redonda e estava cheia de cotão\footnote{.
  Partícula ou felpa, usualmente de tecido, que estava a impedir a
  abertura normal do cofre-forte.}, e nesta ocasião foi pedido à
testemunha que fizesse uma chave que infundisse mais respeito, sendo
maior; pelo que, ele, depoente, preparou uma chave \emph{inteiramente
nova, de boca de estrela,} CONSERVANDO AS ANTIGAS GUARDAS DE LATÃO
INTERNAS, \emph{de modo que a chave primitiva}, CUJA BROCA ERA REDONDA,
conquanto tivesse ficado em poder do tesoureiro (Barroso) era inservível
para poder mais abrir a dita fechadura.''

N.B.\footnote{. Ignoro o exato significado das iniciais, porém, elas
  abrem um parênteses para uma observação e antecedem a mudança para
  outro depoimento.} (Se este depoimento tem valor, por ele prova-se
perfeitamente que, para ser utilizada a primitiva chave, \emph{bastara
dar ao tubo da broca a forma de estrela; trabalho que qualquer
serralheiro faria em meia hora!)}.

-- 97ª. \emph{Sebastião Carlos Navarro de Andrade}, \emph{1º
escriturário da Alfândega}:

"Disse que não sabe por modo algum quem seja o autor ou cúmplice das
subtrações de dinheiro e estampilhas praticadas no cofre da Alfândega,
nem quais os das danificações feitas no mesmo cofre.

"\emph{Viu ou constou-lhe que alguém tivesse visto o tesoureiro Largacha
retirar dinheiro} do cofre e levá-lo consigo?

-- "Nada sei a respeito.

-- "Sabe se o inspetor da Alfândega e o chefe da 2ª seção cumpriam as
obrigações impostas pelos Regulamentos de 1860 e de 1876?\footnote{.
  Respectivamente, decreto nº 2.647, de 19/09/1860, que mandava executar
  o regulamento das alfândegas e Mesas de Rendas; e o decreto nº 6.272,
  de 02/08/1876, que reorganizava as alfândegas e Mesas de Rendas.}

-- "Quanto ao inspetor sei, por ver, que ele não tomava semanalmente
conta do estado dos cofres; pois nunca me constou. Quanto à remessa de
dinheiro, posso dizer que o inspetor fê-la até janeiro; pois durante o
mês de fevereiro estive ausente, com licença. Quanto ao chefe da 2ª
seção, relativo à verificação dos valores recebidos no dia, a fazia
parcialmente, visto como havia verbas de receita que eram conferidas por
mim, com o próprio tesoureiro, de 5 em 5 dias, sendo uma destas a de
estampilhas do selo adesivo; que, quanto à assistência, abertura e fecho
do cofre, se o dito chefe a exercia, não era com regularidade, \emph{e
como eu não dava atenção, não posso asseverar se o fazia ou deixava de
fazer."}

\emph{-- ``Constou-lhe que o tesoureiro Largacha tivesse retirado
dinheiro do cofre e levado consigo?''}

\emph{--} ``Não sei, nem ouvi''.

-- 209ª. \emph{Tiburtino Mondim Pestana} (amanuense externo da
repartição da polícia):

-- ``A testemunha, juntamente com um empregado da Alfândega, e qual,
revistaram todas as bagagens dos passageiros que saíram do porto de
Santos nos dias 18, 20 e 21 de fevereiro próximo passado?''

-- ``No dia 18, partiu o vapor\footnote{. Espécie de embarcação.}
\emph{Rio Grande}, mas sem passageiros; no dia 20, o vapor alemão
\emph{Argentina}, para o Rio da Prata, levando um passageiro alemão e
outro inglês, cujas bagagens a testemunha revistou; no dia 21, como
houvesse muito atropelo de passageiros, as revistas foram feitas nas
suas bagagens, parte por ele e outra {[}parte{]} pelo empregado da
Alfândega, que é oficial de descarga, cujo nome ignora, mas que é de
altura regular, branco e bem barbado, ocupando-se ainda neste mister
duas praças do destacamento do corpo policial, cujos nomes ignora, uma
das quais era italiano e outra crioulo; SENDO QUE DEIXOU DE SER
REVISTADA UMA BAGAGEM QUE VEIO CONDUZIDA E ACOMPANHADA PELO
ORDENANÇA\footnote{. Soldado às ordens pessoal de uma autoridade a quem
  acompanha durante as horas do expediente.} DO DELEGADO DE POLÍCIA, DE
NOME BANDEIRA, QUE DECLAROU''O DELEGADO MANDA DIZER QUE NÃO NECESSITA
REVISTAR ESTA BAGAGEM PORQUE JÁ SOFREU REVISTA EM TERRA, IGNORANDO A
TESTEMUNHA A QUEM PERTENCIA".

-- 218ª. \emph{Tiburtino Mondim Pestana} (o mesmo que já depôs sob
{[}o{]} nº 209).

-- "Quem acompanhava a bagagem que no dia 21 de fevereiro foi conduzida
ao vapor \emph{S. José}, pelo policial Bandeira, em nome do delegado
tenente Pinho, que mandou não fosse revistada, por já tê-lo sido por ele
{[}revistada{]} em terra?

-- ``Foi Leonce Wynem, EMPREGADO DA CASA DE AZEVEDO \& COMPANHIA,
\emph{ignorando a quem pertencia a dita bagagem}''.

N. B.: À casa dos srs. \emph{Azevedo \& Companhia} pertencia, como
sócio, o sr. Rodolpho Wursten; e, por esse vapor, nessa ocasião, seguiu
para a Corte o sr. dr. Galvão; e, segundo um depoimento do sr.
\emph{Sebastião Carlos Navarro de Andrade}, amigo particular do mesmo
doutor, foi ele à Corte com o fim exclusivo de causar surpresa aos seus
parentes em uma festa de família.

-- 206ª. \emph{Tenente Antonio Joaquim de Pinho} (delegado de polícia):

-- ``Quando e por quê, referindo-se às subtrações havidas na Alfândega,
fechando uma das mãos, disse:''o ladrão está aqui"?

-- Não me recordo absolutamente de haver dito ``o ladrão está aqui'', e
isto com a mão fechada; mas na Alfândega, ao dr. juiz municipal e,
\emph{em confiança}, a mais alguma pessoa, disse que suspeitava ter sido
o subtrador\footnote{. Nesse caso, o autor do crime.} o engenheiro
Galvão, encarregado das obras da Alfândega; os fundamentos destas minhas
suspeitas, são, 1º: estar o mesmo engenheiro encarregado das ditas
obras; 2º: \emph{porque os instrumentos apreendidos só podiam ser de uma
pessoa profissional}; 3º: porque com esses instrumentos, creio eu, não
ter sido feita a operação de abertura do cofre, mas com chave própria,
ou com outra igual, tendo-se tirado molde; 4º: porque o dr. Moyses
dissera mais a mim, depoente: -- "Mande agarrar Rodolpho Wursten, porque
é um dos ladrões da Alfândega;

Ao que respondi:

-- "Como é que, sem outra circunstância, hei de mandar segurar Rodolpho?
Ao que o dr. Moysés, respondeu-me:

-- ``Então eu posso fazer um roubo, e se disserem que foi Moyses, não me
manda prender por ser incapaz de assim proceder?!''

Querendo, com isto, significar que não se prendia Rodolpho porque era
incapaz disso.

-- ``A quem a testemunha referiu isto?''

-- ``Não me recordo''.

-- ``Tem relações com Theodoro de Menezes Forjaz e Manoel Geraldo
Forjaz, fiel do tesoureiro da Alfândega?''

-- "Com Manoel Geraldo Forjaz nenhuma relação tenho; com o segundo
apenas de cumprimento, quando o encontro;

-- ``Conhece José Caballero?''

-- ``Não conheço''.

-- ``Que diligências fez por ocasião das subtrações e danos feitos no
cofre da Alfândega?''

-- ``Fiz unicamente um ofício ao inspetor da Alfândega, pedindo o nome
do engenheiro Galvão e uma relação nominal dos empregados na mesma
Alfândega, dos operários, requisição que foi satisfeita; mas como logo
compareceram os drs. juízes municipal e de direito, abstive-me de
prosseguir; sendo que esses papéis devem estar arquivados no cartório da
delegacia''.

- ``Enquanto esteve nesta cidade, não ouviu atribuir os fatos
acontecidos na Alfândega a diversas outras pessoas, além das que
declara?''

-- ``\emph{Ouvi somente falar nesses dois indivíduos, apontando-se
Galvão como chefe"}; dizendo-se que o inspetor era incapaz de praticar o
fato, bem como o tesoureiro.''\emph{Ouvi também falar em Custodio de
Tal, remetido para o Rio pelo dr. Galvão, afim de ser empregado; e que
este foi quem levara a bagagem do engenheiro para o Rio"}, não sabendo
se o dito Custodio fora antes, ou com ele.

-- 219ª. \emph{Leonce Wymen:}

"Disse que é verdade que a bagagem examinada pelo delegado de polícia, e
por ele acompanhada, com o policial, pertencia ao alemão GUILHERME
KRONLSIN.

N. B. \emph{Na lista dos passageiros dessa viagem do vapor S. José não
foi mencionado o nome do sr. Guilherme Kronlsin!...).}

-- 216ª. \emph{Antonio Francisco Bandeira:}

"Com efeito, levei a bagagem \emph{e dei o recado}, que se me atribui,
ignorando, porém, qual o seu dono.

``A bagagem foi levada à casa do delegado tenente Pinho por um menino,
sendo revista por aquele, ignorando ainda a morada deste. E, conquanto
não me lembre de quantos volumes se compunha, todavia posso dizer que
era pequena. A dita bagagem foi acompanhada por um mocinho estrangeiro,
que fala português, cuja nação ignoro; usa de bigode e parece-me que faz
a barba; \emph{e é da casa comercial de Azevedo \& Companhia,} mas
ignoro como ele se chama''.

\_\_\_\_\_\_\_\_\_\_\_\_\_\_\_\_\_\_

TESTEMUNHAS que só foram inquiridas a requerimento e por esforços do
major Largacha, com citação prévia do sr. dr. promotor público, perante
a delegacia de polícia.

-- \emph{Victor Nothmann}, abastado negociante da Capital:

"Disse que o dr. Galvão lhe dissera ser inexplicável que um
\emph{empregado público, que em 1867 pagou} o seu passivo com 10 00
apenas, tenha, hoje, uma fortuna maior de 200:000\$000 {[}réis{]}!
Contando, entre outros prédios, um em S. Vicente, do valor de
46:000\$000 {[}réis{]}, e tendo constantemente dinheiros a prêmio, que,
em avultadas somas, conserva neste giro.

"Disse mais, que o dr. Galvão pôs em dúvida a existência do roubo da
Alfândega, ou, antes, disse que não existiu; porque sendo a
administração da repartição \emph{muito relaxada, não se tendo dado
balanço há muitos anos,} estava o tesoureiro com os dinheiros públicos à
sua disposição, \emph{de modo que as quantias que deviam ter sido
remetidas não existiam em caixa, pois o tesoureiro tinha somente o
dinheiro que se achou espalhado pela sala, e que isso mais se explica
com o telegrama} mandado ao ministro pelo inspetor, que dizia não poder
mandar o dinheiro no dia da expedição do mesmo telegrama, \emph{quando é
certo que dinheiro algum lhe havia pedido o ministro}.

"Disse mais, que Galvão lhe contara estar junto aos autos uma carta do
marquês de S. Vicente\footnote{. José Antonio Pimenta Bueno (1803-1878),
  o \emph{marquês de São Vicente}, nascido em Santos (SP), foi juiz,
  desembargador, ministro do Supremo Tribunal de Justiça, diplomata e
  político de grande prestígio ao longo do século XIX. Foi presidente
  das províncias de Mato Grosso (1836-1838) e São Pedro do Rio Grande do
  Sul (1850), além de ministro da Justiça (1848) e das Relações
  Exteriores (1870-1871).}, em que este, garantindo proteção ao
tesoureiro, lhe prometia o emprego de sua influência.

"Disse mais, que Galvão lhe contara que, falando ao ministro, este lhe
dissera ter falado ao conselheiro Duarte de Azevedo\footnote{. Manuel
  Antonio Duarte de Azevedo (1831-1912), natural de Itaboraí (RJ),
  exerceu diversos cargos da alta burocracia do Império e da República.
  Foi advogado, juiz, deputado, senador, ministro da Justiça e
  presidente das províncias do Piauí (1860-1861), de Alagoas (1861) e do
  Ceará (1861-1862). A par da carreira política e judiciária, foi também
  professor catedrático de Direito Romano da Faculdade de Direito de São
  Paulo.}, que nada mais podia fazer em benefício do seu parente, o
inspetor da Alfândega de Santos, porque este estava muito complicado no
roubo da Alfândega; dizendo-lhe mais o dr. Galvão, que à vista dos
autos, segundo o pensar de pessoas entendidas, estava provada a
criminalidade do tesoureiro.

``Disse, finalmente, \emph{que o dr. Galvão} É ÍNTIMO AMIGO DO INSPETOR
BHERING, QUE ANDAM SEMPRE JUNTOS; e que, ainda há poucos dias, o dr.
Galvão lhe fez outras muitas revelações, que ele, depoente, nem mesmo se
recorda mais, não fazendo reserva do que narrava porque o fez em um trem
de ferro, quando vinha de S. Paulo para Santos, sem que recomendação
alguma fizesse à testemunha de não propalar; e que lhe consta outras
casas de Santos têm recebido daquele doutor revelações semelhantes.''

-- \emph{João Alberto Casimiro da Costa}, empregado no comércio:

"Respondeu que tem ouvido o engenheiro Luiz Manoel Albuquerque Galvão
insinuando que o ladrão da Alfândega não podia ser outro senão o major
Antonio Eustachio Largacha.

"Que isto ouviu na sala do atual inspetor da Alfândega, estando também
presentes Victorino José de Mattos e Antonio Proost de Souza.

"Disse mais, que estranhou sobremaneira e de modo positivo se manifestou
contra o juízo que aquele engenheiro fazia, isto é, contra a insinuação
inconveniente, \emph{e que lhe pareceu algum tanto interessado, porque
ele, depoente, notou alguma cousa de extraordinário na manifestação
daquele engenheiro, o único certamente que ele, depoente, tem visto
pronunciar-se tão asperamente contra o major Largacha}, notando-se que,
dirigindo-se ao engenheiro Galvão, ele, depoente, quase textualmente
serviu-se das frases que está fazendo agora inserir no seu depoimento.

Disse mais, que o engenheiro Galvão, depois dele, depoente, estranhar
que contra a probidade do major Largacha atentasse ele tão fortemente,
passou a fazer narrativas no intuito de demonstrar que muitos homens que
no mundo gozam de fama de honrados, dias aparecem em que, estudados os
fatos e conhecidas as causas, chega-se à realidade de que a
preconizada\footnote{. Recomendada, afamada.} honra era simplesmente uma
história.

"Disse que, referindo essa narração, dizia aquele engenheiro se ter
passado o fato no Rio Grande do Sul, com um tesoureiro, tendo o pai ou
um parente daquele engenheiro entrado no conhecimento dessa questão.

"Disse mais, que à vista dessa narração, aquele engenheiro deixou bem
patente a imputação grave que faz ao major Largacha, pois, não obstante
a impugnação feita por ele, depoente, insistiu aquele engenheiro em seus
assertos, procurando dar-lhes a vida precisa, fazendo aplicação da
questão havida no Rio Grande do Sul, com um tesoureiro convencido de
ladroeira; sendo ainda para notar que, não obstante a narração feita, o
engenheiro Galvão dizia \emph{que não era seu intento prejudicar a
reputação do major Largacha}.

"Disse mais, que, desde essa ocasião, nunca mais esteve com aquele
engenheiro, com quem, pela primeira e última vez, falou em casa do
inspetor, constando-lhe, porém, por ouvir a diversos, SER ELE ENGENHEIRO
O ÚNICO A PROPALAR BOATOS contra a reputação do major Largacha.

N. B.: Depoimentos iguais foram mais prestados, na mesma ocasião, pelos
srs. Henrique Wright e Luiz Antonio de Barros.

-- HÁ MAIS NO PROCESSO quatro longos depoimentos do escriturário da
Alfândega, sr. Sebastião Carlos Navarro de Andrade, contendo alusões que
bem se filiam aos boatos propalados pelo dr. Galvão, e que são ofensivos
da probidade do major Largacha e do inspetor Assis. Esses depoimentos
estão inçados\footnote{. Repletos, cheios.} de contradições gravíssimas,
que tornam a testemunha seriamente suspeita, maiormente considerando-se
a sua amizade íntima com o mencionado dr. Galvão.

Não o transcreveremos agora, por serem extensos; mas o faremos na
impressão completa do processo.

\_\_\_\_\_\_\_\_\_\_\_\_\_\_

A estes depoimentos, dos quais fica provada a singular, calculada e
inexplicável autoria dos boatos assoalhados\footnote{. Propalados,
  divulgados.} contra o tesoureiro Largacha, o sr. dr. Galvão respondeu
imediatamente, pela imprensa, do seguinte modo:

Santos, 27 de junho de 1877.

Sr.~Redator:

Tenho lido no \emph{Diário de Santos} que diversos indivíduos foram
declarar à autoridade haverem partido de mim versões desfavoráveis à
reputação do sr. major Antonio E. Largacha.

Declaro que nunca desejei o cargo de acusador de quem quer que seja, e
muito menos do sr. Largacha, que muito antes da minha chegada a esta
cidade havia escrito ao meu correspondente do Rio de Janeiro
\emph{propondo-se a fornecer madeiras para o novo edifício da Alfândega}
(...)

Acedemos à sua proposta e, quanto a mim, declaro que então nem sequer
sabia que o sr. Largacha \emph{fosse o tesoureiro da Alfândega}.

Algum tempo depois, o sr. administrador das capatazias\footnote{.
  Atividades de movimentação de cargas e mercadorias nas instalações
  portuárias.} ofereceu-me madeiras de construção, dizendo-me que o sr.
Largacha não poderia fornecer-me toda a necessária para as obras da
Alfândega, por não ter madeiras cortadas em estado de serem
imediatamente aplicadas.

A promessa feita pelo sr. Largacha fez-me esperar alguns meses sem
encomendar madeiras a outros, até que vi-me forçado a procurar com
urgência este material, pois se o não fizesse teria o infalível desgosto
de suspender todos os trabalhos em andamento.

À vista disto, resolvi não continuar a comprar madeiras ao sr. Largacha
e paguei-lhe a quantia de 1:788\$450 {[}réis{]}, em que importavam as
que já havia fornecido.

Trouxe esta narração para mostrar que aqui, apesar de não haver o sr.
Largacha entregue nas obras a madeira no prazo por ele prometido, apenas
atribuí esta falta a seus afazeres. Portanto, posso assegurar ao público
que se alguém porventura pretende ter ouvido alguma palavra minha sobre
o sr. Largacha, interpreta bem injustamente os meus sentimentos e eu não
posso ser responsável por interpretações que deem às minhas frases,
pessoas com quem não entretenho relações de qualidade alguma.

Não tenho por costume ocupar-me da reputação alheia, porque fui educado
em princípios inteiramente opostos a isto; princípios que v{[}ocê{]} bem
pode aquilatar\footnote{. Avaliar, julgar.}, pois é filho do magistrado
mais elevado desta cidade.

Relativamente ao roubo da Alfândega, apenas fiz um depoimento perante o
sr. conselheiro Furtado, quando na qualidade de chefe de polícia veio
inquerir sobre tão deplorável acontecimento, e tenho consciência de nada
haver dito em detrimento da probidade de pessoa alguma, pois, prezando a
minha modesta profissão de engenheiro, sinto-me sem aptidão para ser
denunciante.

Que foi roubada a Alfândega, é fato que está no domínio da publicidade.
Increpar\footnote{. Acusar, rotular.} a quem quer que seja por um crime
tão infamante, sem ter razões muito sólidas, é procedimento que não têm
aqueles que gozam da felicidade de conhecer que devem à educação e a
bons exemplos que receberam de seus maiores a segurança e solidez da
estrada que tinham perante a sociedade.

Queira, pois, v{[}ocê{]} fazer-me a fineza de publicar estas linhas em
seu conceituado jornal.

Sou de v{[}ocê{]} amigo atento e obrigado.

\emph{Luiz M. De Albuquerque Galvão}

-- 60ª. \emph{José Theodoro dos Santos Pereira:}

"Que na casa de banhos, à testa\footnote{. À frente, na direção.} da
qual a testemunha se acha, ouviu, ora a um, ora a outro, dizer 'que o
autor da subtração na Alfândega \emph{tinha sido o mesmo que expediu o
telegrama que no Jornal do Commercio}, do Rio de Janeiro (de 20 de
fevereiro), apareceu impresso; e como Rodolpho Wursten \emph{era o
correspondente do jornal e sub-empreiteiro do dr. Galvão} na obra da
nova Alfândega, tinham sido, eles dois, os autores das ditas subtrações,
ignorando, porém, a testemunha, se esta versão é ou não verdadeira; que
no penúltimo domingo, achando-se a testemunha à bordo do \emph{Vapor S.
José}, convidado a jantar pelo comandante do mesmo - Mello -, presente
este, o capitão tenente Nascimento e o imediato do \emph{Vapor},
unicamente, caiu em conversa falar-se das subtrações na Alfândega; e,
nessa ocasião, Nascimento atribuiu tais fatos \emph{ao tesoureiro
Largacha,} dizendo \emph{que ele fazia muitas despesas, fazia muitos
favores, e tinha serraria à vapor}; ao que ele, testemunha, atalhou,
dizendo 'que a serraria não era à vapor, e sim movida por água; e não
adquirida por ele por compra, e sim por herança dos seus maiores; e que
era um homem que vivia com sua família, e sempre bem conduzido; e que se
alguém dizia o contrário é porque era seu inimigo e não o conhecia'; e,
dizendo ainda a testemunha a Nascimento, que se ele assim falava era por
ter ouvido a ALGUÉM e, instando mesmo com ele, para que dissesse de quem
tinha ouvido, Nascimento respondeu, '\emph{que quem assim lho havia dito
fora o engenheiro dr. Galvão, íntimo amigo dele, Nascimento". (}!!!)

-- 54ª. \emph{Dr.~Pedro Augusto Pereira da Cunha:}

"Disse que era voz geral ser autor do telegrama Rodolpho Wursten, e que
este mesmo procurava justificar-se junto de Antonio Largacha, segundo
este contara a ele, testemunha, por empregado seu, que mandou ao mesmo
Largacha, qual não sabe; dizendo a testemunha que a esse tempo o mesmo
Rodolpho \emph{era correspondente do Jornal do Commercio}, ignorando se
ainda é.

``Disse mais, que a opinião pública aponta também o nome do dr. Manoel
Luiz de Albuquerque Galvão, como sócio de Rodolpho, na subtração
praticada na Alfândega, ignorando o depoente quais os dados ou bases que
serviam de fundamento a essa opinião manifestada.''

-- 118ª. \emph{Leopoldo da Camara Lima:}

"Que sabe, em razão de ter ouvido dos empregados, que se contava
dinheiro para remessa ao Tesouro.

``Que quando ele chegou à Alfândega, ÀS 9 1/4 HORAS,\footnote{. Isto é,
  9:15.} dizia-se que a subtração fora de CENTO E CINQUENTA CONTOS; mas,
mais tarde, PELA UMA HORA, mais ou menos, da tarde, do mesmo dia 19 de
fevereiro, o escriturário da tesouraria - Soares -, depois de haver
somado o Livro de Receita, disse ao depoente 'que a falta de dinheiro
montava 174:000\$000 {[}réis{]}; e, dias depois, desempasteladas as
estampilhas, \emph{verificou-se que o alcance subia a 182:000\$}
{[}réis{]}'''.

\_\_\_\_\_\_\_\_\_\_\_\_\_\_\_\_

AFIRMAM pessoas conceituadas que o sr. \emph{Bombardo}, morador em
Santos, declara ter dado depoimento perante a delegacia de polícia,
depoimento que não encontramos no sumário, e que nesse depoimento
dissera:

-- "Que a 17 ou 18 de fevereiro, alta noite, encontrara na rua um
indivíduo que reconheceu, ou parecera-lhe ser o dr. Galvão, engenheiro
que, em voz baixa, conversava com certo serralheiro, antigo morador de
Santos.

"Que esse serralheiro era o mesmo que para a Corte mudara-se, seguindo a
21 de fevereiro, \emph{pelo vapor S. José}, pelo qual, e na mesma
ocasião, também seguira o dito dr. Galvão.

\_\_\_\_\_\_\_\_\_\_\_\_\_

-- \emph{Luiz} \emph{Antonio de Barros:}

"Respondeu, que, depois do roubo da Alfândega, não se recordando se no
dia seguinte, ou se no imediato, estando a almoçar em companhia de
Henrique Wright, ouviu o dr. Galvão, engenheiro, que também estava à
mesa, no \emph{Hotel Central}, dizer 'que se a polícia de Santos fosse
mais ativa, o major Largacha já deveria ter sido metido em prisão, em
razão do roubo havido nos cofres'.

"Que havendo o seu companheiro (Henrique Wright) oposto-se a uma tão
grave asserção, o dr. Galvão desculpou-se, dizendo '\emph{que não fazia
mau juízo do tesoureiro...}'.

Que tem ouvido a diversos que o engenheiro Galvão imputa ao major
Largacha o roubo da Alfândega, desacreditando-o, por este modo, com esta
acusação.

``Que, finalmente, sabe ser o dr. Galvão muito amigo do inspetor Lucas
Ribeiro Bhering, por tê-los amiudadas vezes visto, à noite, nos hotéis
\emph{América} e \emph{Bragança...}''.

-- 59ª. \emph{Benedicto José de Souza:}

*-- ``Que é exato tudo quanto em seu depoimento expôs* Antonio de Padua,
e que, neste ato lhe foi lido, como a verdade do que se provou; \emph{e
bem assim reconhece a chave que lhe foi apresentada, que atualmente
servia no cofre da Alfândega, como a própria que foi feita, com broca
diferente da que a fechadura tinha trazido, e da cruzeta feita pelo
mesmo Antonio de Padua}, NÃO INTERVINDO OUTROS QUAISQUER OFICIAIS NESTE
NEGÓCIO''.

\_\_\_\_\_\_\_\_\_\_\_\_\_\_

\textbf{Observações sobre os exames feitos no edifício e cofre da
Alfândega e depoimentos prestados relativamente ao roubo cometido na
mesma Repartição.}

\textbf{\_\_\_\_\_\_\_\_\_\_\_}

Começaremos a análise da prova em que se fundou (\emph{afundou-se},
seria frase mais assisada\footnote{. Ajuizada, sensata.}, correta e
expressiva) o venerando Acórdão de 19 de Outubro.

Iniciou-se o processo pelo \emph{inquérito policial}, ordenado,
\emph{ex-officio}\footnote{. Realizado por imperativo legal e/ou por
  dever do cargo ou função.}, por o sr. dr. juiz municipal da cidade de
Santos, a 19 de fevereiro deste ano.

O \emph{inquérito policial}, em face das disposições da Lei n° 2.033 de
23 de Setembro de 1861\footnote{. Por erro tipográfico mínimo, trocou-se
  dois dígitos. A data correta da lei citada é 20/09/1871.} e do Decreto
n° 4.824 de 22 de Novembro, do mesmo ano, consiste na reunião das
diligências necessárias para a verificação:

1º: Da existência do \emph{crime comum};

2º: De todas as circunstâncias DO MESMO CRIME;

3º: De todas as circunstâncias sobre os criminosos, autores ou cúmplices
DE TAIS CRIMES (Decreto cit{[}ado{]} n° 4.824 de 22 de Novembro de 1871,
artigos 38, 42).\footnote{. Respectivamente, art. 38. ``Os chefes,
  delegados e subdelegados de polícia, logo que por qualquer meio lhes
  chegue a notícia de se ter praticado algum crime comum, procederão em
  seus districtos às diligências necessárias para verificação da
  existencia do mesmo crime, descobrimento de todas as suas
  circunstâncias e dos delinqüentes''. E art. 42. ``O inquérito policial
  consiste em todas as diligências necessárias para o descobrimento dos
  fatos criminosos, de suas circunstâncias e dos seus atores e
  cúmplices; e deve ser reduzido a instrumento escrito, observando-se
  nele o seguinte (...)''.}

Dessas diligências são:

1º: O corpo de delito direto;

2º: Exames e buscas para apreensão de instrumentos e documentos;

3º: Inquirição de testemunhas, \emph{que houvessem presenciado o fato
criminoso ou tenham razão de sabê-lo};

4º: Perguntas ao réu e ao ofendido;

5º: Em geral, tudo que for útil para esclarecimento do fato e suas
circunstâncias (Decreto n° 4.824 de 22 de Novembro de 1871, artigo
39).\footnote{. Descrição praticamente literal dos parágrafos do art. 39
  do decreto, à exceção dos grifos em itálico, que são originais dos
  autores.}

Os CRIMES COMUNS a respeito dos quais deve-se proceder a inquérito são
aqueles em que cabe a denúncia (Decreto n° 4.824 de 22 de Novembro de
1871, cit{[}ado{]}, artigo 41).\footnote{. Art. 41. ``Quando, porém, não
  compareça logo a autoridade judiciária ou não instaure imediatamente o
  processo da formação da culpa, deve a autoridade policial proceder ao
  inquérito acerca dos crimes comuns de que tiver conhecimento próprio;
  cabendo a ação pública, ou por denúncia, ou a requerimento da parte
  interessada; ou no caso de prisão em flagrante''.}

O inquérito se abre:

1º: \emph{Por queixa};

2º: \emph{Por denúncia};

3º: \emph{Ex-officio, no caso de prisão em flagrante} (Decreto n° 4.824
de 22 de Novembro de 1871, cit{[}ado{]}, artigo 41).

À vista deste demonstração legal irrefutável, porque a evidência da lei
não se contesta, temos que:

\emph{Se o crime era comum}, e logo que se divulgou a sua perpetração no
lugar dele compareceu o dr. juiz municipal, autoridade única competente
para a formação da culpa, e a esta deu começo, ordenando o respectivo
sumário, pelos autos de corpo de delito, ociosa, ilegal, inexplicável e
criminosa foi a presença e a interferência do exmo. sr. conselheiro
chefe de polícia no processo, \emph{ratificando e retificando
policialmente os atos judiciário} do juiz municipal.

\emph{Se o crime era comum}, muito bem procedeu o digno sr. dr. juiz
municipal instaurando a formação da culpa; mas, neste caso,
perguntaremos:

Onde está a denúncia ou a queixa que deveria determinar o seu
procedimento, que não é um ato de arbítrio, se não o rigoroso
cumprimento de uma obrigação que resulta precisamente da estrita
observância da lei?!

Não houve queixa nem denúncia!...

Foi a culpa formada \emph{ex-officio}?

Quando, onde, por quem, de que modo foram os réus presos em flagrante
delito?!

Não houve prisão em flagrante delito!...

Qual, então, o motivo legal que determinou a presença do sr. dr. juiz
municipal na Alfândega?

Qual o texto de lei que justifique o procedimento desse emérito juiz?

Que razões, que princípios de direito, que normas de jurisprudência, que
mistérios judiciários forçaram o esclarecido juiz a suspender a formação
da culpa e a devolver os autos à autoridade policial, que, diante deste
caos informe, nem sequer exercia a faculdade conferida no artigo 60 do
Regulamento nº 120 de 31 de Janeiro de 1842, mantida pelo artigo 9º da
Lei n° 2.033 de 20 de Setembro de 1871 e pelo Decreto n° 4.824 de 1871,
artigo 12?!\footnote{. Respectivamente, art. 60. ``O governo, ou os
  presidentes nas províncias poderão ordenar que os chefes de polícia se
  passem temporariamente para um ou outro termo ou comarca da província,
  quando seja aí necessária a sua presença, ou porque a segurança e
  tranquillidade pública se ache gravemente comprometida; ou porque se
  tenha ali comettido algum, ou alguns crimes de tal gravidade, e
  revestidos de circunstâncias tais, que requeiram uma investigação mais
  escrupulosa, ativa, imparcial ou inteligente; ou finalmente porque se
  achem envolvidos nos acontecimentos que occorrerem pessoas cujo
  poderio e prepotência tolha a marcha regular e livre das Justiças do
  lugar''. Art. 9º. ``Os chefes de polícia poderão ser nomeados dentre
  os desembargadores e juízes de direito, que voluntariamente se
  prestarem, ou dentre os doutores e bachareis formados em Direito, que
  tiverem pelo menos quatro anos de prática do foro ou de administração.
  Quando magistrados, no exercício do cargo policial, não gozarão do
  predicamento de autoridade judiciária; vencerão, porém, a respectiva
  antiguidade e terão os mesmos vencimentos pecuniários, se forem
  superiores aos do cargo de chefe de polícia''. Art. 12. ``Permanece
  salva ao chefe de polícia a faculdade de proceder à formação da culpa
  e pronunciar, no caso do art. 60 do Regulamento nº 120 de 31 de
  Janeiro de 1842, com recurso necessário para o presidente da Relação
  do Distrito, na corte e nas províncias do Rio de Janeiro, S. Paulo,
  Minas, Bahia, Sergipe, Pernambuco, Alagoas, Paraíba e Maranhão; e nas
  outras, para os juízes de direito das respectivas capitais, enquanto
  não se facilitarem as comunicações com as sedes das Relações''.}

Pois a formação da culpa, uma vez encetada\footnote{. Iniciada.}, poderá
ser interrompida pela polícia para a organização de um simples
inquérito?!

-- Glória ao exmo. sr. desembargador Accioli de Brito, neste processo
tudo é monstruoso!...

\emph{Se o crime não era comum}, se os culpados, por secreta indicação
da Tesouraria de Fazenda, em ofícios e relatórios reservados, \emph{até
escritos antes da perpetração do delito}, eram empregados públicos não
privilegiados, se os crimes dos empregados, neste caso, têm foro
especial, se ao dr. juiz de direito da comarca compete exclusivamente a
organização e o julgamento do processo, se a forma e os termos do
processo estão expressamente precisados na lei, é certo, é inconcusso
que o \emph{inquérito policial}, que só tem cabimento no processo dos
\emph{crimes comuns}, foi absurdamente feito, constitui um ato de
arbítrio culposo e está neste processo como prova patente da mais
revoltante monstruosidade jurídica!...

E foi neste inquérito ilegal e monstruoso que o colendo Tribunal da
Relação esteiou-se para proferir o venerando Acórdão de 12 de Outubro!

-- Glória ao exmo. sr. desembargador Accioli de Brito, neste singular
processo tudo é monstruoso!...

Se a organização dos autos de corpo de delito, se a inquirição graciosa
de testemunhas, sem juramento, em ausência de queixa ou de denúncia, ou
de prisão em flagrante delito, de crime, \emph{que é comum}, para
determinar a intervenção do dr. juiz municipal, \emph{que é misto}, para
transformar o delegado de polícia, de autoridade, \emph{em testemunha}
(!...), \emph{que é itinerante}, porque do Juízo Municipal
transportou-se miraculosamente para a Chefia de Polícia, e desta para a
Secretaria da Presidência, na capital, e da Presidência para a
Promotoria de Santos; que é de responsabilidade, porque, afinal, foi ter
ao Juízo de Direito da Comarca; se tudo isto, na parte oficialmente
realizada no Juízo Municipal de Santos, não constitui começo de formação
de culpa; e se, pelo contrário, se pretende que seja mero inquérito
policial, sobe de ponto o absurdo, avulta mais, com espanto, a
monstruosidade, reparo maior determina o arbítrio, mais flagrante é a
violação da Lei; porque ou o crime é comum, e o juiz municipal,
competente para a formação da culpa, é incompetente para fazer
inquéritos policiais, que incumbem aos delegados, subdelegados e chefes
de polícia, ou o crime é de responsabilidade, e, por isso, torna-se
imprestável, por ilegal, o inquérito, que só tem cabimento nos crimes
comuns, nos termos, na forma, e sob as condições prescritas pelas
disposições em vigor.

\_\_\_\_\_\_\_\_\_

Apreciemos agora, com reflexão, calma e imparcialidade, os autos de
corpo de delito feitos no edifício da Alfândega, interna e externamente,
e no cofre-forte, onde estavam depositados os valores confiados à guarda
do tesoureiro; e vejamos de que modo foram coligidos os indícios, pelos
vestígios existentes combinados os fatos, averiguadas as violências e
preparado e julgado este elemento essencial do crime de roubo, pública e
geralmente acusado, em toda cidade de Santos, e que faz objeto deste
sumário.

Foram nomeados quatro peritos: dois mestres de obras para o exame do
edifício; dois serralheiros para o exame do cofre e; notificadas duas
testemunhas para serem presentes a este ato da mais subida gravidade e
importância, base legal do famoso processo.

Uma das testemunhas foi o \emph{sr. dr. Luiz Manoel de Albuquerque
Galvão, engenheiro notável, arquiteto de nomeada, de inteligência
elevada, de perícia provada, diretor habilíssimo}, e, com acerto,
encarregado das obras do novo edifício da Alfândega de Santos,
construindo-se quase paredes-meias\footnote{. Paredes comuns construídas
  na divisa de dois prédios contíguos.} com o antigo, em que funciona
esta repartição, e que teve a santa ingenuidade de aceitar o encargo...

Sem que de leve façamos a mínima perniciosa alusão ao caráter do muito
digno sr. dr. juiz municipal, sem que pretendamos, de modo algum, pôr em
dúvida a sisudez e a retidão do seu ato, e antes acreditando, como em
muito boa fé dizemos, na imprevista existência de um concurso fortuito
de circunstâncias, espantamo-nos desta fatal escolha, ou sinistra
notificação.

Não é intuito nosso, preciso é que o digamos, desde já, com sinceridade,
franqueza e ousadia; não é intuito nosso hastear, aqui, o estandarte
negro da calúnia, para, com astúcia, e vibrando as armas da perfídia,
defendermos a causa nobilíssima dos nossos clientes, à custa do
ignominioso\footnote{. Desonroso, deplorável.} sacrifício de alheias
reputações.

Outro é o nosso fim.

Fazemos reparo deste precipitado açodamento com que a uns, sob o fútil
pretexto da existência de vagos indícios, aliás repelidos com
tenacidade, por a geral opinião do lugar, se suspende, prende, e
pronuncia e demite, enquanto que a outros, embora com razão, ou sem ela,
malsinados pela voz pública como os autores do enorme roubo,
tauxiados\footnote{. Incrustados, embutidos. A expressão ainda carrega a
  ideia de enfeitar -- incrustar metal precioso num outro metal, por
  exemplo --, de modo que a marca gravada possuiria, sarcasticamente, um
  quê de ornamento.} na fronte pela pública reprovação, deixa-se que, à
mercê dos ventos, se façam ao largo, embalados pelas ondas, em a nau do
mistério, tripulada sempre pela indiferença e ao som do murmúrio geral.

Tornando ao que narrávamos.

Causou-nos espanto, a mais viva admiração, o fato, muito de
estranhar-se, de figurar o sr. dr. Galvão como testemunha dos aludidos
exames; espanto e admiração que bem se justificam pelos seguintes fatos:

O povo, de tropel\footnote{. Alvoroçado, agitado.}, apaixonado e
insistente, era, e ainda é unânime, se bem que não assinale, nem decline
as causas, em atribuir ao sr. dr. Galvão o fato horroroso da subtração!

Os senhores delegado de polícia, tenente Pinho e dr. Moyses de Castro o
repetem nos seus depoimentos, que deixamos transcritos. O primeiro é
agente do governo; o segundo é pessoa de elevado conceito e posição.

Os exames, em sua redação, em muitos pontos, acusam a \emph{influência
científica da notável testemunha} sobre assertos dos peritos; fato este
muito natural, porque não se ocultam os raios do Sol em pleno espaço...
Nas respostas aos quesitos, rápidas, fáceis, intuitivas e rudes,
notam-se as incongruências dos peritos, afirmando, com sinceridade, a
existência da violência e do roubo; nas descrições feitas do estado das
cousas, ao inverso das respostas aos quesitos, com regularidade de
forma, e hábito de observação, desenham-se e referem-se, com
insistência, \emph{simulações calculadas e ausência de violências}, que
determinar possam a existência de roubo!...

Por uma casual coincidência, que de outro modo não sabemos qualificar,
fato sr. dr. Galvão gira em derredor desta desastrosa ocorrência, como
uma roda, matematicamente sobre o seu eixo!

O sr. dr. Galvão, que inesperadamente foi testemunha importante nos
autos de exames, foi também chamado a depor no inquérito, naturalmente
porque, na expressão insuspeita da Lei de 1871, \emph{viu} ou
\emph{tinha motivos para saber} quem fosse o autor do crime.

Seu depoimento é dos mais simples; nada sabe, senão \emph{que o
tesoureiro Largacha oferecera-lhe madeiras à venda}!...

Há, porém, amestrados\footnote{. Peritos, hábeis.}, \emph{químicos
judiciários}, que pretendem encontrar veneno nesta declaração
calculada...

Os empregados da Fazenda são proibidos de comerciar; a venda de
madeiras, na espécie considerada, se bem que em aparência, constitui ato
de comércio. O empregado da Fazenda que se dá às práticas do comércio é
infrator voluntário da lei; logo, o tesoureiro Largacha deve ser um
funcionário suspeito às vistas da Administração, mormente\footnote{.
  Sobretudo, principalmente.} nas atuais circunstâncias em que se trata
de \emph{uma simulada subtração} de valores do \emph{cofre da
Alfândega}!...

A \emph{imputação}, porém, não tem a menor procedência; porque o
tesoureiro Largacha é proprietário, e a venda de madeiras extraídas da
sua fazenda, aos olhos da Lei, não constitui ato de comércio.

O sr. Bhering, digno inspetor da Tesouraria da Fazenda, particular e
íntimo amigo do sr. dr. Galvão, que com esse coabitava em Santos, e que
com ele passeava de braço pelas ruas, em um ofício de 8 de abril,
endereçado ao exmo. sr. conselheiro chefe de polícia, aludindo a certo
pagamento que o tesoureiro Largacha fizera ao procurador da Câmara
Municipal, para indenização de despesas feitas com variolosos\footnote{.
  Aquele que sofre de varíola.}, sob pretexto de tal pagamento se ter
realizado sem autorização, insinua interessada e intempestivamente
'\emph{que o tesoureiro, com semelhante e irregular procedimento,
comprometera-se gravemente...'}

Não acreditamos, de maneira alguma, que tais expressões fossem escritas
com solapado\footnote{. Por sentido figurado, dissimulado, disfarçado.}
sentimento; e, antes, com os íntegros juízes da causa, vemos nisto um
ato louvável de acrisolado\footnote{. Apurado, aperfeiçoado.} civismo.
Apenas lamentamos a existência deste fatal acaso, que, à semelhança da
serpente, enquanto enleia\footnote{. Amarra, prende.} a parte superior
do corpo no tesoureiro Largacha, jeitosamente afrouxa a cauda do sr. dr.
Galvão!...

Sempre as coincidências operando maravilhas espantosas!...

Enche-se a cidade de Santos e transborda até a capital que o autor do
roubo da Alfândega é o tesoureiro Largacha; que o cofre não fora
arrombado, se não aberto muito naturalmente, \emph{com a própria chave},
existente em mão desse tesoureiro. Isto repete-se entre os passageiros
nas estradas de ferro e nos hotéis, e até entre as mulheres infelizes de
péssima reputação! Afirma-se que os desmanchos do cofre e os vestígios
encontrados no edifício não passam de industriosa\footnote{. Forjada.}
simulação para encobrir desvios criminosos de quantias,
sub-repticiamente\footnote{. Clandestinamente.} praticados no cofre. E a
população de Santos indica como autor desses boatos ao sr. dr. Galvão e
um seu desenhista que o seguia nas viagens de recreio a S. Paulo,
viagens que repentinamente cessaram!...

E este fato da propagação dos boatos prova-se cabalmente com depoimentos
de testemunhas insuspeitas, inquiridas na delegacia de Santos!...

E o sr. dr. Galvão, sabendo de tal prova, corre espontaneamente à
imprensa e formalmente declara-se caluniado, vítima de precipitadas
interpretações e completamente alheio aos desastrosos boatos! E que ele
apenas conhece de vista ao major Largacha, \emph{por ocasião de
oferecer-lhe madeiras à venda}!...

E, assim, torna à baila a célebre venda de madeiras; e a terrível
coincidência, ainda desta vez, faz com que ao sr. dr. Galvão se
atribuam, com certeza e com verdade, a autoria dos boatos aterradores
contra o tesoureiro!

No dia 19 de fevereiro, logo que em Santos propalou-se a existência do
roubo da Alfândega, quando esta ocorrência pairava em todos os cérebros
e irrompia de todas as bocas, como uma centelha elétrica, quando ainda
não era sabida a importância monetária do roubo, ALGUÉM, que a despeito
de todos os esforços das autoridades, dos acusados e dos seus advogados,
não pôde ainda ser descoberto, transmitiu para a Corte, ao \emph{Jornal
do Commercio}, um telegrama anunciando o roubo?

Nesse telegrama, é a soma subtraída calculada em 180:000\$000 réis...
Mais tarde, verificou-se que a subtração era de 175:000\$000 réis.

Afirma-se (e ainda nenhum boato corria a respeito), \emph{que havia
simulação de arrombamentos com o fim de encobrir-se antigos desvios de
dinheiros!}...

A opinião pública indica ao sr. dr. Galvão e ao sr. Wursten como os
autores do telegrama, onde, porém, está a prova disso?

Como o dr. Galvão, preocupado com os exames na Alfândega, \emph{onde
servia de testemunha}, poderia ter tempo de expedir tal telegrama?

Que interesse imediato teria ele para fazê-lo?

É verdade que a contextura\footnote{. Sequência, encadeamento de ideias,
  argumentos e circunstâncias dentro de um contexto organizado.} do
telegrama referido \emph{coincide com certas opiniões} atribuídas aos
peritos, externadas nos exames da Alfândega, e também com os boatos
espalhados contra o tesoureiro, que as testemunhas juram ter ouvido ao
sr. dr. Galvão. E estes sucessos, ao que parece, não passam de enredos
dramáticos urdidos pelo acaso, e de um modo tão extraordinário, que a
muitos se afiguram como estudado meio de encobrir aos olhos da
autoridade o verdadeiro criminoso!...

A Promotoria Pública requereu ao Juízo todas as diligências precisas
para que, na Corte, se obtivesse a exibição do autógrafo do telegrama.
Nada se conseguiu!...

O major Largacha, pela mediação de um advogado hábil, chamou ao juízo
criminal \footnote{. Isto é, aduziu como prova documental.}o telegrama,
como meio de conseguir a apresentação do autógrafo.

E o \emph{Jornal do Commercio}, folha de vasta e merecida reputação, que
só assume responsabilidade em negócios de alta política, e de bem
ponderadas conveniências, negou-se à exibição requerida e apresentou
como responsável pela publicação do telegrama um improvisado editor!...
E ainda veio a terreiro de lança enristada\footnote{. Erguida,
  levantada.}, procurando tirar partido contra o infeliz tesoureiro!...

E a autoria do telegrama até hoje é um enigma!

Nesse telegrama, dá-se como já conhecida em Santos uma providência
tomada no Tesouro, \emph{minutos depois}, e secretamente comunicada ao
inspetor da Alfândega: \emph{a suspensão do tesoureiro!......}

Se o autor do telegrama não possui o dom da pré-ciência, se não é
adivinhador, ou se não é o alvo predestinado das mais extraordinárias
\emph{coincidências}, devemos supor \emph{que ele tinha dentro do
Tesouro} {[}O{]} Espírito Santo, que o instruía dos mais recônditos
segredos daquela repartição! E os metafísicos criminalistas não estarão
longe de acreditar que o autor do telegrama não ignora quem sejam os
roubadores do cofre da Alfândega...

\_\_\_\_\_\_\_\_\_\_\_

Muito maior interesse oferecem os autos de exames, estudados detidamente
em suas divisões, nos fatos, de per si\footnote{. Por si, isoladamente.},
e principalmente comparados uns aos outros, com calma e reflexão.

Não há contradição, disparate ou absurdo que aí não estejam garbosamente
aposentados, como o vencedor Aníbal em sua tenda\footnote{. Referência a
  Aníbal Barca (247-183 a.C.), general e estadista cartaginês, que é
  considerado um dos maiores estrategistas militares da história
  mundial. O sentido da metáfora, contudo, não se afigura tão claro,
  salvo pela ideia de que todas as informações repousavam nos autos.}.
Certo é, porém, que tudo tem a sua razão de ser e a sua devida
explicação.

QUANTO AO EDIFÍCIO:

-- Nenhum vestígio de violência externamente encontraram os peritos que
determinar possa a existência de escalada na gateira do telhado;
existiam telhas removidas, \emph{mas de dentro para fora}, denunciando a
existência de simulação, pelo que responderam:

-- Ao 1° quesito: que houve violência?

-- Ao 2° quesito: que consiste na remoção das telhas da gateira!!

-- Ao 3° quesito: que por esta violência foi vencido o obstáculo
existente!!!

-- Ao 4° quesito: que havia obstáculo!!!!

O que, com certo pensado atilamento\footnote{. No sentido de
  perspicácia, ou mesmo malícia.}, se lê na \emph{parte descritiva do
exame} não é o mesmo, senão exatamente o contrário do que se contém
\emph{nas respostas dos quesitos!......}

Ali está revelado o cultivado espírito da \emph{testemunha}; aqui a
resposta pesada do \emph{perito.}

Ali revela-se, envolta em a nuvem do mistério, um raio \emph{daquele
refalsado}\footnote{. Fingido.} \emph{telegrama}; aqui a sinceridade
alvar\footnote{. Ingênua.} do operário\footnote{. Antes chamado de
  perito, agora de operário, o que altera de modo significativo a
  imparcialidade do agente.}!...

\emph{As respostas aos quesitos} repelem as \emph{descrições!...}

Encontraram os peritos, no salão do cofre-forte:

-- Uma lima de três quinas, com a ponta partida de fresco;

-- Uma verruma;

-- Um formão pequeno, sem cabo;

-- Um pequeno instrumento em forma de chave, feito de arame de ferro,
com uma das pontas envolvida de arame de cobre, mais fino;

-- Mais um formão sem cabo;

-- Mais um instrumento de arame, semelhante ao já descrito;

-- Um outro dito, também de arame, tendo numa das extremidades 5
círculos de fio de cobre, em forma de flor;

-- \emph{Diversos maços de notas miúdas} esparsos pelo chão;

-- Mais notas miúdas, não emaçadas\footnote{. Empacotadas, envoltas em
  maços.};

-- Folhas de estampilhas de diversos valores;

-- Um vidro de óleo de amêndoas;

-- Uma caixa de folha vazia;

-- Um alicate;

-- Diversos instrumentos de arame, semelhantes aos já descritos;

-- Dois ferros curvos, em forma de gazua.

Todos estes objetos foram arrecadados. Todos eles pertenceram a alguém.
Não eram da repartição. É claro que para ali foram levados; não estavam
lá no dia 17 (sábado), quando fechou-se a repartição. Constituem
indícios veementes ou prova, quando conhecidos os donos, ou a
procedência: \emph{pela obra se revela o mestre}. Seriam eles
trabalhadores em Santos? Teriam vindo de fora? Estas perguntas encerram
fatos de grande alcance.

O processo, a tal respeito, é silencioso como um túmulo!...

Nem uma pesquisa, nem uma diligência, nem uma indagação!

E para quê?

Pois não estava tão claro que o tesoureiro, o inspetor, e o chefe de
seção, conluiados, tinham dado saque ao cofre?!

Pois não está plenamente provado que eles, da noite para o dia,
fizeram-se milionários!?

Para que procurar os ladrões, se já \emph{temos à mão três vítimas
aparelhadas}\footnote{. Forjadas.}?!

Em o ano de 1857, um homem distintíssimo, dos que maiores serviços hão
prestado à causa pública, com civismo e notável desinteresse, nesta
heróica província, o exmo. sr. conselheiro Furtado de Mendonça\footnote{.
  Francisco Maria de Sousa Furtado de Mendonça (1812-1890), nascido em
  Luanda, Angola, foi subdelegado, delegado, chefe de polícia e
  secretário de polícia da província de São Paulo ao longo de quatro
  décadas. Foi, também, professor catedrático de Direito Administrativo
  da Faculdade de Direito de São Paulo. A relação de Luiz Gama com
  Furtado de Mendonça é bastante complexa, escapando, em muito, aos
  limites dos eventos da demissão de Gama do cargo de amanuense da
  secretaria de polícia, em 1869. Para que se ilustre temporalmente a
  relação, tenhamos em vista que à época do rompimento público, aos
  finais da década de 1860, ambos já se conheciam e trabalhavam juntos
  há cerca de duas décadas; e, mais, Gama não rompeu definitivamente com
  Furtado de Mendonça, como erroneamente indica a historiografia, visto
  que em 1879 publicou o artigo \emph{Aos homens de bem}, defesa moral e
  política explícita do legado de Furtado de Mendonça.} foi nomeado
delegado de polícia da capital; e este ato patriótico do governo foi
geralmente considerado ``medida de salvação''!

Era então a bela e importante cidade de S. Paulo infestada de
malfeitores e ladrões e cotidianamente repetiam-se, com ousadia
incrível, os ataques às pessoas e à propriedade. Contado era o dia em
que não amanheciam três ou quatro casas de negócios arrombadas ou
saqueadas. O comércio, principalmente, estava sob o domínio do terror!

Logo que entrou de posse da delegacia, o exmo. sr. conselheiro Furtado
tratou de arrecadar os instrumentos deixados ou esquecidos pelos ladrões
nas casas saqueadas: foram-lhe enviados, pela Secretaria de Polícia,
\emph{um formão e uma baioneta}!

S. Excia., examinando a baioneta, disse: ``Isto é disfarce. Foi deixada
de propósito para desviar as atenções das autoridades.''

E, tomando o formão, acrescentou: ``Isto sim, é instrumento esquecido.''
Com este fio, vou eu fazer ``de Teceu'' neste labirinto.

No fim de oito dias, o novo delegado procedia a rigoroso recrutamento na
capital. E, ao cabo de dois meses, dava por exterminada a
matula\footnote{. Corja, ajuntamento.} de vagabundos e desordeiros e a
quadrilha de larápios!

Em tudo isto andou o tino e o trabalho do delegado. Pelo \emph{formão}
descobriu ele o \emph{dono}; e pelo dono, os seus associados!

Os crimes cessaram. A paz e a ordem restabeleceram-se. O formão foi a
chave; e o dono... ainda existe!.\ldots{}.

Nos mistérios da Alfândega, os formões, as verrumas, os frascos de óleo,
os arames de ferro, os arames de cobre, as gazuas e os alicates, tudo
falhou!!!

QUANTO AO EXAME DA PORTA DA SALA DO COFRE-FORTE:

Da \emph{descrição} colige-se\footnote{. Infere-se, conclui-se.} que a
porta não foi arrombada e que os desmanchos encontrados, aliás, de
recente data, constituem um embuste, adrede\footnote{. Premeditadamente.}
ajeitado, para simular violências. \emph{As respostas} \emph{aos
quesitos}, porém, atestam formalmente a existência de violências
indispensáveis, sem a prática das quais não se poderia ir ao cofre, nem
cometer a subtração!...

Sempre o mesmo dualismo, a mesma divergência, entre o fato e a
consequência, a mesma palpável contradição!

Eis a resposta dos quesitos:

"Ao 1°: Há vestígios de violência à porta;

"Ao 2°: Consistem no \emph{arrombamento do telhado e da porta};

"Ao 3°: Por essa violência foi vencido o obstáculo existente;

"Ao 4°: Que havia obstáculo;

"Ao 5°: (Quanto à porta) Que empregou-se força, instrumento e aparelho
para vencer o obstáculo;

"Ao 6°: Que a força foi a pressão exercida sobre a porta, não podendo
precisar quais os instrumentos ou aparelhos;

"Ao 7°: Que, do lugar em que encontraram as violências até chegar ao em
que se achava o cofre-forte, havia embaraço a vencer;

"Ao 8°: Partindo da gateira à porta, no lado, e desta ao cofre, ao
biombo, em cuja porta havia fechadura com chave e trinco;

``Ao 9° quesito: Pela força e instrumentos desconhecidos e o emprego da
chave do biombo.''

Façamos agora o exame deste exame.

Se o telhado, no lugar da gateira, não foi arrombado, como
cavilosamente\footnote{. Maliciosamente, enganosamente.} pretende-se, e
deu-se o afastamento das telhas \emph{simuladamente}, \emph{de dentro
para fora}, provado está que o subtrator, ou subtratores, não entraram
pelo telhado...

Se não entraram pelo telhado, nem pelas janelas, que estavam fechadas e
foram achadas intactas, nem pela porta principal e única externa do
edifício, este roubo é um fenômeno inextrincável\footnote{.
  Indecifrável.}, uma maravilha, ou milagre!...

O tesoureiro retirou-se da repartição no dia 17; compareceu no dia 18,
mas não entrou; estava em sua casa no dia 19, quando foi o roubo
descoberto!!!

O porteiro tinha as chaves; as portas estavam fechadas, não sofreram a
mínima violência, ninguém o acusa! Por onde, pois, entraram os
criminosos, que andaram \emph{simulando tantos e tantos trabalhosos
fingimentos?!}

De duas, uma: ou deu-se um conluio geral entre todos os empregados da
repartição, ou entre parte importante deles, e todos os estragos foram
preparados, de antemão, no dia 17 de fevereiro, no decurso das horas do
expediente, balela esta singularíssima e extravagante, com a qual ainda
ninguém sonhou, nem mesmo pressentida foi pelo atiladíssimo\footnote{.
  Espertíssimo, muito sagaz.} espírito do sr. dr. Galvão, que
aventa\footnote{. Percebe, prevê.} os roubos com a mesma perspicácia com
que os corvos descobrem a carniça; ou foram os estragos feitos em as
noites de 17 ou 18, e, neste caso, o autor ou autores entraram de fora.

Se entraram de fora, como não se pode deixar de crer, os desmanchos do
teto, na gateira, foram necessários, feitos de fora para dentro, para
dar passagem aos delinquentes; e falsa e simulada é a declaração
contrária, que muito de ia existe nos autos; e se os ladrões entraram
pela gateira, é de todo ponto claro que tiveram necessidade indeclinável
de arrombar a porta do salão para chegar ao cofre.

Se há simulações, elas visam um fim, e, por isso, e por isso mesmo
explicam o fato da sua existência.

Os ladrões eram hábeis, inteligentes, práticos, de alta condição social
e intrépidos; consideraram antes de perpetrar o crime; realizaram-no com
preucauções, \emph{armaram os meios de defesa}, e conseguiram os seus
intuitos, PORQUE DESVIARAM AS VISTAS E TODA A ATENÇÃO DOS JUÍZES!...

É deste modo único que a inteligência, o estudo e a boa razão podem,
unidos, explicar as grosseiras contradições existentes nos autos de
exames, contra a verdade dos fatos, em que figura, com a maior
inconveniência, e representando papel singularíssimo DE TESTEMUNHA,
\emph{entre pedreiros e carpinteiros, em posição inferior}, o distinto
engenheiro e considerado arquiteto, diretor de obras importantíssimas da
nova Alfândega de Santos, dr. Luiz Manoel de Albuquerque Galvão!

Isto não pressentiu, nem percebeu, nem viu, o ilustrado sr.
desembargador Accioli de Brito, relator do processo. Porque não o
estudou. E estas causas não se improvisam.

Veja o público, aprenda, aprecie, e trema diante deste modo sisudo,
grave e escrupuloso pelo qual se decide da honra, da fortuna e do
direito dos cidadãos!

QUANTO AO PRIMEIRO EXAME DO COFRE-FORTE:

Eis aqui o ponto de maior importância e para o qual devem convergir as
vistas e toda a atenção do observador judicioso.

Afirmam os peritos, neste exame, respondendo aos quesitos propostos pelo
juiz:

Que o cofre sofreu violência;

Que por essa violência foi vencido o obstáculo existente;

Que a violência, além de força empregada, teve o auxílio certo de
instrumentos apropriados.

Esta afirmação é incontestável: o cofre foi arrombado.

Cumpre, entretanto, notar que os instrumentos deixados propositalmente
na sala do cofre, pelos subtratores, serviram para os atos preparatórios
da subtração, e ali ficaram calculadamente, como elementos da planejada
defesa dos atilados, avisadíssimos roubadores.

A violação do cofre, porém, foi realizada com instrumento especial,
cautamente\footnote{. Cautelosamente, cuidadosamente.} preparado,
\emph{por oficial habilíssimo}, a aperfeiçoado e ajustado pela própria
fechadura; instrumento que, só por si, importaria o descobrimento do
fabricante, pelo que o subtrator cautamente o levou consigo, deixando
apenas gazuas imprestáveis e outros instrumentos que os peritos não
conheceram!...

As pontas de lima quebrada, as de pregos, e outros objetos introduzidos
no canhão da fechadura, o foram depois da subtração. E todos estes
fatos, \emph{praticados com certo desazo}\footnote{. No sentido de
  desleixo.}, que antes revelam \emph{propósito} do que descuido ou
negligência, acusam \emph{uma simulação tão delicada}, em seus efeitos,
que um juiz de espírito agudo não pode aceitá-la de chofre, sem reservas
muito sérias para meditadas ponderações.

A simulação existe, está patente, e é irrecusável.

O cofre foi aberto com instrumento especial, \emph{apropriado}, e não
com a própria chave, que estava em poder do tesoureiro.

O autor da abertura do cofre é o subtrator.

O subtrator é o autor da simulação.

Prossigamos no exame:

Respondendo ao 3° quesito do segundo exame, série 2ª, (\emph{Segurança
do edifício, parte interna},) declaram os peritos:

-- "Quem PENETROU na Alfândega \emph{denota conhecimento do
'edifício'".}

E, em seguida, respondendo aos quesitos propostos pelo dr. promotor
público, quanto ao terceiro, dizem:

-- ``Que antes de encravado pela lima (o canhão) quebrada dentro do
canhão, podiam passar pela fenda, nele existente, \emph{os ferros
encontrados soltos na fechadura}.''

A chave do cofre foi exibida, pelo tesoureiro, no ato de exame.

``Quem \emph{penetrou} na Alfândega denota conhecimento do 'edifício'''!

Isto é intuitivo, ao alcance do mais leve intuito\footnote{. Escopo.}, e
nem podia constituir matéria de quesito, pois, alguém se arrojaria a
invadir um estabelecimento público, da maior importância, guardado, e
nele cometer um roubo, dos mais audaciosos, sem conhecer as disposições
do edifício, sem plano ou sem cálculo?!

Que dificuldade teria o atrevido subtrator em adquirir o conhecimento
indispensável das disposições internas do prédio, que era cotidianamente
frequentado por centenares\footnote{. O mesmo que centenas.} de pessoas,
mais ou menos preocupadas com os seus afazeres e interesses?

Não está a repartição da Alfândega estabelecida em um edifício amplo,
antigo, de \emph{conhecida construção}, e funcionando em poucos e
grandes salões?

Os interessados não iam diariamente à boca do cofre fazer os seus
pagamentos?

Não era este edifício continuamente \emph{devassado} por uma turba de
operários de todas as condições e, em grande parte desconhecidos,
empregados em as obras da nova Alfândega, cuja paredes elevadas, com
relação ao velho edifício, já estavam postas a cavaleiro\footnote{.
  Refere-se, possivelmente, a uma técnica de construção.}?

É inacreditável que qualquer empregado da repartição, principalmente os
mais subalternos, se prestasse a auxiliar a perpetração do roubo.

Para conhecer as disposições do edifício, era unicamente bastante a
simples observação visual.

Vêm aqui de molde mais algumas considerações, antes que passemos ao
segundo exame do cofre.

Se a fechadura, com os objetos que foram postos dentro da respectiva
caixa, não podia funcionar, é claro que tais objetos ali foram colocados
depois de aberto o cofre, e por \emph{mero disfarce, e para, como tal,
ser tido à primeira vista}.

Os instrumentos de arame de ferro e, principalmente, os de cobre,
verruma, alicate e limas, foram de grande utilidade para a realização do
roubo.

Sobre este ponto nada perguntaram os juízes, nem disseram os peritos,
incapazes de ver e de apreciar fatos de tamanha gravidade.

O arame de cobre, que é um corpo dúctil\footnote{. Flexível.} e de fácil
maleabilidade, e sobre tudo amoldável, serviu para o exame detido da
fechadura e, por tal meio, conseguiram conhecer, \emph{com acurado
trabalho e notável perícia}, o movimento mecânico da fechadura, a forma
de certas cavidades e a posição das molas, as suas dimensões e força,
para ser calculada, com precisão, a resistência de que deveria dispor a
gazua. A grande quantidade de óleo empregado serviu para facilitar os
movimentos do trinco e linguetas sobre as peças correspondentes.

O trabalho deveria ser muito menor do que o calculado; porque não só a
fechadura é de má qualidade, com o que não contavam os \emph{operários},
ou estragada, como foi, em parte, grosseiramente modificada em um
conserto que sofreu pelo serralheiro Antonio de Padua.

O cofre foi remetido aberto, vindo a chave dentro, pela razão de que se
não fecham por si -- com uma só chave --, quando esse cofres
\emph{trazem, pelo menos, três chaves}. Mas chegou a Santos fechado, por
terem corrido não só o trinco, mas até as linguetas!... Para que isto
acontecesse preciso fora, ou que se desse estrago na fechadura, ou que
esta fosse mal construída. Porque as linguetas de uma fechadura de
qualidade não se movem se não mecanicamente, por a força propulsora da
chave ou de instrumento semelhante perfeitamente adaptado. O serralheiro
Antonio de Padua, que abriu o cofre em sua oficina e CONSERTOU A
FECHADURA, para a segurança desta, como ele próprio declara em seu
importante depoimento, \emph{fez emenda pior que o soneto}, pois, que,
encontrando \emph{cinco molas de metal} que funcionavam sucessivamente,
quer na abertura, quer no fechamento do cofre, inutilizou este movimento
de arte, reduzindo-as a duas, por adesão, pois que, unindo as duas
superiores, delas formou uma. E, por igual processo, das três inferiores
formou outra!

Ora, ninguém que tenha bom senso acreditará que uma fechadura é tanto
mais difícil de abrir quanto menos complicado é o seu maquinismo.

Um ex-oficial do sr. Antonio de Padua veio a juízo e jurou, \emph{depois
de examinar ocularmente} \emph{a chave do cofre e a fechadura}, que tal
chave e fechadura eram idênticas! Que os consertos realizados por
Antonio de Padua, \emph{e só por ele}, sem intervenção de mais ninguém,
eram os mesmos, estavam intactos, sem a mínima alteração!! Que nenhum
outro oficial interveio na operação do conserto!!! E que ele sabe de
tudo isto, e atesta, com juramento, não só a identidade das peças, se
não também a inalterabilidade das obras!!!!...

O alicate era indispensável: por ele aprimorou-se a forma da gazua, que
foi preparada, amoldada e ajustada, \emph{não pela chave}, mas pela
fechadura. As limas serviram para o mesmo efeito, e para ao instrumento
dar-se a conveniente perfeição. A espiral da verruma grande deve ter
servido para imprimir certa e apropriada forma ao arame de cobre, para o
trabalho preparatório de \emph{sondagem} da fechadura: nada do que se
encontrou era imprestável. E a prova inconcussa\footnote{.
  Incontestável, indiscutível.} do arrombamento está manifestamente
estampada no alargamento do canhão da fechadura, \emph{para serem
manejados} mais de um instrumento conjuntamente e para abrir passagem à
gazua.

A ignorância dos peritos é espantosa e constitui indiretamente\footnote{.
  Insinuação dos peritos?} a base larga em que assentam os cálculos
atiladíssimos da prevista e acertada defesa dos ladrões. \emph{O
artista} que executou o plano de abertura do cofre é dotado de
extraordinária perícia, cuja sagacidade contrasta, com grandíssima
vantagem, com a inépcia exuberante dos peritos: É OFICIAL DO SEU OFÍCIO!

Vamos agora tratar do 2° exame, feito a requerimento do digno sr. dr.
Promotor público, mediante o concurso dos senhores:

Adolpho Sydow,\footnote{. Mantenho a grafia conforme o orginal,
  ressalvando que, em escritos posteriores, que o leitor verá na
  sequência, o mesmo Adolpho Sydow reaparece -- e assina -- com essa
  mesma grafia, bem como com Adolpho Sidow.} \emph{serralheiro}, {[}e{]}

Frederico Guilherme Hersztberg, \emph{maquinista.}

Dos demais não trataremos, porque constituem papel sujo.

Este mesmo exame é uma vergonha judiciária, é uma deformidade legal, um
disparate forense e o eterno atestado da imbecilidade dos peritos.
Ouçamo-los. Sejam eles mesmos os juízes. Lavrem eles próprios a sua
fatal condenação!

"Por experimentarem, colocaram a fechadura no seu respectivo lugar, isto
é, na sua caixa grande, e \emph{retirado o canhão do seu orifício
natural}, foi introduzida uma gazua, \emph{das coligidas pelo juiz}; e,
depois de muito esforço, A LINGUETA DESCEU, ou antes, TORNOU À POSIÇÃO
que deveria tomar para abrir-se o cofre.

...............................\ldots{}.

...............................\ldots{}.

Unidas, porém, à caixa grande, a folha externa ou principal, e colocado
o canhão no seu lugar, ficando assim a fechadura como devera estar,
quando intacto o cofre, verificaram ser impossível a introdução das
gazuas coligidas \emph{por serem de diâmetro superior à largura da
entrada da fechadura, que está abaixo do canhão}.

Fizeram de um dos instrumentos de arame de ferro uma gazua, com forma
correspondente à parte da mola em que devia funcionar e, introduzida na
fechadura, não conseguiram, apesar de muitos esforços, fazer descer a
lingueta, o que verificaram com uma luz introduzida na caixa grande"!...

Destas afirmações categóricas, baseadas no exame e na experiência,
resulta a seguinte prova:

1°: Que as gazuas achadas na Alfândega vieram preparadas de fora e não
serviram para abertura do cofre, porque tinham muito maior diâmetro do
que o espaço do canhão da fechadura, o que prova que o fabricante da
gazua não conhecia a fechadura;

2°: Que a que serviu, se bem que de menor diâmetro, ainda excedia o
espaço do canhão, que para o previsto fim fora convenientemente
dilatado, como garantem os peritos;

3°: Que a parte mecânica da gazua \emph{era tão bem calculada, e de tal
modo atesta a perícia do seu autor,} que, preparando-a, em ausência da
fechadura, suposta a disparidade entre o diâmetro da mesma gazua e o
espaço do canhão, atinou com a forma, e tanto que ela serviu para mover
as linguetas da fechadura, mesmo em mãos inábeis!...

4°: Que sobe de ponto este fato espantoso quando se atenta que a
experiência dos peritos é feita por pessoas que não dispõe{[}m{]} dos
mesmos recursos intelectuais e da mesma proficiência que o dito
fabricante;

5°: Que a incapacidade os peritos, resultante de inaptidão, provém de
que, preparando eles uma gazua, modelada pela própria fechadura, não
conseguiram mover as linguetas, ao passo que o fizeram \emph{com a
imperfeita gazua deixada} pelo subtrator.

6°: Que assim provada a evidente incapacidade dos peritos para o
delicado mister de que se encarregaram, imprestáveis e fúteis são as
suas experiências e inaceitáveis as suas conclusões.

\_\_\_\_\_\_\_\_\_\_\_\_\_\_\_\_\_

Havia em Santos um serralheiro (o seu nome consta dos autos),
considerado geralmente peritíssimo em seu ofício. Mas, infeliz por
desfavores da sorte, dava-se, com excesso, às libações
báquicas\footnote{. Por sentido figurado, bebedeiras, bacanais.}, e
vivia em extrema penúria.

Este homem, a quem referiu-se o sr. Bombardo, no depoimento que se
afirma ter prestado perante o delegado de polícia, sr. alferes
Fernandes, a despeito da completa escassez de meios, quatro dias depois
da perpetração do roubo, pelo \emph{vapor S. José}, que partiu no dia 21
de fevereiro, repentinamente e sem que alguém o esperasse, fez-se de
vela para o Rio de Janeiro e não mais voltou!...

Há por aí quem mussite\footnote{. Cochiche, fofoque.}, de ouvido em
ouvido, e certos estamos que os boatos partiram da mesma pessoa que, por
motivos que o povo explica, andou propalando torpezas contra o
tesoureiro, que só este, para tirar-se de gravíssima responsabilidade,
tinha interesse nas simulações descobertas e indicadas nos autos de
exames. Porque provindo a falta do dinheiro no cofre de desvios
anteriores, só por ela era responsável o tesoureiro, só ele teria
interesse em fingir o roubo, cuja autoria a mais ninguém pode ser
atribuída. Isto, porém, é tão grosseiro, tão chato, tão infame e tão
vil, que só em alienados beócios pode ser desculpado.

É este boato propalado industriosamente\footnote{. Por sentido figurado,
  astuciosamente, maliciosamente.} pelos autores do roubo, e só na
simplicidade dos néscios\footnote{. Ignorantes, estúpidos.} pode
encontrar eco.

Os próprios peritos (os \emph{famosos} do segundo exame), postos diante
da realidade, nutriram a presunção de que só pela própria chave poderia
o cofre ser aberto!

É certo, entretanto, e eles ignoram porque não leem, que na Europa há
fabricantes que anunciam abertura de burras\footnote{. Cofres.} com
rapidez, e sem estrago das fechaduras, para o caso de extravio ou perca
de chaves.

E para convencimento dos srs. peritos e de outros que não leem, nem
sabem, mas que têm o inveterado\footnote{. Arraigado.} hábito pessimista
de tudo contrariar a esmo, por tolice ou capricho, damos, em seguida, a
tradução de um artigo.

REVISTA INDUSTRIAL -- TURIM

Nº 543 - Janeiro de 1873

O MISTÉRIO DOS COFRES

"Na exposição de Londres esteve, por muito tempo, um cofre de fechadura
de segredo muito complicada, segundo se propalava. E parece que o
fabricante tanto presumia da sua obra que julgava, por ela, ter
descoberto a pedra filosofal!

"Um dia anunciou ele que dava o prêmio de duzentas libras a quem fosse
capaz de descortinar o segredo e abrir o cofre sem a chave própria.

"Apareceu um artista, serralheiro, sem nomeada, que pouco se recomendava
pelos modos, e menos ainda pelo trajo.

"Seguro de que obteria o prêmio no caso de êxito, deixou-se revistar.
Levava consigo apenas alguns pedaços de arame!

-- ``O cofre tem algum fecho além da fechadura?'' Perguntou o
desconhecido.

-- ``Não'', tornou-lhe o fabricante.

-- ``Está fechado?''

-- ``Sim''.

-- ``Peço que se retirem e que me concedam meia hora para preparar os
meus instrumentos.''

-- "Retiraram-se, mas deixaram guardada a porta, para que ninguém mais
ali penetrasse.

"No fim de 30 minutos, eram todos chamados pelo desconhecido, que, ao
vê-los, exclamou: 'cuidei que fosse cousa mais séria!...

O cofre estava aberto de par em par!'".

Estes habilidosos são raros. Mas há deles em todos os países, para
flagelo dos que precisam ter cofres para guardar dinheiro.

\_\_\_\_\_\_\_\_\_\_\_\_\_\_\_\_

QUANTO AOS DEPOIMENTOS

Está provado que a voz pública unanimemente, com ou sem fundamento,
indica como autores do roubo da Alfândega os srs. dr. Galvão, Rodolpho
Wursten, Custodio de Tal, um serralheiro e...

Está provado que o dr. Galvão, por seu turno, desde que foi descoberto o
roubo, oficiosamente, \emph{e só ele}, por todos os meios ao seu
alcance, o atribuiu graciosamente ao tesoureiro Largacha!

Está provado que na fatura dos exames, procurou-se, com habilidade
notável, encartar\footnote{. Introduzir com esperteza, com astúcia.} a
ideia de que as violências do cofre e do edifício não existiram
realmente, e que as observadas não passam de preparado simulacro para
encobrir a verdade, \emph{que era a abertura do cofre com a própria
chave}!

Está provado que, nos autos de exame, figurou o dr. Galvão como
\emph{testemunha}!...

Está provado que, na redação da parte descritiva dos exames, apesar de
todas as cautelas guardadas adrede, se manifesta a perícia de alguém,
cujo conhecimentos científicos estão superiores aos do serralheiro, do
pedreiro e do carpinteiro!

Está provado que, entre as respostas dos quesitos e os assertos das
\emph{descrições}, há disparidades e contradições extraordinárias,
afirmando-se, \emph{nas descrições}, a existência de calculada simulação
e, \emph{nas respostas aos quesitos}, a existência real de violências e
de roubo!

Está provado que, no telegrama secreto, expedido misteriosamente de
Santos, ``às 11 horas do dia 19 de fevereiro'', ao \emph{Jornal do
Commercio}, foi mencionada esta circunstância de simulação, enquanto sob
segredo de justiça se procedia aos exames, \emph{que concluíram-se alta
noite}! Que se anunciou a importância da quantia subtraída {[}de{]}
180:000\$000 réis, quando é certo que \emph{só pela uma hora da tarde}
foi verificado tal resultado! Que se menciona a suspensão do tesoureiro,
que foi ordenada pelo Governo em telegrama reservado, 10 ou 15 minutos
depois da expedição daquele outro telegrama de Santos!!

Está provado que a população de Santos, em peso, atribui aquele
telegrama aos srs. dr. Galvão e Rodolpho Wursten; e era opinião geral,
em Santos, ser o sr. Wursten o correspondente do \emph{Jornal do
Commercio} em Santos, naquela época!...

Está demonstrado, com incontestável evidência, que entre os boatos
espalhados pelo dr. Galvão e os \emph{seus íntimos}, inclusive Sebastião
Navarro e o inspetor da tesouraria, as \emph{descrições dos exames}
aludidos e aquele celebérrimo\footnote{. Superlativo de célebre, algo
  como muitíssimo célebre.} telegrama, há uma filiação misteriosa, um
liame\footnote{. Vínculo, ligação.} industrioso\footnote{. Ardiloso,
  astucioso.}, uma trama sibilina\footnote{. Obscura, enigmática.}, que
revelam a determinação de um acordo, de um concerto de elevado alcance,
de um plano preconcebido, cujo resultado é: responsabilizar-se o
tesoureiro Largacha pelo roubo da Alfândega!...

Está provado que os subtratores do dinheiro da Alfândega, depois da
abertura do cofre, trataram de encobrir as violências, por meio de
preparadas simulações e de modo tal, que estas dessem na vista com o
determinado intuito de desviar as vistas da autoridade e encobrir os
culpados!

Está provado que entre os subtratores \emph{havia serralheiro perito},
especialista em trabalhos de \emph{fechaduras de qualidade}; e tal prova
resulta, de modo irrecusável, dos instrumentos encontrados no salão da
Alfândega. Instrumentos que não foram conhecidos nem apreciados pelos
peritos, que, com isto, exibiram prova inconcussa da sua incapacidade!

Está provado que nem uma diligência se fez para descobrimento da
procedência de tais instrumentos, diligências da maior importância, e
que deveriam ter sido das primeiras, cuidadosamente ordenadas pela
autoridade!

Está provado que, em Santos, havia um serralheiro perito, extremamente
pobre e vicioso \emph{que, três dias depois da perpetração do roubo},
inesperadamente se mandou mudar para o Rio de Janeiro!

Está provado que, pelo mesmo paquete, seguiu com destino ao mesmo lugar
o sr. dr. Galvão!

Está provado que, nessa ocasião, sofreram minuciosa revista a bordo as
bagagens de todos os passageiros.

Está provado que uma bagagem houve, privilegiada, que, nessa ocasião,
deixou de sofrer revista a bordo e foi revistada, com surpresa e por
favor, na casa do delegado de polícia, de onde saiu acompanhada pelo
respectivo ordenança, com recomendação \emph{de passar, porque já estava
examinada}!...

Está provado que essa bagagem saiu da casa do sr. Rodolpho Wursten --
\emph{Azevedo \& Companhia}!

Está provado que essa bagagem \emph{incógnita} passou como pertencente a
\emph{Guilherme Kronlsin}, alemão, e que tal nome não está mencionado na
lista de passageiros dessa viagem do paquete \emph{S. José}!...

Está provado que nessa mesma ocasião seguiram com destino à Corte o dr.
Galvão, o serralheiro e a incógnita bagagem!...

Está provado que todas as autoridades, desde o inspetor da tesouraria,
de olhos fechados, obstinam-se em fazer crer, a despeito de quanto fica
relatado, que o ladrão da Alfândega \emph{só podia ser} o tesoureiro
Largacha!!

Está provado que contra Largacha NENHUM INDÍCIO EXISTE, e que este foi
suspenso, processado, \emph{pronunciado}, demitido, revogando-se, para
isso, uma judiciosa sentença de não pronúncia proferida pelo douto e
honrado juiz de direito da Comarca de Santos!

Está provado que contra aqueles que são indigitados pela pública opinião
e contra os quais pesam indícios graves, que de modo algum foram ainda
destruídos, se não deu até hoje o menor procedimento!!!

Está provado que o inspetor Antonio Justino de Assis foi pronunciado
pelo colendo Tribunal da Relação como incurso no artigo 154 do Código
Criminal\footnote{. Art. 154. ``Deixar de cumprir, ou de fazer cumprir
  exatamente qualquer lei, ou regulamento. Deixar de cumprir, ou fazer
  cumprir, logo que lhe seja possível, uma ordem, ou requisição legal de
  outro empregado''.}, e que é equívoca, senão ininteligível, tal
pronúncia, porquanto essa disposição de lei contém diversas e distintas
hipóteses, como sejam:

-- Deixar de cumprir;

-- Deixar de fazer cumprir, com exatidão, qualquer lei ou regulamento;

-- Deixar de cumprir;\footnote{. Não se trata de erro tipográfico. Os
  autores reiteram os comandos normativos da ordem que o art. 154
  estabelece.}

-- Deixar de fazer cumprir, \emph{logo que lhe seja possível}, uma ordem
ou requisição legal de outro empregado.

Está provado, por mais este fato inexplicável, além de outros muitos,
que ficam considerados e, principalmente, por as desarrazoadas razões do
venerando Acórdão de 19 de Outubro, que a pronúncia do tesoureiro e do
inspetor da Alfândega não tira os seus fundamentos do corpo dos autos,
onde nenhum existem que a possam justificar. Mas, fora extorquida pela
intriga e por boatos espalhados à socapa\footnote{. Por sentido
  figurado, o mesmo que maliciosamente.}, por funcionários públicos que
esforçam-se pelo comprometimento daqueles dois empregados, e pela
salvação de indivíduos apontados como criminosos, dos quais a inocência
foi elevada à categoria de postulado, e o postulado convertido em dogma,
e contra os quais ainda nenhum procedimento foi iniciado!

Não está provado! Mas nós garantimos sob nossa responsabilidade que o
sr. dr. Galvão, em Araraquara,\footnote{. Cidade do interior paulista
  que dista 270 km da capital.} caluniou covarde e grosseiramente ao
respeitável engenheiro sr. dr. Pimenta Bueno. Que a calúnia teve por
alvo 'desacreditar' e tirar o prestígio ao ilustre ofendido, afim de que
o governo imperial se visse forçado a demiti-lo do lugar de chefe de uma
expedição científica, que então desempenhava. Que o dr. Galvão assim
procedia visando, para si, a nomeação do alto cargo que ocupava aquele
seu distinto colega de profissão. Que, sendo processado, o dr. Galvão
foi ao Tribunal do Júri e ali, com desplante inacreditável, retratou-se,
com espanto do auditório, de tudo quanto havia dito e escrito contra o
dr. Pimenta Bueno, que desistiu da acusação! Que tão extraordinário,
imprevisto e degradante foi este procedimento, que o dr. Galvão teve de
retirar-se de Araraquara para evitar os efeitos do público desagrado de
que se tornou alvo.

Que, isto posto:

Está provado não só que o dr. Galvão não é digno de ser crido nos boatos
que divulga contra o tesoureiro, nem tampouco os seus amigos, tão
suspeitos como ele, como principalmente porque em semelhante
procedimento repulsivo ele revela oculto interesse, pouco recomendável
ante a dignidade e a honradez. Porque, em face dos seus precedentes, não
é crível que ele diga mal do tesoureiro só pelo satânico prazer de
difamá-lo quando é certo que mal o conhece.

Está provado que os roubadores da Alfândega conheciam, de antemão, o
conceito em que eram tidos perante o Tesouro Público Nacional os
empregados da Alfândega de Santos, em razão das \emph{informações
secretas} do sr. inspetor da tesouraria. Que as simulações preparadas no
cofre e no edifício estão de harmonia com este prévio conhecimento. Que
a preparada e propalada imputação do roubo atribuída ao tesoureiro
estriba-se em estas informações. Que os subtratores, à semelhança do
Argos da fábula\footnote{. Na mitologia grega, Argos Panoptes foi um
  gigante que tinha cem olhos e, mesmo dormindo, mantinha metade de seus
  olhos abertos e atentos. No contexto, a metáfora representa,
  ironicamente, a vigilância dos ladrões, que obervavam repartições
  administrativas, policiais e judiciárias.}, até hoje, observam o que
vai pelos tribunais e não ignoram o que se passa no Tesouro, na Corte e
na Tesouraria, em S. Paulo. Que, pela mediação de amigos, fazem chegar
nos ouvidos dos juízes boatos contrários à reputação do tesoureiro. Que
toda a base da defesa dos roubadores, perfeitamente combinada, e ainda
de melhor modo realizada, consiste nas opiniões manifestadas ao governo,
\emph{em reserva}, pelo sr. inspetor da tesouraria. Que foi o sr. dr.
Galvão quem, por diversas vezes e pela mediação de amigos, pretendeu
imodestamente elogiar-se, a si mesmo, para de sobre si arredar as
imputações criminosas que lhe eram feitas com relação ao roubo da
Alfândega, por artigos mandados a certo \emph{jornal} desta cidade. Que
foi o sr. dr. Galvão quem, na \emph{Província de São Paulo}, nº 709, de
26 de junho, ainda pela medição de um amigo, \emph{repeliu
astuciosamente} a autoria das indignidades atribuídas ao tesoureiro da
Alfândega, enquanto alguns empregados da tesouraria, a exemplo do seu
inspetor, inseparável do sr. dr. Galvão, segredavam, de casa em casa,
com criminosa deslealdade, que o autor do roubo era o major Largacha.

Está provado que os roubadores, os principais, são pessoas de elevada
condição civil, de inteligência pouco vulgar, de trato social, dotados
de atividade, amestrados e de incontestável influência e prestígio!

Está provado, finalmente, que o venerando Acórdão de 19 de Outubro é
iníquo, porque julgou contra a verdade dos fatos. É injusto, porque
violou a expressa disposição da lei. É desumano, porque, sem provas, sem
indícios, e por meras suspeitas injustificáveis, sem madureza, sem
ponderação, sem o mínimo fundamento, sujeitou à prisão, dificultou meios
de defesa, negando \emph{habeas-corpus}, negando fiança legal,
interpretando absurdamente o direito, perseguindo sem conveniência,
torturando, com surpresa e sem razão, homens que, pela sua boa
reputação, pela sua idade, pela sua posição e pelo alto conceito de que
sempre foram dignos, estavam no caso de merecer os rigores da justiça,
que não é, por certo, o vilipêndio da inocência e o desprestígio
caprichoso do cidadão.

E para a prova de tudo quanto temos dito e escrito, aí ficam os
documentos para serem cotejados com o venerando Acórdão, que
reimprimimos.

-- "Acórdão em Relação, etc, que relatados e discutidos estes autos na
forma da lei, dão provimento ao recurso interposto
\emph{ex-officio}\footnote{. Realizado por imperativo legal e/ou por
  dever do cargo ou função.} pelo juiz de direito de Santos, do despacho
de não pronúncia a fl.~8, 6 v{[}erso{]}, que proferia a favor dos réus
Antonio Eustachio Largacha, tesoureiro; Antonio Justino de Assis,
inspetor; João Baptista de Lima, chefe de seção da Alfândega da mesma
cidade. Quanto aos dois primeiros réus, e quanto ao terceiro, negam
provimento. Julgam procedente a denúncia do promotor público PELA
VIOLAÇÃO DO COFRE da Alfândega e o desfalque nele encontrado da quantia
de 185:650\$679 réis, em moeda e estampilhas, contra o dito tesoureiro.
Porquanto a mesma denúncia acha-se DEVIDAMENTE BASEADA NO CORPO DE
DELITO CONSTANTE DOS EXAMES A FL.\footnote{. Os desembargadores não
  indicaram o número da folha.} NAS PARTICIPAÇÕES E BALANCETES do cofre
pelas competentes autoridades da repartição fiscal, reconhecendo, aliás,
os réus, o elemento material do crime com relação ao cofre e aos valores
do mesmo subtraídos. No entanto, \emph{deu-se na causa um concurso de
indícios de suma gravidade sobre a criminalidade do mesmo tesoureiro,}
como incurso em peculato, no qual assaz se fundamentasse (sic) sua
pronúncia. Verificou-se, pelos exames judiciais a fls. 12 e 23, não
consignar-se da inspeção exterior do edifício da Alfândega vestígio
algum de violência e escalada que nele se praticasse, apenas em um vão
do telhado acharam-se as telhas que o cobriam corridas para o lado de
baixo, mas isso parecendo ter sido feito por pessoa da parte de dentro.
No interior do mesmo edifício, RASTRO DE PESSOA ALGUMA, ou de sinal de
violência nos compartimentos superiores até a porta que abre-se para o
salão do cofre não foram também descobertos. Essa porta, \emph{que
estava aberta} e que havia provavelmente sofrido força, mas conservava
ainda a chave na fechadura o lado de dentro. Alguns ferros ou
instrumentos, que encontraram esparsos pelas proximidades do cofre,
\emph{préstimo algum poderiam ter para abrir o mesmo}. O cofre estava
aberto, mas somente com vestígios de violência na parte exterior do
canhão da fechadura, achando-se o aparelho interior intacto. Por
consequência, ou com a própria chave, ou alguma moldada pelo feitio dela
e gazua apropriada à ENGENHOSA fechadura, fora o cofre aberto. A
informação do perito na Corte, a fl.~566, diz que, a menos de não ter-se
obtido a forma da fechadura e chave do fabricante, \emph{não poderia
mesmo algum profissional falsificar uma chave ou gazua que abrisse tal
cofre.} (!!!)

Ora, o dito tesoureiro era o único que tinha a chave do cofre, o que ele
próprio, bem como o seu fiel\footnote{. Isto é, o inspetor Antonio
  Justino de Assis, auxiliar de Largacha na alfândega de Santos.}, a
fl.~492 v{[}erso{]} e 493 confirmam. Acresce que o os supramencionados
vestígios traem uma simulação de roubo como para supor-se abertura do
cofre por efração, e não com a chave. A conclusão a tirar-se de
\emph{tal prova circunstancial} é que a suspeita mais grave e veemente
recai sobre o dito tesoureiro de ter sido autor desse crime. Se pode
escapar pela tangente admissível de que houvesse chave falsa para
abrir-se o cofre, as provas do processo estabelecem que \emph{é uma
hipótese que deve ficar a cargo do indiciado, dando-se-lhe os meios mais
amplos de uma justificação em processo plenário} (!!!), ainda mais
sob{[}re{]} outros indícios. Quanto ao denunciado inspetor da Alfândega,
\emph{o julgam como negligente em ter deixado de cumprir e fazer cumprir
o Regulamento da Alfândega com relação ao enorme prejuízo do Tesouro}
(!!!). Porquanto, prova-se com os documentos a fl.~123\footnote{. Ou
  fl.~125.}, 128, e depoimentos a fl.\footnote{. Folha sem numeração.},
que, devendo remeter por conta dos saldos da Alfândega, pelo paquete da
linha do Sul, de fevereiro último, a quantia de 180:000\$000 {[}réis{]},
deixou de fazer essa remessa alegando depois um pretexto para
desculpar-se. Mas não o julgam{[}os{]} conivente no peculato cometido
pelo dito tesoureiro, por não provar-se a conexão moral dessa sua
negligência com este último crime. Portanto, revogam a não pronúncia
constante do despacho recorrido, para julgar, como julgam, procedente a
denúncia contra os ditos tesoureiro e inspetor da Alfândega de Santos. E
os pronunciam: o primeiro, como incurso no art. 170; o segundo, no art.
154 do Código Criminal. Aquele, sujeito à prisão nos termos do Decreto
de 5 de Dezembro de 1849; e a livramento, ambos, pagas as custas pelos
réus. Sustentam o despacho de não pronúncia a favor do terceiro réu,
pelos seus fundamentos, conforme a direito e ao que consta dos
autos.\footnote{. Respectivamente, art. 170. ``Apropriar-se o empregado
  público, consumir, extraviar, ou consentir que outrem se aproprie,
  consuma ou extravie, em todo ou em parte, dinheiro ou efeitos
  públicos, que tiver a seu cargo''. Art. 154. ``Deixar de cumprir, ou
  de fazer cumprir exatamente qualquer lei, ou regulamento. Deixar de
  cumprir, ou fazer cumprir, logo que lhe seja possível, uma ordem, ou
  requisição legal de outro empregado''. Sobre as hipóteses de prisão
  conforme o decreto de 1849, art. 2°. ``Em especial observância do Tít.
  3º, § 2º, e Tít. 7º, §§ 9º, 10º e 11º do referido Alvará
  {[}28/06/1808{]}, o ministro e secretario de estado dos Negócios da
  Fazenda e presidente do Tribunal do Tesouro Público Nacional, na
  corte, e os inspetores das tesourarias nas províncias, podem e devem
  ordenar a prisão dos tesoureiros, recebedores, coletores, almoxarifes,
  contratadores e rendeiros quando forem remissos ou omissos em fazer as
  entradas dos dinheiros a seu cargo nos prazos que pelas leis e
  regulamentos lhes estiverem marcados''. Art. 3°. ``Para se efetuarem
  estas prisões nos casos do artigo antecedente, o presidente do Tesouro
  na corte ordenará, e os inspetores das tesourarias nas províncias
  deprecarão por seus ofícios às autoridades judiciárias que as mandem
  fazer por seus oficiais, e lhes remetam as certidões delas''. Art. 4°.
  ``Estas prisões assim ordenadas serão sempre consideradas meramente
  administrativas, destinadas a compelir os tesoureiros, recebedores,
  coletores ou contratadores ao cumprimento de seus deveres, quando
  forem omissos em fazer efetivas as entradas do dinheiro público
  existente em seu poder; e por isso não obrigarão a qualquer
  procedimento judicial ulterior''. Art. 5°. ``Verificadas as prisões, o
  presidente do Tesouro e os inspetores das tesourarias marcarão aos
  presos um prazo razoável para dentro dele efetuarem as entradas do
  dito dinheiro públicos a seu cargo, e dos respectivos juros devidos na
  conformidade do art. 43 da Lei de 28 de Outubro de 1848''. Art. 6°.
  ``Se os tesoureiros, recebedores, coletores e contratadores depois de
  presos não verificarem a entrada do dinheiro público no prazo marcado,
  se presumirá terem extraviado, consumido ou apropriado o mesmo
  dinheiro e, por conseguinte, se lhes mandará formar culpa pelo crime
  de peculato, continuando a prisão no caso de pronúncia e mandando-se
  proceder civilmente contra seus fiadores''.}

São Paulo, 19 de Outubro de 1877.

A. L. Gama\footnote{. Agostinho Luiz da Gama (?-1880), nascido na
  província do Mato Grosso, foi político e magistrado. Exerceu os cargos
  de juiz municipal, juiz de direito e desembargador do Tribunal da
  Relação de São Paulo. Foi chefe de polícia das províncias da Bahia,
  Pernambuco e na Corte (Rio de Janeiro), além de presidir a província
  de Alagoas.}, P{[}residente{]}.

A. Brito\footnote{. Luiz Barbosa Accioli de Brito (1825-1900) nasceu no
  Rio de Janeiro (RJ), foi juiz municipal e de órfãos, juiz de direito,
  desembargador e ministro do Supremo Tribunal de Justiça.}, vencido
sobre a pronúncia do segundo réu, julgando-o incurso no art. 170 do
Código, em vista das provas.

Mendonça Uchôa.\footnote{. Ignacio José de Mendonça Uchôa (1920-1910),
  nascido na província de Alagoas, foi promotor público, juiz municipal
  e de órfãos, juiz de direito, desembargador dos tribunais da relação
  de Porto Alegre e de São Paulo, além de procurador da Coroa, Soberania
  e Fazenda Nacional e ministro do Supremo Tribunal de Justiça.}

J. P. Villaça.\footnote{. Joaquim Pedro Villaça (1817-1897), nascido na
  província de São Paulo, foi promotor público, juiz municipal e de
  órfãos, juiz de direito, desembargador dos tribunais da relação de
  Ouro Preto e de São Paulo, onde também foi presidente do tribunal,
  além de ministro do Supremo Tribunal de Justiça.}"

S. Paulo, 14 de novembro de 1877.

Os advogados,

DR. RIBEIRO CAMPOS.\footnote{. José Emílio Ribeiro Campos foi
  jornalista, fundador e redator do \emph{Diário de Santos} (1872),
  promotor público e advogado.}

L. GAMA.

\textbf{16. 1. AOS SRS. ADVOGADOS RIBEIRO CAMPOS E LUIZ GAMA
{[}réplica{]}}\footnote{. In: \emph{A Província de S. Paulo} (SP), Seção
  Livre, 20/11/1877, p.~2.}

\textbf{*didascália*}

\emph{Surge uma réplica contra a obra jurídica de Gama e Ribeiro Campos.
Réplica que, na verdade, teria como alvo mesmo o caráter -- e o
conhecimento -- de Gama. O serralheiro Sidow, um dos peritos auxiliares
da Justiça, revoltou-se contra os advogados de Largacha, que, segundo
ele, ``por excessivo e mal entendido calor na defesa de seus clientes'',
o teriam injuriado e feito juízo equivocado de seu trabalho e da sua
reputação. Embora breve, a carta de Sidow é bastante reveladora de
relações -- e inimizades -- antigas. Gama e Sidow eram velhos
conhecidos. Assim, o próprio motivo levantado por Sidow -- o tal
excessivo e mal entendido calor dos advogados no patrocínio da defesa
--, era apenas uma camada de verniz retórico numa história que já vinha
de muito tempo, desde a adolescência de Gama, tempo em que viveu
escravizado na casa do alferes Cardozo, no centro de São Paulo.
Cinicamente, Sidow dizia-se ``pronto a aceitar dos srs. advogados
Ribeiro Campos e Luiz Gama, algumas lições de direito, sendo que, este
último, recordando-se dos tempos idos, pode completar a minha educação
artística dando-me algumas lições de sapataria, em retribuição do que
dar-lhe-ei algumas lições de mecânica e serralheria''. A menção ao
passado de sapateiro de Gama --que ele não renegava, antes ainda se
orgulhava -- fugia das raias do processo -- evidente sinal de que o
contendor acusava o golpe -- e convertia-se em agressão pessoal e
barata. É certo, porém, que Gama não deixaria barato. }

\begin{center}\rule{0.5\linewidth}{\linethickness}\end{center}

No suplemento distribuído ontem pela \emph{Província de S. Paulo}, estes
srs. advogados, por excessivo e mal entendido \emph{calor} na defesa de
seus clientes, no processo da Alfândega de Santos, entenderam que podiam
assacar\footnote{. Imputar.} injúrias ao desconhecido e modesto artista
que se prestou desinteressadamente a ser perito, por ordem do chefe da
casa onde se achava como mestre.

Com interesse ou sem ele, a minha decisão seria sempre a mesma, não
admitindo eu que os srs. advogados, para inocentar os seus clientes,
queiram, insultando, desmoralizar o parecer que foi dado.

Não tenho tempo para discussões, nem quem m'as\footnote{. Me as.} pague
para sustentá-las; cumpre-me, porém, declarar aos srs. advogados que são
mais imbecis aqueles que revolvendo sempre no monturo\footnote{.
  Entulho, montueira, depósito de lixo.} das leis da chicana\footnote{.
  Nesse caso, palavra que representa pejorativamente o direito e suas
  ordens normativas, atores, costumes e instrumentos.}, têm a
veleidade\footnote{. Fantasia, presunção.} de julgar que podem confundir
ao artista que sabe de seu ofício, e que não pode permitir que os srs.
advogados entrem em matéria em que são leigos e inteiramente ignorantes.

Devolvo-lhes, portanto, o apelido de imbecil que ``como bom filho à casa
torna.''

Como, porém, sou modesto e direi mais, condescendente, estou pronto a
aceitar dos srs. advogados Ribeiro Campos\footnote{. José Emílio Ribeiro
  Campos foi jornalista, fundador e redator do \emph{Diário de Santos}
  (1872), promotor público e advogado.} e Luiz Gama, algumas lições de
\emph{direito}, sendo que, este último, recordando-se dos tempos idos,
pode completar a minha educação artística dando-me algumas lições de
sapataria, em retribuição do que dar-lhe-ei algumas lições de mecânica e
serralheria.

A César o que é de César.

S. Paulo, 19 de Novembro de 1877.

ADOLPHO SYDOW.

\textbf{17. AO SR. ADOLPHO SIDOW}\footnote{. In: \emph{A Província de S.
  Paulo} (SP), Seção Livre, 21/11/1877, p.~2.}

\textbf{*didascália*}

\emph{Tão logo a réplica de Sidow fora publicada, Gama redigiu a sua. A
contenda atinge níveis de aspereza raras vezes vistos nos jornais. Gama
mantinha a afirmação de que o ``imperito sr. Sidow'' havia lavrado um
``esdrúxulo parecer'', mas aumentava o fervor dizendo que Sidow ``fora
um perito imbecil'' e a prova para tal era o tal exame do cofre feito
pelo serralheiro, discutido, aliás, no artigo precedente. Porém, se é
verdade que Gama continuava a investida sobre o ``improvisado perito'',
rebatendo agora que Sidow carecia até ``de capacidade para saber o que
fez; e, mais simples, para entender o que leu'', era ao passado que ele
trataria de responder de modo ainda mais incisivo. Para isso, Gama
abriria sua caixinha de segredos e contaria um ou dois lances que
dormitavam ocultos para a historiografia. Ele revela o nome de seu
mestre sapateiro, a pessoa que o ensinou e guiou pelos caminhos do
ofício que por muitos anos exerceu. O ``velho e honrado Marcellino Pinto
do Rêgo, meu amigo e meu digno mestre'', afirmava Gama, era uma figura
tão exemplar -- e tão central ao desenvolvimento intelectual do jovem
Gama --, que trazer o seu nome para o debate demonstrava quão ridícula
era a pretensão de tentar ofender-lhe chamando-o de sapateiro. É
verdade, sr. Sidow, que fui sapateiro (...), fiz sapatos para alguns
parentes de S. S.``, relembrava Gama, destacando na sequência, em grande
estilo, que no presente ele era um dos mais afamados advogados da cidade
de São Paulo. O arremate, contudo, é digno das laudas da história.
Vejamos:''Sou advogado nesta mesma terra em que S. S. foi e ainda é
serralheiro, sr. Sidow; e, permita que eu lhe diga, sem ânimo de
ofendê-lo, e só para glória dos artistas que se distinguem pelo talento:
sou advogado entre advogados, nesta mesma cidade em que S. S. é um parvo
Dulcamara entre os modestos serralheiros". }

\begin{center}\rule{0.5\linewidth}{\linethickness}\end{center}

Acabo de ler as torpezas que a S. S. aprouve\footnote{. Dignou.}
\emph{mandar escrever} contra mim, à míngua de habilitações para fazê-lo
por si, e que vêm insertas na \emph{Província} de hoje.

Sim senhor; está obra de malho\footnote{. Há duas possibilidades que se
  adequam (e complementam): um tipo de martelo, de cabeça pesada, que se
  pega com as duas mãos, ou, por sentido figurado, ataque veemente,
  calúnia.} e sobremodo digna de quem a inspirou; entretanto, eu nunca
pensei que a pesada bigorna do ferreiro se pudesse firmar \emph{sobre o
monturo das leis da chicana!...}\footnote{. Gama retorque utilizando uma
  expressão do artigo anterior assinado por Sidow.}

Eu disse, qualificando o esdrúxulo parecer do imperito sr. Sidow, e o
repetirei sempre, que S. S. \emph{fora um perito imbecil}, e dei a prova
do meu asserto: é a cópia do seu parecer.

E o sr. Sidow, que julga tão hábil jurista, como apurado serralheiro
entendeu, de si para si, que eu pensadamente fiz injúria ao seu caráter
de artista!... É que o sr. Sidow, a despeito do seu mal entendido
orgulho, carece de capacidade para saber o que fez; e, mais simples,
para entender o que leu.

Eu não me fiz cargo de escrever para o sr. Sidow; se não para um público
ilustrado.

Rejeito as lições de SERRALHERIA (!), que me oferece o imodesto sr.
Sidow, e muito lhe agradeço o favor; não que eu desdenhe a nobilíssima
profissão; mas porque tenho em menosprezo as parcas habilitações do sr.
Sidow, que pode ser improvisado perito por todos os juízes da província;
que poderá subir à elevada categoria de \emph{Salomão de bigorna e
malho}; mas que nunca será meu mestre.

Lições de SERRALHERIA, sr. Sidow, bastam para sua imortalidade, as que
S. S. deixou estampadas naquele memorável exame do cofre da alfândega de
Santos!...

Também não lhe posso dar lições do meu ofício de sapateiro; não só
porque S. S. dá exuberante prova da sua nativa rudeza, como
principalmente porque não estou disposto a desasná-lo\footnote{.
  Ensiná-lo, corrigi-lo.} à \emph{tira-pé}.\footnote{. Correia de couro
  com que os sapateiros seguram o calçado sobre a forma.}

Agora os meus respeitosos cumprimentos.

É verdade, sr. Sidow, que fui sapateiro; e, ali na travessa do Rosário
ainda mora o velho e honrado Marcellino Pinto do Rêgo, meu amigo e meu
digno mestre.

Fui sapateiro, sr. Sidow, ofícios às vezes igual ao do ferrador; fiz
sapatos para alguns parentes de S. S.; hoje sou \emph{advogado} e bem
conhecido nesta cidade de S. Paulo.

Sou advogado nesta mesma terra em que S. S. foi e ainda é serralheiro,
sr. Sidow; e, permita que eu lhe diga, sem ânimo de ofendê-lo, e só para
glória dos artistas que se distinguem pelo talento: sou advogado entre
advogados, nesta mesma cidade em que S. S. é um parvo\footnote{. Idiota,
  imbecil.} Dulcamara\footnote{. A referência revela outro traço da
  assombrosa erudição de Gama. Personagem de \emph{L'elisir d'amore}
  (1832) -- ópera cômica de autoria do compositor italiano Gaetano
  Donizetti (1797-1848), com libreto do poeta genovês Felice Romani
  (1788-1865) --, o ``médico ambulante'' Dulcamara foi um charlatão que
  prometia mundos e fundos e dizia ter a cura -- mediante dinheiro... --
  para todos os males da terra. A ópera-bufa tem por cenário uma pequena
  aldeia no País Basco do século XVIII. Ao chegar no vilarejo, o
  charlatão Dulcamara passa a anunciar licores mágicos, elixires
  milagrosos, entre outras extravagâncias, alcançando sucesso através do
  seu curioso ofício de ludibriar pessoas simples. A associação entre
  Sidow e Dulcamara é bastante sugestiva. Com a elegância da alusão
  literária, Gama impinge a pecha em Sidow de um tipo de mentiroso que
  ganhava a vida fazendo propaganda de si mesmo sem ter competência
  alguma no que dizia ter pleno domínio.} entre os modestos
serralheiros.

S. Paulo, 20 de Novembro de 1877.

L. GAMA.

\textbf{17. 1. AO SR. LUIZ GAMA {[}réplica{]}}\footnote{. In: \emph{A
  Província de S. Paulo} (SP), Seção Livre, 22/11/1877, p.~2.}

\textbf{*didascália*}

\emph{Sidow volta à carga. Se antes a agressão barata, devidamente
respondida, havia passado de qualquer limite do razoável, pode-se dizer
que a estupidez de Sidow, por sua vez, não possuía limites. Nesse texto,
o deplorável ataque que Sidow desfere não encontra precedente tão
perverso na história dos achincalhes que Gama sofreu na imprensa de São
Paulo. E olha que Gama enfrentara todo tipo de ruindade humana nas
páginas dos jornais, a exemplo do poderoso senhor de escravizados -- e
torturador -- Raphael Tobias de Aguiar, na célebre ``\emph{Questão do
pardo Narciso}''. Sidow e quem o assessorava -- por que não dizer? --
agredia Gama por todos os flancos: insinuava que ele não teria o
conhecimento que dizia ter, haja vista alguém ``detrás da cortina'' ter
de lhe soprar o caminho; sugeria que alguém pagava suas contas (aliás,
Tobias de Aguiar o acusara do mesmo...); que Gama vivia de marrar e
berrar, i.e., que falava demais; que desprezava e ridicularizava quem
não concordava consigo próprio, etc. e etc. Porém, certamente nenhum
ataque foi tão vil quanto o escárnio sobre a orfandade de Gama. ``Bem é
que S. S. fosse o ferrador de meus parentes'', dizia Sidow, ``já que
nunca encontrou os seus para fazer obra mais perfeita''. A covardia sem
rival passava das raias do insulto e transformava-se na mais virulenta e
odiosa agressão dirigida contra o trauma pessoal, sobre o qual Gama mais
de uma vez confessou ter chorado. }

\begin{center}\rule{0.5\linewidth}{\linethickness}\end{center}

Veio hoje o sr. Luiz Gama à imprensa e começou de dizer que foi obra de
encomenda o artigo a que responde.

Disso não lhe dou satisfações, assim como muita gente não as dá, quando
em autos e em defesa de causas sob o seu patronato se inspiram (sic)
n'um \emph{vulto} que detrás da cortina lhes (sic) serve de
espírito-santo de orelha.

Cá e lá más fadas há.

Não estou, como já disse no meu anterior artigo, para sustentar
discussões e não tenho quem m'as pague; berre e marre S. S. até quando
quiser e contra minha humilde individualidade; não destruirá com isso a
opinião que a meu respeito fazem distintos profissionais.

Não lhe agradeço o não estar disposto a \emph{desasnar-me à tira-pé},
pois que da minha parte estou com a bigorna pronta a ir de encontro às
suas marradas\footnote{. Cabeçadas ou, no sentido pejorativo da
  contenda, chifradas. Esse é mais um elemento que explicita a
  estereotipação animalizada que oponentes de Gama lançavam contra ele.}.

Bem é que S. S. fosse o ferrador\footnote{. O termo apropriado, parece,
  seria sapateiro. No entanto, é provável que o emprego do termo tal
  como lido carregasse conotação especialmente pejorativa.} de meus
parentes já que nunca encontrou os seus para fazer obra mais perfeita.

Não há ninguém que o não conheça como advogado, não há dúvida, mas
também não há quem não saiba que para S. S. todos os peritos são imbecis
desde que são (sic) contrários à parte que S. S. patrocina.

Obriga-me a responder-lhe assim a sua \emph{linguagem
guindada}\footnote{. Empolada.}, que empregou no seu artigo de hoje;
tome lá o troco; de sua parte podem vir os maiores insultos e os maiores
impropérios, nada responderei.

S. Paulo, 21 de Novembro de 1877.

ADOLPHO SIDOW.

\textbf{18. AO ILMO. SR. ADOLPHO SIDOW}\footnote{. In: \emph{A Província
  de S. Paulo} (SP), Seção Livre, 23/11/1877, p.~2.}

\textbf{*didascália*}

\emph{A resposta de Gama à agressão covarde de Sidow foi ao seu estilo
``Getulino'' ou ``Polichinello'', i.e., pelo vezo da arte satírica que
tão bem manejava. Como tantas vezes fizera, vestiu a carapuça do bode --
o que marra e berra --, defendia sua ``ilustre raça'' e, provavelmente
muito bem informado do histórico familiar do contendor, como lhe era de
praxe, chamava a atenção para a ``catinga de casa'' de Sidow, modo sem
dúvida extrovertido para expor a mistura racial que constituía a própria
família de seu ofensor. Ao final, em outra tirada sarcástica que compõe
o texto magistral, apelava retoricamente não à toa ao juiz de órfãos --
como a devolver a agressão passada -- para que este recolhesse da rua
aquele infeliz idiota que estava a lhe atazanar. Touché, Gama! }

\begin{center}\rule{0.5\linewidth}{\linethickness}\end{center}

O sr. Sidow despiu incontestávelmente o siso, se é que algum dia o
teve...

Tornou à espora; e, \emph{manhoso} como d'antes não perdeu o
sestro\footnote{. Vício, hábito.} de agredir-me com fereza; e, como da
vez primeira, brindou o público com uma novidade a mais a meu respeito:
o sr. Sidow, depois de inquirir da minha ilustre raça, descobriu e
anuncia, com afano -- \emph{que eu sou bode!...}

Esta descoberta, entretanto, denuncia \emph{catinga de casa.\ldots{}}

O sr. Sidow já tem CARNEIROS na família; se não fora a sua imbecilidade
congênita, {[}ilegível{]} que -- de \emph{carneiro à bode} é
insignificante {[}ilegível{]} distinção......

{[}Ilegível{]} ao mais que, por pedido seu {[}ilegível{]} {[}ilegível{]}
escreveu o seu \emph{letradaço}, respondo {[}ilegível{]}:

-- ORDENAÇÕES DO LIVRO 4º, TÍTULO 103.

``Mandamos que tanto que o juiz dos órfãos souber que em sua jurisdição
há algum sandeu\footnote{. Idiota.}, que por causa de sua
sandice\footnote{. Idiotice, estupidez.} possa fazer mal ou dano algum,
o faça aprisionar, em maneira que não possa fazer mal a
outrem.''\footnote{. A citação, ligeiramente adaptada, confere com o
  original.}

Digne-se o exmo. sr. dr. juiz dos órfãos de cumprir o seu dever, e todos
estaremos livres de sofrer o sr. Sidow.

L. GAMA.

\textbf{19. {[}PETIÇÃO NO PROCESSO DA ALFÂNDEGA DE SANTOS{]}}\footnote{.
  In: \emph{Jornal do Commercio} (RJ), Roubo da Alfândega de Santos,
  17/12/1877, p.~1.}

\textbf{*didascália*}

\emph{Embora dispersa da montanha de papeis que compunha o caso
Largacha, a presente petição demonstra como Gama foi longe -- inclusive
geograficamente -- para defender seu cliente. O local da assinatura da
petição -- ``S. João do Rio Claro'' -- comprova que Gama viajou até
recônditos longínquos do interior paulista para que pudesse
``devidamente instruir sua queixa, por crime de calúnia, contra o dr.
Luiz Manoel de Albuquerque Galvão''. Em dezembro de 1877, já muita coisa
se esclarecia no que antes era tudo mistério sobre o caso Largacha.
Figuras como Albuquerque Galvão passariam a ser implicadas no roubo da
alfândega de Santos, coisa que, se se dependesse exclusivamente das
autoridades policiais e judiciárias de São Paulo, jamais ocorreria. A
ida de Gama até Rio Claro, evento notável tanto em sua advocacia quanto
no curso da investigação paralela que conduzia havia alguns meses,
pretendia encontrar uma peça-chave do quebra-cabeça do roubo da
alfândega: o alemão Guilherme Kroulein, ator diretamente implicado em
eventos determinantes daquele fevereiro de 1877. Não se sabe o que teria
levado Gama até aquela distante praça além Campinas, mas é certo que lá
esteve e peticionou ao delegado de polícia local para que mandasse,
``com a máxima possível urgência, em segredo de justiça'', inquirir o
dito alemão Kroulein. }

\begin{center}\rule{0.5\linewidth}{\linethickness}\end{center}

Ilmo. Sr.~Delegado de Polícia.

O major Antonio Eustachio Largacha, tesoureiro que foi da repartição da
alfândega da cidade de Santos, ali residente, e nesta por seu advogado
infra escrito, para que possa devidamente instruir sua queixa, por crime
de calúnia, contra o dr. Luiz Manoel de Albuquerque Galvão, vem,
respeitosamente, nos termos de direito, requerer à V. S. que seja
servido, com a máxima possível urgência, em segredo de justiça, mandar
que seja inquirido Guilherme Kroulein sobre os seguintes fatos:

1º: Se ele (Kroulein) seguiu para Corte pelo paquete nacional \emph{S.
José} a 21 de Fevereiro deste ano.

2º: O que sabe direta ou circunstancialmente com relação ao roubo da
alfândega de Santos.

3º: O que sabe relativamente a imputação de semelhante roubo ou
relativamente às pessoas a quem {[}se{]} atribui o mencionado crime.

4º: O que se passou em um hotel, em Santos, entre ele e Frank,
relativamente ao referido roubo, e o que mais lhe constar.

Nos termos do Decreto nº 4824 de 22 de Novembro de 1871, art. 39, nº 3,
o suplicante pede deferimento de justiça e ERM.\footnote{. Art. 39. "As
  diligências a que se refere o artigo antecedente compreendem:

  § 3º Inquirição de testemunhas que houverem presenciado o fato
  criminoso ou tenham razão de sabê-lo".}

S. João do Rio Claro\footnote{. Rio Claro (SP), município do interior
  paulista, a 170 km da capital, foi um polo cafeeiro que concentrou
  altas taxas de trabalho escravo em meados do século XIX.}, 4 de
Dezembro de 1877,

\emph{Luiz Gama.}

\textbf{20. O ROUBO DA ALFÂNDEGA DE SANTOS}\footnote{. In: \emph{A
  Província de S. Paulo} (SP), Seção Livre, 10/01/1878, pp.~2-3.}

\textbf{*didascália*}

\emph{``Absolvo o réu Antonio Eustachio Largacha da acusação que contra
ele foi intentada, e mando que se expeça alvará de soltura em seu
favor''. O apagar das luzes do ano de 1877 reservou uma excelente
notícia para Gama, Ribeiro Campos, Largacha e seus dois companheiros de
alfândega, também responsabilizados pelo roubo milionário que agitava a
província e o país. A sentença do juiz de direito de Santos, Alberto
Bezamat, reproduzida ao fim do artigo, acolhia parte substancial dos
argumentos que Gama e Ribeiro Campos sustentaram em diversas peças --
fossem elas internas ao processo ou específicas para a audiência da
imprensa, alçada, por estratégia que tanto notabilizou o estilo da
advocacia de Gama, como jurisdição de defesa de direitos. O anúncio de
Gama e Ribeiro Campos, contudo, estava longe de cantar vitória. ``Está
terminado o processo de responsabilidade'', diziam os autores, muito
embora eles soubessem que, num passe de mágica jurídica, o que seria
``irrevogável'' poderia acabar reformado. De todo modo, a vitória no
tribunal é comemorada com sobriedade -- dispensando elogios ao juiz
Bezamat --, e ânimo de luta para as etapas vindouras. Habilmente, os
advogados sinalizavam que a etapa vindoura que eles ansiosamente
aguardavam era a restauração da honra pessoal dos réus absolvidos. É
evidente que eles sabiam que a parada dura que enfrentariam seria na
manutenção dos direitos de Largacha e companhia no Tribunal da Relação
de São Paulo, cujo julgamento ainda estaria por vir. Porém, era tempo de
mirar o futuro e avisar aos quatro cantos do mundo que a sede de justiça
era o motor de todos eles. Daí um desejo de justiça não pela metade, mas
por inteiro. ``Não bastam aos acusados, às vítimas infelizes do acaso,
as sinceras manifestações, em uma só palavra, por uma só voz, de uma
população tão ilustrada quão severa e magnânima, sem distinção de
posições, de convicções políticas, e de nacionalidades; eles querem
mais: pretendem o julgamento do país inteiro; exigem uma sentença
nacional; querem a reintegração solene e completa de todos os seus foros
de homens honrados, de funcionários íntegros, que sempre mantiveram
ilesos''. }

\begin{center}\rule{0.5\linewidth}{\linethickness}\end{center}

Está terminado o processo de responsabilidade, que foi ordenado, por
crime de peculato, \emph{mediante inquérito policial} \emph{(!!!)}
contra o inspetor daquela repartição -- comendador Antonio Justino de
Assis; tesoureiro -- major Antonio Eustachio Largacha; e chefe de seção
-- João Baptista de Lima: está concluído o famoso prólogo desse mistério
de iniquidade\footnote{. Perversidade, injustiça.}!...

O verbo da lei, a expressão do direito, o assento da verdade, o juízo
imparcial do julgador fez-se ouvir; a sentença foi proferida; tornou-se
irrevogável.

Agora a opinião de todos os homens honestos perante a prova
irrefragável\footnote{. Irrefutável, incontestável.}, perante o
monstruoso processo, em que depuseram 190 TESTEMUNHAS, \footnote{. Foram
  mais de duzentos e trinta depoimentos, entre as quase duzentas
  testemunhas.}escolhidas pelos defensores da lei, pelos
levitas\footnote{. Por sentido figurado, sacerdotes.} da justiça, e
inqueridas, em máxima parte, secretamente, a portas fechadas, no recesso
da polícia, nos arcanos impenetráveis das íntimas indagações.

Não bastam aos acusados, às vítimas infelizes do acaso, as sinceras
manifestações, em uma só palavra, por uma só voz, de uma população tão
ilustrada quão severa e magnânima, sem distinção de posições, de
convicções políticas, e de nacionalidades; eles querem mais: pretendem o
julgamento do país inteiro; exigem uma sentença nacional; querem a
reintegração solene e completa de todos os seus foros\footnote{.
  Privilégios.} de homens honrados, de funcionários íntegros, que sempre
mantiveram ilesos.

Para este julgamento preparam a publicação do processo, por inteiro, com
todos os dados coligidos pela autoridade, e com os esclarecimentos
prestados pela defesa, o que só com vagar e muito trabalho realizaremos.

Depois da devassa policial, vastíssimo arsenal de incoerências
manifestas e de contradições, que se atropelam; depois dos exames
prolongados e repetidos, difusos nas descrições, ermos\footnote{.
  Deserto, vazio.} de conceitos, incompreensíveis na forma,
inconcluedentes em todos os pontos; depois das suspensões
administrativas, pelas quais sacrificou-se brutalmente a dignidade de
velhos e eméritos funcionários, para dar azo\footnote{. Motivo, causa.}
à intumescida\footnote{. Dilatada, inchada.} vaidade e à
soprada\footnote{. Insuflada, propagada.} filáucia\footnote{. Presunção
  exacerbada.} de orgulhosos chefes; depois da prisão indébita, com
infração da lei, em que os mandatários fizeram alarde de todos os erros,
para lisonjear a protérvia\footnote{. Petulância, desfaçatez.} dos
mandantes; depois da negação incompreensível de ordem de
\emph{habeas-corpus}, da revogação caprichosa da sentença de
não-pronúncia, sem a necessária destruição dos seus fundamentos
jurídicos; depois de uma sentença judicial de pronúncia, por infração do
artigo 170 do Código Criminal\footnote{. Art. 170. ``Apropriar-se o
  empregado público, consumir, extraviar, ou consentir que outrem se
  aproprie, consuma ou extravie, em todo ou em parte, dinheiro ou
  efeitos públicos, que tiver a seu cargo''.}, obrigando absurdamente o
acusado à prisão, nos termos de um decreto do Poder Executivo!... Depois
da negação de fiança, firmada em tais fundamentos, consignados em um
venerando despacho do colendo Tribunal da Relação!... Depois do
esquecimento do direito, da tortura da lei, do atropelo das fórmulas, do
amesquinhamento da infelicidade, e da negação da justiça, raiou o dia da
redenção, fez-se a luz nos tribunais, foi pronunciada a palavra de
ordem, fez-se ouvir a sentença.

Tudo não está concluído, porém; tudo não está demonstrado; há sombras
que cumpre desvendar;há indícios que cumpre averiguar; há provas que
devem ser tiradas a limpo;os autores do roubo da Alfândega devem ser
expostos ao público, à luz meridiana; há no processo calculadas
falsidades que devem ser desmascaradas; a calúnia desfaçada\footnote{.
  Desavergonhada, descarada.} ainda campeia com ousadia;, afrontando as
vítimas, zombando da justiça e da lei, escarnecendo\footnote{.
  Menosprezando, desdenhando.} da moralidade pública; mas a calúnia é um
crime, seus autores são conhecidos, os ofendidos têm direitos, os
tribunais saberão cumprir o seu dever.

Não queremos o sedicioso\footnote{. Aqui no sentido de perturbado,
  insano.} domínio das prevenções\footnote{. Hostilidade gratuita,
  antipática, preconceituosa.}, não queremos os processos secretos, não
queremos as detenções por supostos indícios: queremos a prova dos fatos
e da ignomínia\footnote{. Humilhação, grande desonra pública.} com que
se pretendeu macular aos nossos constituintes, sob pena de
infligida\footnote{. Aplicada, imposta.} ser aos inventores ardilosos
das torpezas a pena infamante do caluniador.

Como advogados cumprimos conscienciosamente\footnote{. Conduta honesta
  com responsabilidade e cuidado.} o nosso dever; agora devem os homens
honestos de todo o país cumprir também o seu.

S. Paulo, 8 de Janeiro de 1878.

Os advogados,

DR. RIBEIRO CAMPOS.

LUIZ GAMA.

\_\_\_\_\_\_\_

Antonio Luiz Ribeiro, escrivão do juízo de direito nesta cidade de
Santos e seu termo. Certifico que a fl.~892 do processo crime de
responsabilidade em que são: a justiça, autora, e Antonio Eustachio
Largacha, réu, consta a sentença de teor seguinte.

Vistos e examinados estes autos, etc., Deles consta que o réu Antonio
Eustachio Largacha é acusado de haver extraviado do cofre da Alfândega
desta cidade a quantia de 185:650\$679 réis, sendo 177:031\$279 réis em
dinheiro e 8:619\$400 réis em estampilhas, que tinha a seu cargo como
tesoureiro. Considerando que não está provado que foi o réu o autor da
subtração da referida quantia, pois conquanto se verifique do inquérito,
que, no dia 19 de Fevereiro do corrente ano, por ocasião de achar-se
aberto o mesmo cofre, não se encontrou vestígio algum de violência e
escalada no exterior do edifício da Alfândega, e que o cofre forte da
mesma apresentava apenas sinais de violência no exterior do canhão da
fechadura, devendo, por isso, ter sido aberto, ou com chave apropriada à
fechadura, ou com gazua moldada pelo feitio da chave, fl.~71,
v{[}erso{]}, não são porém esses fatos outra cousa mais que meros
indícios, que perdem toda a sua intensidade porquanto dos autos se vê
estar provado:

Primeiro: que podia-se penetrar no edifício por meio de escalada sem
deixar vestígio algum -- fls. 48, v., 840 e 844;

Segundo: que uma porta interior, que deita para o salão onde se achava o
cofre forte, foi encontrada com sinais de ter sido forçada -- corpo de
delito, fl.~50, exame a fl.~641;

Terceiro: que o mesmo cofre foi consertado quando chegou da Inglaterra e
não se pode abrir aqui -- depoimento do serralheiro Antonio de Padua do
Coração de Jesus, a fl.~255, e de Benedicto José de Souza, a fl.~266 e
outros documentos a fls. 262 e 263, sendo por aquele serralheiro feita
mais tarde uma chave nova, apenas diferente da primitiva, não nos dentes
que correspondem às peças interiores da fechadura e, por conseguinte, ao
segredo, mas simplesmente na broca, que era cilíndrica e passou a ter
forma de estrela, fls. 255 e 266;

Quarto: que a chave primitiva, vindo com o cofre, ficou em poder do
tesoureiro de então, hoje falecido, não se sabendo onde a mesma existe,
nem se tendo procedido a diligência alguma para esse fim, aliás
importante por seu alcance.

Sendo assim, certos como são esses pontos, nenhum valor tem, como
indício contra o réu, a falta de vestígios de escalada, pois era fácil
não os ter deixado quem no edifício penetrasse, e nem tampouco o fato de
ser o réu tesoureiro, Antonio Eustachio Largacha, quem tinha consigo a
nova chave do cofre, que a ninguém a confiara, desde que uma outra chave
existia, por onde se podia modelar uma gazua, caso, modificada a forma
da broca no sentido do novo canhão (de estrela), não pudesse ela mesma
ser empregada.

Sobreleva notar que a fechadura depois do conserto referido, muito
perdeu em sua especialidade, porquanto as cinco peças interiores que
tinham movimento distinto e deviam ser movidas simultaneamente para
abrir e fechar o cofre, foram reduzidas a duas, como se vê no corpo de
delito a fls. 53 e 60, v., e pela inspeção ocular e exame da mesma
fechadura, sendo também visivelmente claro que com o conserto
prejudicou-se a complicação das peças interiores. Isto posto, nenhum
préstimo têm as informações dos peritos ouvidos na Corte, fls. 576 e
seguintes, tanto mais quanto referindo-se eles a uma fechadura em
perfeito estado, não têm seus ditos aplicação à do cofre de que se
trata, sendo ainda para estranhar que, consultados os ditos peritos e
estando a fechadura em poder da autoridade policial que investigava o
caso, não fosse ela remetida para ser examinada.

Não menos digno de reparo é que julgando-se necessário informações do
fabricante e tratando-se de um cofre de Hobbs \& Comp., fl.~62, se tenha
consultado Chubbs \& Son, por haver o intermediário da compra do cofre
da Alfândega, Henrique Nottron, declarado não ter tido transações com
Hobbs \& Comp., fl.~601, o que não é exato como se vê dos documentos a
fls. 261 e 263.

Considerando ainda que, do corpo de delito consta que por ocasião do
exame no dia 19 de Fevereiro foram encontradas removidas as telhas de
uma antiga abertura nas ripas do telhado (\emph{gateira}),parecendo
terem-na sido de dentro para fora, mas que semelhante indício de
simulação não produz contra o réu.

Primeiro: porque os próprios peritos declararam mais tarde que as telhas
encontradas removidas podiam tê-lo sido da parte de fora -- fl.~642, v.,
(\emph{corpo de delito});

Segundo: porque, mesmo não sendo assim, quem penetrasse no edifício,
pela gateira, poderia tê-la de coberto para entrar e aí depois
recomposto o telhado, e uma vez perpetrado o crime, removido da parte de
dentro as mesma telhas, para sair;

Terceiro: porque a aceitar-se como real a simulação, a hipótese a que
conduz tal indício, é ilógica ou incompreensível.

De fato: se as telhas da gateira foram removidas da parte de dentro para
simular que ali fora o lugar de entrada, pois não se deu a escalada como
indica a falta de vestígios; se no edifício não houve violências que
denotassem a entrada de alguém, quem descobriu a gateira deixou-se ficar
no edifício no dia 17 de Fevereiro antes de fechar-se a repartição.

Por onde, porém, saiu esse indivíduo, que não foi encontrado no dia 19
pelas autoridades que, como se vê dos autos, compareceram logo, nem
pelos empregados que se conservaram junto à porta que estava aberta?

Pelas portas ou janelas? Não, porque dos autos consta que foram elas
fechadas no dia 17 e não se abriu a repartição no dia 18, encontrando-se
todas no dia 19 como haviam ficado.

Pela gateira? Também não, porque se da não existência de vestígios se
conclui a não entrada por ela, é forçoso concluir-se igualmente a não
saída, sendo, como é certo, que tanto importava caminhar da parede do
edifício da nova Alfândega, do lado do Quartel, até a gateira, como
desta àquela.

Considerando mais, que o fato de serem as violências simuladas não
produz ainda conta o réu, porque, se é verdade que, a ter sido ele o
autor do delito, lhe aproveitaria a simulação, não menos verdade é que
reconhecida essa circunstância, muito o comprometia; e, portanto, estava
no interesse de quem abrisse cofre com gazua, ou com outra chave que não
a do réu, ``deixar sinais de simulação'', facilmente reconhecíveis, que
indigitando\footnote{. Indicando, apontando.} o mesmo réu, desviassem as
vistas das autoridades do verdadeiro culpado e induzissem a proceder
desde logo contra o réu acusado.

Considerando que resultou do inquérito que o tesoureiro réu, Antonio
Eustachio Largacha, aumentara consideravelmente sua fortuna de modo
inexplicável, atendendo-se de um lado aos bens que houve por heranças
aos que trouxe sua mulher, e são excluídos da comunhão por contrato
antenupcial, aos vencimentos do seu cargo, etc., etc.; e de outro, as
aquisições de vários prédios, a construção e reconstrução de outros, a
ter ele montado uma serraria, comprado uma lancha a vapor, etc., e as
suas despesas.

Considerando, porém, que ficou plenamente provado nos autos que menos
exatas foram aquelas averiguações e destituídas de fundamento são as
conclusões, que, sob este ponto, o mesmo inquérito autorizou, porquanto
dos prédios que ali se dizem terem sido adquiridos pelo réu, alguns
existem que nunca lhe pertenceram ou estiveram sequer em seu nome,
outros são de propriedade de sua mulher, havidos por herança, e outros
adquiridos por ela com o seu rendimento antes de o réu ser tesoureiro da
Alfândega (exceptuando um do valor de 1:273\$200 réis), figurando apenas
na coleta para o pagamento do imposto em nome do réu; existindo um só
prédio por ele e sua mulher construído na vila de São Vicente\footnote{.
  A primeira vila da América portuguesa, São Vicente localiza-se na
  baixada Santista (SP).} -- documento a fl.~505, a fls. 649 a 661, e
depoimento a fl.~467e outros.

Considerando mais ainda sob esta relação, que as despesas e rendimentos
do réu e sua mulher não foram no inquérito calculados com exatidão e que
a demonstração que faz o mesmo réu da receita e despesas com os prédios,
serraria, edificação da casa na vila de São Vicente, lancha a vapor,
etc., e dos seus vencimentos durante o tempo que serviu como tesoureiro
da Alfândega desta cidade, oito anos e meio, apresenta um saldo de
84:846\$060 réis, para fazer face às despesas do seu tratamento e de sua
família.

Considerando que, assim é improcedente a suspeita que naquele
pressuposto se fundara de ter o réu se apropriado do dinheiro a seu
cargo, ou para atender às suas despesas.

Considerando que é inaceitável, em vista dos autos, a hipótese de
desvios anteriores por parte do réu, de dinheiro a seu cargo, porquanto,
além do balanço geral a que se procedeu no cofre em 26 de Setembro do
ano passado (certidão da Alfândega a fl.~677), eram de 5 em 5 dias
remetidos para o Tesouro Nacional balancetes demonstrativos do estado do
mesmo cofre, e as remessas mensais dos saldos neles existentes
confirmavam a exatidão daqueles balancetes, sendo que em Janeiro do
corrente ano recebeu e cumpriu o réu saques a esgotar o cofre -- fls.
673, v., 675, v., (certidão da Alfândega).

Considerando que, das 190 testemunhas do inquérito nenhuma atribuiu ao
réu autoria ou participação no crime praticado no cofre da Alfândega,
nada constando contra o mesmo réu, a não ser o depoimento a fl.~498 da
testemunha Sebastião Carlos Navarro de Andrade, primeiro escriturário
daquela repartição, e que essa mesma testemunha declara ``nada saber''
quanto ao referido crime, e só diz que o réu fazia empréstimos pequenos
de dinheiro {[}à{]} seus colegas e à estranhos, no que cai em
contradição com o seu depoimento no sumário a fl.~754 e na delegacia de
polícia a fl.~662, e ficou provado não ser exato a fls. 356, 741 e
outras.

Considerando que as testemunhas do sumário nada depõem contra o réu e,
pelo contrário, são lhe favoráveis e o abonam em seu caráter e
reputação, e que o mesmo se dá com as testemunhas do plenário.

Considerando que está provado pelo depoimento do fiel\footnote{. No
  sentido de fâmulo, funcionário subalterno e diligente.} do tesoureiro,
Manoel Geraldo Forjaz, a fls. 842 e seguintes, e do segundo
escriturário, Manoel de Jesus Couto, a fls. 846 e seguintes, ambos
maiores de toda excepção e que trabalhavam no biombo onde estava o cofre
e o tesoureiro que, no dia 17 de Fevereiro do corrente ano, o réu,
durante as horas do serviço, esteve a contar, a emaçar\footnote{.
  Empacotar, organizar em maços.} e rotular dinheiro para a remessa que
no dia 19 devia ser feita para o Tesouro Nacional.

Considerando que o dito fiel afirma que no dia 17 ficaram contados,
emaçados, rotulados e guardados no cofre cento e vinte contos de réis
(parte dos quais ele recontou), além de grande quantidade por contar e
emaçar; o que também foi visto pelo dito 2º escriturário, fl.~877, v.,
pelo chefe da 2ª seção, fl.~30, pelo guarda-mór, fl.~360, e é confirmado
pelo dito do atual tesoureiro a fl.~872, e de outros empregados que o
mesmo ouviram dizer na repartição, fls. 362, 366, 367, v., 372, v., 376,
378, v., 761, 769, e outras.

Considerando que está provado pelos depoimentos a fls. 843, 846, 847,
v., que o réu no dia 17 de Fevereiro, depois de ter fechado o cofre e
verificado o fato\footnote{. Conjunto de bens móveis, nesse caso, as
  instalações da Alfândega.}, como de costume, retirou-se da repartição
quando encerrado o expediente e em companhia de alguns colegas.

Considerando que está provado pelos depoimentos de fls. 840, 843, v.,
872, 878 e outras, que o réu no dia 18 de Fevereiro, tendo notícia da
chegada do paquete\footnote{. Navio mercante que prestava serviço de
  correio e transporte de valores, mercadorias e passageiros.} \emph{Rio
Grande}, da linha subvencionada do Sul, que era esperado no dia 19, e
devia ser o portador da remessa do saldo do cofre da Alfândega,
apresentou-se com o seu fiel, a quem chamou à porta da repartição para
fazer o serviço que lhe competia, caso a remessa tivesse de efetuar-se
naquele dia.

Considerando o que foi alegado e provado pelo réu em sua resposta a fl.
619, em sua defesa no plenário e o mais que dos autos consta:

Absolvo o réu Antonio Eustachio Largacha da acusação que contra ele foi
intentada, e mando que se expeça alvará de soltura em seu favor
(argumento do art. 6º do Decreto nº 657 de 5 de Dezembro de 1849, última
parte, ibid. -- continuando a prisão no caso de pronúncia) e se lhe dê
baixa na culpa; pagas pela municipalidade as custas.\footnote{. Art. 6°.
  ``Se os tesoureiros, recebedores, coletores e contratadores depois de
  presos não verificarem a entrada do dinheiro público no prazo marcado,
  se presumirá terem extraviado, consumido ou apropriado o mesmo
  dinheiro e, por conseguinte, se lhes mandará formar culpa pelo crime
  de peculato, continuando a prisão no caso de pronúncia e mandando-se
  proceder civilmente contra seus fiadores''.}

Dos autos consta mais, que o réu Antonio Justino de Assis é acusado de
ter, como inspetor da Alfândega desta cidade, deixado de cumprir a ordem
do exmo. ministro da Fazenda, presidente do Tribunal do Tesouro
Nacional, de 20 de Março de 1876, e o art. 105, § 20, do regulamento que
baixou com o decreto nº 6.272 de 2 de Agosto do ano passado,\footnote{.
  Art. 105. "O inspetor é o chefe superior da alfândega. Incumbe-lhe
  especialmente:

  § 20. Tomar conhecimento semanalmente do estado dos cofres, e fazer
  efetivas as ordens sobre a remessa dos dinheiros, que neles existirem,
  à repartição competente".} por não haver remetido para o mesmo
Tesouro, no dia 18 de Fevereiro do corrente ano, pelo paquete \emph{Rio
Grande}, da linha subvencionada do Sul, o saldo existente no cofre da
mesma Alfândega até aquela data.

Considerando que o réu não foi denunciado por esse crime, para por ele
responder, nos termos do art. 159 do Código do Processo {[}Criminal{]},
e arts. 398 e 399 do Regulamento nº 120 de 31 de Janeiro de
1842.\footnote{. Respectivamente, art. 159. ``As Relações, e mais juízes
  a quem compete a formação da culpa, logo que for presente uma queixa
  ou denúncia concludente contra qualquer empregado público da sua
  competência, fará ouvir a este por escrito; depois do que,
  proceder-se-á no termos da pronúncia''. Art. 398. ``Logo que se
  apresentar uma queixa ou denúncia legal e regularmente formalizada, o
  juiz de direito a mandará autuar, e ordenará, por seu despacho, que o
  denunciado seja ouvido por escrito, salvo verificando-se algum dos
  casos em que o não deve ser, conforme o art. 160 do Código do Processo
  Criminal''. Art. 399. ``Para esta audiência, expedirá ordem ao mesmo
  denunciado, diretamente ou por intermédio do juiz municipal
  respectivo, acompanhada da queixa ou denúncia, e documento com
  declaração dos nomes do accusador e das testemunhas, a fim de que
  responda no prazo improrrogável de quinze dias''.}

Considerando, não obstante, porém, que está provado pelo depoimento a
fls. 871, 876 e outras, e é público e notório que os paquetes daquela
carreira\footnote{. Linha.}, pelos quais na forma da referida ordem se
fazem as remessas de saldo do cofre da Alfândega, entram sempre neste
porto no dia 19 de cada mês, e que no dia 18 de Fevereiro entrou sem ser
esperado -- fls. 871, v., 876, 197, 199, 353 e outras.

Considerando que, conquanto no mencionado dia 18, o mesmo paquete se
demorasse neste porto, aproximadamente o tempo que devia demorar-se se
entrasse no dia 19, era aquele dia domingo, a repartição estava fechada
e os empregados teriam de ser chamados para o trabalho da remessa, o que
consumiria mais tempo que o que se costumava gastar estando eles na
repartição -- depoimentos citados.

Considerando que, assim, não se pode dizer que o tempo de que disporia o
réu na segunda-feira, 19, para providenciar sobre a remessa, era igual
ao de que dispunha no domingo, 18.

Considerando que o réu, ao saber da entrada do dito paquete no dia 18,
dirigiu-se à agência para informar-se da partida do mesmo e procurou
saber se ela podia ser adiada, mostrando diligência com relação à
remessa do saldo existente no cofre da repartição que debaixo de sua
inspeção estava -- depoimentos a fls. 197, 199, 872 e outros.

Considerando que o réu, não efetuando a mesma remessa no dia 18, por
julgar insuficiente o tempo para o trabalho que tinha de fazer-se,
telegrafou ao exmo. ministro da fazenda no sentido de realizá-la por um
dos paquetes da Companhia de Navegação Paulista -- certidão do telegrama
a fls. 823, v., documento a fls. 124, 125 e 133.

Considerando que, com o procedimento exposto, provado dos autos, o réu
não manifestou descuido, frouxidão, negligência ou omissão dos deveres
impostos pela supradita ordem e regulamento.

Considerando o mais que foi alegado e provado pelo réu, e o que dos
autos consta:

Absolvo o réu Antonio Justino de Assis da acusação contra ele intentada
e mando que se lhe dê baixa na culpa, pagas as custas pela
municipalidade.

E como dos presentes autos conste matéria que pode interessar a justiça
no descobrimento do autor, ou autores, e partícipes do crime perpetrado
no cofre da Alfândega desta cidade, mando que o escrivão extraia cópias
das seguintes peças:

Portaria de fl.~45, certidão de fl.~45, v., auto de corpo de delito a
fl.~46, petição de fl.~57, auto de fl.~58, com respectivos julgamentos e
termos de fl.~62, v.

Auto de exame a fl.~155, 274, 339 e 524, com os respectivos julgamentos.

Depoimento de fls. 182, v., 191, v., 196, 427, 428, , 430, v., 432, 441,
v., 515 e 545; petição de fl.~95, ofício de fl.~98 e certidão de fl.
523.

Ofício e documento de fls. 259 a 263 e depoimento de fl.~601.

Depoimentos a fls. 527, 548, 533, 530, 428, 531, 495, 548, 535 e 536.

Certidão a fls. 622 e seguintes.

Auto de corpo de delito de fls. 641 e seguintes.

Depoimento de fls. 838, v., 841, v., 851, 856, v., 862, v., e documentos
de fls. 882 e seguintes: feito o que remeta ao meritíssimo juiz
municipal do termo, para os fins convenientes.

Santos, aos 29 de Dezembro de 1877.

ALBERTO BEZAMAT.

\textbf{21. ROUBO DA ALFÂNDEGA DE SANTOS}\footnote{. In: \emph{A
  Província de S. Paulo} (SP), Seção Livre, 29/03/1878, p.~2.}

\textbf{*didascália*}

\emph{Gama responde as recentes incursões de Albuquerque Galvão nos
jornais. A posição de Gama, Ribeiro Campos e Largacha após a vitória de
final de dezembro de 1877, contudo, não exigia deles a permanência na
trincheira da imprensa. Uma vez obtida a sentença favorável aos direitos
de Largacha, quem agora deveria correr para reverter o curso do processo
era Albuquerque Galvão e os demais implicados nas apurações de indícios
e provas de autoria do roubo da alfândega de Santos. Ainda assim, Gama
resolve redarguir Galvão em alguns pontos, destacando, no entanto, que a
força normativa de duas sentenças em favor de Largacha restava fora de
discussão. Gama discutiria outros pontos. Desde a epígrafe, diria ao
público que o seu oponente estaria desassossegado. Estaria à busca de
qualquer elemento que colocasse sua versão a salvo. Uma certidão do
ministro da Fazenda, por exemplo, faria as vezes de prova cabal da sua
narrativa. Gama, portanto, trata de alertar aos desprevenidos que aquela
era tão só uma certidão administrativa ordinária. Mas ao fim, Gama
voltava com tudo, sendo esta talvez a razão central do escrito. Avisava
ele a Galvão, ``permita-me que lhe dê um conselho gratuito: mude de
rumo, porque perde a jornada. Os felizes ladrões da Alfândega de Santos
hão de morrer impunes!... Filhos de alcaide não vão à cadeia. Um dos
roubadores do cofre da Alfândega, dias depois da colossal proeza, por a
mediação de alguém, de modo especial e cauteloso, na Corte, depositou em
certo estabelecimento a quantia de réis...''. Gama simplesmente avançava
o sinal e dizia em alto e bom som um traço absolutamente marcante do
ladrão da Alfândega -- um filho de prefeito! --, e ainda agregava uma
informação sobre o paradeiro do dinheiro roubado. Iriam as autoridades
policiais e judiciárias atrás das hipóteses ventiladas por quem conhecia
o inquérito e os processos como a palma de sua mão?}

\begin{center}\rule{0.5\linewidth}{\linethickness}\end{center}

\emph{... O desassossego de espírito é indício de enfermidade.}

O sr. dr. L. M. De Albuquerque Galvão, alma de Euclides em corpo de
Calafate\footnote{. Embora o tom sarcástico salte aos olhos, os sentidos
  da metáfora escapam de algum modo ao leitor contemporâneo, afinal, a
  figura do calafate, profissional especializado em tapar fendas e
  buracos, não nos é tão próxima hoje em dia. O contraste, todavia, é
  chamativo. A referência ao célebre Euclides de Alexandria, escritor e
  matemático grego que viveu no séc. III a. C., como expressão da
  inteligência que animava o modesto corpo de calafate, guarda uma
  ironia afiada. Pode-se ler, em síntese, que Gama via em Galvão um
  sujeito que, sem cálculo e inteligência alguma, obrava
  desajeitadamente para tapar furos e buracos de sua narrativa sobre o
  crime da alfândega de Santos.}, ``que não tolera fendas sem estopa'',
publicou na \emph{Província} de hoje algumas magras considerações, muito
suas, relativamente ao seu pesadelo -- o audacioso roubo da Alfândega de
Santos -- que ``ele'' poética e calculadamente qualifica de
``simulado''...............

E acompanha essas suas desinteressadas considerações de um longo ofício,
que há de todo sair à luz em volumosos trechos, firmado pelo exmo. sr.
dr. Sebastião José Pereira,\footnote{. Sebastião José Pereira
  (1834-1881), nascido em São Paulo (SP), foi advogado, juiz de direito
  e presidente da província de São Paulo (1875-1878).} ex-Presidente da
Província. Ofício que vem a lume, assim, à guisa de parto estupendo,
porém, que nada mais é do que a simples reprodução do Relatório-policial
do exmo. sr. conselheiro Furtado de Mendonça\footnote{. Francisco Maria
  de Sousa Furtado de Mendonça (1812-1890), nascido em Luanda, Angola,
  foi subdelegado, delegado, chefe de polícia e secretário de polícia da
  província de São Paulo ao longo de quatro décadas. Foi, também,
  professor catedrático de Direito Administrativo da Faculdade de
  Direito de São Paulo. A relação de Luiz Gama com Furtado de Mendonça é
  bastante complexa, escapando, em muito, aos limites dos eventos da
  demissão de Gama do cargo de amanuense da secretaria de polícia, em
  1869. Para que se ilustre temporalmente a relação, tenhamos em vista
  que à época do rompimento público, aos finais da década de 1860, ambos
  já se conheciam e trabalhavam juntos há cerca de duas décadas; e,
  mais, Gama não rompeu definitivamente com Furtado de Mendonça, como
  erroneamente indica a historiografia, visto que em 1879 publicou o
  artigo \emph{Aos homens de bem}, defesa moral e política explícita do
  legado de Furtado de Mendonça.}, e que já foi cabalmente refutado no
auto do processo, em todos os seus pontos, mediante irrecusável prova
legal.

O atiladíssimo\footnote{. Pode ser lido, sarcasticamente, como
  escrupuloso e correto ou como espertíssimo, muito perspicaz.} sr. dr.
Galvão, que não só desta peça oficial, como de outras muitas, e há longo
tempo, tinha particular conhecimento, vem, hoje, de ânimo estudado,
exibí-la, como curiosa novidade ou surpreendente maravilha, para ele...

E, para dar maior encanto ao seu deslumbrador sucesso, publica também a
petição que fizera ao governo imperial, para pública obtenção de tal
documento, e gaba-se com sonoroso entono\footnote{. Orgulho, vaidade.}
do favorável despacho que alcançara, não só do ministro da Fazenda,
``mera entidade governamental'', senão do exmo. sr. conselheiro dr.
Gaspar da Silveira Martins!\footnote{. Gaspar da Silveira Martins
  (1835-1901), natural de Cerro Largo, Uruguai, foi advogado, magistrado
  e político. Eleito deputado e senador por sucessivos mandatos, também
  foi ministro da Fazenda (1878-1879) e presidente da província de São
  Pedro do Rio Grande do Sul (1889).}

E todo este alarde ``por causa de um despacho concedendo uma
certidão''!...

Ao ler esta parte do precioso escrito do sr. dr. Galvão, recordei-me de
certa cavatina\footnote{. Termo musical para uma curta canção cantada
  por um personagem numa ópera.} em que o impagável ``dr. Dulcamara''
anuncia estrepitosamente\footnote{. Ruidosamente, com estardalhaço.} pós
maravilhosos para matar mosquitos!...\footnote{. Personagem de
  \emph{L'elisir d'amore} (1832) -- ópera cômica de autoria do
  compositor italiano Gaetano Donizetti (1797-1848), com libreto do
  poeta genovês Felice Romani (1788-1865) -- o ``médico ambulante''
  Dulcamara foi um charlatão que prometia mundos e fundos e dizia ter a
  cura -- mediante dinheiro... -- para todos os males da terra. A
  ópera-bufa tem por cenário uma pequena aldeia no País Basco do século
  XVIII. Ao chegar no vilarejo, o charlatão Dulcamara passa a anunciar
  licores mágicos, elixires milagrosos, entre outras extravagâncias,
  alcançando sucesso através do seu curioso ofício de ludibriar pessoas
  simples. A associação entre Galvão e Dulcamara é bastante sugestiva.
  Com a elegância da alusão literária, Gama impinge a pecha de falacioso
  a Galvão, que induziria o público ao erro ao ostentar uma certidão
  qualquer como se ela significasse atestado de sua idoneidade. As
  expressões ``curiosa novidade'', ``surpreendente maravilha'', ``parto
  estupendo'', ``sonoroso entono'' e ``deslumbrador sucesso'', que Gama
  aplica a Galvão no corpo do texto, são condizentes com uma possível
  leitura do personagem Dulcamara.}

Desde já, com critério, e sem doesto, cumpre corrigir um grave erro em
que, por descuido ou por indústria\footnote{. Pelo contexto, se sugere
  que a indústria, a astúcia, resultou, parodoxalmente, na falta de
  perspicácia.}, caiu o precavido sr. dr. Galvão.

Não é certo que o dr. Alberto Bezamat seja ``o único juiz'' que lavrou
sentença absolutória em favor do major A. E. Largacha. Antes dele, já o
exmo. juiz de direito de Santos, dr. Marcos de Souza, hoje
desembargador, o havia absolvido. E a sentença do sr. dr. Bezamat corre
impressa, até hoje, sem a menor contestação.

Agora duas palavras mais para terminar.

Se o sr. dr. Galvão, com estas suas custosas publicações, não tem em
mira defender-se, porém malsinar os respeitabilíssimos ladrões da
Alfândega, permita-me que lhe dê um conselho gratuito: mude de rumo,
porque perde a jornada. Os felizes ladrões da Alfândega de Santos hão de
morrer impunes!...

Filhos de alcaide não vão à cadeia.

Um dos roubadores do cofre da Alfândega, dias depois da colossal proeza,
por a mediação de alguém, de modo especial e cauteloso, na Corte,
depositou em certo estabelecimento a quantia de réis...
................\ldots{}

Pôs no ``seguro'' os teres\footnote{. Bens.} e a ``reputação''.

S. Paulo, 27 de março de 1878.

O advogado, LUIZ GAMA.

\textbf{O HOMICÍDIO}

\textbf{*didascália*}

\emph{Ao longo de seis textos, Gama discute na imprensa três casos que
tratavam do crime de homícidio ou de tentativa de homicídio. Em todos os
casos, Gama era o advogado de defesa em processos de} habeas-corpus
\emph{que correram ou na primeira instância judiciária ou no Tribunal da
Relação de São Paulo. Cada um dos supostos crimes, por sua vez, teve
processamento inicial em cidades distantes da capital, respectivamente,
conforme data de publicação, em Jaú, depois em Pirassununga e, por
derradeiro, em Ribeirão Preto, todas elas no interior da então província
e hoje estado de São Paulo. Isso indica como a advocacia de Gama era bem
conhecida em diversas localidades. Por exemplo, quando João Franco de
Moraes Octavio, homem branco e fazendeiro de Ribeirão Preto, precisou
cuidar de sua liberdade, sem dúvida que optou em confiar em quem melhor
podia lhe tirar da encalacrada judicial em que estava metido. No
entanto, a escolha por Gama não foi imediata. O fazendeiro Octavio,
mediante outro advogado que não Gama, já havia havia recorrido três
vezes por sua soltura. Em todos os recursos, contudo, sua soltura foi
negada. No quarto e último} habeas-corpus\emph{, Gama foi o seu
advogado. Resultado? Votação unânime dos desembargadores da Relação de
São Paulo pela soltura de Octavio. Como se verá, Gama mitigou a validade
da prova testemunhal e descaracterizou elementos estruturantes da
formação de culpa de seu cliente. Ganhou a causa. Além dessa,
seguramente ganhou a de Pirassununga. Embora não se saiba o desfecho do
processo de Jaú, é próvável que o resultado tenha sido o mesmo dos
demais aqui citados, haja vista seu posicionamento nos jornais. Em todo
caso, temos aqui argumentos consistentes em causas criminais com o mesmo
tipo penal defendidas por Gama no biênio 1878-1879. }

\begin{center}\rule{0.5\linewidth}{\linethickness}\end{center}

\textbf{22. JAÚ}\footnote{. In: \emph{A Província de S. Paulo} (SP),
  Seção Livre, 12/10/1878, p.~1.}

\textbf{*didascália*}

\emph{Embora não haja maiores detalhes sobre a suposta tentativa de
homicídio de que versa o artigo, Gama afirma já ter discutido o mérito
da causa no processo, ou seja, não deixa dúvida de que era o advogado do
acusado. Por estratégia processual, contudo, Gama parece não ter voltado
à imprensa para discutir esse caso. A advertência na última linha é
fatal: ``se o quiserem, discuti-la-ei pela imprensa''. Ao que apurei
nenhum figurão pagou para ver. }

\begin{center}\rule{0.5\linewidth}{\linethickness}\end{center}

Acabo de ler na \emph{Província} de hoje uma correspondência firmada
pelo sr. Antonio Benedicto de Campos Arruda, na qual se nota o trecho
seguinte:

``O processo dos Assiz, o subdelegado tratou do inquérito e o juiz
competente do processo, sendo os réus acompanhados pelo seu advogado, o
sr. Luiz Gama, e não me consta que houvesse qualquer ato vexatório,
antes pelo contrário as autoridades têm sido benignas a tal ponto que
\emph{desde o inquérito ficou provada a tentativa de morte e os réus
acharam-se em plena liberdade!}''

Em que pese ao sr. Arruda, declaro-lhe que a sua afirmação é inexata.

A tentativa de homicídio atribuída aos irmãos Assiz é uma visão
eleitoral que está plenamente conhecida.

Já discuti-a, nos autos; se o quiserem, discuti-la-ei pela imprensa.

S. Paulo, 11 de outubro de 1878.

O advogado,

LUIZ GAMA.

\textbf{23. FRANCISCO ALDO DE OLIVEIRA {[}I{]}}\footnote{. In: \emph{A
  Província de S. Paulo} (SP), Seção Livre, 23/01/1879, p.~2.}

\textbf{*didascália*}

\emph{Gama defende publicamente seu cliente dos ataques que sofria na
imprensa. Francisco Aldo de Oliveira era acusado e havia sido preso como
autor do homicídio de um ex-juiz de direito de Bagé, Rio Grande do Sul.
A redação da} Tribuna Liberal\emph{, especialmente, saudava a sentença
do juiz municipal de Pirassununga, São Paulo, onde Oliveira fora
capturado e julgado. Gama pontua que o juiz ``analisou de improviso a
fisionomia do suposto réu (...) até ao ponto de adivinhar as
intenções!'', isto é, tomou caracterísiticas físicas como marcas
conclusivas da criminalidade do acusado. O homicídio em Bagé ocorreu em
outubro de 1876; a prisão do suspeito, por sua vez, em finais de 1878 ou
janeiro de 1879. Gama lutava pela soltura de seu cliente, sustentando
que a ``prisão de Francisco Aldo de Oliveira, realizada pelo honrado sr.
dr. juiz municipal de Pirassununga, é mais do que um erro; é mais do que
um desastre judicial; é mais do que um absurdo legal; é uma
monstruosidade; é um crime''. O réu, Gama avançava na crítica, deveria
ser o juiz, na medida em que este transgredia deliberadamente as
disposições normativas que o obrigariam a agir de outro modo.}

\begin{center}\rule{0.5\linewidth}{\linethickness}\end{center}

A ilustrada redação da \emph{Tribuna Liberal}, em a notícia que ontem
deu, do descustodiamento\footnote{. Aqui empregado como soltura,
  liberação.}, por concessão de \emph{habeas-corpus}, de Francisco Aldo
de Oliveira, revelou certos arreliques\footnote{. Questiúnculas, coisas
  de pouca importância.} censórios, com fumos de impertinente
parcialidade, que seguramente se não coadunam com os seus louváveis
intentos de reto e sisudo julgador.

Meteu-se a Lavater\footnote{. Johann Kaspar Lavater (1741-1801), nascido
  em Zurique, Suíça, foi escritor, filósofo e teólogo protestante.
  Alcançou notoriedade com estudos sobre fisionomia, publicando o
  tratado, em livre tradução, \emph{A arte de conhecer os homens pela
  fisionomia} (1775-1778).}; analisou de improviso a fisionomia do
suposto réu e abalançou-se até ao ponto de adivinhar as intenções!...

Eu não discuto circunstâncias efêmeras, ocorrências casuais despidas de
importância objetiva, e até ridículas.

Como é do meu dever, encaro os fatos em toda sua magnitude; considero-os
em sua sindesmologia\footnote{. Ramo da anatomia que estuda os
  ligamentos e as articulações.} legal; os avalio por os seus feitos
sociais, jurídicos e políticos; e, em face do país, rendendo graças ao
colendo Tribunal da Relação, declaro: a prisão de Francisco Aldo de
Oliveira, realizada pelo honrado sr. dr. juiz municipal de
Pirassununga\footnote{. Pirassununga (SP), cidade do interior paulista,
  dista aproximadamente 200 km da capital.}, é mais do que um erro; é
umais do que um desastre judicial; é mais do que um absurdo legal; é uma
monstruosidade; é um crime.

Preciso é que ponhamos de parte as anfibologias\footnote{. Ambiguidades,
  duplicidade de sentidos.} cortesãs e que designemos as coisas
claramente, pelos seus próprios nomes.

O juiz é a encarnação viva da lei; garantia inabalável dos direitos
individuais: sua missão é a justiça. Aquele que transgride
propositalmente as suas disposições, que ofende, em vez de proteger, e
que atropela, em vez de garantir, é um déspota, um verdugo, ou um réu.

Isto é que, por a dignidade da causa que defende, por amor dos
princípios da sua bandeira, deveria ter observado cimeira\footnote{. Em
  alto nível.}, e profligado\footnote{. Criticado, fustigado.}, sem
reservas, a ilustrada redação da \emph{Tribuna}; e não os equívocos de
palavras, os erros de nome, as mudanças de apelido\footnote{. A depender
  do teor do processo pode significar alcunha ou sobrenome.}, as
infelicidades, as misérias ou as defecções fisionômicas do
paciente,\footnote{. Pelo contexto, mudança drástica de aparência.} para
quem já eram carga sobeja\footnote{. Bastante, de sobra.} o rigor das
injustiças, e a humildade da posição.

S. Paulo, 21 de Janeiro de 1879.

LUIZ GAMA.

\textbf{24. FRANCISCO ALDO DE OLIVEIRA {[}II{]}}\footnote{. In: \emph{A
  Província de S. Paulo} (SP), Seção Livre, 25/01/1879, pp.~1-2.}

\textbf{*didascália*}

\emph{Neste artigo, já vitorioso no Tribunal da Relação, onde conseguiu
a soltura de seu cliente, Gama categoriza as razões jurídicas da
ilegalidade do mandado de prisão expedido pelo juiz municipal de
Pirassununga (SP). Buscando a doutrina criminal do senador Pimenta
Bueno, referência abalizada para assuntos dessa natureza, além da
legislação processual aplicável ao caso, Gama dispara dezessete
perguntas ao público, em geral, e aos defensores da ordem do juiz de
Pirassununga, em particular. Dezessete perguntas que, lidas à contraluz
dos excertos legais e doutrinários trazidos, faziam ruir a fundamentação
do mandado de prisão. ``Nada disto se sabe; tudo são conjecturas; tudo
são suspeitas'', bradava Gama como quem pergunta e quer a resposta,
acrescentando, ainda, que ``o próprio dr. juiz municipal de
Pirassununga, no mandado que expediu, decreta a prisão por suspeitas que
tem de que Francisco Aldo seja criminoso!''. Ou seja, Gama pinça uma
citação textual do mandado de prisão em que o juiz admite que se
fundamenta tão somente em suspeitas; e não em provas. Por término, Gama
voltava as baterias contra a redação da} Tribuna Liberal\emph{, que,
numa contradição que não o espantava, ``em nome dos princípios liberais
(...), tece elogios ao juiz violento!'' Como visto, após dezessete
interrogações vieram duas exclamações, uma direcionada ao juiz violento
e outra aos liberais partidários do... juiz violento! }

\begin{center}\rule{0.5\linewidth}{\linethickness}\end{center}

Em que pese à ilustrada redação da \emph{Tribuna}, torno à
liça\footnote{. Arena.}, e pela última vez, para mostrar o descabimento
das censuras que injustamente faz, por a concessão de
\emph{habeas-corpus} a Francisco Aldo de Oliveira, e dos suspeitos
encômios\footnote{. Elogios.} com que calculadamente galardoa\footnote{.
  Premia.} o sr. juiz municipal de Pirassununga\footnote{. Pirassununga
  (SP), cidade do interior paulista, dista aproximadamente 200 km da
  capital.}.

Cinjamo-nos à questão; pouco nos importa saber se Francisco Aldo muda de
apelidos\footnote{. A depender do teor do processo pode significar
  alcunha ou sobrenome.}, \emph{com a mesma facilidade com que muda de
camisas; se é médico ambulante; se tem corrido seca e meca}\footnote{.
  Remete ao adágio português que assim se refere a quem viaja muitas
  terras.}\emph{;} ou se tem boa ou \emph{má cara}...

Francisco Aldo foi preso como autor de homicídio na pessoa do ex-juiz de
direito de Bagé, em S. Pedro do Sul, \emph{perpetrado em Outubro de
1876}.

É verdadeira esta imputação?

Onde a prova?

Realizou-se a prisão de Francisco Aldo sem infração da lei?

Foram respeitados seus direitos individuais?

Não está provado, de modo algum, que Francisco Aldo seja autor do crime
que gratuitamente se lhe atribui.

A sua prisão realizou-se com a mais escandalosa violação da lei.

-- "A prisão, antes de culpa formada, à exceção de flagrante delito, só
pode ter lugar nos crimes inafiançáveis, por mandado escrito do juiz
competente para a formação da culpa, ou à sua requisição, precedendo,
neste caso, ao mandado ou à requisição, declaração de duas testemunhas,
QUE JUREM DE CIÊNCIA PRÓPRIA, ou prova documental, de que resultem
veementes indícios contra o culpado, ou declaração deste CONFESSANDO O
CRIME (Lei nº 2.033 de 20 de Setembro de 1871, art. 13, § 2º).\footnote{.
  Versão adaptada, porém, condizente com teor normativo do parágrafo 2º,
  art. 13.}

-- "Em matéria criminal, a \emph{confissão} do culpado só é regular e
admissível quando \emph{feita em juízo competente, sendo livre, e
coincidindo com as circunstâncias do fato} (senador Pimenta Bueno,
Código de Processo Criminal, art. 94).\footnote{. Não encontrei a
  citação a Pimenta Bueno. Possivelmente, Gama tenha se referido ao
  célebre \emph{Apontamentos sobre o processo criminal brasileiro}
  (1857), de José Antonio Pimenta Bueno (1803-1878), conhecida
  autoridade política que desempenhou os cargos de juiz, desembargador
  (1844-1847), ministro da Justiça (1849), presidente do Conselho de
  Ministros (1870-1871) e senador do Império (1853-1878). O conteúdo
  normativo do artigo citado, por sua vez, embora adaptado, também
  confere com o original.}

-- ``A prisão não poderá ser ordenada, ou requisitada, nem executada, de
réu não pronunciado, se houver decorrido um ano, depois da perpetração
do delito.'' (Lei nº 2. 033, citada, art. 13, § 4º; Decreto nº 4. 824 de
22 de Novembro de 1871, art. 29, § 3º).\footnote{. Respectivamente, art.
  13. ``O mandado de prisão será passado em duplicata. O executor
  entregará ao preso, logo depois de efetuada a prisão, um dos
  exemplares do mandado com declaração do dia, hora e lugar em que
  efectuou a prisão, e exigirá que declare no outro havê-lo recebido;
  recusando-se o preso, lavrar-se-á auto assinado por duas testemunhas.
  Nesse mesmo exemplar do mandado, o carcereiro passará recibo da
  entrega do preso com declaração do dia e hora. § 4º Não terá lugar a
  prisão preventiva do culpado se houver decorrido um ano depois da data
  do crime''. Art. 29. ``Ainda antes de iniciado, o procedimento da
  formação da culpa ou de quaisquer diligências do inquérito policial, o
  promotor público, ou quem suas vezes fizer, e a parte queixosa,
  poderão requerer, e a autoridade policial representar, acerca da
  necessidade ou conveniência da prisão preventiva do réu indiciado em
  crime inafiançável, apoiando-se em prova de que resultem veementes
  indícios de culpabilidade, ou seja, confissão do mesmo réu, ou
  documento ou declaração de duas testemunhas; e, feito o respectivo
  autuamento, a autoridade judiciária competente para a formação da
  culpa, reconhecendo a procedência dos indícios contra o arguido
  culpado e a conveniência de sua prisão, por despacho nos autos a
  ordenará, ou expedindo mandado escrito, ou requisitando por
  comunicação telegráfica, por aviso geral na imprensa ou por qualquer
  outro modo que faça certa a requisição. § 3º Não poderá ser ordenada
  ou requisitada, nem executada, a prisão de réu não pronunciado, se
  houver decorrido um ano depois da perpetração do crime''.}

Onde a prova de que seja Francisco Aldo autor do homicídio que se lhe
atribui?

Onde os depoimentos de duas testemunhas, que jurassem \emph{de ciência
própria} sobre o fato criminoso?

Onde os documentos comprobatórios do crime?

Onde a \emph{confissão} do culpado?

Perante que juiz competente foi ela feita?

Quais as circunstâncias do fato criminoso com que ela coincide?

Quando e porque autoridade foi Francisco Aldo processado?

Quando e por quem foi ele pronunciado?

Que juiz expediu precatória\footnote{. Carta precatória. Instrumento
  pelo qual um juiz de uma jurisdição pede ao juiz de outra jurisdição
  que cumpra um mandado ou sentença sua.} para a sua prisão?

A quem foi ela dirigida?

Quem viu essa precatória ou requisição?

Onde se acha ela?

Houve requisição telegráfica ou pela imprensa?

Nada disto se sabe; tudo são conjecturas; tudo são suspeitas; e o
próprio dr. juiz municipal de Pirassununga, no mandado que expediu,
\emph{decreta a prisão por suspeitas que tem de que Francisco Aldo seja
criminoso!...}

Entretanto, é certo que a ilustrada redação da \emph{Tribuna}, em face
da lei, dando largas ao arbítrio, em nome dos princípios liberais, que
defende, qualifica de precipitado o procedimento da Relação, e tece
elogios ao juiz violento!...

S. Paulo, 24 de Janeiro de 1879.

LUIZ GAMA.

\textbf{25. FRANCISCO ALDO DE OLIVEIRA {[}III{]}}\footnote{. In: \emph{A
  Província de S. Paulo} (SP), Seção Livre, 22/02/1879, p.~2.}

\textbf{*didascália*}

\emph{Embora tenha afirmado que não voltaria mais a público tratar dessa
questão, haja vista a vitória obtida no tribunal e a exposição de suas
razões jurídicas, um fato novo fez Gama dar uma palavra a mais. O chefe
de polícia do Rio Grande do Sul havia comunicado ao seu correspondente
chefe de polícia paulista que Francisco Aldo de Oliveira não tinha
participação alguma no homicídio do ex-juiz de Bagé e, mais, ``que
contra ele, ou contra outro, por tal motivo, jamais fora deprecada
prisão''. Certamente Gama havia diligenciado tal pedido de
esclarecimento. Seria, caso acessada em tempo hábil, prova fatal no
Tribunal da Relação. Não foi, contudo, necessária. Havia outros
elementos probantes em sua defesa. A razão para Gama voltar à imprensa
sobre esse caso, além de dar ao público uma notícia relevante, foi
acenar aos redatores da Tribuna de que eles cometeram uma injutiça e
deviam, portanto, ``uma justa retificação'' à vítima e ao tribunal. }

\begin{center}\rule{0.5\linewidth}{\linethickness}\end{center}

Quando o colendo Tribunal da Relação concedeu \emph{habeas-corpus} a
este cidadão, preso como assassino do ex-juiz de direito de Bagé, na
província de São Pedro do Rio Grande do Sul, qualificando de ilegal a
sua detenção, {[}e{]} o mandou pôr em liberdade, a ilustrada redação da
\emph{Tribuna}\footnote{. \emph{Tribuna Liberal.}} qualificou de
imprudente este procedimento da Relação, que atirava ao seio da
sociedade um grande criminoso, como tal reconhecido, e contra quem, pelo
juízo competente, havia sido expedida ordem de prisão.

Está terminada a questão.

O sr. dr. chefe de polícia do Rio Grande acaba de declarar ao desta
província que Francisco Aldo de Oliveira não tivera parte no assassinato
do dr. juiz de direito de Bagé; e que contra ele, \emph{ou contra
outro}, por tal motivo, jamais fora deprecada\footnote{. Ato escrito
  pelo qual um juiz, ou autoridade competente, pede a outro que cumpra
  algum mandado ou ordene alguma diligência.} prisão.

Os dignos redatores da \emph{Tribuna}, por sua própria dignidade, devem
ao sr. Francisco Aldo e ao colendo Tribunal uma justa retificação.

S. Paulo, 21 de fevereiro de 1879.

LUIZ GAMA.

\textbf{26. RIBEIRÃO PRETO}\footnote{. In: \emph{A Província de S.
  Paulo} (SP), Seção Livre, 30/04/1879, p.~2.}

\textbf{*didascália*}

\emph{Assinado por} Afro\emph{, o artigo é uma das provas que evidenciam
de modo concludente a autoria de Gama. Um fazendeiro de Ribeirão Preto,
à época uma das localidades do interior paulista mais distante de sua
capital, foi acusado e preso pelo crime de tentativa de homicídio. Após
três} habeas-corpus \emph{denegados no juízo local, a causa do
fazendeiro João Franco de Moraes Octavio chega ao Tribunal da Relação de
São Paulo. Quem apresenta a petição e sustenta oralmente a demanda de}
habeas-corpus\emph{? Luiz Gama. Quem apresenta a perspectiva da defesa
ao público da capital?} Afro\emph{. Ambos,} Afro \emph{e Gama, em
escritos diferentes, comentam o mesmo trecho da ``prova exibida, em
juízo policial''.} Afro \emph{e Gama usam até de uma frase praticamente
igual. ``Foi isto elevado à categoria de tentativa de homicídio'',
disse} Afro\emph{, ao passo que Gama dizia que aquela prova fragílissima
``foi elevada à categoria de tentativa de homicídio''.} Afro\emph{,
nesse caso, introduz Gama ,que, dias depois, tomaria assento na tribuna
da imprensa, agora em nome próprio, para defender a causa de seu
cliente.}

\begin{center}\rule{0.5\linewidth}{\linethickness}\end{center}

Na vila do Ribeirão Preto foi preso, pelo delegado de polícia, à
requisição do juiz municipal, o fazendeiro João Franco de Moraes
Octavio, como \emph{mandante de tentativa de homicídio}, na pessoa do
coletor\footnote{. Diz-se do funcionário do Ministério da Fazenda
  encarregado do lançamento e/ou arrecadação de tributos.} da mesma
vila, Antonio Bernardino Velloso.

É a seguinte a prova exibida, em juízo policial, contra o detido:

José Pedro de Almeida declara que seu ex-patrão, João Franco de Moraes
Octavio, o incumbira de dar um tiro em Antonio Bernardino Velloso; e que
ele, Almeida, em vez de dar o tiro, como havia prometido a seu amo, e
tivera intenção de fazer, fora à Velloso e lhe denunciara o caso,
pondo-o ao fato de toda a ocorrência.

Foi isto elevado à categoria de tentativa de homicídio; o inquérito
policial foi secreto e exigido em ofício reservado pelo dr. juiz de
direito da comarca; e, em virtude deste inquérito, da exigência do juiz
de direito e de depoimento de José Pedro de Almeida, foi Moraes Octavio
recolhido à prisão!...

Interpôs por três vezes recurso de \emph{habeas-corpus} perante o dr.
juiz de direito, e foi desatendido!...

Hoje, o cidadão Luiz Gama interpôs o mesmo recurso perante o colendo
Tribunal da Relação do distrito, onde foi ouvida com admiração a leitura
dos documentos exibidos e concedida unanimemente a ordem de
\emph{habeas-corpus}, por entenderem os dignos desembargadores \emph{que
nem existe o crime arguido}.

Veja o público, e lamentem os jurisperitos o modo porque\footnote{. Pelo
  qual.} se administra justiça no interior da província.

\emph{Afro}.

\textbf{27. RIBEIRÃO PRETO} -- \textbf{RESPOSTA AO PÉ DA
LETRA}\footnote{. In: \emph{Jornal da Tarde} (SP), Ineditoriais,
  11/05/1879, p.~2.}

\textbf{*didascália*}

\emph{No meio do litígio em torno da legalidade da prisão de João Franco
Moraes Octavio, Luiz Gama sai em defesa de seu cliente, que estava
preso, respondendo os irmãos Camargo -- um deles, juiz em Ribeirão Preto
e responsável pela manutenção da prisão de Octavio. Embora a contenda
tenha ganhado as páginas da} Província de S. Paulo\emph{, Gama publica
no} Jornal da Tarde\emph{, optando em lançar a ``resposta ao pé da
letra'' ainda no mesmo dia, haja vista que o} Jornal da Tarde\emph{,
como o nome indica, era um periódico vespertino e saía publicado horas
depois da} Província\emph{. De fato, o julgamento de Octavio pelo
Tribunal da Relação estava na pauta. Gama conseguiu a soltura de seu
cliente naquele mesmo dia. Teria o artigo influenciado os
desembargadores que unanimemente votaram pela ilegalidade da prisão e,
portanto, pela imediata soltura de Octavio? É bastante provável. Gama
detalha aspectos do sumário criminal e da formação da culpa do suposto
autor do crime. A argumentação de Gama desmonta a peça de acusação. Uma
única testemunha ocular deporia em desfavor de Octavio e, segundo o
depoimento de outras testemunhas no inquérito, aquela mesma testemunha
ocular não seria digna de maior crédito. Gama fulmina: ``Esta fantástica
anedota, publicada por um ébrio, completamente nua de circunstâncias
verossímeis, foi elevada à categoria de tentativa de homicídio e deu
lugar à prisão, que se pretende justificar, de João Franco de Moraes
Octavio!!!'' A verve do publicista, contudo, cede lugar à sobriedade do
jurista. Num rol de perguntas fundadas na razão jurídica e na
interpretação legal, Gama devolve aos contendores (e leitores)
evidências de que Octavio não estava preso legalmente. O resultado saiu
no mesmo dia. Octavio fora da cadeia.}

\begin{center}\rule{0.5\linewidth}{\linethickness}\end{center}

O exmo. sr. dr. J. A. de Camargo, em defesa do seu estimável irmão, o
sr. dr. Hyppolito de Camargo, juiz de direito da comarca do Ribeirão
Preto, e no intuito de refutar um artigo, firmado pelo digno sr. dr.
Candido Pereira Barreto, relativo à violenta prisão do sr. João Franco
de Moraes Octavio, realizada com flagrante violação da lei, fez hoje
inserir, na \emph{Província},\footnote{. Cf., por exemplo, \emph{A
  Província de S. Paulo (SP)} de 09/05/1879, 11/05/1879 e 30/05/1879.} o
injurídico despacho proferido pelo sr. dr. Hyppolito de Camargo, em que
pretende moralizar a criminosa prisão que fora decretada pelo \emph{juiz
municipal suplente}, e uma consulta assinada pelos eminentes mestres --
drs. Ramalho\footnote{. Joaquim Ignacio Ramalho (1809-1902), nascido em
  São Paulo (SP), foi presidente da província de Goiás (1845-1848) e
  diretor da Faculdade de Direito de São Paulo (1891-1902). Professor
  reconhecido, publicou obras jurídicas, a exemplo de \emph{Elementos de
  processo criminal para uso das Faculdades de Direito do Império}
  (1856) e \emph{Praxe brasileira} (1869), que Gama, entre outros
  advogados, usualmente citava em suas petições.} e Francisco Justino de
Andrade.\footnote{. Francisco Justino Gonçalves de Andrade (1821-1902),
  nascido na Ilha da Madeira, Portugal, formou-se e fez carreira
  jurídica em São Paulo. Foi professor de Direito Natural e Direito
  Civil, alcançando notoriedade nesse último campo como autor de
  diversos livros doutrinários.}

A defesa do sr. Candido Barreto é, para mim, um dever; e é para
cumpri-lo que venho à imprensa.

Antes da lição de direito, a verdade dos fatos.

\emph{José Pedro de Almeida}, agregado de João Franco de Moraes Octavio,
\emph{denunciou} que o mesmo Octavio o imcumbira de \emph{dar umas
pauladas, ou matar ao coletor}\footnote{. Diz-se do funcionário do
  Ministério da Fazenda encarregado do lançamento e/ou arrecadação de
  tributos.} \emph{Antonio Bernardino Velloso}; e que ele, em vez de
cometer o crime (?!), dera aviso a Velloso.

Foi inquerido \emph{José Pedro}, \emph{como} \emph{testemunha}, que
exerceu o tríplice papel de \emph{depoente, denunciante} e
\emph{co-réu}, se é que o fato constitui delito...

\emph{José Pedro} É A ÚNICA PESSOA \emph{que refere a suposta
ocorrência; as demais testemunhas ouviram dele}.

Eusebio de Carvalho e Thomaz de Aquino, TESTEMUNHAS DA ACUSAÇÃO,
afirmaram, no sumário, \emph{que José Pedro é bêbado habitual e indigno
de crédito!...} e nenhuma outra testemunha o abona.

Esta fantástica anedota, publicada por um ébrio, completamente nua de
circunstâncias verossímeis, foi elevada à categoria de \emph{tentativa
de homicídio} e deu lugar à prisão, que se pretende justificar, de João
Franco de Moraes Octavio!!!...

Agora a lição de direito.

``A prisão, antes de culpa formada, à exceção de flagrante delito,
\emph{só pode ter lugar} nos crimes inafiançáveis, por mandado escrito
do juiz competente para a formação da culpa, ou à sua requisição,
PRECEDENDO, neste caso, \emph{ao mandado ou à requisição, declaração de
duas testemunhas,} QUE JUREM DE CIÊNCIA PRÓPRIA'' (Lei nº 2.033 de 20 de
Setembro de 1871, art. 13, § 2º; Decreto nº 4.824 de 22 de Novembro de
1871, art. 19).\footnote{. Respectivamente, art. 13. "O mandado de
  prisão será passado em duplicata. O executor entregará ao preso, logo
  depois de efetuada a prisão, um dos exemplares do mandado, com
  declaração do dia, hora e lugar em que efetuou a prisão, e exigirá que
  declare no outro havê-lo recebido; recusando-se o preso, lavrar-se-á
  auto assinado por duas testemunhas. Nesse mesmo exemplar do mandado, o
  carcereiro passará recibo da entrega do preso com declaração do dia e
  hora.

  § 2°. À exceção de flagrante delito, a prisão antes da culpa formada
  só pode ter lugar nos crimes inafiançáveis, por mandado escrito do
  juiz competente para a formação da culpa ou à sua requisição; neste
  caso, precederá ao mandado ou à requisição declaração de duas
  testemunhas, que jurem de ciência própria, ou prova documental de que
  resultem veementes indícios contra o culpado ou declaração deste
  confessando o crime``. Art. 19.''Além das atribuições subsistentes,
  compete aos juízes de paz: § 1°. Processar e julgar as infrações de
  posturas municipais; § 2°. Obrigar a assinar termos de segurança e bem
  viver, não podendo, porém, julgar as infrações de tais termos; §
  3°.Conceder a fiança provisória".}

Peço agora ao exmo. sr. dr. Camargo que me conteste os fatos,
comprovando imediatemente os seus assertos;

Que, perante o direito, demonstre que o fato atribuído a Moraes Octavio
constitui crime de \emph{tentativa de morte};

Que a declaração de \emph{José Pedro}, perante a lei, \emph{equivale ao
depoimento de duas testemunhas, que jurem de ciência própria};

Que a prisão de Moraes Octavio foi regular, e que a lei não foi
flagrantemente violada;

Que o despacho proferido pelo sr. dr. Hyppolito de Camargo tem senso
jurídico;

Que, finalmente, o parecer firmado pelos exmos. srs. drs. Ramalho e
Justino de Andrade tem aplicação ao caso vertente.

Sou amigo do sr. dr. Hyppolito de Camargo, cuja honestidade não pode ser
posta em dúvida; sou apreciador da sua inteligência, como literário e
como jurista; em minha palavra, e na sinceridade da minha consciência,
nunca lhe faltaram encômios\footnote{. Elogios.}; nesta questão, porém,
quanto a mim, a sua defesa descansa exclusivamente na pureza
incontestável das suas intenções.

S. Paulo, 11 de Maio de 1879.

LUIZ GAMA.

\textbf{LADRÃO QUE ROUBA LADRÃO}

\textbf{*didascália*}

\emph{No curso da monumental defesa que Gama fez do funcionário público
Antonio Lagarcha no caso do roubo da alfândega de Santos, surge na
crônica forense da capital um certo} Afro:.\emph{, assim mesmo, com a
conhecida pontuação maçônica no arremate da assinatura. A causa era de
liberdade.} Afro:.\emph{, portanto, tratava da demanda de liberdade de
uma mulher escravizada que foi alforriada mediante o pagamento de uma
alta quantia. Julgando-se lesado em seu direito de propriedade, um tal
``prejudicado'', assim se auto-intitulava, foi à imprensa reclamar
daquela causa de liberdade, em particular, bem como do modo pelo qual se
andava decidindo causas de liberdade em São Paulo.} Afro:. \emph{não
deixou por menos. Ciente do processo e dos bastidores inerentes ao
andamento da causa, devolveu ao ``prejudicado'', agora na alcunha de
``prejudicador'', de que a legalidade da alforria era incontestável, ao
contrário do passado e do presente duvidoso de homens da laia dele,
homem branco, escravizador e afeito a outros negócios escusos e
criminosos. O ``prejudicado'', por sua vez, voltou à carga com inédita
virulência, formada tanto pelo racismo quanto pela soberba dos que se
julgam impuníveis.} Afro:.\emph{, contudo, deu a palavra final sobre o
caso que tinha como argumento jurídico de fundo o conflito entre
direitos de liberdade e direitos de propriedade. O ``prejudicado''
atacava a lisura do meio como foi constituído o pecúlio da mulher
alforriada, alegando ter sido produto de roubo.} Afro:.\emph{, após
defender a legalidade do modo pelo qual o dinheiro foi adquirido,
sarcástica e sobriamente contra-atacava dizendo que o ``prejudicado''
não só teria enriquecido por uma séria de roubos, como também queria
roubar no próprio arbitramento da então alforrianda. A moral da história
quem dava era} Afro:.\emph{. Afinal, se o ``prejudicado'' havia feito
fortuna através da pilhagem e, ``sabendo que o dinheiro era roubado'',
queria tomar parte no suposto butim, o que seria ele? Ao que} Afro:.
\emph{concluía fatalmente, não com uma citação do Código Criminal, mas
sob a licença poética que o uso do pseudônimo conferia ao autor e com o
ditado popular que, aplicado ao contexto, deixava a vitória no tribunal
ainda mais saborosa: ``Bem diz o povo: 'Ladrão que rouba ladrão / Tem
cem anos de perdão'''. Em outras palavras, ainda que o escravizado
tivesse roubado o dinheiro que foi doado para a alforria, o que} Afro:.
\emph{não acede, estava ele devolvendo a paga de uma propriedade formada
sob os auspícios de roubo pregresso. Daí a quadrinha popular na boca do
defensor da legalidade da alforria. Quem ousaria contestar}
Afro:.\emph{?}

\textbf{28. ESCÂNDALO -- I}\footnote{. In: \emph{A Província de S.
  Paulo} (SP), Seção Livre, 13/09/1877, p.~2.}

\textbf{*didascália*}

\emph{O auto-intitulado ``prejudicado'', um homem branco e senhor de
escravizados, estava revoltado pelo modo como se deram (e se davam, de
modo geral) as alforrias mediante pagamento no juízo de direito de São
Paulo. Um caso concreto -- a alforria de uma mulher escravizada --
motivou o ``prejudicado'' a vir a público. Ele alegava que um
escravizado doou o valor da alforria da mulher através de recursos
obtidos por meios ilícitos. O ``prejudicado'', em suma, estava
preocupado em prejudicar potenciais direitos de liberdade. A réplica,
como se veria, mostrava que essa causa de liberdade era
instrumentalizada por alguém bastante experiente em firmar o
entendimento jurídico criticado pelo ``prejudicado'', a saber, o rito da
petição inicial seguida de depósito e arbitramento favorável ao
demandante.}

\begin{center}\rule{0.5\linewidth}{\linethickness}\end{center}

Declarada livre uma escrava pertencente a um indivíduo residente nesta
capital, é conveniente que todos saibam quanto caminha adiantada a
\emph{especulação}, aliás honrosíssima.

Um escravo que não possuía pecúlio\footnote{. Patrimônio, quantia em
  dinheiro que, por lei (1871), foi permitido ao escravizado constituir
  a partir de doações, legados, heranças e diárias eventualmente
  remuneradas.} para sua liberdade, firmando título de doação a fim de
ser preenchida a prescrição legal!, quando provas existem que a quantia
exibida e depositada por determinação judicial não foi adquirida por
meios lícitos! É extraordinário!

Se a liberdade deve ser protegida, não pode todavia ser ao ponto de
causar violação do direito de propriedade, garantido amplamente pela
Constituição; entretanto, é uso inveterado\footnote{. Estabelecido,
  arraigado.} no foro desta capital o arbitramento em prejuízo do
senhor.

Feliz foro para tão vantajoso \emph{negócio!}

Os depositários, que também são \emph{determinados amigos},
aproveitam-se dos serviços e concorrem para a boa vida dos \emph{felizes
depositados}.

\emph{Um prejudicado.}

\textbf{28. 1. ESCÂNDALO -- I {[}réplica{]}}\footnote{. In: \emph{A
  Província de S. Paulo} (SP), Seção Livre, 15/09/1877, p.~2.}

\textbf{*didascália*}

\emph{A resposta é tanto sóbria quanto sarcástica.} Afro:.
\emph{demonstrava estar bastante a par das ``questões manumissórias no
foro da capital''. Ao estilo de um advogado recém-vitorioso na corte,}
Afro:. \emph{limitava-se a dizer que a alforria era legítima e legal
porque, em síntese, o ``pecúlio foi doado por quem podia fazê-lo; porque
adquiriu o dinheiro pelo seu trabalho''. Contudo, devolvia na réplica
que conhecia mais do que a causa de liberdade da alforriada: conhecia o
passado do ``prejudicador'', que teria adquirido posses e propreidades
``surrupiando o alheio''. }

\begin{center}\rule{0.5\linewidth}{\linethickness}\end{center}

Resposta ao sr. \emph{Prejudicado} que, na \emph{Província} de ontem,
publicou um artigo relativamente a questões manumissórias\footnote{.
  Formas processuais em que se demanda a liberdade.} no foro da capital.

O pecúlio\footnote{. Patrimônio, quantia em dinheiro que, por lei
  (1871), foi permitido ao escravizado constituir a partir de doações,
  legados, heranças e diárias eventualmente remuneradas.} foi doado por
quem podia fazê-lo; porque adquiriu o dinheiro pelo seu trabalho.

Saiba o sr. prejudicador que nem todos adquirem fortuna surrupiando o
alheio, \emph{como alguém}; nem todos têm a felicidade de contar
{[}com{]} tuteladas ricas, com as quais se casem vantajosamente; nem
tanta falta de pudor, como... que recebe (e acha pouco!!!) 1:200\$ pela
alforria \emph{da mãe do seu sobrinho carnal!...}

Só homens desta laia podem, sem fundamento, invectivar\footnote{.
  Afrontar através de linguagem insultante.} juízes honestos.

\emph{Afro:.}

\textbf{29. ESCÂNDALO -- II}\footnote{. In: \emph{A Província de S.
  Paulo} (SP), Seção Livre, 16/09/1877, p.~3.}

\textbf{*didascália*}

\emph{A tréplica do tal ``prejudicado'' ilustra de maneira categórica os
interesses, ideias e inimigos com que} Afro\emph{-Gama costumava lidar
na trincheira pelo direito à liberdade. Refutando prolongar a discussão
sob explícito pretexto racista, o auto-intitulado ``prejudicado'' dá
outros elementos sobre a causa de liberdade que o faziam vir a público
criticar o juízo de direito da capital; e estende quais os pressupostos
de sua reclamação, a saber, a suposta constituição fraudulenta do
pecúlio e ``o roubo como meio legal de adquirir a propriedade''. Essas
duas chaves de leitura seriam exploradas na contestação feita por}
Afro:.\emph{, que, pelo teor do ataque que recebia, não era um ocasional
defensor de uma eventual demanda de liberdade; demonstrava conhecer o
processo por dentro, estava a par da liberalidade de terceiros na
constituição do pecúlio da mulher escravizada e tinha em sua biografia
alguns dos traços estruturantes da ofensa racial e política que o
``prejudicado'' tentava, afinal e sem sucesso, prejudicar. }

\begin{center}\rule{0.5\linewidth}{\linethickness}\end{center}

Apareceu \emph{Afro:.} pela \emph{Província} {[}de{]} nº 775 de
16\footnote{. Em realidade, como visto, \emph{Afro} publicou na edição
  do dia 15/09/1877.} do corrente, e pelo que se nota é \emph{pedreiro
livre}, não obstante ser escravo.\footnote{. É de se notar o escárnio
  nas entrelinhas -- entre o grifo e a ofensa. A designação de
  ``pedreiro livre'' pode ser lida como eufemismo para trabalhador
  pobre, não escravizado, e sem insígnia alguma -- canudo, anel, título
  -- que o qualificasse como bacharel ou fidalgo. A definição seguinte
  -- ``não obstante ser escravo'' -- é implacável com o estatuto
  jurídico com que faziam perseguir o ex-escravizado que, embora
  alcançasse a liberdade, seria comumente estereotipado pela marca da
  escravização pregressa.}

Para declarar que o \emph{pecúlio}\footnote{. Patrimônio, quantia em
  dinheiro que, por lei (1871), foi permitido ao escravizado constituir
  a partir de doações, legados, heranças e diárias eventualmente
  remuneradas.} \emph{foi doado por quem podia fazê-lo, porque adquiriu
o dinheiro pelo seu trabalho}, não precisava vir à imprensa, porquanto
todos percebem que o \emph{trabalho} a que se refere é sobremodo lícito,
máxime\footnote{. Principalmente, especialmente.} para aqueles que
consideram o roubo como meio legal de adquirir a propriedade.

Quanto às outras \emph{cantigas} do pedreiro livre, petas, petas,
petas,\footnote{. Mentiras, engodos.} não merecem resposta; mesmo por
não ser honrosa qualquer discussão com escravo, e outros canalhas,
apenas diremos: ``só homens de alma tão negra podem, mentindo, levantar
tantas calúnias.''\footnote{. Se a abjeta qualificação do primeiro
  parágrafo parecia excessiva, no arremate o auto-intitulado
  ``prejudicado'' fez o aparentemente impossível: a um só tempo subiu o
  tom da injúria e baixou o nível da contenda. Insistiu na marca
  pregressa da escravização de \emph{Afro} como pretexto para fugir da
  discussão; somou o adjetivo ``canalha'' ao seu já prejudicado ataque
  e, como cusparada final, assacou uma ideia -- ``só homens de alma tão
  negra'' -- que expressava o racismo que dominava sua reflexão. A
  réplica a esse texto, que viria na edição seguinte -- o leitor a lerá
  --, demonstra que \emph{Afro} não deixaria esse ponto sem resposta.}

\emph{Um prejudicado.}

\textbf{29. 1. ESCÂNDALO -- II {[}réplica{]}}\footnote{. In: \emph{A
  Província de S. Paulo} (SP), Seção Livre, 18/09/1877, p.~2.}

\textbf{*didascália*}

\emph{A discussão teria fim com a palavra de} Afro:.\emph{. Às injúrias
do ``prejudicado'', aqui tratado como ``homem branco'' e ladrão,} Afro:.
\emph{reiteraria um ponto-chave de sua primeira réplica, a saber, a
imoralidade senhorial no arbitramento de uma alforria mediante pagamento
da própria mãe de um seu sobrinho carnal, o que explicitava relações
familiares social e juridicamente espúrias. Uma vez que o argumento
central do ``prejudicado'' era de que o pecúlio seria fraudulento,}
Afro:. \emph{retorquia-lhe que, mesmo ``sabendo que o dinheiro era
roubado'', teria ele, o auto-intitulado ``prejudicado'', pedido mais
dinheiro ainda. Ou seja, enquanto era possível lucrar com o suposto
roubo, o ``prejudicado'' estava pronto para a rapina; não mais o sendo,
escandalizou-se com a infâmia de que estava antes contente em
participar. Batendo nesse ponto,} Afro:. \emph{castigava a imoralidade
da figura do senhor da escravizada que alcançou a liberdade,
acrescentando que, ``se o roubo não fosse causa lícita'', o
``prejudicado'' não teria feito fortuna ``naquele celebérrimo contrato
para o Mato Grosso!''} Afro\emph{-Gama voltava ao passado do ``homem
branco'' para contestá-lo onde certamente lhe doía mais. E tirava o chão
de futuros ``prejudicados'' que não passavam de prejudicadores.}

\begin{center}\rule{0.5\linewidth}{\linethickness}\end{center}

Sr.~\emph{Prejudicado}.

Com que então o negrinho doador do pecúlio\footnote{. Patrimônio,
  quantia em dinheiro que, por lei (1871), foi permitido ao escravizado
  constituir a partir de doações, legados, heranças e diárias
  eventualmente remuneradas.} é \emph{cativo} e é \emph{ladrão}!!

E V. S., \emph{homem branco} (!), \emph{sabendo que o dinheiro era
roubado}, pediu 4:000\$000 pela alforria da \emph{mãe de seu sobrinho}!!

Que nobreza de sentimentos!!!

Que pureza de consciência!!!

Bem diz o povo:

"Ladrão que rouba ladrão

Tem cem anos de perdão"

Ora, sr. \emph{Prejudicado}, estes juízes são mesmo uns
\emph{corruptos!...} e os protetores dos escravos uns especuladores
vis!...

Confesse, sr. \emph{Prejudicado}, se o roubo não fosse causa lícita, V.
S. não teria dado \emph{passos} adiante do seu ex-sócio naquele
celebérrimo\footnote{. Superlativo de célebre, algo como muitíssimo
  célebre.} contrato para o Mato Grosso!...

Tanto não fez o \emph{negro ladrão!}

\emph{Afro:.}

\textbf{CARTAS AO "MISTER JOS BONIFÁCIO E A OUTROS MISTERS}

\textbf{*didascália*}

\emph{``Procurem o Luiz Gama -- original; porque há, por aí, em
cavilosas brumas, um Luiz Gama -- de imitação...'' A bem humorada tirada
demonstra a maneira franca e confortável com que Gama tratava os
leitores. Própria, aliás, de quem dominava os macetes do ofício da
escrita. Nada, a essa altura da partida, que nos surpreendenda. Nessa
seção, contam-se onze cartas escritas por Gama. Entre elas, cartas
privadas e cartas públicas; cartas com endereçamento amplo --
``\emph{Aos homens de bem}'', ``\emph{Carta aos cidadãos franceses}'',
etc. --, ou direcionado -- ``\emph{Carta a José Bonifácio}'',
``\emph{Carta a Francisco Antonio Duarte}'', etc. Todas, sublinha-se,
firmadas em seu nome próprio. Lidas em conjunto, considerado obviamente
o triênio desse volume, pode-se observar uma rede pessoal e profissional
variada e eclética. Há o célebre professor de direito e político José
Bonifácio, o mister José, numa referência que atesta senão amizade ao
menos alguma proximidade entre colegas; mas há também a declaração
pública de apreço ao modesto cabo de esquadra Francisco Antonio Duarte,
num sinal que soa evidente de uma amizade construída ainda nos tempos de
quartel, quando Gama servia como oficial da Força Pública. É de se
destacar que duas de suas cartas são dirigidas aos seus médicos. Cada
uma, todavia, datada de um ano diferente -- 1878, 1879. Pode-se
entender, lidas a contrapelo, que tratam de uma mesma e grave
enfermidade que insistentemente acompanhava Gama desde seu súbito
aparecimento, provavelmente em fevereiro 1878. Seria a diabetes que o
fulminou em agosto de 1882? Não se sabe com toda precisão. Contudo, a
partir de então, pode-se conjecturar que Gama passou a lidar com o
trabalho e a vida com novas e indesejadas preocupações. }

\textbf{30. CARTA A JOSÉ BONIFÁCIO}\footnote{. In: Biblioteca Mário de
  Andrade, Seção de Obras Raras, Correspondências de José Bonifácio, P 9
  D 77, 20 de Maio de 1877.}

\textbf{*didascália*}

\emph{A carta é reveladora da relação pessoal e profisisonal que Gama
tinha, em 1877, com José Bonifácio. Entre possíveis observações nesse
sentido, notem a forma como Gama pede que Bonifácio lhe envie livros.
``Tenho pressa'', finalizava Gama, sugerindo nas entrelinhas que a
urgência se devia a algum estudo ou artigo que estivesse escrevendo. }

\begin{center}\rule{0.5\linewidth}{\linethickness}\end{center}

1877

Maio 20

Mister José.

Saúde, bom apetite, paz de espírito, e áureas inspirações.

Não me faça como o Bernardo, que, de tudo quanto lhe pedi, arranjou,
apenas, o improvimento\footnote{. Decisão desfavorável no mérito de um
  recurso processual, nesse caso mais exatamente de uma Revista no
  âmbito do Supremo Tribunal de Justiça.} de uma Revista, no Supremo
Tribunal de Justiça....\ldots{}

Manda vir, para mim, e me remeta, se na Corte não houver, o Dicionário
Jurídico Pereira e Souza\footnote{. Escrito pelo jurista português
  Joaquim José Caetano Pereira e Souza (1756-1819), \emph{Esboço de hum
  diccionario juridico, theoretico e practico} (1825) foi um importante
  livro jurídico-pragmático para o advogado atuar no foro.};

A obra sobre acontecimentos políticos do Brasil, pelo Antonio Pereira
Rebouças\footnote{. Antonio Pereira Rebouças (1798-1880), natural de
  Maragogipe (BA), foi advogado, jurista e político. Eleito deputado
  sucessivas vezes, foi autor de destaque nas letras jurídicas e na
  historiografia do século XIX. A obra a que Gama se refere é
  \emph{Recordações da Vida Parlamentar: moral, jurisprudência, política
  e liberdade constitucional}, publicada em dois volumes em 1870.};

Mello Freire\footnote{. Pascoal José de Mello Freire dos Reis
  (1738-1798) foi jurista, historiador do direito, professor de Direito
  Civil da Faculdade de Direito da Universidade de Coimbra e
  desembargador da Casa de Suplicação.}; Direito Civil em
português\footnote{. Referência provável à obra \emph{Instituições de
  Direito Civil Português} (1789-1794).}; -- Creio que a tradução é de
Pernambuco\footnote{Ao invés de tradução, talvez deva-se ler edição.}.

Tenho pressa.

Disponha do

Am.o.

Luiz.

\textbf{31. BILHETE PARA JOSÉ BONIFÁCIO}\footnote{In: Biblioteca Mário
  de Andrade, Seção de Obras Raras, Correspondências de José Bonifácio,
  Cx. 02, P 9 D 77, s. d.}

\textbf{*didascália*}

\emph{Mais uma carta cifrada -- e uma peça a mais para o quebra-cabeça
das relações pessoais e profissionais de Gama. Quem seria o Fructuoso e
do que trataria o folheto por ele confiado? }

\begin{center}\rule{0.5\linewidth}{\linethickness}\end{center}

Mister,

O Fructuoso confiou-me o folheto.

Leia-o com atenção.

Seu ami{[}g{]}o,

Luiz.

\textbf{32. CARTA A JOSÉ BONIFÁCIO}\footnote{In: Biblioteca Mário de
  Andrade, Seção de Obras Raras, Correspondências de José Bonifácio, Cx.
  02, P 9 D 75, s. d.}

\textbf{*didascália*}

\emph{Breve e eloquente, essa carta a José Bonifácio tem um conteúdo
cifrado. Não se sabe exatamente quais as razões do ``tão grande
empenho'' que Gama tomou em favor do ``estudante Tito Antonio da
Cunha''. Ainda assim, a carta é uma peça interessantísima no
quebra-cabeça das relações pessoais de Gama. Qual seria o requerimento
do aluno Tito que seria deliberado na congregação da Faculdade de
Direito de São Paulo? }

\begin{center}\rule{0.5\linewidth}{\linethickness}\end{center}

Exmo.

Peço-lhe encarecidamente que seja favorável, amanhã, na
congregação\footnote{Órgão colegiado decisório da Faculdade de Direito
  de São Paulo}, ao estudante Tito Antonio da Cunha\footnote{Ainda não
  encontrei informações detalhadas desse personagem, nem mesmo indícios
  concretos que teriam levado Gama a interceder por ele. Uma pista, no
  entanto, é digna de nota: na ``lista das faltas dos estudantes da
  Faculdade de Direito dadas até o fim de Maio de 1870'', um certo Tito
  Antonio da Cunha aparece entre os cinco alunos mais faltosos de uma
  turma de trinta e nove quartanistas (cf. \emph{Correio Paulistano}
  (SP), 16/06/1870, p.~4. Qual o motivo para tão enérgico empenho em
  favor de um estudante é uma pergunta em aberto.}.

Depois lhe direi porque tomo tão grande empenho.

Seu amigo obrigadíssimo,

Luiz.

\textbf{33. CARTA AOS REDATORES DA \emph{PROVÍNCIA}}\footnote{In:
  \emph{A Província de S. Paulo} (SP), Seção Livre, 06/11/1877, p.~2.}

\textbf{*didascália*}

\emph{Um dos mais significativos textos da militância republicana de
Gama. Fazendo um balanço crítico dos últimos dez anos do movimento
republicano paulista e brasileiro, Gama denuncia a capitulação e
cooptação de antigos correligionários para uma agenda reformista
inconsistente, moderada e tutelada pelos partidos do Império. ``Somos
radicais; este é o nosso estandarte'', dizia Gama, lamentando, antes de
tudo, a fraqueza programática da ala majoritária do Partido Republicano
e, por outra parte, valorizando a tenacidade da ``minguada fração do
grande Partido Republicano'', que clamava com todas as letras:
``Queremos a reforma pela revolução; temos princípios, temos programa''.
Gama respondia o editorial do jornal} Tribuna Liberal \emph{como se
lesse nele a voz de um antigo aliado. Ainda que não saibamos
precisamente a quem individualmente ele replicava, pode-se ler o artigo
como uma página da história do movimento republicano e um testemunho
valioso dado por um veterano de suas fileiras. Reconhecida liderança
política, ainda que fora dos mecanismos institucionais de poder, a
exemplo do parlamento e dos gabinetes do Executivo, Gama reunía como
poucos condições para fazer um balanço dessa natureza. }

\begin{center}\rule{0.5\linewidth}{\linethickness}\end{center}

Quando, em tempos passados, que não muito se distanciam do presente, nos
reunimos sob a bandeira, e à luz dos princípios da democracia pura,
cristã e socialista, animavam-nos dois grandes pensamentos, tínhamos
duas grandes ideias: derruir a monarquia, em nome do país e da
civilização; estabelecer a República em nome da liberdade.

Então constituíamos um partido, o partido nacional, o partido radical, o
partido da revolução; não se media a sua força pelo número dos
congregados, senão pelo arrojo das concepções, pela firmeza da vontade,
pela singularidade da abnegação, pelo trasonismo\footnote{Atrevimento,
  ousadia excessiva.} das manifestações.

Éramos demolidores das obras do despotismo sob todas as formas
conhecidas, e construtores de uma nacionalidade inteiramente livre.

Declaramos\footnote{Pelo contexto e pelos tempos verbais adotados em
  diversos parágrafos, faz mais sentido que se leia como
  ``Declarávamos''.} guerra formal aos partidos militantes do Império,
mórbidas coortes\footnote{Tropa, força armada.} de
valetudinários\footnote{Débeis, doentes.} Druidas\footnote{Sacerdote,
  por metonímia, aquele investido de algum cargo político ou judiciário
  por delegação do imperador.}, entibiados\footnote{Enfraquecidos,
  debilitados.} há muito pelo fumo do incenso e da mirra\footnote{Resina
  vegetal aromática usada como incenso e recurso medicinal.} dos
palácios; condenamos, sem detença\footnote{Delonga.}, o parlamento e as
assembleias, sáfaras\footnote{Toscas, grosseiras.} chancelarias do rei e
dos seus agaloados\footnote{Indivíduo que usa galão no vestuário.
  Espécie de adorno que sinaliza condecoração, distinção de patente,
  privilégio ou classe.} sátrapas\footnote{Déspotas, tiranos.}; e
votamos à execração pública essa mentira codificada pela hipocrisia, que
a ironia dos poderosos, por mero escárnio, qualificou de --
\emph{Constituição política do império}.

Queríamos construir, depois da luta, da completa derrota e do
aniquilamento indispensável dos nossos adversários, sob a égide de uma
ditadura provisória e necessária, ilustrada e intransigente, inspirada
pelo direito, dirigida pela razão, e dominada pela justiça, não sobre
ruínas, porque tudo seria removido, até os alicerces, mas \emph{em uma
superfície plana}, o edifício moderno da nova sociedade, sem municípios
atrelados, sem magistratura cômica, sem parlamentos subservientes, sem
eleitores autômatos, sem ministérios de fâmulos\footnote{Criados
  domésticos, serviçais subservientes.}, sem religiões de estado, sem
ciência oficial e professores tutelados, sem regimentos
monocráticos\footnote{Autocráticos, absolutos.}, sem exército de
janízares\footnote{Capangas, guarda-costas de déspota.}, e sem escravos,
porque estava proscrito\footnote{Extinto.} o senhor.

Tínhamos um programa infinito, encerrado em uma só palavra -- PROGRESSO
--; alterável todos os dias, porque um programa político não é um
evangelho sedicioso\footnote{Insurgente, indisciplinado.}, que desafia à
revolta, quando não planta o indiferentismo, pelo estoico\footnote{Inflexível,
  rígido.} sofisma da sua desastrosa imutabilidade; é o relatório fiel
das necessidades públicas, e de todas as aspirações legítimas; e da sua
restrita satisfação dependem a felicidade dos povos, a segurança dos
governos, a tranquilidade dos estados, e a conservação das instituições;
em política não há dogmas; os deuses foram-se com a mitologia.

Eis o que, há dez anos, pretendiam os validos\footnote{Prezados.}
republicanos do Brasil.

Hoje, este programa pertence a uma minguada fração do grande Partido
Republicano, disseminado em todo o país.

Somos radicais; este é o nosso estandarte.

Escrevemos estas linhas em resposta ao editorial da \emph{Tribuna
Liberal} de hoje.

Queremos a reforma pela revolução; temos princípios, temos programa.

-- ``Somos homens, enfim, temos futuro!''

S. Paulo, 3 de Novembro de 1877.

L. GAMA.

\textbf{34. CARTA PÚBLICA AOS SEUS MÉDICOS}\footnote{In: \emph{A
  Província de S. Paulo} (SP), Noticiário, Luiz Gama, 27/02/1878, p.~2.}

\textbf{*didascália*}

\emph{Ao dirigir publicamente ``um sincero voto de profunda gratidão'' a
três médicos que o acompanharam na ``grave enfermidade'' de que sofria,
Gama revelava que vivia tempos difíceis. Pelo tom da carta, pode-se
notar que passou por maus bocados. Qual teria sido a moléstia? Que
doença o acometera? Teria sido a diabetes que o vitimou fatalmente em
1882? Ou possuía outra comorbidade? Não se sabe com toda a exatidão. O
que se vê, todavia, é que escapou com vida de uma grave enfermidade e
expressava seu reconhecimento e gratidão aos médicos que, pode-se dizer,
o salvaram de uma situação complicada.}

\begin{center}\rule{0.5\linewidth}{\linethickness}\end{center}

Ilmos. Srs. Redatores.\footnote{A carta é precedida pela seguinte nota
  da redação do jornal: ``Este distinto cidadão, já restabelecido da
  grave enfermidade que há dias sofreu, nos envia as seguintes linhas''.}

Devo aos meus respeitáveis amigos, distintos médicos, drs. Jayme
Serva\footnote{Jayme Soares Serva (1843-1901), baiano, natural de
  Salvador, onde se formou em medicina em 1867. Foi voluntário da pátria
  durante os combates na Guerra do Paraguai e de lá voltou com a patente
  de major médico. Fez carreira médica em São Paulo.}, Clímaco
Barbosa\footnote{Clímaco Barbosa (1839-1912), natural de Salvador (BA),
  foi médico, político e jornalista, sendo redator-proprietário da
  \emph{Gazeta do Povo (SP)} no início da década de 1880. Na qualidade
  de perito e avaliador, colaborou com Luiz Gama em diversas ações
  judiciais.} e Adolpho Gad\footnote{Não encontrei referências
  detalhadas sobre Adolpho Gad, contudo, pode-se dizer que foi um médico
  dinamarquês, formado pela Universidade de Copenhague, especializado em
  oftalmologia, que se radicou em São Paulo, onde fundou e chefiou o 1º
  Serviço de Moléstias dos Olhos da Santa Casa de Misericórdia de São
  Paulo, entre 1885-1892. Aparentemente irrelevante, o indicativo de sua
  especialização médica -- que se encontra em dezenas de anúncios de
  jornais -- sinaliza um traço importantíssimo para compreender o quadro
  clínico de Luiz Gama: diagnosticado com diabetes, provavelmente sofria
  de sintomas da ``grave enfermidade'' relacionada à visão, razão pela
  qual um médico oftalmologista o acompanhava.}, um sincero voto de
profunda gratidão pelo muito interesse e notável perícia com que
trataram-me na grave enfermidade de que fui repentinamente acometido; e
imploro-vos a graça de consentirdes que, pelo vosso conceituado jornal,
eu dê publicidade a este meu voto de reconhecimento e gratidão.

S. Paulo, 26 de fevereiro de 1878.

Vosso respeitador e amigo,

LUIZ GAMA.

\textbf{35. CARTA AOS CIDADÃOS FRANCESES}\footnote{In: \emph{A Província
  de S. Paulo} (SP), Noticiário, 08/09/1878, p. 2. Prêambulo que
  contextualiza o mote da carta de Gama: \emph{``\textbf{O quatro de
  Setembro} - Damos aqui a carta que o cidadão Luiz Gama, ausente,
  enviou aos cidadãos franceses, reunidos no Hotel da Paz, a 4 do
  corrente, para solenizar o aniversário da instituição de governo
  republicano em França. S. Paulo, 4 de setembro de 1878.''}}

\textbf{*didascália*}

\emph{Talvez por complicações de saúde, Gama não compareceu ao jantar
festivo em comemoração ao aniversário da República da França. Promovido
por cidadãos franceses, o evento também reunía militantes do Partido
Republicano Paulista, entre outros entusiastas das relações amistosas
entre Brasil e França. Contudo, mesmo não se fazendo presente, Gama
endereçou uma carta pública aos cidadãos franceses, provavelmente lida
na solenidade realizada no Hotel da Paz, na qual renovava suas
convicções republicanas por meio de um repertório de metáforas políticas
bastante eloquente. Transitando entre temporalidades distintas, Gama
escreve uma peça que, embora laudatória, acena para a luta -- e utopia
-- internacional pela emancipação e união dos povos.}

\begin{center}\rule{0.5\linewidth}{\linethickness}\end{center}

Lá está 92, a esplêndida epopéia

Escrita por um povo à luz de cem batalhas,

Lá está 92 para provar que a ideia

Não morre com metralhas!\footnote{\emph{Victoria da França} (1870),
  versos de Guerra Junqueiro publicados pela Livraria Internacional de
  Ernesto Chardron, Porto-Braga, 20 pp.}

G. JUNQUEIRO\footnote{Abílio Manuel Guerra Junqueiro (1850-1923),
  nascido em Ligares, Portugal, foi político, jornalista e diplomata,
  além de poeta e escritor que alcançou grande notoriedade na vida
  cultural luso-brasileira.}.

Não faltei ao convite; aqui estou.

Vim na palavra; e vim para saudar-vos.

Saúdo, em vós, a República, na República, a liberdade, e na liberdade, o
eterno luzeiro dos povos.

Comemorais, nesta esplêndida festa, o maior dos prodígios populares, o
terceiro estabelecimento da República em França; relembrais, ao clarão
do mais notável dos séculos, o mais importante dos acontecimentos que
registra a história da humanidade.

Este fato aviventa aos cidadãos; os servos tauxia{[}m{]}\footnote{Coram,
  enrubescem.} de opróbrio\footnote{Grande vergonha.} e faz{[}em{]}
estremecer os tiranos.

Esta imponente reunião, estes irrompimentos de júbilo, este civismo
inquebrantável, que tão brilhante se eleva, constituem o hino sagrado da
grande vitória da justiça e da verdade; são os cânticos matinais
entoados por milhões de vozes ao despontar do Sol no oriente: o Sol é a
República; o oriente, a França.

Mas... silêncio!

Estes cânticos sublimes são por Vós entoados perigosamente, na melhor
porção da velha Turquia americana...

Cautela!... Não acordeis, com as vossas abundâncias de alegria, os
súditos felizes, que repousam, em calma, e sonham com a infalibilidade
do Rei...

Moderai as vozes.

Vossos cânticos patrióticos partiram de além século; passaram por sobre
túmulos reais; rememoram as lutas homéricas da revolução;
reboaram\footnote{Ecoar com estrondo, retumbar.} em Versalhes quando se
derruía um trono; foram ouvidos na Convenção\footnote{Referência ao
  regime político denominado Convenção Nacional, que vigorou entre 1792
  e 1795, fundando a Primeira República Francesa.}, quando julgaram o
neto de São Luiz; e na \emph{praça da revolução}, quando o
decapitaram.\footnote{Refere-se a Luís XVI (1754-1793), rei francês
  deposto e decapitado em decorrência dos julgamentos da Revolução
  Francesa.}

Vossos cânticos foram escritos com o sangue divino do Cristo das
monarquias; assinalam o despertar do mundo; recordam a data de uma
sentença imortal; traduzem a verdadeira Ilíada\footnote{No sentido de
  epopeia, associando com o contexto revolucionário francês a
  \emph{Ilíada} da Grécia Antiga, narrativa épica dos acontecimentos da
  Guerra de Tróia.}; chamam-se -- 93 ou a emancipação do povo.

Aqui, porém, neste vastíssimo incógnito paraíso, os republicanos são
como Corifeus\footnote{Metáfora que remete ao personagem-chave do teatro
  grego, indicando, nesse contexto, a proeminência dos republicanos
  brasileiros.} olímpicos, trajam cândida pretexta; querem a liberdade
por a conquista da inteligência; por armas têm a pena e a palavra; são
suas trincheiras a imprensa e a tribuna; têm por baluarte as urnas; por
facho as eleições; e por bandeira a lei do orçamento. Descansa a
liberdade em berço de esmeraldas; medra\footnote{Cresce, desenvolve.}
por entre flores de retórica; brilham ao seu colo rubis e diamantes:
temos uma democracia erótica.

Aqui os republicanos são os Serafins\footnote{No sentido de mensageiros.}
da paz; e a paz é a base da liberdade.

Aqui convivem o Império com a República; a democracia tem por emblema a
coroa; o povo é Rei!

Cautela!...

Onde o povo é Rei os livres falam de manso; porque não é um direito, é
um crime a revolução.

..................................................................

Há na superfície do globo dois pontos culminantes -- Filadélfia e Paris
--, e são como dois pedestais enormes construídos pela natureza.

Sobre um está Washington\footnote{George Washington (1732-1799) foi um
  comandante militar, líder político e estadista, sendo o primeiro
  presidente da República dos Estados Unidos da América (1789-1797).};
no outro Thiers\footnote{Adolphe Thiers (1797-1877), natural de
  Marselha, França, foi advogado, jornalista, historiador e estadista.
  Foi presidente da França (1871- 1873).}; são dois marcos do destino,
erguidos no seio do infinito; um pela América, o outro pela Europa.

Um dia, esses colossos se abraçarão, à face do oceano; e as
nacionalidades livres formarão os -- Estados Unidos do mundo.

Em nome do futuro e da liberdade eu vos saúdo.

Vosso correligionário e amigo,

LUIZ GAMA.

\textbf{36. {[}CARTA DA COMISSÃO POPULAR EM HOMENAGEM A JOAQUIM
LEBRE{]}}\footnote{In: \emph{A Província de S. Paulo} (SP), Noticiário,
  Manifestação, 24/12/1878, p.~2. Antes de reproduzir a íntegra do
  ofício da Comissão Popular, \emph{A Província} noticiou o contexto da
  homenagem. Vejamos:

  ``O sr. Joaquim Lopes Lebre recebeu anteontem em sua casa, à noite, um
  grande número de pessoas que, reunidas e com uma banda de música,
  foram cumprimentá-lo pela graça que lhe fez o governo de Portugal
  enviando-lhe o título de barão. Uma comissão entregou ao agraciado o
  ofcício que abaixo transcrevemos por cópia, sendo em seguida servida
  uma esplêndida mesa de doces. Houve entusiásticos brindes e discursos
  e calorosas saudações, tomando parte nelas os srs. dr. Zeferino
  Candido, Luiz Gama, Cardim, J. M. Lisboa, coronel Rodovalho, Domingos
  Coelho, Serpa e outros cidadãos. O sr. Joaquim Lopes Lebre,
  profundamente comovido, agradeuceu a todas aquelas manifestações de
  apreço''.}

\textbf{*didascália*}

\emph{Assinada por uma Comissão Popular de três membros, Gama entre
eles, a carta pública homenageava um português que havia sido agraciado,
pelo rei de Portugal, com o título nobiliárquico de barão. Embora o
escopo da carta seja incomum com outros escritos de Gama, haja vista
nunca ter homenageado a titulação de qualquer fidalgo, a carta tem
evidentes marcas estílisticas próprias da autoria de Gama, sugerindo,
por outra parte, que o ``testemunho público do muito apreço e admiração
em que o povo tem o alto caráter'' de Joaquim Lebre demonstrava o bom
trânsito que Gama possuía com a colônia portuguesa na capital paulista.}

\begin{center}\rule{0.5\linewidth}{\linethickness}\end{center}

S. Paulo, 22 de Dezembro de 1878.

Ao muito digno sr. Joaquim Lopes Lebre.

A honradez, o trabalho e a perseverança constituem três virtudes que
formam o homem de bem.

V. Excia. para constante prática destas virtudes e pelos generosos
sentimentos do seu coração magnânimo, dando a mão aos fracos, remediando
aos pobres e socorrendo aos infelizes, constituiu-se, para com os homens
de bem, um êmulo digno da maior consideração; para os fracos, um seguro
protetor; e para os infelizes, pelo seu notável desinteresse, um
verdadeiro pai.

O governo de S. M. El-Rei de Portugal, conferindo à V. Excia o título de
barão de S. Joaquim, pagou à V. Excia. uma dívida de honra, contraída
pelos bond portugueses, que não podiam saldá-la.

Os assinados nesta carta, mal interpretando, por sua minguada
inteligência, os sentimentos nobilíssimos dos amigos de V. Excia.,
estrangeiros e nacionais, vêm dar testemunho público do muito apreço e
admiração em que o povo tem o alto caráter de V. Excia. E de quanto
aprecia o ato de benemerência de que, com tanto acerto, acaba de ser
alvo.

Digne-se V. Excia., pois, de aceitar esta pequena prova da mais elevada
consideração dos de V. Excia. criados atentos respeitadores.

Manoel Antonio Ferreira do Valle.

Manoel José Maia.

Luiz G. P. da Gama.

Membros da Comissão Popular.

\textbf{37. AGRADECIMENTO}\footnote{In: \emph{A Província de S. Paulo}
  (SP), Seção Livre, 15/02/1879, p.~2.}

\textbf{*didascália*}

\emph{Carta de agradecimento ao seu médico particular, seu conterrâneo e
amigo, Jayme Serva. Indica, por um lado, que a saúde de Gama inspirava
sérios cuidados no início de 1879 e, por outro lado, que possuía uma
estreita relação de amizade com o baiano que, assim como Gama, fazia
carreira profissional em São Paulo.}

\begin{center}\rule{0.5\linewidth}{\linethickness}\end{center}

Ao meu distinto e honrado amigo sr. dr. Jayme Serva\footnote{Jayme
  Soares Serva (1843-1901), baiano, natural de Salvador, onde se formou
  em medicina em 1867. Foi voluntário da pátria durante os combates na
  Guerra do Paraguai e de lá voltou com a patente de major médico. Fez
  carreira médica em São Paulo.} devo um público testemunho de gratidão
que, destarte, apresso-me de saldar.

Prevalecendo-me da oportunidade, venho dar de minha boa razão por faltas
antigas; e sinceramente agradecer aos bons amigos e dignos cavalheiros,
que longo fora nomear, as provas inequívocas de apreço de que, por atos
de nímia e espontânea delicadeza, lhes sou devedor.

S. Paulo, 14 de Fevereiro de 1879.

L. GAMA.

\textbf{38. AOS HOMENS DE BEM}\footnote{In: \emph{A Província de S.
  Paulo} (SP), Seção Livre, 20/06/1879, p.~2.}

\textbf{*didascália*}

\emph{Às vésperas do seu aniversário de 49 anos de idade, Gama publica
um testemunho em abono ao caráter e à boa-fé do seu ``venerando mestre e
amigo'', Furtado de Mendonça, famosa autoridade policial de São Paulo,
que havia sido demitido de seu posto em meio a uma crise política. A
carta pode ser lida como uma pá de cal na interpretação que se revela
equivocada do suposto rompimento definitivo que teria se dado entre
ambos, dez anos antes, em 1869. O desagravo de Gama em benefício de
Furtado de Mendonça explicita que mantinham uma relação afetuosa mesmo
depois do conflito deflagrado na década anterior. O testemunho, contudo,
tinha outro agraciado, o também amigo Lins de Vasconcellos, com quem
trabalhou -- em lados opostos ou no mesmo lado da banca -- diversas
vezes. ``Aos homens de bem'', todavia, não era uma carta com
direcionamento genérico ou escrita para simples satisfação de seus
amigos. A carta tinha o objetivo de abaixar a fervura da crise política
denunciando, por um lado, os ``ataques sorrateiros'' que Lins de
Vasconcellos vinha sofrendo; e, por outro lado, solidarizando-se com ele
e com o delegado Furtado de Mendonça, que, envolvido na crise, foi
injustamente demitido pelo presidente da província. }

\begin{center}\rule{0.5\linewidth}{\linethickness}\end{center}

Acostumado a sentir, como meus, as ofensas e os desastres de que são
vítimas os meus amigos, tenho presenciado, com profunda mágoa, as
levianas indelicadezas, se não indignidades, de que se hão servido
inimigos, que parecem jactar-se\footnote{Gabar-se, vangloriar-se.} do
próprio menosprezo, contra o digno e ilustrado sr. dr. Lins de
Vasconcellos.\footnote{Luiz de Oliveira Lins de Vasconcellos
  (1853-1916), nascido em Maceió (AL), foi um advogado, promotor público
  e político, chegando a exercer a presidência da província do Maranhão
  (1879-1880). Na advocacia foi um colaborador em diversas demandas de
  liberdade junto a Luiz Gama, muito embora também tenha atuado, em
  matéria comercial, no polo oposto de Gama.}

Os ataques sorrateiros; as inventivas\footnote{Invencionices, fantasias.}
pungentes; a ofensa ao caráter, sem respeito até à vida íntima de um
homem solteiro, até hoje impoluta\footnote{Sem mancha, honesta.}, não
têm sido poupados.

Esquecem-se de que o ódio e a paixão são cegos e têm o dom da
ubiquidade...\footnote{Onipresença.}

Agora mesmo, entre os pasquins\footnote{Jornal ou panfleto crítico,
  satírico. Pelo contexto da frase, Gama também a estende para além do
  texto escrito, incluindo discursos e discussões.} escritos e verbais
que circulam, atinge-se a ponto mais odioso: procura-se envolver nesta
cena lamentável, própria das aldeias, e das sociedades viciosas, os
nomes de inimigos da vítima, incapazes, pelo seu caráter e critério, de
associarem-se a semelhantes desregramentos; e este é o motivo que
obriga-me imperiosamente a romper o silêncio diante de fatos tão
contristadores\footnote{Desoladores, que entristecem.} que enlutam os
corações nobres.

\_\_\_\_\_

Neste momento (duas e meia horas da tarde) garantem-me que fora demitido
do cargo de delegado de polícia da capital, o meu venerando mestre e
amigo, o exmo. sr. conselheiro Furtado de Mendonça\footnote{Francisco
  Maria de Sousa Furtado de Mendonça (1812-1890), nascido em Luanda,
  Angola, foi subdelegado, delegado, chefe de polícia e secretário de
  polícia da província de São Paulo ao longo de quatro décadas. Foi,
  também, professor catedrático de Direito Administrativo da Faculdade
  de Direito de São Paulo. A relação de Luiz Gama com Furtado de
  Mendonça é bastante complexa, escapando, em muito, aos limites dos
  eventos da demissão de Gama do cargo de amanuense da secretaria de
  polícia, em 1869. Para que se ilustre temporalmente a relação,
  tenhamos em vista que à época do rompimento público, aos finais da
  década de 1860, ambos já se conheciam e trabalhavam juntos há cerca de
  duas décadas; e, mais, Gama não rompeu definitivamente com Furtado de
  Mendonça, como erroneamente indica a historiografia, visto o presente
  artigo, \emph{Aos homens de bem}, que é uma espécie de defesa moral e
  política da carreira de Furtado de Mendonça.}; e que o fato desta
demissão filia-se a outros em que se há envolvido, indébita ou
levianamente, o nome do sr. dr. Lins!

Não quero antecipar defesas, se bem que, de modo algum, eu arrecei-me
das ocorrências e das suas consequências naturais e, sem fazer a mínima
censura ao fato da demissão, que parte do primeiro magistrado da
província, perguntarei, apenas, aos velhos, aos dignos, e aos honrados
paulistas:

Quem, nesta briosa província, mais se dedicou à causa pública do que o
conselheiro Furtado?

Quem melhor defendeu o lar, a segurança e a fortuna dos habitantes desta
cidade?

Quem, com mais segura observância da lei, deu caça aos criminosos e
garantiu os interesses dos oprimidos?

Quem, com maior lealdade, tem servido à causa da justiça, sem
envolver-se nem prejudicar interesses políticos?

Quem melhor do que ele, ou com maior dedicação e desinteresse,
sacrificou a sua saúde e o seu bem-estar no inglório e onerosíssimo
serviço da polícia, desde o ano de 1842?...

Diante de certas catástrofes sociais, os patriotas sinceros, consultando
a própria consciência, encontram justificação para {[}o{]} quanto veem e
contemplam nesta brilhante sentença:

O Estado é como as valas ou as piras dos campos de batalha, onde se
enterram ou queimam tanto os corpos dos heróis, como os dos pusilâmines.

S. Paulo, 19 de Junho de 1879.

LUIZ GAMA.

\textbf{39. CARTA A FRANCISCO ANTONIO DUARTE}\footnote{In: \emph{Jornal
  da Tarde} (SP), Noticiário, 5/7/1879, p.~2. Tiramos da \emph{Tribuna}
  de hoje: ``Nas mãos do exmo. sr. dr. presidente da província prestou
  ontem juramento de cavalheiro da ordem da Rosa o cabo do corpo de
  permanentes, Francisco Antonio Duarte, que em Itu defendeu a cadeia do
  assalto do povo, por ocasião dos acontecimentos daquela cidade, de que
  os nossos leitores têm notícia. O sr. Luiz Gama brindou o condecorado
  com a venera da ordem, que lhe pregou ao peito o comandante daquele
  corpo, enviando com ela a seguinte carta.''}

\textbf{*didascália*}

\emph{Curta e direta, a carta pública ao amigo Francisco Duarte tem a
grande valia de jogar luz a um dos aspectos pouco conhecidos da vida de
Gama: o tempo de quartel, isto é, a carreira militar. A homenagem de
Gama, portanto, revela esse traço estruturante de um homem que foi
soldado e dizia conservar ``certas manias inerentes à farda''. }

\begin{center}\rule{0.5\linewidth}{\linethickness}\end{center}

S. Paulo, 3 de Julho de 1879.

Meu caro tenente-coronel.

Fui soldado e conservo ainda certas manias inerentes à farda\footnote{Aqui
  Gama revela uma curta e valiosa informação pessoal que, ainda antes da
  \emph{Carta a Lúcio de Mendonça}, atesta sua formação militar.}; uma
delas é a predileção pelos antigos colegas.

Li nos jornais que fora condecorado com o hábito da imperial ordem da
Rosa, como cavalheiro, o cabo de esquadra Francisco Antonio Duarte.

Remeto-vos, com esta carta, uma venera\footnote{Insígnia, medalha
  condecorativa, distintivo de honra.} para que o presenteeis com ela.

É uma homenagem que presto ao seu civismo e ao seu nobre caráter.

Vosso amigo obrigadíssimo,

\emph{L. Gama}.

\textbf{40. CAUTELA!} \footnote{In: \emph{Jornal da Tarde} (SP),
  Ineditoriais, 02/07/1879, p.~2.}

\textbf{*didascália*}

\emph{``O artigo é simplesmente saboroso de se ler. O humor finíssimo e
os causos que conta merecem ser lidos linha por linha. Um negociante de
fora de São Paulo vai até a casa de Luiz Gama para lhe cobrar um certo
dinheiro. Lá chegando, estando à presença do líder abolicionista, não o
reconhece como Luiz Gama, haja vista ter feito o negócio com outra
pessoa que assim se apresentava.''Compreendi então o caso``, disse o
nosso Luiz Gama, prevenindo o público de que havia um golpe na praça,
isto é, de que alguém andava se apresentando como Luiz Gama para assim
colher créditos e dinheiro dos desavisados. Aconteceu ao comerciante de
Guarulhos, que foi bater à porta de Gama exigindo-lhe o pagamento de uma
certa quantia, e a um preso que pedia-lhe um recurso criminal que
possibilitaria sua soltura. O arremate certeiro prevenia os inocentes, é
verdade, mas atacava o estelionatário, certamente atento aos passos
do''Luiz Gama - original". }

\begin{center}\rule{0.5\linewidth}{\linethickness}\end{center}

Hoje apareceu, em minha casa, à rua do Rosário, nº 10, sobrado, o sr.
José Alves Ferreira, residente na Conceição dos Guarulhos, e exigiu a
quantia de 240\$, importância de um crédito que confiara a Luiz Gama,
para cobrar.

Apresentei-me surpreendido, pela novidade, ao sr. Ferreira; e ele mais
surpreendido ficou com a minha presença, dizendo que havia tratado com
\emph{Luiz Gama}, e não comigo!...

Compreendi então o caso; e fiquei sabendo que, em S. Paulo, há um só
Luiz Gama, \emph{em dois volumes}; que eu sou uma entidade dupla; tenho
duas formas, dois tamanhos, duas cores e um só nome!

Há dias, um preso, na cadeia, agradeceu-me serviços que eu lhe prestara,
em uma sua apelação, em razão de uma carta que trouxera-me; e que
\emph{há dois meses}, mandara pagar-me os 200\$, por mim exigidos!... e
pedia-me que desse pronto andamento à sua revista!...\footnote{Espécie
  de recurso, cabível para instâncias superiores, em que se discute
  divergência de interpretação do direito.}

Só então eu tive conhecimento de tal negócio, que havia sido cuidado por
\emph{outro} \emph{Luiz Gama}, que não eu!...

Ora, visto está que um destes dois \emph{volumes é incorreto},
\emph{clandestino e de falsa erudição}; que pode ir ter a mãos
desastradas, que, em momento de cólera, o \emph{desencaderne}; e também
pode dar-se o fracasso de, por equívoco, \emph{desencadernado ser bom
volume}!...

Peço, pois, às pessoas que tiverem de tratar negócios com Luiz Gama, que
sejam cautelosas, que procurem, com cuidado, o verdadeiro, para evitar
prejuízos, que lhes poderá causar o falso.

Procurem o Luiz Gama -- \emph{original}; porque há, por aí, em
cavilosas\footnote{Enganosas, fraudulentas.} brumas, um \emph{Luiz Gama}
-- \emph{de imitação}...

Eu não tenho agentes; não autorizei pessoa alguma para contratar em meu
nome.

S. Paulo, 1º de Julho de 1879.

LUIZ GAMA.

\begin{center}\rule{0.5\linewidth}{\linethickness}\end{center}
